% !TeX encoding = UTF-8
% !TeX spellcheck = en_US
% !TeX TXS-program:bibliography = biber -l zh__pinyin --output-safechars %

\documentclass[a4paper,10pt]{article}

\newcommand{\hwNumber}{3}

% Templates: 82ccb576e4df24e5eac4194b76230be360b4f733

% to be `\input` in subfolders,
% ... therefore the path should be relative to subfolders.

\usepackage[UTF8
	,heading=false
	,scheme=plain % English Document
]{ctex}
\usepackage{indentfirst}

\input{../.modules/basics/macros.tex}
\input{../.modules/preamble_base.tex}
\input{../.modules/preamble_notes.tex}

\newcommand{\legacyReference}{{
%	\clearpage\par
%	\quad\clearpage
	\renewcommand{\midquote}{\textbf{PAST WORK, AS TEMPLATE}}
	\newparagraph
}}

% Settings
\counterwithout{equation}{section}
\mathtoolsset{showonlyrefs=false}
%\DeclareTextFontCommand{\textbf}{\sffamily}
\renewcommand{\midquote}{\quad}

% Spacing
\geometry{footnotesep=2\baselineskip} % pre footnote split
\setlength{\parskip}{.5\baselineskip}
\renewcommand{\baselinestretch}{1.15}

%Title
	\posttitle{
		\hfill\Large\ccbyncsajp
		\par\end{flushleft}%
		\vspace*{-.7ex}\hrule%
	}
	\preauthor{\vspace{-1.5ex}%
		\flushleft\itshape%
	}
	\postauthor{\hfill}
	\predate{\noindent\ttfamily Compiled @ }
	\postdate{\vspace{.5ex}}

	\title{Advanced QFT \textnumero\hwNumber}
	\author{\signature Bryan}
	\date{\today}

% List
	\setlist*{
		listparindent=\parindent
		,labelindent=\parindent
		,parsep=\parskip
		,itemsep=1.2\parskip
	}

\input{../.modules/basics/biblatex.tex}

%%% ID: sensitive, do NOT publish!
%\InputIfFileExists{../id.tex}{}{}
\usepackage{cancel}

\begin{document}
\maketitle
\pagestyle{headings}
\pagenumbering{arabic}
\thispagestyle{empty}

\vspace*{-1.5\baselineskip}

\section{BRST Symmetry}
	The BRST transformation of $c^a$ ghost is:
	\begin{equation}
		\var{c^a} = \frac{1}{2}\,
			f\id{^a_{bc}} c^b c^c \Lambda,\quad
		\var{c} = \var{c^a} T_a
		= \frac{1}{2}\,\bqty{c,c\Lambda}
	\end{equation}
	$D_\mu = \pdd{\mu} + A^a_\mu\,T_a$, $T_a$ acts on $c^a$ by adjoint representation: $
		\pqty{T_a}\id{^c_b}\, c^b
		= f\id{^c_{ab}} c^b
	$, i.e.\ %
	\begin{gather}
		T_a\cdot c
		= \pqty{T_a}\id{^c_b}\, c^b\, T_c
		= f\id{^c_{ab}} T_c\,c^b
		= [T_a, T_b]\,c^b
		= [T_a, c], \\
		D_\mu c
		= \pdd{\mu} c + [A_\mu, c]
		= [D_\mu, c], \\
		D_\mu \var{c}
		= \pdd{\mu} \var{c}
			+ \bqty\big{
				A_\mu, \var{c}
			},\\[1ex]
	\begin{aligned}
		(D_\mu \var{c})^a
		&= \pdd{\mu} \var{c}^a
			+ \bqty\big{
				A_\mu, \var{c}
			}^a \\
		&= \pdd{\mu} \var{c}^a
			+ A_\mu^c\,\bqty\big{
				T_c, T_b
			}^a \var{c}^b \\
		&= \pdd{\mu} \var{c}^a
			+ A_\mu^c\,f\id{^a_{cb}}
				\var{c}^b \\
		&= \pdd{\mu} \var{c}^a
			+ A_\mu^c\,\pqty{T_c}\id{^a_b}
				\var{c}^b \\
		&= D_\mu (\var{c}^a),
	\end{aligned}\\[.5ex]
	\text{i.e.}\quad
		(D_\mu \var{c})^a
		- \frac{1}{2}\,D_\mu \pqty{
				f\id{^a_{bc}} c^b c^c \Lambda
			}
		= (D_\mu \var{c})^a
		+ \frac{1}{2}\,D_\mu \pqty{
				f\id{^a_{bc}} c^b \Lambda c^c
			}
		= 0
	\end{gather}
	
%	Note that the commutator $[,] \equiv [,]_\circ$ is defined with composition of operators, i.e.\ $
%		[\pdd{\mu}, c]
%		= \pdd{\mu}\circ c - c\circ \pdd{\mu}
%		= (\pdd{\mu} c)
%	$. By Jacobi identity, we have:
%	\begin{equation}
%	\begin{aligned}
%		D_\mu \var{c}
%		&= \frac{1}{2}\,\bqty\big{
%			D_\mu, [c,c\Lambda]
%		} \\
%		&= \frac{1}{2}\,\pqty\Big{
%			\bqty\big{
%				[D_\mu,c], c\Lambda
%			}
%			- \bqty\big{
%				[D_\mu,c\Lambda], c
%			}
%		} \\
%		&= \bqty\big{
%				[D_\mu,c], c\Lambda
%			}
%		= [D_\mu c,c\Lambda]
%	\end{aligned}
%	\end{equation}
\section{Relativistic Particle}
	\vspace*{-2\baselineskip}
	\begin{gather}
		L_q
		= L + L_{gf} + L_{gh},\\[1ex]
		L = \frac{1}{2e} \pqty{
				\frac{1}{c_0}\dv{X}{t}
			}^2 - \frac{e}{2}\,m^2 c_0^4,\\
		L_{gh} = - e\dot{b}c
	\end{gather}
	
	\begin{itemize}
	\item For $t\mapsto t' = t - \xi(t)$, we have gauge transformation: $
		\var{X^\mu}
		= \xi\dot{X}^\mu,\ %
		\var{e}
		= \dv{t} \pqty\big{e\xi},\ 
		\var{L}
		= \dv{t} \pqty\big{\xi L}
	$, replace $\xi\mapsto c\Lambda$, and we have BRST transformation:
	\begin{equation}
		\var{X^\mu}
		= c\Lambda\dot{X}^\mu
		= c\dot{X}^\mu \Lambda,\quad
		\var{e}
		= \dv{t} \pqty\big{ec\Lambda}
		= \dv{t} \pqty\big{ec} \Lambda
	\end{equation}
	
	\item Assume nilpotency, and we have:
	\begin{gather}
	\begin{aligned}
		0 = \var_\Lambda
			\var_{\Lambda'} X^\mu
		&= \pqty\big{
				(\var_\Lambda {c})\,\dot{X}^\mu
				+ c\,\var_\Lambda \dot{X}^\mu
			} \Lambda' \\
		&= \pqty{
				(\var_\Lambda {c})\,\dot{X}^\mu
				+ c\,\pqty\big{
					\dot{c} \dot{X}^\mu
					+ \cancel{c\ddot{X}^\mu}
				} \Lambda
			} \Lambda' \\
		&= \pqty{
				\var_\Lambda {c} 
				+ c\dot{c} \Lambda
			}\,\dot{X}^\mu \Lambda',
	\end{aligned}\\[1ex]
	\allowdisplaybreaks
	\boxed{
		\var_\Lambda c
		= - c\dot{c} \Lambda
	}\\
	\begin{aligned}
		\var_\Lambda
			\var_{\Lambda'} e
		&= \dv{t} \pqty\big{
				(\var_\Lambda e)\, c
				+ e\,\var_\Lambda c
			} \Lambda' \\
		&= \dv{t} \pqty{
				\dv{t} \pqty\big{ec} \Lambda c
				- e\,c\dot{c}\Lambda
			} \Lambda' \\
		&= \dv{t} \pqty\big{
				e \dot{c} \Lambda c
				- e\,c\dot{c}\Lambda
			} \Lambda' = 0
	\end{aligned}
	\end{gather}
	
	\item The BRST transformation for $c$ is also nilpotent:
	\begin{equation}
	\begin{aligned}
		\var_\Lambda
			\var_{\Lambda'} c
		&= -\pqty\big{
				(\var_\Lambda c)\,\dot{c}
				+ c\,\var_\Lambda \dot{c}
			} \Lambda' \\
		&= -\pqty\big{
				- c\dot{c}\Lambda\,\dot{c}
				- c\,c\dot{c}\Lambda
			} \Lambda' = 0
	\end{aligned}
	\end{equation}
	
	\item Gauge fixing $
		f = e(t) - 1 = 0
	$ can be imposed by:
	\begin{gather}
		\delta\bqty\big{f}
		\sim \int \DD{d(t)}\,
			\exp\pqty{
				i\int \dd{t} d(t)\,f(t)
			},\\[.5ex]
		L_{gf}
		= d(t) f(t)
		= d(t)\,\pqty\big{
				e(t) - 1
			}
	\end{gather}
	The quantum action is $
		S_q = \int \dd{t} L_q,\ %
		L_q
		= L[X,e] + L_{gf}[e,d] + L_{gh}[e,b,c]
	$. We want $S_q$ to be BRST invariant, which will help determine transformation rules for $b,d$; consider: 
	\begin{equation}
	\begin{aligned}
		\var \pqty{
			L_{gf} + L_{gh}
		}
		&= (e-1) \var{d} + d\var{e}
			- \var \pqty\big{e\dot{b}c} \\
		&= (e-1) \var{d}
			+ d(t)\,
				\dv{t}\pqty\big{ec}\,\Lambda
			+ \pqty\big{ec} \var{\dot{b}} \\
		&= (e-1) \var{d}
			+ \dv{t} \pqty\big{ec}\,\pqty\big{
				d(t)\,\Lambda
				- \var{b}
			}
			+ \dv{t} \pqty\big{ec \var{b}}
	\end{aligned}
	\label{eq:var_aux_fields}
	\end{equation}
	We find a natural choice of $
		\var{b},\var{d}
	$:
	\begin{equation}
		\var{d} = 0,\quad
		\var{b} = d(t)\,\Lambda,\quad
		\var \pqty{
			L_{gf} + L_{gh}
		} = \dv{t} \pqty\big{ec \var{b}}
	\end{equation}
	
	\item The complete quantum action is BRST invariant, since:
	\begin{gather}
		\var{L}
		= \dv{t} \pqty\big{\xi L}_{
				\xi\mapsto c\Lambda
			}
		= \dv{t} \pqty\big{cL} \Lambda,\quad
		\var \pqty{L_{gf} + L_{gh}}
		= \dv{t} \pqty\big{ec \var{b}},\\[.5ex]
		\var{S_q}
		= \int \dd{t} \var{L_q}
		= \int \dd{t} \pqty\Big{
			\var{L}
			+ \var \pqty{L_{gf} + L_{gh}}
		} = 0
	\end{gather}
	
	\item Note that $
		\fdv{S_q}{d} = 0
		\ \Rightarrow\ %
		f = e - 1 = 0
	$. Moreover, 
	\begin{gather}
		\fdv{S_q}{e} = 0
		\quad\Longrightarrow\quad
		- \frac{1}{2}\,\pqty{
				\frac{1}{e^2 c_0^2} \dot{X}^2
				+ m^2 c_0^4
			} + d - \dot{b}c = 0,\\[.8ex]
%		\textsl{On shell:}\quad
		d = d[X,b,c]
		= \frac{1}{2}\,\pqty\big{
				\tfrac{1}{c_0^2} \dot{X}^2
				+ m^2 c_0^4
			} + \dot{b}c
	\end{gather}
	Therefore, it is convenient to consider the reduced Lagrangian $
		L_q[X,b,c] = (L_q)_{
				e=1,\, d=d[X]
			}
	$, where $e,d$ are integrated out\footnote{
		Reference: \textit{Polchinski}. 
	}. The symmetries are thus reduced to:
	\begin{gather}
	\allowdisplaybreaks
		\var{X^\mu}
		= c\dot{X}^\mu \Lambda,\quad
		\var{b}
		= d[X,b,c] \Lambda,\quad
		\var{c}
		= - c\dot{c} \Lambda,
	\label{eq:reduced_sym} \\
		\var{L_q}
		= \dv{t} \pqty\Big{
				cL\Lambda + ec\,\var{b}
			}_{\!e=1} \!\!
		= \dv{t} \pqty{
				cL_{e=1}
				+ \frac{c}{2}\,\pqty\big{
						\tfrac{1}{c_0^2} \dot{X}^2
						+ m^2 c_0^4
					}
			} \Lambda
		= \dv{t} \pqty{
				c\,\pqty\big{
					\tfrac{1}{c_0} \dot{X}
				}^{\!2\,}
			} \Lambda,\\[1ex]
		L_q
		= \frac{1}{2}\,\pqty\big{
				\tfrac{1}{c_0} \dot{X}
			}^{\!2\,}
			- \frac{1}{2}\,m^2 c_0^4
			- \dot{b} c
	\end{gather}
	
	On the other hand, the \textit{on-shell} variation is given by:
	\begin{gather}
		\var_0{L_q}
		= \dv{t} \pqty{
				\pdv{L_q}{\dot{X}^\mu} \var{X^\mu}
				+ \pdv{L_q}{\dot{b}} \var{b}
			}
		= \dv{t} \Bqty{c\,\pqty{
				\pqty\big{
					\tfrac{1}{c_0} \dot{X}
				}^{\!2\,}
				+ d[X,b,c]
			}} \Lambda,
	\end{gather}
	The $\dot{b}c$ term in $d[X,b,c]$ is killed by the $c$ multiplication: $c\,d[X,b,c] = c\,d[X]$. 
	Therefore, the canonical BRST charge $Q$ is given by:
	\begin{gather}
		0 = \var_0{L_q} - \var{L_q}
		= \dv{Q}{t} \Lambda
		= \dv{t} \pqty\big{c\,d[X]} \Lambda,\\[1ex]
		Q = c\,d[X]
		= \frac{c}{2}\,\pqty\big{
				\tfrac{1}{c_0^2} \dot{X}^2
				+ m^2 c_0^4
			}
	\end{gather}
	
	\item Note that $
		L_{gh}
		= -\dot{b} c
		= b\dot{c} - \dv{t} \pqty\big{bc}
		\sim b\dot{c}
	$; for future convenience, let's replace $
		(-\dot{b} c)\mapsto b\dot{c}
	$ in the Lagrangian, and we have:
	\begin{equation}
		p_\mu = \pdv{L}{\dot{X}^\mu}
		= \tfrac{1}{c_0^2} \dot{X}_\mu,\quad
		p_c = \pdv{L}{\dot{c}}
		\equiv \pqty\big{b\dot{c}}
			\lvec{\pdv{\dot{c}}}
		= b
	\end{equation}
	Here we adopt the ``right'' derivative convention, in this case the Hamiltonian:
	\begin{equation}
		H = p_\mu \dot{X}^\mu
			+ p_c\,\dot{c} - L_q
		= \frac{c_0^2}{2}\,p_\mu p^\mu
			+ \frac{1}{2}\,m^2 c_0^4
		= \frac{1}{2}\,\pqty{
				p^2 c_0^2 + m^2 c_0^4
			}
	\end{equation}
	
	\item We have:
	\begin{equation}
		Q = \frac{c}{2}\,\pqty\big{
				\tfrac{1}{c_0^2} \dot{X}^2
				+ m^2 c_0^4
			}
		= \frac{c}{2}\,\pqty{
				p^2 c_0^2 + m^2 c_0^4
			}
		= cH
	\end{equation}
	After canonical quantization, $p_\mu, p_c$ and $H$ are promoted to Hermitian operators, and:
	\begin{equation}
		Q^2 = cH\cdot cH = 0
	\end{equation}
	
	\item
%	The $bc$ system generates a two-state system: 
%	\begin{equation}
%		\newcommand{\spaced}[1]{\ \ #1\ \ }
%		0
%		\xleftarrow{\spaced{b}}
%		\ket{\downarrow}
%		\xrightleftharpoons[\spaced{b}]{c}
%		\ket{\uparrow}
%		\xrightarrow{\spaced{c}}
%		0
%	\end{equation}
%	Therefore, all states are (at least) doubly degenerate. Note that $\ket{\uparrow}$ and $\ket{\downarrow}$ states are isomorphic and orthogonal to each other, hence we need only one set of states, e.g.\ $\ket{\psi} = \ket{\psi,\downarrow}$, to describe physical process\footnote{
%		Reference: \textit{Polchinski}. 
%	}; this can be achieved by imposing an additional condition, e.g.\ $b\ket{\psi} = 0$. 
%\pagebreak[3]
%	
%	Furthermore, 
	Note that:
	\begin{gather}
		[p_\mu,X^\nu]
		= -i\delta^\nu_\mu,\quad
		[p_\mu,\mcal{F}(X)]
		= -i\pdd{\mu} \mcal{F}(X),
%		\\[.5ex]
%		\frac{c}{2}\,[p^2 c_0^2,X^\mu]
%		= -c_0^2\,\frac{c}{2}\,[X^\mu,p_\nu p^\nu]
%		= -c_0^2\,\frac{c}{2}\,\pqty{
%				i\,2p^\mu
%			}
%		= -ic\,(c_0^2\,p^\mu)
%		= -ic\,\pdd{t} X^\mu,\\
%		[Q,F(X)]
%		= c\,\pqty{
%				-i\pdd{t} + m^2 c_0^4
%			}\, F(X),
	\end{gather}
	i.e.\ $p_\mu$ acts on $\mcal{F}(X)$ by $X$--derivative; from the path integral perspective, we have:
	\begin{gather}
		\allowdisplaybreaks
		\ave[\big]{p_\mu\,\mcal{F}(X)}
		= \int\DD{p}\,\DD{X}\,\DD{b}\,\DD{c}\,
			e^{iS[p,X,b,c]}\,
			p_\mu\,\mcal{F}(X), \\
		S[p,X,b,c] = \int \dd{t} \pqty{
				p_\mu \dot{X}^\mu
				+ \dot{b}c
				- H[p]
			},\\
		\int \dd{t'} \fdv{S}{X^\mu(t')}
		= \int \dd{t'} \int \dd{t} 
			p_\mu\,\pdd{t}\,\delta(t-t')
		\sim -\int \dd{t'} \dot{p}_\mu(t')
		= -p_\mu,\\
	\begin{aligned}
		\ave[\big]{p_\mu\,\mcal{F}(X)}
		&= \int \dd{t'}
			\int\DD{p}\,\DD{X}\,\DD{b}\,\DD{c}\,
			\pqty{
				i\,\fdv{X^\mu(t')}\,
				e^{iS[p,X,b,c]}
			}\,\mcal{F}(X) \\
		&= \int\DD{p}\,\DD{X}\,\DD{b}\,\DD{c}\,
			e^{iS[p,X,b,c]}\,
			\int \dd{t'}
			\pqty{
				-i\,\fdv{X^\mu(t')}\,
			} \mcal{F}(X),
	\end{aligned}\\
		p_\mu\,\mcal{F}(X)
		\sim \int \dd{t'}
			\pqty{
				-i\,\fdv{X^\mu(t')}\,
			}\,\mcal{F}(X)
		= -i\,\pdv{X^\mu} \mcal{F}(X)
	\end{gather}
	
	For $
		e^{ik_\mu X^\mu} \ket{0},\ %
		\mcal{F}(X) = e^{ik_\mu X^\mu}
	$, it is $Q$--closed iff.\ %
	\begin{gather}
	\begin{aligned}
		0 = Q\,e^{ik_\mu X^\mu} \ket{0}
%		= [Q,e^{ik_\mu X^\mu}] \ket{0}
%		= c\,\pqty{
%				-ik_\nu p^\nu + m^2 c_0^4
%			} \pqty\big{
%				e^{k_\mu X^\mu} \ket{0}
%			} \\
%		&= c\,\pqty{
%				-ik_\nu\,(-ik^\nu)
%				+ m^2 c_0^4
%			} \pqty\big{
%				e^{k_\mu X^\mu} \ket{0}
%			} \\
		&= \frac{c}{2}\,\pqty{
				p^2 c_0^2 + m^2 c_0^4
			} e^{ik_\mu X^\mu} \ket{0} \\
		&= \frac{c}{2}\,\pqty{
				k^2 c_0^2 + m^2 c_0^4
			} \pqty\big{
				e^{ik_\mu X^\mu} \ket{0}
			},
	\end{aligned}\\[1ex]
		k^2 c_0^2 = k_\mu k^\mu c_0^2
		= - E^2 + \vb{k}^2 c_0^2
		= - m^2 c_0^4,
	\end{gather}
	Or $
		E^2 = \vb{k}^2 c_0^2 + m^2 c_0^4
	$. This is the dispersion relation of a relativistic particle. 
	\end{itemize}

\printbibliography[%
%	title = {参考文献} %
	,heading = bibintoc
]
\end{document}
