% !TeX encoding = UTF-8
% !TeX spellcheck = en_US
% !TeX TXS-program:bibliography = biber -l zh__pinyin --output-safechars %

\documentclass[a4paper,10pt]{article}

\newcommand{\hwNumber}{1}

% Templates: 82ccb576e4df24e5eac4194b76230be360b4f733

% to be `\input` in subfolders,
% ... therefore the path should be relative to subfolders.

\usepackage[UTF8
	,heading=false
	,scheme=plain % English Document
]{ctex}
\usepackage{indentfirst}

\input{../.modules/basics/macros.tex}
\input{../.modules/preamble_base.tex}
\input{../.modules/preamble_notes.tex}

\newcommand{\legacyReference}{{
%	\clearpage\par
%	\quad\clearpage
	\renewcommand{\midquote}{\textbf{PAST WORK, AS TEMPLATE}}
	\newparagraph
}}

% Settings
\counterwithout{equation}{section}
\mathtoolsset{showonlyrefs=false}
%\DeclareTextFontCommand{\textbf}{\sffamily}
\renewcommand{\midquote}{\quad}

% Spacing
\geometry{footnotesep=2\baselineskip} % pre footnote split
\setlength{\parskip}{.5\baselineskip}
\renewcommand{\baselinestretch}{1.15}

%Title
	\posttitle{
		\hfill\Large\ccbyncsajp
		\par\end{flushleft}%
		\vspace*{-.7ex}\hrule%
	}
	\preauthor{\vspace{-1.5ex}%
		\flushleft\itshape%
	}
	\postauthor{\hfill}
	\predate{\noindent\ttfamily Compiled @ }
	\postdate{\vspace{.5ex}}

	\title{Advanced QFT \textnumero\hwNumber}
	\author{\signature Bryan}
	\date{\today}

% List
	\setlist*{
		listparindent=\parindent
		,labelindent=\parindent
		,parsep=\parskip
		,itemsep=1.2\parskip
	}

\input{../.modules/basics/biblatex.tex}

%%% ID: sensitive, do NOT publish!
%\InputIfFileExists{../id.tex}{}{}

\begin{document}
\maketitle
\pagestyle{headings}
\pagenumbering{arabic}
\thispagestyle{empty}

\vspace*{-1.5\baselineskip}

\section{Symmetry \& Noether's Theorem}
\subsection{2D $\sigma$-Model}
	\vspace{-.8\baselineskip}
	\begin{equation}
		\mcal{L} = -\frac{1}{2}\,
			\eta_{\alpha\beta}\,
			\eta_{\mu\nu}\,
			\pd^\alpha X^\mu\,
			\pd^\beta X^\nu
		= -\frac{1}{2}\,
			\pd^\alpha X_\mu\,
			\pd_\alpha X^\mu,\quad
		X^\mu\in\mbb{R}^{1,D-1}
	\end{equation}
	\begin{itemize}
	\item For $
		\var{X}^\mu
		= a^\mu + \lambda\id{^\mu_\nu} X^\nu
	$, the Lagrangian (density) transforms as follows:
	\begin{equation}
	\begin{aligned}
		\var{\mcal{L}}
		&= -\pd^\alpha X_\mu\,
			\pd_\alpha \var{X^\mu} \\
		&= -\pd^\alpha X_\mu\,
			\pd_\alpha \pqty{
				a^\mu
				+ \lambda\id{^\mu_\nu} X^\nu
			} \\
		&= -\pd^\alpha X_\mu\,
			\pqty{
				\pdd{\alpha} a^\mu
				+ X^\nu\,
					\pdd{\alpha} \lambda\id{^\mu_\nu}
				+ \lambda\id{^\mu_\nu}\,
					\pdd{\alpha} X^\nu
			} \\
		&= -\pd^\alpha X_\mu\,\pdd{\alpha} a^\mu
		- \pd^\alpha X^\mu\,
			\pdd{\alpha} X^\nu\,
			\lambda_{\mu\nu}
		- X^\nu\,\pd^\alpha X^\mu\,
			\pdd{\alpha} \lambda_{\mu\nu} \\
		&= -\pd^\alpha X_\mu\,\pdd{\alpha} a^\mu
		- \pd^\alpha X^\mu\,
			\pdd{\alpha} X^\nu\,
			\lambda_{(\mu\nu)}
		- X^\nu\,\pd^\alpha X^\mu\,
			\pdd{\alpha} \lambda_{\mu\nu}
		\label{eq:lorentz_variation}
	\end{aligned}
	\end{equation}
	Since $a^\mu$ and $\lambda\id{^\mu_\nu}$ are independent, imposing $\var{L} = 0$ yields $\pdd{\alpha} a^\mu = 0,\,a = \mrm{const}$. Furthermore, if $\var{L} = 0$ is to hold for arbitrary $X^\mu$ fields, then $\pdd{\alpha} \lambda_{\mu\nu} = 0,\,\lambda_{(\mu\nu)} = 0$, i.e.\ $\lambda_{\mu\nu}$ is constant and anti-symmetric over its indices. 
	
%	\item Variation over any generic $\var{X}$ yields:
%	\begin{equation}
%	\begin{aligned}
%		\var{S} = \int \dd[2]{x}
%			\var{\mcal{L}}
%		&= - \int \dd[2]{x}
%			\pd^\alpha X_\mu\,
%			\pd_\alpha \var{X^\mu} \\
%		&= \int \dd[2]{x}
%			\pqty{\pd^\alpha \pdd{\alpha} X_\mu}
%			\var{X^\mu}
%		- \int \dd[2]{x} \pdd{\alpha} \pqty\big{
%			\pd^\alpha X_\mu\,
%			\var{X^\mu}
%		} \\
%	\end{aligned}
%	\end{equation}
%	This gives the equation of motion (EOM) $
%		\pd^\alpha \pdd{\alpha} X = 0
%	$. 
	
	\item Promote $
		\var{X}\mapsto \epsilon(x)\var{X}
		= \epsilon(x)\,\pqty{
			a^\mu + \lambda\id{^\mu_\nu} X^\nu
		}
	$, with $\epsilon(x)$ some localized bump function; using \eqref{eq:lorentz_variation} and considering \textit{on-shell} variation, we have:
	\begin{equation}
	\begin{aligned}
		0 = \var{S}
		&= - \int \dd[2]{x} \pqty{
			\pd^\alpha X_\mu\,
				a^\mu\,
				\pdd{\alpha}\epsilon
			+ X^\nu\,\pd^\alpha X^\mu\,
				\lambda_{\mu\nu}\,
				\pdd{\alpha}\epsilon
		} \\
		&= - \int \dd[2]{x} \pqty{
			\pd^\alpha X_\mu\,
				a^\mu\,
			+ X_{[\nu}\,\pd^\alpha X_{\mu]}\,
				\lambda^{[\mu\nu]}\,
		}\,\pdd{\alpha}\epsilon
	\end{aligned}
	\end{equation}
	It is then evident (after partial integration) that the following currents are conserved; they are the Noether currents associated with $a^\mu$ and $\lambda^{[\mu\nu]}$:
	\begin{equation}
		j^\alpha_\mu
		= -\pd^\alpha X_\mu,\quad
		j^\alpha_{\mu\nu}
		= -X_{[\nu}\,\pd^\alpha X_{\mu]}
		= \frac{1}{2}\,\pqty{
			X_\mu\,\pd^\alpha X_\nu
			- X_\nu\,\pd^\alpha X_\mu
		}
	\end{equation}
	
	Conserved charge $
		Q = \int\dd[2]{x} j^0(x)
	$, we have:
	\begin{equation}
		P_\mu = - \int\dd{x^1} \pd^0 X_\mu
		= \int\dd{x^1} \pd_0 X_\mu,\quad
		M_{\mu\nu}
		= \frac{1}{2}\,\int\dd{x^1}\pqty{
			X_\nu\,\pd_0 X_\mu
			- X_\mu\,\pd_0 X_\nu
		}
	\end{equation}
	They can be interpreted as spacetime momentum and spacetime angular momentum. 
	\qedfull
	\end{itemize}
%\pagebreak[3]
\subsection{Real Scalar in $(3+1)$\,D}
	\vspace{-.8\baselineskip}
	\begin{equation}
		\mcal{L}
		= -\frac{1}{2}\,
			\pd^\mu\phi\,
			\pdd{\mu}\phi
		-\frac{1}{2}\,m^2\phi^2
	\end{equation}
	\begin{itemize}
	\item For $\phi$: scalar, under $
		x' = \lambda\circ x
	$, $\phi(x)\mapsto \phi'(x)$, while:
	\begin{equation}
		\phi'(x') = \phi(x)
		\ \Longrightarrow\ %
		\phi'(x) = \phi(\lambda^{-1}\circ x)
	\end{equation}
	For $\lambda\sim \lambda\id{^\mu_\nu}$: Lorentz transformation, $
		\eta_{\mu\nu}
			\lambda\id{^\mu_\rho}
			\lambda\id{^\nu_\sigma}
		= \eta_{\rho\sigma}
	$, or equivalently, $
		(\lambda^{-1})\id{^\mu_\nu}
		= \lambda\id{_\nu^\mu}
	$. Therefore,
	\begin{equation}
		\phi'(x^\mu) = \phi(\lambda^{-1}\circ x^\mu)
		= \phi(x^\nu\lambda\id{_\nu^\mu})
	\end{equation}
%\pagebreak[3]
	
	\item Under $
		x'^\mu = \lambda\id{^\mu_\nu} x^\nu
%		,\ %
%		\phi(x)\mapsto\phi'(x)
	$, we have:
	\begin{equation}
	\begin{aligned}
		\mcal{L}'(x')
		&= - \frac{1}{2}\,
			\pd'^\mu\phi'(x')\,
			\pdd{\mu}'\phi'(x')
		- \frac{1}{2}\,m^2\phi'^2(x') \\
		&= - \frac{1}{2}\,
			\pd'^\mu\phi(x)\,
			\pdd{\mu}'\phi(x)
		- \frac{1}{2}\,m^2\phi^2(x) \\
		&= - \frac{1}{2}\,
			\eta^{\mu\nu}
			\pdv{x^\rho}{x'^\mu}\,
			\pdd{\rho}\phi(x)\,
			\pdv{x^\sigma}{x'^\nu}\,
			\pdd{\sigma}\phi(x)
		- \frac{1}{2}\,m^2\phi^2(x) \\
		&= - \frac{1}{2}\,
			\eta^{\rho\sigma}
			\pdd{\rho}\phi(x)\,
			\pdd{\sigma}\phi(x)
		- \frac{1}{2}\,m^2\phi^2(x) \\[.5ex]
		&= \mcal{L}(x)
	\end{aligned}
	\end{equation}
	Here we've used $
		\eta^{\mu\nu}
		\pdv{x^\rho}{x'^\mu}\,
		\pdv{x^\sigma}{x'^\nu}
		= \eta^{\mu\nu}
			\lambda\id{_\mu^\rho}
			\lambda\id{_\nu^\sigma}
		= \eta^{\rho\sigma}
	$. Furthermore, $
		S' = \int \dd[4]{x} \mcal{L}'(x)
		= \int \dd[4]{x'} \mcal{L}'(x')
		= \int \dd[4]{x'} \mcal{L}(x)
		= \int \dd[4]{x} \mcal{L}(x)
		= S
	$, hence the action is invariant under Lorentz transformation. 
	
	\item Consider an infinitesimal Lorentz transformation: $\lambda \sim \idty + \omega$, then $
		\eta_{\mu\nu}
			\lambda\id{^\mu_\rho}
			\lambda\id{^\nu_\sigma}
		= \eta_{\rho\sigma}
	$ implies that $\omega_{\mu\nu}$ is anti-symmetric: $
		\omega_{\mu\nu} + \omega_{\nu\nu} = 0
	$. For $
		\var{x}^\mu
		= \omega\id{^\mu_\nu} x^\nu
	$, we have:
	\begin{equation}
		\var{\phi} = - \pdv{\phi}{x^\mu}
			\var{x}^\mu
		= - \omega\id{^\mu_\nu} x^\nu\,
			\pdd{\mu}\phi
%		= - \omega_{\rho\sigma} x^\sigma\,
%			\pd^\rho \phi
	\end{equation}
	
	To obtain the corresponding Noether charges, we can simply repeat the operations done in our previous problem; alternatively, we can try to derive a general recipe\footnote{
		References: \arxiv{1601.03616} and \textit{Tong:} \url{http://damtp.cam.ac.uk/user/tong/qft.html}
	}: for $
		\mcal{L} = \mcal{L}(\phi,\pdd{\mu}\phi)
	$ and $
		S = \int \dd[4]{x} \mcal{L}
	$, we have:
	\begin{equation}
	\begin{aligned}
		\var{S}
		&= \int\dd[4]{x} \var{\mcal{L}} \\
		&= \int\dd[4]{x} \pqty{
			\pdv{\mcal{L}}{\phi} \var{\phi}
			+ \pdv{\mcal{L}}{(\pdd{\mu}\phi)}
				\var{\pdd{\mu}\phi}
		} \\[.5ex]
		&= \int\dd[4]{x} \pqty{
				\pdv{\mcal{L}}{\phi}
				- \pdd{\mu}
				\pdv{\mcal{L}}{(\pdd{\mu}\phi)}
			} \var{\phi}
		+ \int\dd[4]{x}
			\pdd{\mu} \pqty{
				\pdv{\mcal{L}}{(\pdd{\mu}\phi)}
				\var{\phi}
			}
	\end{aligned}
	\end{equation}
	If we vary $S$ w.r.t.\ a symmetry of the system, we will have $\var{\mcal{L}} = \pdd{\mu} K^\mu$ some total derivative; when on-shell, such variation gives the conserved current with boundary term $K^\mu$:
	\begin{equation}
		j^\mu
		= \pdv{\mcal{L}}{(\pdd{\mu}\phi)}
			\var{\phi} - K^\mu
	\end{equation}
	
	Back to our Lorentz transformation $
		\var{\phi}
		= -\omega\id{^\mu_\nu} x^\nu\pdd{\mu}\phi
	$, we have symmetry variation:
	\begin{equation}
		\var{\mcal{L}}
		= - \omega\id{^\mu_\nu} x^\nu
			\pdd{\mu} \mcal{L}
		= - \pdd{\mu} \pqty{
			\omega\id{^\mu_\nu} x^\nu
			\mcal{L}
		}
	\end{equation}
	We can write this down without explicit calculations, since we know $\mcal{L}$ itself is a Lorentz scalar, and that's how scalar transforms under Lorentz transformations. \\
	This gives a boundary term $
		K^\mu = -\omega\id{^\mu_\nu} x^\nu \mcal{L}
	$, and the Noether current and its corresponding conserved charge can be calculated as follows:
	\begin{gather}
		j^\mu = -\omega\id{^\sigma_\nu} x^\nu
		\pqty{
			\pdv{\mcal{L}}{(\pdd{\mu}\phi)}
				\,\pdd{\sigma} \phi
			- \delta^\mu_\sigma \mcal{L}
		},\\[1ex]
		Q = \int \dd[3]{x} j^0
		= - \omega\id{^\sigma_\nu} \int \dd[3]{x}
		x^\nu \pqty{
			\pdd{0}\phi\,\pdd{\sigma}\phi
			- \delta^0_\sigma \mcal{L}
		},
	\end{gather}
	
	Note that $\omega\id{^\mu_\nu}$ is arbitrary, therefore $Q$ can be decomposed into independent charges:
	\begin{equation}
		Q
		= \frac{1}{2}\,
			\omega_{\mu\nu} M^{\mu\nu},\quad
		M^{\mu\nu}
		= - \int \dd[3]{x}
			2x^{[[\mu} \pqty{
				\pdd{0}\phi\,\pd^{\nu]}\phi
				- \eta^{\nu] 0} \mcal{L}
			},
	\end{equation}
	The indices of $M^{\mu\nu}$ are anti-symmetrized to match the degrees of freedom in $\omega_{\mu\nu}$. Note that the $\mcal{L}$ term only appears when one of the indices is 0. 
	
	Note that the canonical momentum:
	\begin{equation}
		\varpi
		= \pdv{\mcal{L}}{\dot{\phi}}
		= \dot{\phi}
		= \pd_0\phi
	\end{equation}
	It is thus natural to re-organize $M^{0i}$ in the following way:
	\begin{equation}
	\begin{aligned}
		M^{0i} = -M^{i0}
		&= - \int \dd[3]{x} \pqty{
			\varpi\,\pqty{
				x^0 \pd^i
				- x^i \pd^0
			}\,\phi
			- x^i \mcal{L}
		} \\
		&= - \int \dd[3]{x} \pqty{
			x^0 \varpi\,\pd^i \phi
			+ x^i \pqty\big{
				\varpi\dot{\phi} - \mcal{L}
			}
		} \\
		&= - \int \dd[3]{x} \pqty{
			x^0 \varpi\,\pd^i \phi
			+ x^i \mcal{H}
		} \\
	\end{aligned}
	\end{equation}
	
	We've obtained an interesting result: the expression for the boost generator $M^{i0}$ contains the Hamiltonian density $\mcal{H}$, weighted by the radial distance $x^i$. This is natural since a boost does indeed contains time evolution for excitations away from the origin. It's an important result utilized by the so-called \textit{Rindler decomposition}; in fact, $M^{i0}$ becomes the Hamiltonian for an accelerated observer in the Rindler patch\footnote{
		See the lecture notes of Tom Hartman: \http{hartmanhep.net/topics2015/gravity-lectures.pdf} or Daniel Harlow \arxiv{1409.1231}. 
	}. 
	
	For $M^{ij}$, we have:
	\begin{equation}
		M^{ij}
		= - \int \dd[3]{x}
			\dot{\phi}\,\pqty{
				x^i \pd^j
				- x^j \pd^i
			}\,\phi
	\end{equation}
	This is interpreted as the angular momentum of the field $\phi$. Suppose $\phi$ is a wave packet localized around $\vb{x}$ with momentum $\approx \vb{k}$, then we have the classical angular momentum up to some factor:
	\begin{equation}
		M^{ij}
		\sim \pqty{
				x^i k^j
				- x^j k^i
			} \int \dd[3]{x}
				E \phi^2,
	\quad
		E = \sqrt{\vb{k}^2 + m^2}
	\end{equation}
	
	The $
		\int \dd[3]{x} E \phi^2
	$ factor in the above expression is an $\order{1}$ normalization constant for a particle-like wave packet; to see this, note that $\phi\in\mbb{R}$ has a phase factor $
		\phi
		\sim a\,e^{+ik\cdot x} + a^\dagger e^{-ik\cdot x}
		\sim \cos\,(k\cdot x)
	$,
	\begin{gather}
	\small
		E = \int \dd[3]{x} \mcal{H}
		= \int \dd[3]{x} \pqty{
				\frac{1}{2} \dot{\phi}^2
				+ \frac{1}{2} (\grad{\phi})^2
				+ \frac{1}{2} m^2 \phi^2
				+ \cdots
			}
		\sim \pqty{E^2 + \vb{k^2} + m^2 + \cdots}
			\int \dd[3]{x} \frac{1}{2}\,\phi^2,
	\end{gather}
	\vspace{-.7\baselineskip}
	\begin{equation}
		E\int \dd[3]{x} \phi^2 \sim 1,
	\end{equation}
	Indeed, we have: $
		M^{ij}
		\sim \pqty{
				x^i k^j
				- x^j k^i
			}
	$. 
	
	Canonical quantization:
	\begin{gather}
		[\phi(\vb{x}),\varpi(\vb{y})]
		= [\phi(\vb{x}),\dot{\phi}(\vb{y})]
		= i\delta(\vb{x} - \vb{y})
	\end{gather}
	Other equal-time commutators between $\phi,\varpi$ all just vanish. Operator products are then regularized by normal ordering: $M\mapsto \normorder{M}$, which can be explicitly implemented by normal ordering of the oscilator modes:
	\begin{equation}
		\phi(x)
		= \int \frac{\dd[3]{k}}{(2\pi)^3}
			\frac{1}{\sqrt{2E_k}}
			\pqty{
				a_{k} e^{ik\cdot x}
				+ a^\dagger_{k} e^{-ik\cdot x}
			},
	\quad
		[a_{k}, a^\dagger_{k'}]
		= (2\pi)^3 \delta(\vb{k} - \vb{k}')
	\end{equation}
	The $k$ dependence in $a_k, E_k$ is, in fact, only a $\vb{k}$ dependence; we've dropped the boldface in the subscripts simply for convenience.
	
	For example, the first term in $M^{0i} = -M^{i0}$ can be expanded as:
	\begin{equation}
	\begin{aligned}
		- x^0 \int \dd[3]{x}
			\varpi\,\pd^i \phi
		= x^0 \int \frac{\dd[3]{k}}{(2\pi)^3}\,
				k^i a^\dagger_k a_k
		= x^0 P^i
	\end{aligned}
	\end{equation}
	Here $P^\mu$ is the momentum operator on the Hilbert space, promoted from the classical $-i\pd^\mu$. $M^{0i}$ is thus further reduced to:
	\begin{equation}
		M^{0i} = -M^{i0}
		= x^0 P^i - \int \dd[3]{x}
			x^i \mcal{H}
	\end{equation}
	
	We note that this result is almost the classical $
		x^0 P^i - x^i P^0
	$, but here $x^i P^0$ is replaced with the integral over energy density $\mcal{H}$. The result can be nicely re-written with the stress tensor $T^{\mu\nu}$; just run the Noether's procedure with $\var{x^\mu} = \epsilon^\mu$, then we shall obtain:
	\begin{equation}
	\begin{gathered}
		j'^\mu
		= \epsilon_\nu T^{\mu\nu},
	\quad
		T^{\mu\nu}
		= \pd^\mu\phi\,\pd^\nu\phi
			+ \eta^{\mu\nu} \mcal{L},
	\\
		Q' = \epsilon_\mu P^\mu,
	\quad
		P^\mu
		= \int \dd[3]{x} T^{\mu 0}
		= \int \dd[3]{x} \pqty{
			\pd^0\phi\,\pd^\mu\phi
			+ \eta^{0\mu} \mcal{L}
		}
	\\
		M^{\mu\nu}
		= \int \dd[3]{x}
			2x^{[\mu} T^{\nu]0}
	\end{gathered}
	\end{equation}
	The quantization $M\mapsto \normorder{M}$ is thus reduced to the quantization of $T^{\nu 0}$, weighted by a $x^\mu$ factor. 
	
	First let's look at $T^{00} = \mcal{H}$; note that:
	\begin{equation}
		\int \dd[3]{x}
			x^i e^{i\vb{k}\cdot\vb{x}}
		= (2\pi)^3 \pqty{
				-i\,\pdv{k_i}
			} \delta(\vb{k})
	\end{equation}
	One can then check explicitly with mode expansion that, up to normal ordering, we have\footnote{
		See a similar result in \url{https://physics.stackexchange.com/q/27906}. Moreover, the charge can be computed at arbitrary time slice $t$, but the $t$-dependence ($\sim e^{\pm iEt}$) drops out in the final result, due to the on-shell condition $E_k^2 = \vb{k}^2 + m^2$ and symmetries, e.g.\ $\int \dd[3]{k} k^i = 0$. Note that $\pdv{E_k}{k_i} = \frac{k^i}{E_k}$. 
	}:
	\begin{gather}
		H = \int \dd[3]{x} \mcal{H}
		= \int \frac{\dd[3]{k}}{(2\pi)^3}\,
			E_k\, a^\dagger_k a_k,
	\\
		\mcal{O}^i = \int \dd[3]{x} x^i \mcal{H}
		= \int \frac{\dd[3]{k}}{(2\pi)^3}\,
			E_k\, a^\dagger_k
				\pqty{+i\,\pdv{k_i}}
			a_k,
	\end{gather}
	
%\pagebreak[3]
	At first glance, derivative of $a_k = a_{\vb{k}}$ with respect to $k^i$ seems puzzling; however, note that $\pqty\big{+i\,\pdv{k_i}}$ is precisely the $x^i$ operator in ``momentum-space'', and one can make sense of it by considering a generic $n$-particle state:
	\begin{equation}
		\ket{\Psi}
		= \int \dd[3]{k_1}\cdots \dd[3]{k_n}
			\Psi(\vb{k}_1,\cdots,\vb{k}_n)
			\,a^\dagger_{k_1}
			\cdots a^\dagger_{k_n} \ket{0}
	\end{equation}
	Where $\Psi(\vb{k}_1,\cdots,\vb{k}_n)$ is the $n$-particle wave function; using $
		a_k a^\dagger_{k'}
		= a^\dagger_{k'} a_k
			+ (2\pi)^3\delta(\vb{k} - \vb{k'})
	$ recursively, we get in the end that:
	\begin{equation}
		\mcal{O}_i \ket{\Psi}
		= \int \dd[3]{k_1}\cdots \dd[3]{k_n}
			\Bqty{
				\sum_{m = 1}^n
				E_{k_m}
				\pqty{+i\,\pdv{k^i_m}}\,
				\Psi(\vb{k}_1,\cdots,\vb{k}_n)
			}
			\,a^\dagger_{k_1}
			\cdots a^\dagger_{k_n} \ket{0}
	\end{equation}
	
	We see that indeed $\mcal{O}_i$ acts as $
		E_k \pqty{+i\,\pdv{k^i}}
	$ on the momentum-space $n$-particle wave function, consistent with the result of ordinary quantum mechanics; ...
	
	
	\sidenote{TODO: Detailed analysis! HINT: Ward identity!}
	
	Notice that $
		x^{[\mu} \pd^{\nu]}
		= \frac{1}{2}\,\pqty{
			x^\mu \pd^\nu - x^\nu \pd^\mu
		} = \frac{1}{2} D^{\mu\nu}
	$ is the Killing vector fields of $\mbb{R}^{3,1}$, hence they naturally follow the commutation relations of $\mfrak{so}(3,1)$ (up to a constant coefficient, or an isomorphism)\footnote{
		I would like to thank \textit{林般} for pointing this out. 
	}. We have:
	\begin{equation}
	\begin{aligned}
		[M^{\mu\nu}, M^{\rho\sigma}]
		&= \int \dd[3]{x}
			\int \dd[3]{y} \bqty{
				\dot{\phi} D^{\mu\nu}\phi(x),
				\dot{\phi} D^{\rho\sigma}\phi(y)
			} \\
		&= \int \dd[3]{x}
			\dot{\phi}\,\bqty{
				 D^{\mu\nu}, D^{\rho\sigma}
			}\,\phi
	\end{aligned}
	\end{equation}
	Similar holds for $M^{i0}$. Therefore, $M^{\mu\nu}$'s indeed form the Lie algebra $\mfrak{so}(3,1)$. 
	\qedfull
	
	\end{itemize}

\printbibliography[%
%	title = {参考文献} %
	,heading = bibintoc
]
\end{document}
