% !TeX document-id = {dcff3c7d-6b8b-4238-98c6-67a1f1289604}
% !TeX encoding = UTF-8
% !TeX spellcheck = en_US
% !TeX TXS-program:bibliography = biber -l zh__pinyin --output-safechars %

\documentclass[a4paper,10pt]{article}

\newcommand{\hwNumber}{2}

% Templates: 82ccb576e4df24e5eac4194b76230be360b4f733

% to be `\input` in subfolders,
% ... therefore the path should be relative to subfolders.

\usepackage[UTF8
	,heading=false
	,scheme=plain % English Document
]{ctex}
\usepackage{indentfirst}

\input{../.modules/basics/macros.tex}
\input{../.modules/preamble_base.tex}
\input{../.modules/preamble_notes.tex}

\newcommand{\legacyReference}{{
%	\clearpage\par
%	\quad\clearpage
	\renewcommand{\midquote}{\textbf{PAST WORK, AS TEMPLATE}}
	\newparagraph
}}

% Settings
\counterwithout{equation}{section}
\mathtoolsset{showonlyrefs=false}
%\DeclareTextFontCommand{\textbf}{\sffamily}
\renewcommand{\midquote}{\quad}

% Spacing
\geometry{footnotesep=2\baselineskip} % pre footnote split
\setlength{\parskip}{.5\baselineskip}
\renewcommand{\baselinestretch}{1.15}

%Title
	\posttitle{
		\hfill\Large\ccbyncsajp
		\par\end{flushleft}%
		\vspace*{-.7ex}\hrule%
	}
	\preauthor{\vspace{-1.5ex}%
		\flushleft\itshape%
	}
	\postauthor{\hfill}
	\predate{\noindent\ttfamily Compiled @ }
	\postdate{\vspace{.5ex}}

	\title{Advanced QFT \textnumero\hwNumber}
	\author{\signature Bryan}
	\date{\today}

% List
	\setlist*{
		listparindent=\parindent
		,labelindent=\parindent
		,parsep=\parskip
		,itemsep=1.2\parskip
	}

\input{../.modules/basics/biblatex.tex}

%%% ID: sensitive, do NOT publish!
\InputIfFileExists{../id.tex}{}{}

\begin{document}
\maketitle
\pagestyle{headings}
\pagenumbering{arabic}
\thispagestyle{empty}

\vspace*{-1.5\baselineskip}

\section{Local Transformation}
	\vspace{-.5\baselineskip}
	\begin{equation}
		\var{A^a_\mu}
		= \pdd{\mu} \lambda^a(x)
			+ f\id{^a_{bc}} A^b_\mu \lambda^c(x),
	\label{eq:A_gauge_var}
	\end{equation}
	Here $f_{abc}$ is the totally anti-symmetric structure constant for a semi-simple Lie algebra $\mfrak{g}$, with generators $\{T_a\}_a$ and normalized Killing form $\delta_{ab}$. 
	
	\begin{itemize}
	\item The field strength is defined as follows:
	\begin{equation}
	\begin{aligned}
		F_{\mu\nu}
		\equiv F^a_{\mu\nu} T_a
		= [D_\mu,D_\nu]
		&= \bqty\big{
			\pdd{\mu} + A_\mu,
			\pdd{\nu} + A_\nu
		} \\
		&= \dd{A} + A\wedge A \\
		&= \pdd{\mu} A_\nu
			- \pdd{\nu} A_\mu
			+ f\id{^a_{bc}} A^b_\mu A^c_\nu\,T_a
	\end{aligned}
	\end{equation}
	Adjoint indices $a,b,\cdots$ are sometimes suppressed by contracting with $T_a$'s. By exploiting the anti-symmetric property of $f\id{^a_{bc}}$, along with the Jacobi identity, we get the infinitesimal transformation:
	\begin{gather}
	\qquad
	\begin{aligned}
		\var{F^a_{\mu\nu}}
		&= \pdd{\mu} \var{A^a_\nu}
			- \pdd{\nu} \var{A^a_\mu}
			+ f\id{^a_{bc}}
				\var{(A^b_\mu A^c_\nu)} \\[.5ex]
		&= f\id{^a_{bc}} \pqty\Big{
			\lambda^c \pqty{
				\pdd{\mu} A^b_\nu
				- \pdd{\nu} A^b_\mu
			} + \pqty{
				A^b_\nu\,\pdd{\mu} \lambda^c
				- A^b_\mu\,\pdd{\nu} \lambda^c
			} + \var{(A^b_\mu A^c_\nu)}
		} \\
		&= f\id{^a_{bc}} \pqty\Big{
			\lambda^c \pqty{
				F^b_{\mu\nu}
				- f\id{^b_{de}} A^d_\mu A^e_\nu
			} + \pqty{
				A^b_\nu\,\pqty{
					\smash{\underline{\var{A^c_\mu}}}
					- f\id{^c_{de}}
						A^d_\mu\,\lambda^e
				}
				- A^b_\mu\,\pqty{
					\smash{\underline{\var{A^c_\nu}}}
					- f\id{^c_{de}}
						A^d_\nu\,\lambda^e
				}
			} + \underline{\var{(A^b_\mu A^c_\nu)}}
		} \\
		&= {f\id{^a_{bc}}} \pqty\Big{
			\lambda^c \pqty{
				F^b_{\mu\nu}
				- {f\id{^b_{de}}}
					A^d_\mu A^e_\nu
			} - \pqty{
					{f\id{^c_{de}}}
						A^b_\nu A^d_\mu\,\lambda^e
					- {f\id{^c_{de}}}
						A^b_\mu A^d_\nu\,\lambda^e
			}
		} \\
		&= \underline{f\id{^a_{bc}}} \pqty\Big{
			\lambda^c \pqty{
				F^b_{\mu\nu}
				- \underline{f\id{^b_{de}}}
					A^d_\mu A^e_\nu
			} - \pqty{
					\underline{f\id{^c_{de}}}
						A^b_\nu A^d_\mu\,\lambda^e
					- \underline{f\id{^c_{de}}}
						A^b_\mu A^d_\nu\,\lambda^e
			}
		} \\
		&= f\id{^a_{bc}}
			\lambda^c F^b_{\mu\nu} \\[-4ex]
	\end{aligned}
	\quad
	\end{gather}
	
	When contracted with $T_a$, this yields:
	\begin{gather}
		\var{F_{\mu\nu}}
		= \lambda^c F^b_{\mu\nu}\,f\id{^a_{bc}} T_a
		= \lambda^c F^b_{\mu\nu}\,[T_b,T_c]
		= F_{\mu\nu}\cdot\lambda
			- \lambda\cdot F_{\mu\nu},\\[.8ex]
		\lambda = \lambda^c(x)\, T_c,\quad
		F_{\mu\nu} = F^b_{\mu\nu}\, T_b,\\[1.2ex]
		F_{\mu\nu}
		\ \longmapsto\ %
			e^{-\Lambda^a(x)\,T_a}\,F_{\mu\nu}\,
			e^{\Lambda^a(x)\,T_a}
	\end{gather}
	The exponentiation is valid even for local $
		\lambda = \lambda(x)
	$, since it is produced by integrating along the fiber direction $\lambda\to\Lambda$, not the spacetime direction $x$. This is the finite transformation w.r.t.\ $\Lambda(x)$. 
	
	\item For any matter field $\psi$ furnishing a representation of $\mfrak{g}$, we have:
	\begin{gather}
		T_a \psi = (T_a)\id{^i_j}\,\psi^j,\quad
		\var{\psi} = -\lambda^a(x)\,T_a\psi,\\[.5ex]
		\psi
			\ \longmapsto\ %
			e^{-\Lambda^a(x)\,T_a}\,\psi,\\
		D_\mu \psi
			\ \longmapsto\ %
			e^{-\Lambda^a(x)\,T_a}\,D_\mu\psi,
	\end{gather}
	In fact, \eqref{eq:A_gauge_var} is chosen to ensure that $D_\mu \psi$ transforms gauge covariantly just like $\psi$. Therefore, 
	\begin{gather}
	\begin{aligned}
		D_{\mu}
= \pdd{\mu} + A_\mu
		\ &\longmapsto\ %
			e^{-\Lambda^a(x)\,T_a}
			\circ D_{\mu}\circ
			e^{\Lambda^a(x)\,T_a} \\
		&\qquad\quad
		= e^{-\Lambda} \circ \pqty{
				\pdd{\mu} + A_\mu
			} \circ e^\Lambda\\
		&\qquad\quad
		= e^{-\Lambda}
			\circ \pdd{\mu} \circ e^\Lambda
			+ e^{-\Lambda} A_\mu\,e^\Lambda,\quad
		\Lambda = \Lambda^a(x)\,T_a,
	\end{aligned}\\[1ex]
		A_\mu
		\ \longmapsto\ %
			e^{-\Lambda}\,(\pdd{\mu} e^\Lambda)
			+ e^{-\Lambda} A_\mu\,e^\Lambda
%		\\
%		&\qquad\quad
		= T_a\,\pdd{\mu} \Lambda^a(x)
			+ e^{-\Lambda} A_\mu\,e^\Lambda
	\end{gather}
	
	\item $F^2\equiv F\wedge F$, we have:
	\begin{equation}
	\begin{aligned}
		F^2
		&= (\dd{A} + A\wedge A)
			\wedge (\dd{A} + A\wedge A) \\
		&= \dd{A} \wedge \dd{A}
			+ \dd{A} \wedge A \wedge A
			+ A \wedge A \wedge \dd{A}
			+ A \wedge A \wedge A \wedge A
	\end{aligned}
	\end{equation}
	The last term is proportional to $
		\epsilon_{abcd}\,T^a\,T^b\,T^c\,T^d
	$, hence its trace will vanish; therefore,
	\begin{gather}
	\begin{aligned}
		\tr F^2
		&= \tr \pqty{
			\dd{A} \wedge \dd{A}
			+ \dd{A} \wedge A \wedge A
			+ A \wedge A \wedge \dd{A}
		} \\
		&= \tr \pqty{
			\dd\pqty{\dd{A} \wedge A}
			+ \frac{2}{3}\,\dd\pqty{
				A \wedge A \wedge A
			}
		} \\
		&= \dd \tr \pqty{
			\dd{A} \wedge A
			+ \frac{2}{3}\,A \wedge A \wedge A
		} = \dd{\omega},
	\end{aligned}\\
		\omega = \tr \pqty{
			\dd{A} \wedge A
			+ \frac{2}{3}\,A \wedge A \wedge A
		}
	\end{gather}
	\vspace{-1.8\baselineskip}
	\end{itemize}
	\qedfull
	\vspace{-1\baselineskip}
\section{Relativistic Particle}
	\vspace{-.35\baselineskip}
	\begin{equation}
		L = \frac{1}{2e} \pqty{
			\frac{1}{c}\dv{X}{t}
		}^2 - \frac{e}{2}\,m^2 c^4
	\end{equation}
	
	\begin{itemize}
	\item For $t\mapsto t' = t - \xi(t)$, we have $
		X'(t') = X(t)
	$, therefore:
	\begin{equation}
		\var{X^\mu}
		= -\var{t} \dv{X^\mu}{t}
		= \xi(t)\,\dot{X}^\mu,
	\end{equation}
	Or more explicitly, $
		X^\mu(t)
		\mapsto X^\mu(t) + \xi(t)\,\dot{X}^\mu
	$. 
	
%	\item We have:
%	\begin{align}
%		\var{S}
%		= \int\dd{t} \var{L}
%		&= \int\dd{t} \Bqty{
%			\frac{1}{ec^2}\,
%				\dot{X}_\mu
%				\var{\dot{X}^\mu}
%			- \frac{\var{e}}{2}\,
%				\frac{1}{e^2 c^2} \dot{X}^2
%			- \frac{\var{e}}{2}\,m^2 c^4
%		}
%		\label{eq:particle_var_generic}\\[1ex]
%		&= - \int\dd{t} \Bqty{
%			\dv{t} \pqty{
%				\frac{1}{ec^2}\,
%				\dot{X}_\mu
%			} \var{X^\mu}
%			+ \frac{\var{e}}{2}\,\pqty{
%				\frac{1}{e^2 c^2} \dot{X}^2
%				+ m^2 c^4
%			}
%		} \notag\\[.5ex]
%		&\qquad + \int \dd\mspace{.5mu}\pqty{
%			\frac{1}{ec^2}\,
%				\dot{X}_\mu
%				\var{X^\mu}
%		}
%		\label{eq:particle_var_integrated}
%	\end{align}
%	This gives the equations of motion (EOMs): $
%		\dv{t} \pqty\big{
%			\frac{1}{ec^2}\,
%			\dot{X}_\mu
%		} = 0,\ %
%		\pqty{\frac{1}{c} \dv{X}{t}}^2
%		= -e^2 m^2 c^4
%	$. As we can see, the auxiliary field $e(t)$ is not dynamical; Fixing $e = 1$ is equivalent to setting $t = \tau$: the proper time, or affine parametrization for the massless case. 
	
	\item We have:
	\begin{equation}
	\begin{aligned}
		\var{L}
		&= \frac{1}{ec^2}\,
				\dot{X}_\mu
				\var{\dot{X}^\mu}
			- \frac{\var{e}}{2}\,
				\frac{1}{e^2 c^2} \dot{X}^2
			- \frac{\var{e}}{2}\,m^2 c^4 \\[.5ex]
		&= \frac{1}{ec^2}\,
				\xi\dot{X}_\mu\ddot{X}^\mu
			+ \frac{1}{ec^2}\,
				\dot{\xi}\dot{X}^2
			- \frac{\var{e}}{2}\,
				\frac{1}{e^2 c^2} \dot{X}^2
			- \frac{\var{e}}{2}\,m^2 c^4
	\end{aligned}
	\end{equation}
	For $
		S = \int\dd{t} {L}
	$ to be invariant, $\var{L}$ should be reduced to a total derivative, which can then be reduced to some vanishing boundary terms. 
	
	Consider $
		\var{e} = \dv{t}\pqty{e\xi}
		= \dot{e}\xi + e\dot{\xi}
	$, and we have:
	\begin{equation}
	\begin{aligned}
		\var{L}
		&= \frac{1}{ec^2}\,
				\xi\dot{X}_\mu\ddot{X}^\mu
			+ \frac{1}{2ec^2}\,
				\dot{\xi}\dot{X}^2
			- \frac{\dot{e}}{2e^2 c^2}\,
				\xi \dot{X}^2
			- \dv{t}\pqty{
				\frac{1}{2}\,e\xi\,m^2 c^4
			} \\[.8ex]
		&= \dv{t} \Bqty{\pqty{
			\frac{1}{2ec^2}\dot{X}^2
			- \frac{e}{2}\,m^2 c^4
		}\,\xi}
		= \dv{t} \pqty\big{\xi L}
	\end{aligned}
	\end{equation}
	Indeed we get a total derivative; therefore,
	\begin{gather}
		\var{e} = \dv{t}\pqty\big{e\xi},\quad
		\var{S} = \int \var{L}
		= \int \dd\mspace{.5mu} \pqty\big{
			\xi L
		} = 0
	\end{gather}
	
	\item $e(t)$ can be seen as a gauge field coupled to $X$, which captures the $t$--reparametrization redundancy through the gauge transformation parameter $\xi(t)$. A natural gauge choice is fixing $f = e(t) - 1 \equiv 0$, which is equivalent to setting $t = \tau$: the proper time, or affine parametrization for the massless case. 
\pagebreak[3]
	
	The gauge invariant path integral is constructed as follows:
	\begin{equation}
	\begin{aligned}
		\mcal{Z}
		&= \frac{1}{\int\!\DD{\xi}\,}
			\int \DD{X} \DD{e}  e^{iS} \\
		&= \frac{1}{\int\!\DD{\xi}\,}
			\int \DD{X} \DD{e}  e^{iS}
			\int \DD{f} \delta\bqty\big{
				f
			} \\
		&= \frac{1}{\int\!\DD{\xi}\,}
			\int \DD{X} \DD{e}  e^{iS}
			\int \DD{\xi} \delta\bqty\big{
				f_\xi
			} \det \fdv{f_\xi}{\xi} \\
		&= \frac{1}{\int\!\DD{\xi}\,}
			\int \DD{\xi}
			\int \DD{X} \DD{e}  e^{iS}\,
			\delta\bqty\big{
				f_\xi
			} \det \fdv{f_\xi}{\xi} \\
		&= \frac{1}{\int\!\DD{\xi}\,}
			\int \DD{\xi}
			\int \DD{X_\xi} \DD{e_\xi}
				e^{iS_\xi}\,
			\delta\bqty\big{
				f_\xi
			} \det \fdv{f_\xi}{\xi}
			\bigg|_{e_\xi} \\
		&= \frac{1}{\int\!\DD{\xi}\,}
			\int \DD{\xi}
			\int \DD{X} \DD{e}
				e^{iS}\,
			\delta\bqty\big{f}
			\det \fdv{f_\xi}{\xi}
			\bigg|_{\,\xi=0} \\
		&= \int \DD{X} \DD{e}  e^{iS}\,
			\delta\bqty\big{f}
			\det \fdv{f_\xi}{\xi}
			\bigg|_{\,\xi=0}
	\end{aligned}
	\end{equation}
	
	Here the gauge-transformed quantities are marked with a $\xi$ subscript. Note that in the final expression, $f$ is \textit{not} integrated out and can be \textit{any} possible gauge-fixing function, i.e.\ $
		\pqty{
			\delta\bqty\big{f}
			\det \fdv{f_\xi}{\xi}
		}_{\!\xi=0}
	$ is in fact $f$-independent. 
	
	These are the first steps of the Faddeev--Popov (FP) procedure; it achieves several things at once: first it imposes a gauge-fixing $f = 0$, and then it removes the gauge redundancy with the help of FP determinant $
		\pqty\big{
			\det \fdv{f_\xi}{\xi}
		}_{\!\xi=0}
	$, while implicitly imposing the constraints resulted from the gauge-fixing process. The constraints implemented by $
		\pqty\big{
			\det \fdv{f_\xi}{\xi}
		}_{\!\xi=0}
	$ can be made explicit with the help of BRST formalism.
	
	The gauge-fixing term $
		\delta\bqty\big{f}
	$ can be replaced by a Gaussian packet with width parameter $\zeta$. More rigorously, up to an overall constant coefficient, we have the following equivalence:
	\begin{gather}
		\delta\bqty\big{f}
		\sim \delta\bqty\big{f - f_0}
		\sim \int \DD{f_0}\,
			\exp\pqty{
				-\frac{i}{2\zeta}\int \dd{t} f_0^2
			}\,\delta\bqty\big{f - f_0}
		= \exp\pqty{i\int \dd{t} L_{gf}},\\[.5ex]
		L_{gf}
		= -\frac{1}{2\zeta} f^2
		= -\frac{1}{2\zeta} \,(e-1)^2,
	\end{gather}
	Here $f_0$ is some gauge-invariant shift of $f$, namely $(f_0)_\xi = f_0$. $f_0$ can be seen as a non-dynamical auxiliary field that enforce the gauge fixing, much similar to a Lagrange multiplier.
	On the other hand, the determinant can be evaluated using Faddeev--Popov (FP) ghosts $b,c$\,:
	\begin{gather}
		\det P \sim \int \DD{b}\,\DD{c}\,
			\exp \pqty{
				i\int\dd{t} \int\dd{t'}
					b(t)\cdot P(t,t')\cdot c(t')
			},\\[.8ex]
		\fdv{f_\xi(t)}{\xi(t')}\,
			\bigg|_{\,\xi=0} \!\!
		= \fdv{\xi(t')} \,\bigg|_{\,\xi=0}
			\pqty{
				e + \dv{t} \pqty\big{e\xi} - 1
			}_{\!(t)}
		= \dv{t} \pqty\Big{
				e(t)\,\delta(t - t')
			},\\[1.5ex]
	\begin{aligned}
		\det \fdv{f_\xi(t)}{\xi(t')}\,
			\bigg|_{\,\xi=0} \!\!
		&\sim \int \DD{b}\,\DD{c}\,
			\exp \pqty{
				i\int\dd{t} \int\dd{t'}
				b(t)\,\pqty\Big{
					e(t)\,\delta(t - t')
				}\,c(t')
			} \\[-.5ex]
		&\sim \int \DD{b}\,\DD{c}\,
			\exp \pqty{
				-i\int\dd{t} e\dot{b}c
			},
	\end{aligned}\\[.8ex]
		L_{gh} = -e\dot{b}c
	\end{gather}
	
	In summary, we have:
	\begin{gather}
		\mcal{Z}
		= \int \DD{X}\,\DD{e}\,\DD{b}\,\DD{c}\,
			e^{iS_{q}},\quad
		S_q = \int \dd{t} L_q,\\
		L_q
		= L + L_{gf} + L_{gh}
		= L - \frac{1}{2\zeta} \,(e-1)^2
			- e\dot{b}c,
	\end{gather}
	$S_q$ is the quantum action under the gauge-fixing condition $f = e(t) - 1 = 0$. 
	\qedfull
	\end{itemize}
\section{2D $\sigma$-Model}
	\vspace{-.3\baselineskip}
	\begin{equation}
		\mscr{L}
		= -\frac{1}{2}\,
			\pdd{\alpha} X^\mu\,
			\pdd{\beta} X_\mu
			\sqrt{-h}\,h^{\alpha\beta},\quad
		X\colon \Sigma^{1,1}\to\mbb{R}^{D-1,1}
	\end{equation}
	
	\begin{itemize}
	\item The action is diff-invariant; under $
		\sigma^\alpha
		\mapsto \sigma^\alpha + \xi^\alpha
	$, we have:
	\begin{equation}
		\var{X^\alpha}
		= \ldv{\xi} X^\alpha,\quad
		\var{h^{\alpha\beta}}
		= \ldv{\xi} h^{\alpha\beta}
	\end{equation}
	$\ldv{\xi}$ is the Lie derivative along $\xi^\alpha$. Note that $
		0 = \var\pqty{
			h_{\alpha\beta} h^{\beta\gamma}
		}
	$, hence we have:
	\begin{gather}
		\var{h_{\alpha\beta}}
		= -h_{\alpha\alpha'} h_{\beta\beta'}
			\var{h^{\alpha'\beta'}}
		= \xi^\gamma \pdd{\gamma} h_{\alpha\beta}
			+ (\pdd{\alpha} \xi^\gamma)\,
				h_{\gamma\beta}
			+ (\pdd{\beta} \xi^\gamma)\,
				h_{\alpha\gamma}
		= \ldv{\xi} h_{\alpha\beta},\\[1ex]
		\var{\sqrt{-h}}
		= \frac{1}{2} \sqrt{-h}\,
			h^{\alpha\beta} \var{h_{\alpha\beta}},
	\end{gather}
	
	Furthermore, we have $
		\ldv{\xi} \dd{X}
		= \dd (\ldv{\xi} X)
	$, i.e.\ $
		\pdd{\alpha} \var{X}
		= \pdd{\alpha} \ldv{\xi} X
		= \pdd{\alpha} \pqty\big{
			\xi^\gamma \pdd{\gamma} X
		}
		= \ldv{\xi} \pqty\big{
			\pdd{\alpha} X
		}
	$. Note that due to the $\sqrt{-h}$ factor, $\mscr{L}$ is not a scalar but a \textit{scalar density}. For convenience, define $
		\mscr{L} = \widetilde{\mscr{L}}\sqrt{-h}
	$, then $
		\widetilde{\mscr{L}}
		= -\frac{1}{2}\,h^{\alpha\beta}\,
			\pdd{\alpha} X^\mu\,
			\pdd{\beta} X_\mu
	$ is a scalar; using chain rule, we obtain:
\pagebreak[3]
	\begin{equation}
	\begin{aligned}
		\var{\mscr{L}}
		&= \sqrt{-h}
				\var{\widetilde{\mscr{L}}}
			+ \widetilde{\mscr{L}}
				\var{\sqrt{-h}} \\
		&= \sqrt{-h}
				\ldv{\xi} \widetilde{\mscr{L}}
			+ \widetilde{\mscr{L}}
				\var{\sqrt{-h}} \\
		&= \sqrt{-h}\,
				\xi^\gamma\pdd{\gamma}
				\widetilde{\mscr{L}}
			+ \widetilde{\mscr{L}}
				\var{\sqrt{-h}} \\
		&= \pdd{\gamma} \pqty\Big{
				\xi^\gamma
				\widetilde{\mscr{L}}
				\sqrt{-h}
			} 
			- \widetilde{\mscr{L}}\,\pqty{
				\sqrt{-h}\,
					(\pdd{\gamma} \xi^\gamma)
				+ \xi^\gamma\,
					(\pdd{\gamma} \sqrt{-h})
			}
			+ \widetilde{\mscr{L}}
				\var{\sqrt{-h}} \\
		&= \pdd{\gamma} \pqty\big{
				\xi^\gamma \mscr{L}
			} 
			- \widetilde{\mscr{L}}\,\pqty{
				\sqrt{-h}\,
					(\pdd{\gamma} \xi^\gamma)
				+ \xi^\gamma\,
					(\pdd{\gamma} \sqrt{-h})
				- \var{\sqrt{-h}}
			} \\
		&= \pdd{\gamma} \pqty\big{
				\xi^\gamma \mscr{L}
			} 
			- \widetilde{\mscr{L}}
			\sqrt{-h}\,\pqty{
				\pdd{\gamma} \xi^\gamma
				+ \frac{1}{2}\,
					\xi^\gamma\,
					h^{\alpha\beta}
					\pdd{\gamma} h_{\alpha\beta}
				- \frac{1}{2}\,
					h^{\alpha\beta}
					\var{h_{\alpha\beta}}
			} \\
		&= \pdd{\gamma} \pqty\big{
				\xi^\gamma \mscr{L}
			} 
			- \widetilde{\mscr{L}}
			\sqrt{-h}\,\pqty{
				\pdd{\gamma} \xi^\gamma
				- \frac{1}{2}\,
				h^{\alpha\beta}\,\pqty\Big{
					(\pdd{\alpha} \xi^\gamma)\,
						h_{\gamma\beta}
					+ (\pdd{\beta} \xi^\gamma)\,
						h_{\alpha\gamma}
				}
			} \\
		&= \pdd{\gamma} \pqty\big{
				\xi^\gamma \mscr{L}
			} 
			- \widetilde{\mscr{L}}
			\sqrt{-h}\,\pqty\Big{
				\pdd{\gamma} \xi^\gamma
				- \pdd{\gamma} \xi^\gamma
			} \\
		&= \pdd{\gamma} \pqty\big{
				\xi^\gamma \mscr{L}
			}
	\end{aligned}
	\end{equation}
	We see that $\var{\mscr{L}}$ is a total derivative, hence $
		\var{S} = \int \dd[2]{\sigma}
			\var{\mscr{L}}
		= 0
	$, i.e.\ the action is diff-invariant. 
	
	\item The action is Weyl invariant; with $
		\var{h^{\alpha\beta}}
		= -\lambda(\sigma)\,h^{\alpha\beta}
	$, we have:
	\begin{equation}
	\begin{aligned}
		\var\pqty{
			\sqrt{-h}\,h^{\alpha\beta}
		}
		&= \sqrt{-h}
				\var{h^{\alpha\beta}}
			+ h^{\alpha\beta}
				\var{\sqrt{-h}} \\
		&= \sqrt{-h}\,h^{\alpha\beta} \pqty{
			- \lambda
			- \frac{1}{2}\,
				h_{\alpha'\beta'}
				\var{h^{\alpha'\beta'}}
		} \\
		&= \sqrt{-h}\,h^{\alpha\beta} \pqty{
			- \lambda
			+ \frac{1}{2}\,\lambda\,
				h_{\alpha'\beta'}
				h^{\alpha'\beta'}
		} \\
		&= \sqrt{-h}\,h^{\alpha\beta} \pqty{
			- \lambda
			+ \frac{2}{2}\,\lambda
		} \\
		&= 0
	\end{aligned}
	\end{equation}
	Here we've used the fact that $
		h_{\alpha\beta} h^{\alpha\beta}
		= \delta^\alpha_\alpha
		= 2
	$. Therefore, $
		\var{\mscr{L}}
		= -\frac{1}{2}\,
			\pdd{\alpha} X^\mu\,
			\pdd{\beta} X_\mu
			\var\pqty{
				\sqrt{-h}\,h^{\alpha\beta}
			}
		= 0
	$, i.e.\ the action is Weyl invariant. 
	
	\item FP quantization of this system follows the same recipe as the point particle case above: 
	\begin{gather}
		\mcal{Z}
		= \int \DD{X}\,\DD{h}\,\DD{b}\,\DD{c}\,
			e^{iS_{q}},\quad
		S_q = \int \dd[2]{\sigma} \mscr{L}_q,\\
		\mscr{L}_q
		= \mscr{L} + \mscr{L}_{gf} + \mscr{L}_{gh}
	\end{gather}
	Given gauge fixing: $
		f^{\alpha\beta}
		= h^{\alpha\beta} - h^{\alpha\beta}_{(0)}
	$\,, we have:
	\begin{equation}
		\mscr{L}_{gf}
		= -\frac{1}{2\zeta}\,
			f^{\alpha\beta}
			f_{\alpha\beta} \sqrt{-h}
		= -\frac{1}{2\zeta}\,\pqty{
				h^{\alpha\beta}
				- h^{\alpha\beta}_{(0)}
			} \pqty{
				h_{\alpha\beta}
				- h_{\alpha\beta}^{(0)}
			} \sqrt{-h}
	\end{equation}
	
%	Alternatively, we can introduce a non-dynamical field $B$ ...
	
	The FP ghost term $\mscr{L}_{gh}$ is given by functional determinant; we have:
	\begingroup
	\allowdisplaybreaks[0]
	\begin{align}
		\fdv{f^{\alpha\beta}_\xi(\sigma)}
			{\xi^\gamma(\sigma')}\,
			\bigg|_{\,0} \!\!
		&= \fdv{\xi^\gamma(\sigma')}\,
			\bigg|_{\,0}
			\pqty{
				\ldv{\xi} h^{\alpha\beta}
				- \lambda h^{\alpha\beta}
			}_{\!(\sigma)} \\[.5ex]
		&= \delta(\sigma-\sigma')\,
				\pdd{\gamma} h^{\alpha\beta}
			- \delta^\alpha_\gamma\,\pd^\beta
				\delta(\sigma-\sigma')
			- \delta^\beta_\gamma\,\pd^\alpha
				\delta(\sigma-\sigma') \\
		&= - \delta^\alpha_\gamma\,\nabla^\beta
				\delta(\sigma-\sigma')
			- \delta^\beta_\gamma\,\nabla^\alpha
				\delta(\sigma-\sigma'), \\
%%%%%%%%%%%%%%%%%%%%%%%%%%%%%%%%
%		\lambda
%%%%%%%%%%%%%%%%%%%%%%%%%%%%%%%%
		\fdv{f^{\alpha\beta}_\xi(\sigma)}
			{\lambda(\sigma')}\,
			\bigg|_{\,0} \!\!
		&= - \delta(\sigma-\sigma')\,
				h^{\alpha\beta},
	\end{align}
	\endgroup
	Here we've replaced $\pd$ with $\nabla$ which commutes with the metric $h_{\alpha\beta}$. 
	Define $
		\Xi^\Gamma
		= (\xi^\gamma,\lambda)
	$ to combine all gauge parameters, and use fermionic FP ghosts: $b_{\alpha\beta},\ c^\Gamma = (c^\gamma,c')$ to contract the indices; after some integration by parts, we have:
	\begin{equation}
		\det \fdv{f^{\alpha\beta}_\xi(\sigma)}
			{\Xi^\Gamma(\sigma')}\,
			\bigg|_{\,0} \!\!
		\sim \int
			\DD{b_{\alpha\beta}}\,
			\DD{c^\gamma}\,\DD{c'}\,
			\exp \pqty{
				i\int \dd[2]{\sigma}
				\sqrt{-h}\,b_{\alpha\beta}\,
				\pqty\Big{
					- \nabla^\beta c^\alpha
					- \nabla^\alpha c^\beta
					- h^{\alpha\beta} c'
				}
			}
	\end{equation}
%\pagebreak[4]
	
	To simplify the action, it is common\footnote{
		References: \textit{Tong:} \url{http://damtp.cam.ac.uk/user/tong/string.html}, and also \textit{Polchinski}. 
	} to integrate $c'$ out, which constrains $b_{\alpha\beta}$ to be symmetric traceless: $
		b_{\alpha\beta} h^{\alpha\beta}
		= b^\alpha_\alpha = 0
	$. The resulting $b_{\alpha\beta}$ has 2 degrees of freedom, same as $c^\gamma$. 
	
	In the end, we have:
	\begin{equation}
		\det \fdv{f^{\alpha\beta}_\xi(\sigma)}
			{\Xi^\Gamma(\sigma')}\,
			\bigg|_{\,0} \!\!
		\sim \int
			\DD{b_{\alpha\beta}}\,
			\DD{c^\gamma}\,
			\exp \pqty{
				-2i\int \dd[2]{\sigma}
				\sqrt{-h}\,
				b_{\alpha\beta}\,
				\nabla^\alpha c^\beta
			}
	\end{equation}
	Therefore, FP quantization with $
		f^{\alpha\beta}
		= h^{\alpha\beta} - h^{\alpha\beta}_{(0)}
	$ yields:
	\begin{gather}
		\mcal{Z}
		= \int \DD{X}\,\DD{h^{\alpha\beta}}\,
			\DD{b_{\alpha\beta}}\,\DD{c^\gamma}\,
			e^{iS_{q}},\quad
		S_q = \int \dd[2]{\sigma} \mscr{L}_q,\quad
		\mscr{L}_q
		= \mscr{L} + \mscr{L}_{gf} + \mscr{L}_{gh},
		\\[.5ex]
		\mscr{L}_{gf}
		= -\frac{1}{2\zeta}\,
			f^{\alpha\beta}
			f_{\alpha\beta} \sqrt{-h}
		= -\frac{1}{2\zeta}\,\pqty{
				h^{\alpha\beta}
				- h^{\alpha\beta}_{(0)}
			} \pqty{
				h_{\alpha\beta}
				- h_{\alpha\beta}^{(0)}
			} \sqrt{-h},\\[.5ex]
		\mscr{L}_{gh}
		= -2b_{\alpha\beta}\,
			\nabla^\alpha c^\beta \sqrt{-h}
	\end{gather}
	\qedfull
	\end{itemize}

\printbibliography[%
%	title = {参考文献} %
	,heading = bibintoc
]
\end{document}
