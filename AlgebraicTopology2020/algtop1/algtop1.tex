% !TeX encoding = UTF-8
% !TeX spellcheck = en_US
% !TeX TXS-program:bibliography = biber -l zh__pinyin --output-safechars %

\documentclass[a4paper,10pt]{article}

\newcommand{\hwNumber}{1}

\input{../algtop_preamble.tex}
\input{../.modules/basics/biblatex.tex}

%%% ID: sensitive, do NOT publish!
%\InputIfFileExists{../id.tex}{}{}

\begin{document}
\maketitle
\pagestyle{headings}
\pagenumbering{arabic}
\thispagestyle{empty}

	\begin{enumerate}
	\item \textbf{Equivalence of categories is fully faithful}: 
	
	$F\colon \mcal{C}\to\mcal{D}$ equivalence of categories, i.e.\ $\exists\ G\colon\mcal{D}\to\mcal{C}$, s.t.
	\begin{equation}
		G\circ F \simeq \idty_{\mcal{C}},\quad
		F\circ G \simeq \idty_{\mcal{D}}
	\end{equation}
	Here $\mquote{\simeq}$ means naturally isomorphic as functors, i.e., 
	\begin{equation}
		\exists\ \tau\colon
			G\circ F\Rightarrow\idty_{\mcal{C}},\quad
		\sigma\colon
			F\circ G\Rightarrow\idty_{\mcal{D}}%
		\,\colon\ \text{natural isomorphisms}
	\end{equation}
	
	By the definition of natural transformation, for $f\in\Hom_{\mcal{C}}(A,B)$, we have:
	\begin{center}
	\begin{tikzcd}[row sep=3em,column sep=4em]
	G\circ F(A)
		\arrow{r}{G\circ F(f)}
		\arrow[swap]{d}{\tau_A} & 
	G\circ F(B)
		\arrow{d}{\tau_B} \\
	A
		\arrow{r}{f} &
	B
	\end{tikzcd}
	\begin{equation}
		\tau_B
			\circ (G\circ F)(f)
			\circ \tau^{-1}_A
		= f,\quad
		\forall\ f\in\Hom_{\mcal{C}}(A,B)
	\end{equation}
	\end{center}
	Here $\tau_{A,B}$ are isomorphisms, which means that $G\circ F$ must be a bijection between hom-sets, which further implies that $F$ is injective and $G$ is surjective. 
%	(onto $
%		\Hom_{\mcal{C}}(G\circ F(A),G\circ F(B))
%	$)
	Switch the roles of $F,G$, we find that $G$ is injective and $F$ is surjective. Therefore, $F,G$ are both fully faithful. \qedfull
	
	\item \textbf{Forgetful functors to \Cat{Set} are often representable}:
	
	For $F\colon \Cat{Group}\to\Cat{Set}$, consider the free group generated by a single element $\mbb{Z}$. We have:
	\begin{equation}
	\begin{aligned}
		\Hom(\mbb{Z},-)\colon\
			\Cat{Group}\ &\longto\ \Cat{Set} \\
			G\ &\longmapsto\ \Hom(\mbb{Z},G)
	\end{aligned}
	\end{equation}
	This is a covariant functor representable by $\mbb{Z}$. 
	
	On the other hand, $\Hom(\mbb{Z},G)$ consists of group homomorphisms:
	\begin{equation}
		\Hom(\mbb{Z},G)
		= \Bqty{
			\begin{aligned}
				\mbb{Z} &\to G \\[-1.5ex]
				1 &\mapsto g
			\end{aligned}
			\,\bigg|\,
			g\in G
		}
	\end{equation}
	More specifically, to fix any $\mbb{Z}\to G$, we need only assign its generator\footnote{
		Note that $0\in\mbb{Z}$ is the group identity of addiction group $\mbb{Z}$, not $1\in\mbb{Z}$. 
	} $1\mapsto g$. Image of any other $\mbb{Z}$ element is generated automatically from the group law, without further specifications. This means that the hom-set is in one-to-one correspondence with $G$ elements (as a set). Therefore, $F\cong\Hom_{\Cat{Group}}(\mbb{Z},-)$, i.e.\ forgetful $F\colon\Cat{Group}\to\Cat{Set}$ is representable by $\mbb{Z}$. \qed
\pagebreak[3]
	
	Similarly, for $F\colon \Cat{Ring}\to\Cat{Set}$, the free object generated by some generic element $x$ is $\mbb{Z}[x]$, the polynomial ring in one variable; we have:
	\begin{equation}
		F\cong\Hom_{\Cat{Ring}}(\mbb{Z}[x],-),\quad
		\Hom_{\Cat{Ring}}(\mbb{Z}[x],R)
		= \Bqty{
			\begin{aligned}
				\mbb{Z}[x] &\to R \\[-1.5ex]
				x &\mapsto r
			\end{aligned}
			\,\bigg|\,
			r\in R
		}
	\end{equation}
	\textit{Lesson}: Forgetful $\Cat{Cat} \to \Cat{Set}$ are often representable by the free object in \Cat{Cat}. \qedfull
	
	\item \textbf{Properties of contractible space}:
	
	\begin{enumerate}
	\item $X$ contractible: $\idty_X \simeq f_0\colon X\to X$ some constant map, $f_0(X) = \{x_0\}$. We can restrict the codomain of $f_0$ so that $f_0\colon X\to\{x_0\}$, in this way we have:
	\begin{subequations}
	\renewcommand{\theequation}{\theparentequation.\arabic{equation}}
	\begin{align}
		X\xrightarrow{f_0} \{x_0\}
			\hookrightarrow X
		&\simeq \idty_X,
		\label{eq:homotopy_on_X}\\
		\{x_0\}\hookrightarrow X
			\xrightarrow{f_0} \{x_0\}
		&\simeq \idty_{\{x_0\}},
	\end{align}
	\label{eq:homotopic_equiv}
	\end{subequations}
	This means that $f_0\colon X\to\{x_0\}$ isomorphic in $\Cat{hTop} = \Cat{Top}/\simeq$, which is precisely the definition of homotopic equivalence $X\simeq\{x_0\}$. ($\Rightarrow$)
	
	On the other hand ($\Leftarrow$), if $X\simeq\{x_0\}$, there exists some $f_0\colon X\to\{x_0\}$ that fulfills \eqref{eq:homotopic_equiv}. We can then extend the codomain s.t.\ $f_0\colon X\to X$, in this way \eqref{eq:homotopy_on_X} reads $f_0 \simeq \idty_X$, i.e.\ $X$ is contractible. 
	Therefore, $X$ contractible iff.\ homotopic equivalent to a single point. \qedfull[(a)]
	
	\item $\forall\ X$: Topological space, we can define its \textit{cone} as\footnote{
		See \wikiref{https://en.wikipedia.org/wiki/Cone\_(topology)}{Cone (topology)}. 
	}:
	\begin{equation}
		CX = (X\times I) / (X\times\{0\}),\quad
		I = [0,1]
	\end{equation}
	i.e.\ gluing together one end of the cylinder $X\times I$. Naturally $X\subset CX$ as a subspace; now we show that $CX$ is contractible. Using (a), we need only show that $\idty_{CX}\simeq f_0$ some constant map. 
	
	In fact, any point in $CX$ can be uniquely labeled by $[x,h]\in X\times I$, with the exception of the vertex $v \sim [x,0]\sim [x',0],\ \forall\ x,x'\in X$. We can then construct a homotopy $F$ by shrinking the cone towards the vertex $v$:
	\begin{equation}
	\begin{gathered}
		F\colon CX\times I\to CX,\quad
		F\pqty\big{[x,h],t} = [x,h\cdot t],\\
		F|_{CX\times 0} = v = \mrm{const},\quad
		F|_{CX\times 1} = \idty_X
	\end{gathered}
	\end{equation}
	This confirms that $\idty_{CX}\simeq v$: constant map. By (a), $CX$ is contractible. \qedfull[(b)]
\pagebreak[3]
	
	\item For $Y\simeq \{y_0\}$ contractible, given any $g\colon X\to Y$, we can deform the image $g(X)\subset Y$ to a single point, hence $g\simeq y_0$: constant map. 
	More precisely, we have:
	\begin{equation}
		\exists\ G \colon X\times I\to Y,\quad
		\text{s.t.}\quad
		G|_{X\times 0} = y_0 = \mrm{const},\quad
		G|_{X\times 1} = g
	\end{equation}
	Such $G$ can be explicitly constructed using $\idty_Y\simeq y_0$:
	\begin{gather}
		F \colon Y\times I\to Y,\quad
		F|_{Y\times 0} = y_0 = \mrm{const},\quad
		F|_{Y\times 1} = \idty_Y,
		\label{eq:contractible_explicit}\\[1ex]
		G(x,t) = F\pqty\big{g(x),t}
%		\left\lbrace
%		\begin{aligned}
%			\,& F\pqty\big{g(x),1-2t},
%			&& 0\le t\le\tfrac{1}{2}\\
%			& F\pqty\big{g'(x),2t-1},
%			&& \tfrac{1}{2}\le t\le 1\\
%		\end{aligned}
%		\right.
	\end{gather}
\pagebreak[3]
	
	In summary, we have proven that $g\simeq y_0,\ \forall\ g\in\Hom_{\Cat{Top}}(X,Y)$. By definition, this means that $\Hom_{\Cat{hTop}}(X,Y) = \Hom_{\Cat{Top}}(X,Y)\big/_{\simeq} = \Bqty\big{[y_0]}$ a single point. \qedfull[(c)]
	
	\item For $X\simeq \{x_0\}$ contractible, similar to \eqref{eq:contractible_explicit}, we have homotopy $F\colon X\times I \to X$. Given any $f\colon X\to Y$, the composition $f\circ F\colon X\times I \to Y$ yields $f\simeq f(x_0)$: constant map. 
	
	Furthermore, for $Y$: path connected, there is a path $\gamma\colon I\to Y$ connecting $f(x_0)$ and some $y_0\in Y$, therefore $f(x_0) \simeq y_0\colon X\to Y$ constant maps. More precisely, we have:
	\begin{equation}
		\gamma\colon I\to Y,\quad
			\gamma(0) = y_0,\quad
			\gamma(1) = f(x_0),\\
		G\colon X\times I\to Y,\quad
			G(x,t) = \gamma(t)
	\end{equation}
	Which gives $f(x_0) \simeq y_0,\ \forall\,f$, independent of the choice of $f$. This means that $f\simeq f(x_0) \simeq y_0$: constant map, therefore $\Hom_{\Cat{hTop}}(X,Y) = \Bqty\big{[y_0]}$ a single point. \qedfull[(d)]
	
	\end{enumerate}
	
	\item \textbf{Example of homotopic inequivalence}\footnote{
		This proof is produced thanks to helpful insights from \textit{谷夏} and \textit{於子雄}. 
	}:
	\begin{equation}
	\begin{gathered}
		X = \{0\} \cup \Bqty{
			\frac{1}{n} \,\bigg|\, n\in\mbb{Z}_+
		},\quad
		Y = \{0\} \cup \mbb{Z}_+\\[.5ex]
		X,Y\subset\mbb{R}\colon
		\text{subspace topology}
	\end{gathered}
	\end{equation}
	Assume $X\simeq Y$, then similar to \eqref{eq:homotopic_equiv}, we have $
		Y\xrightarrow{g}X
		\xrightarrow{f}Y
		\simeq\idty_Y
	$. However, note that $Y$ has discrete topology, in such case any map $f\circ g$ homotopic to $\idty_Y$ must be $\idty_Y$ itself: $f\circ g = \idty_Y$. 
	
	More specifically, consider:
	\begin{equation}
		F \colon Y\times I\to Y,\quad
		F|_{Y\times 0} = f\circ g,\quad
		F|_{Y\times 1} = \idty_Y
	\end{equation}
	Any point $n\in Y$ is both open and closed, therefore its pre-image $F^{-1}(n)\subset Y\times I$ is also both open and closed, and by $F|_{Y\times 1} = \idty_Y$ we know that $F(y,1) = y,\ (y,1)\in F^{-1}(y)$, therefore the only possibility is that $F(\{y\}\times I) = y$, i.e.\ $f\circ g=\idty_Y$, which implies that $g$ is injective and $f$ is surjective. 
	
	However, $f\colon X\to Y$ cannot be surjective due to the complication around $0\in X$. Consider $f^{-1}\pqty\big{f(0)}\ni 0$, since $f(0)\in Y$ both open and closed, $f^{-1}\pqty\big{f(0)}\subset X$ must also be both open and closed. But any open set $U\subset X$ is induced via subspace topology $X\subset\mbb{R}$; for $0\in U\subset X\subset\mbb{R}$, $U$ must contain $\infty$-many elements: 
	\begin{equation}
		\Bqty{
			\frac{1}{n} \,\bigg|\, n\ge N_0
		}
		\subset U
		\subset f^{-1}\pqty\big{f(0)},\quad
		\text{for some $N_0$, for any $U\ni x$}
	\end{equation}
	Hence $f(X) = f(0)\cup f(
		\Bqty{
			\frac{1}{n} \,|\, n< N_0
		}
	),\ f(X)\subset Y$ a finite set, i.e. $f\colon X\to Y$ is never surjective. Therefore, $X\not\simeq Y$ by contradiction. \qedfull
\pagebreak
	
	\item \textbf{Fundamental group of topological group is abelian}\footnote{%
		This proof is produced with the help of \https{math.stackexchange.com/q/727999}. 
	}:
	
	From a categorical point of view, the fundamental group $\pi_1(G)$ of a topological group $G$ can be seen as a functor:
	\begin{equation}
		G\in\Cat{TopGroup}
		\ \xhookrightarrow{\quad}\ \Cat{Top}
		\ \xrightarrow{\ \pi_1\,}\ \Cat{Group}
		\ni \pi_1(G)
	\end{equation}
%	
	$\Cat{TopGroup}\subset\Cat{Top}$ is a subcategory with additional group structure, i.e.\ $(G,\cdot)\in\Cat{TopGroup}$ is a \textit{group object}\footnote{
		See \wikiref{https://en.wikipedia.org/wiki/Group\_object}{Group object}. 
	} in \Cat{Top}, with $\mquote{\cdot}$ denoting its product operation $(\cdot)\colon G\times G\to G$. Correspondingly, $\pi_1(\Cat{TopGroup})$ should be \textit{group objects of \Cat{Group}}, which have an \textit{additional} group structure $(\star) = \pi_1(\cdot)$, along with the usual group product $\mquote{*}$ in \Cat{Group}. 
	
	In total, we have three different group structures (represented by their product operation):
	\begin{align}
		(\cdot)\colon\ %
			&G\times G\to G,\\
		(*)\colon\ %
			&\pi_1(G)\times\pi_1(G)\to\pi_1(G),\\
		(\star) = \pi_1(\cdot)\colon\ %
			&\pi_1(G)\times\pi_1(G)\to\pi_1(G),
	\end{align}
	Note that $\pi_1(G) = \mop{Aut}_{\Pi_1(G)}\idty_G$, i.e.\ loop classes $[\gamma]$ in $G$; $(*)$ is defined as joining two loops, while $(\star) = \pi_1(\cdot)$ is defined as the translation of loop classes by pointwise group product $(\cdot)$, 
	\begin{equation}
		[\gamma_1]\star[\gamma_2]
		= [\gamma_1\cdot\gamma_2]
	\end{equation}
	
	With the above definitions, we observe that:
	\begin{equation}
		([\gamma_1]\star[\gamma_2])
		* ([\eta_1]\star[\eta_2])
		= ([\gamma_1]*[\eta_1])
		\star ([\gamma_2]*[\eta_2])
		\label{eq:distributive_law}
	\end{equation}
	By definition, they are both equal to $
		[(\gamma_1\cdot\gamma_2)*(\eta_1\cdot\eta_2)]
	$. What's surprising is that by using only the group axioms and ``distributive law'' \eqref{eq:distributive_law}, we can show that $(\star)$ and $(*)$ must always coincide: $(\star) = (*)$, and they have to be in fact, commutative. This is the \textit{Eckmann–Hilton argument}\footnote{
		See \wikiref{https://en.wikipedia.org/wiki/Eckmann\%E2\%80\%93Hilton\_argument}{Eckmann–Hilton argument}. 
	}. 
	
	Proof of this argument is straight-forward; first, observe that the units of the two operations coincide:
	\begin{equation}
		1_{\star}
		= 1_{\star}\star 1_{\star}
		= (1_{*} * 1_{\star})
			\star (1_{\star} * 1_{*})
		\xlongequal{\eqref{eq:distributive_law}}
			(1_{*}\star 1_{\star})
			* (1_{\star}\star 1_{*})
		= 1_{*} * 1_{*}
		= 1_{*}
	\end{equation}
	Further manipulation using \eqref{eq:distributive_law} confirms that the two operations coincide and are commutative:
	\begin{align}
	\begin{aligned}
		[\gamma] * [\eta]
		&= (1\star [\gamma]) * ([\eta]\star 1)
		\xlongequal{\eqref{eq:distributive_law}}
			(1 * [\eta])\star ([\gamma] * 1) \\
		&= [\eta]\star [\gamma] \\
		&= ([\eta] * 1)\star (1 * [\gamma])
		\xlongequal{\eqref{eq:distributive_law}}
			([\eta]\star 1) * (1\star [\gamma]) \\
		&= [\eta] * [\gamma]
	\end{aligned}
	\end{align}
	
	In summary, we find that the group objects in $\Cat{Group}$ are indeed abelian groups, which means that $\pi_1(G)$ for $G\in\Cat{TopGroup}$ must be abelian. \qedfull
	
	\end{enumerate}

\printbibliography[%
	title = {参考文献} %
	,heading = bibintoc
]
\end{document}
