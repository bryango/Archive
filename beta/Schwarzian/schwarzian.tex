% !TeX document-id = {b5392a94-51a3-49d1-9ba5-698bc09f9d35}
% !TeX encoding = UTF-8
% !TeX spellcheck = en_US
% !TeX TXS-program:bibliography = biber -l zh__pinyin --output-safechars %

\documentclass[a4paper
	,10pt
%	,twoside
]{article}

% to be `\input` in subfolders,
% ... therefore the path should be relative to subfolders.

\usepackage{iftex}
\ifPDFTeX
\else
	\usepackage[UTF8
		,heading=false
		,scheme=plain % English Document
	]{ctex}
\fi
%\ctexset{autoindent=true}
\usepackage{indentfirst}

\input{../.modules/basics/macros.tex}
\input{../.modules/preamble_base.tex}
\input{../.modules/preamble_beamer.tex}
\input{../.modules/basics/biblatex.tex}


%Misc
	\usepackage{lilyglyphs}
	\newcommand{\indicator}{$\text{\clefG}$}
	\newcommand{\indicatorInline}{$\text{\clefGInline}$}

\newcommand{\legacyReference}{{
%	\clearpage\par
%	\quad\clearpage
	\def{\midquote}{\textbf{PAST WORK, AS TEMPLATE}}
	\newparagraph
}}

% Settings
\counterwithout{equation}{section}
\mathtoolsset{showonlyrefs=false}
%\DeclareTextFontCommand{\textbf}{\sffamily}

% Spacing
\geometry{footnotesep=2\baselineskip} % pre footnote split
\setlength{\parskip}{.5\baselineskip}
\renewcommand{\baselinestretch}{1.15}


%% List
%	\setlist*{
%		listparindent=\parindent
%		,labelindent=\parindent
%		,parsep=\parskip
%		,itemsep=1.2\parskip
%	}


\addtobeamertemplate{navigation symbols}{}{%
    \usebeamerfont{footline}%
%    \usebeamercolor[fg]{footline}%
    \hspace{1em}%
    \large\insertframenumber/\inserttotalframenumber
}

\makeatletter
\setbeamertemplate{headline}
{%
    \begin{beamercolorbox}[wd=\paperwidth,colsep=1.5pt]{upper separation line head}
    \end{beamercolorbox}
    \begin{beamercolorbox}[wd=\paperwidth,ht=2.5ex,dp=1.125ex,%
      leftskip=.3cm,rightskip=.3cm plus1fil]{title in head/foot}
      \usebeamerfont{title in head/foot}\insertshorttitle
    \end{beamercolorbox}
    \begin{beamercolorbox}[wd=\paperwidth,ht=2.5ex,dp=1.125ex,%
      leftskip=.3cm,rightskip=.3cm plus1fil]{section in head/foot}
      \usebeamerfont{section in head/foot}%
      \ifbeamer@tree@showhooks
        \setbox\beamer@tempbox=\hbox{\insertsectionhead}%
        \ifdim\wd\beamer@tempbox>1pt%
          \hskip2pt\raise1.9pt\hbox{\vrule width0.4pt height1.875ex\vrule width 5pt height0.4pt}%
          \hskip1pt%
        \fi%
      \else%  
        \hskip6pt%
      \fi%
      \insertsectionhead
    \end{beamercolorbox}
% Code for subsections removed here
}
\makeatother
\input{../.modules/basics/biblatex.tex}

\title{Finding finite transformations}
\addbibresource{schwarzian.bib}

%%% ID: sensitive, do NOT publish!
\InputIfFileExists{id.tex}{}{}

\makeatletter
\newcommand{\nobeginpar}{\@beginparpenalty=10000}
\makeatother

\usepackage{cancel}
%\counterwithout{equation}{section}

\begin{document}
\maketitle
\pagenumbering{arabic}
\thispagestyle{empty}

	Sometimes we encounter infinitesimal transformations and we'd like to integrate them to some finite transformations. The trick of doing this is to use the fact that transformations should \textbf{compose nicely}; this idea is well introduced in classical texts of group theory such as \textcite{wybourne1993classical}; here we shall use it to construct the finite transformations\footnote{
		I'd like to thank \cjk[gkai]{\textit{孙悠然}} for re-inventing this method and explaining it to me. Another good reference for this is \url{http://personalpages.to.infn.it/~billo/didatt/gruppi/gruppi.html}, or more specifically, \url{http://personalpages.to.infn.it/~billo/didatt/gruppi/liegroups.pdf}. 
	}. 
	
	In the case of the Schwarzian derivative, a derivation following the same recipe can be found in \textcite{Blumenhagen:2013fgp}. 

\setlength{\parskip}{.1\baselineskip}
\tableofcontents
\setlength{\parskip}{\parskipnorm}

\section{Stress Tensor and the Schwarzian Derivative}
	We will work with a well-known example: the transformation of the stress tensor $T(z)$ in a 2D CFT. We know that the $TT$ OPE takes the following form:
	\begin{equation}
		T(z)\,T(0)
		\sim \frac{c/2}{z^4}
			+ \frac{h\,T(0)}{z^2}
			+ \frac{\pd T(0)}{z},
	\quad h = 2
	\end{equation}
	This is related to the algebra of its modes, namely the Virasoro algebra with central charge $c$. Using Ward identity, we can further relate this to the infinitesimal transformation of $T$, under the conformal map $z\mapsto z+\epsilon$:
	\begin{equation}
		\var_\epsilon {T(z)}
		= - \operatorname*{Res}_{z'\to z}
				\epsilon(z')\,T(z')\,T(z)
		= - \epsilon\,\pd T
			- h\,\pd\epsilon\,T
			- \frac{c}{12}\,\pd^3 \epsilon,
	\quad h = 2
	\end{equation}
	
	We see that $T(z) = T_{zz}(z)$ is almost like a tensor: the first two terms in the infinitesimal transformation are from translations and rotations, as expected, but there is a third \textit{anomalous} term which is \textbf{independent of $T$}. Thus the integrated, finite transformation of this term should be a \textbf{functional of $w(z)$} that is independent of $T$, i.e.
	\begin{equation}
		- \frac{c}{12}\,\pd^3 \epsilon
	\longrightarrow
		- \frac{c}{12}\, S[w]_{(z)}
		\equiv - \frac{c}{12} \Bqty{w,z},
	\quad
		S[w]_{(z)} = S[w,\pd_z w,\cdots]
	\end{equation}
\pagebreak

\noindent
	The full integrated transformation of $T$ can then be written as:
	\begin{equation}
		T_{zz}(z)
		= T_{ww}(w) \pqty{\pdv{w}{z}}^{\!2}
			+ \frac{c}{12} \Bqty{w,z},
	\end{equation}
	\vspace{-.6\baselineskip}
	\begin{equation}
		T_{ww}(w) \dd{w}^2
		= T_{zz}(z) \dd{z}^2
			- \frac{c}{12} \Bqty{w,z} \dd{z}^2
	\end{equation}
	
	To find $S = S[w]$, consider \textbf{two successive transformations:}
	\begin{equation}
		z \xmapsto{\ w\ } w \xmapsto{\ v\ } v
	\end{equation}
	Here we've abused the symbols $w,v$ to label both the transformations and their images: $w\colon z\mapsto w(z)$, $v\colon w\mapsto v = v(w) = v(w(z))$. 
	To say that \textbf{$S[w]_{(z)}$ compose nicely} under successive transformations is to say that $S[w]_{(z)} = \Bqty{w,z}$ satisfies the following \textbf{chain rule:}
	\begin{equation}
		\Bqty{v,z} \dd{z}^2
		= \Bqty{v,w} \dd{w}^2
		+ \Bqty{w,z} \dd{z}^2
	\label{eq:schwarzian_chain}
	\end{equation}
%\pagebreak
	
	Generally, we can think of the chain rules as \textit{representations} of the transformation group; in this case the group product is given by the composition $v\circ w$. For example, the usual chain rule is a representation on the cotangent bundle by the pullback $
		(v\circ w)^*
		= w^* \circ v^*
	$; we have:
	\begin{equation}
		(v\circ w)^* \dd{v}
		= \dv{v}{z} \dd{z}
		= \dv{v}{w} \dv{w}{z} \dd{z}
		= w^* \circ v^* \dd{v}
	\end{equation}
	The Schwarzian derivative is just another representation of the chain rule, on the bundle of symmetric 2-differentials $\dd{z}^2$. 
	
	To actually solve for the functional dependence $S = S[w]_{(z)}$, we consider the special case:
	\begin{equation}
		v = w + \var{w},
	\quad \var{w} = \epsilon
	\end{equation}
	Going back to the chain rule, we then have:
	\begin{equation}
		\pqty{S[w + \epsilon] - S[w]}_{(z)} \dd{z}^2
		= S[w + \epsilon]_{(w)} \dd{w}^2
		= \pd_w^3 \epsilon \dd{w}^2
	\end{equation}
	\vspace{-.8\baselineskip}
	\begin{equation}
		\var{S}
		= \pqty{\pdv[3]{w} \var{w}} (\pd_z w)^2
	\label{eq:functional_eq}
	\end{equation}
	
	We see that the chain rule can be re-written as functional differentials; in this form it has \textbf{no explicit dependence of $z$, and it's a functional equation for $
		S[w]_{(z)} = S[w,\pd_z w,\cdots]
	$}. This reveals the nature of the above 2-step transformation $z\mapsto w\mapsto v$: with $z\mapsto w$ we construct the function $w(z)$, while with $w\mapsto v = w + \var{w}$ we consider the variation of $S[w]_{(z)}$ w.r.t.~$w(z)$. 
	
	\newparagraph
	Now suppose that $w(z)$ is constructed by composing infinitesimal transformations $z\mapsto z + \epsilon$; namely, $w = w_t(z)$ is the integral curve of a series of infinitesimal transformations along $\epsilon$, starting from $z$. From the perspective of differential geometry, \textbf{$w = w_t(z)$ is the flow of vector field $\epsilon$}, and $t$ is the flow parameter. This is precisely the \textbf{exponential map} that brings $t\epsilon$ in the Lie algebra to $w_t(z)$ in the Lie group. 
	
	The crucial observation is that \textbf{the functional equation \eqref{eq:functional_eq} should hold for any $w = w_t(z)$}. Along the flow, we have:
	\begin{equation}
		\var{w_t} = \var{t} \pdd{t} w_t
	\end{equation}
	We then have:
	\begin{equation}
		\pdd{t} S
		= (\pd_z w_t)^2
			\pdv[3]{w_t}\,\pdd{t} w_t
	\end{equation}
	
	To solve the above equation, we try to complete the right-hand side into a total $t$-differential: $
		\pdd{t} S = \pdd{t} ({\cdots})
	$. Note that:
	\begin{equation}
		\pdv{w} = \frac{1}{\pd w}\pdv{z},
	\quad
		\pdv[3]{w}
		= \frac{1}{(\pd w)^3} \pd^3
			- 3 \frac{\pd^2 w}{(\pd w)^4} \pd^2
			+ \pqty{
				3\frac{(\pd^2 w)^2}{(\pd w)^5}
				- \frac{\pd^3 w}{(\pd w)^4}
			} \pd
	\end{equation}
	Namely, we can trade the functional derivative w.r.t.~$w$ to regular derivative $\pd = \pdd{z}$, with a $\frac{1}{\pd w}$ factor. 
	From now on we shall drop the subscript $t$ for $w = w_t$, but the $t$-dependence is not forgotten. We thus have:
	\begin{equation}
	\begin{aligned}
		\pdd{t} S
		&= \frac{1}{\pd w}
				\pd_t \pd^3 w
			- 3 \frac{\pd^2 w}{(\pd w)^2}
				\pd_t \pd^2 w
			+ \pqty{
				3\frac{(\pd^2 w)^2}{(\pd w)^3}
				- \frac{\pd^3 w}{(\pd w)^2}
			} \pd_t \pd w \\
		&= \pdd{t} \frac{\pd^3 w}{\pd w}
			- 3 \frac{\pd^2 w}{(\pd w)^2}
				\pd_t \pd^2 w
			+ 3\frac{(\pd^2 w)^2}{(\pd w)^3}
				\pd_t \pd w \\
		&= \pdd{t} \frac{\pd^3 w}{\pd w}
			- \frac{3}{2}\,\pdd{t} 
				\pqty{\frac{\pd^2 w}{\pd w}}^{\!\!2} \\
	\end{aligned}
	\end{equation}
	
	Indeed, the right-hand side can be arranged into a total $t$-derivative; we have:
	\begin{equation}
		S[w](z)
		\equiv \Bqty{w,z}
		= \frac{\pd^3 w}{\pd w}
			- \frac{3}{2}
			\pqty{\frac{\pd^2 w}{\pd w}}^{\!\!2}
	\end{equation}
	This is the \textit{Schwarzian derivative}. The $t$-integration constant is fixed by requiring that $\Bqty{z,z} = 0$. We see that to find the finite transformation, instead of explicit integrating $\var_\epsilon T$, we can \textbf{assume that there is a flow $S[w]_{(z)}$ where $w = w_t(z)$; by completing the total derivative $\pdd{t} S$ along $w_t(z)$ with arbitrary $t$, we can recover the functional dependence $S = S[w]$}. 
	
\section{Linear Dilaton}
	Let's look at another example: the linear dilaton $X$ with infinitesimal transformation:
	\begin{equation}
		\var{X}
		= - \epsilon\,\pd X
			- \bar{\epsilon}\,\pdbar X
			- \frac{\alpha'V}{2}\,\pqty{
				\pd\epsilon
				+ \pdbar\bar{\epsilon}
			}
	\label{eq:dilaton_transformation}
	\end{equation}
	Here we've dropped the spacetime index $X^\mu \leadsto X$. We see that it is again in the form of translation plus a shift, and the shift is independent of the field $X$. 
	
	To understand this infinitesimal transformation, first recall that $
		\pqty{
			\pd\epsilon
			+ \pdbar\bar{\epsilon}
		}
	$ is precisely the contribution of Weyl rescaling in the conformal symmetry; we have:
	\begin{equation}
		0 = \var{g_{ab}}
		= \pqty{2\var{\omega} - \mcal{L}_\epsilon}\,
			g_{ab}
		= 2\,\pqty{
				\var{\omega} g_{ab}
				- \nabla_{[a} \epsilon_{b]}
			},
	\quad
		2\var{\omega}
		= \frac{2}{d}\,\nabla\cdot\epsilon,
	\label{eq:conformal_Killing}
	\end{equation}
	This is in fact the \textit{conformal Killing equation}; the generators of conformal symmetry $\epsilon$ is the solution to this equation, which holds for general dimension $d$. 
	Here we have $d = 2$, and conformal generators are given by $\epsilon(z), \bar{\epsilon}(\bar{z})$. If we further restrict to flat metric, then we have $
		2\var{\omega}
		= \pd\epsilon
			+ \pdbar\bar{\epsilon}
	$. 
	
	Therefore, the infinitesimal transformation is saying that $X$ transforms in a non-trivial manner under Weyl rescaling:
	\begin{equation}
		\var_\omega g_{ab} = 2\var{\omega} g_{ab},
	\quad
		\var_\omega X
		= - \frac{\alpha' V}{2}\, 2\var{\omega}
	\end{equation}
	Actually, we have a similar result for the stress tensor transformation discussed above, which is associated with \textbf{Weyl anomaly}:
	\begin{equation}
		\var_\omega T(z)
		= - \frac{c}{12}\,\pd^2 (2\var{\omega})
	\end{equation}
	
	In fact, from $\var_\omega {X}$ we can already integrate over $\omega$ and recover the finite transformation for $\var_\omega {X}$, using the finite version of $
		2\var{\omega}
		= \pd\epsilon
			+ \pdbar\bar{\epsilon}
	$, namely the conformal Killing equation \eqref{eq:conformal_Killing}:
	\begin{equation}
		g'_{ab}
		= g_{ab}
		= e^{2\omega} \pdv{\sigma^c}{\sigma'^a}
			\pdv{\sigma^d}{\sigma'^b}\,g_{cd},
	\quad
		e^{-2\omega}
		= \frac{1}{d}\,
			g^{ab}
			\pdv{\sigma^c}{\sigma'^a}
			\pdv{\sigma^d}{\sigma'^b}
			\,g_{cd}
	\end{equation}
	For $d = 2$, note that $g_{z\bar{z}} = g_{\bar{z}z} = 2$, after carefully summing over indices, we have:
	\begin{equation}
		e^{-2\omega}
		= \frac{1}{2}\, g^{w\bar{w}}
			\pdv{z}{w}
			\pdv{\bar{z}}{\bar{w}}\,
			g_{z\bar{z}} \cdot 2
		= \pdv{z}{w} \pdv{\bar{z}}{\bar{w}},
	\quad
		e^{2\omega}
		= \pdv{w}{z} \pdv{\bar{w}}{\bar{z}}
		= \abs{\pd w}^2,
	\end{equation}
	\vspace{-.5\baselineskip}
	\begin{equation}
		X'(w)
		= X(z) - \frac{\alpha' V}{2}\, 2\omega
		= X(z) - \frac{\alpha' V}{2}\, \ln e^{2\omega}
		= X(z) - \frac{\alpha' V}{2}\, \ln \abs{\pd w}^2
	\label{eq:finite_Weyl}
	\end{equation}
\subsection{Worldsheet Dilaton Action}
	We see that under Weyl rescaling, $X$ gets an $X$-independent shift, proportional to the Weyl exponent. What does this mean physically? In fact, this is a ``remnant'' of the worldsheet dilaton term in bosonic string theory \cite{Polchinski:1998rq}:
	\begin{equation}
		S^\Phi = \frac{1}{4\pi\alpha'}
			\int \dd[2]{\sigma} \sqrt{g}\,
				\alpha' R\,\Phi(X)
	\end{equation}
	Although this term vanishes in the flat limit $R \to 0$, its effects persist in the stress tensor, since the stress tensor depends on the functional derivative of $g^{ab}$, and this does not vanish when $R \to 0$ \cite{Tong:2009np}:
	\begin{equation}
		T^\Phi_{ab}
		= -4\pi \frac{1}{\sqrt{g}} \fdv{S^\Phi}{g^{ab}}
		= \pqty{
				\cdv{a} \cdv{b} {}
				- g_{ab} \nabla^c \nabla_c
			}\,\Phi
		\xrightarrow{\ g_{ab} = \delta_{ab}\ }
		\pqty{
				\pd_a \pd_b
				- \delta_{ab}\, \pd^c \pd_c
			}\,\Phi
	\end{equation}
	
	Note that $T_{ab}$ is not traceless, namely the dilaton coupling $S^\Phi$ is generally \textbf{not Weyl invariant} at the classical level, unless $\pd^c \pd_c \Phi = 0$, or $\Phi(X) \equiv \Phi_0$ constant. 
	However, note that $R\,\Phi(X)$ comes with an $\alpha'$ in the Lagrangian; therefore, its contribution mixes with $\order{\alpha'}$ loop corrections from the other terms, and it's possible that the \textbf{total combined Weyl anomaly cancels} in the presence of a non-trivial $\Phi(X)$, thus restoring Weyl invariance. 
	
	In fact, the full theory is Weyl invariant if the \textit{dilaton profile} $\Phi(X)$, the spacetime metric $G_{\mu\nu}(X)$ and gauge field $B_{\mu\nu}(X)$ all have vanishing beta functions:
	\begin{equation}
		\beta^G_{\mu\nu}
		= \beta^B_{\mu\nu}
		= \beta^\Phi = 0
	\label{eq:beta_constraints}
	\end{equation}
	Which gives the spacetime equations of motion for $G_{\mu\nu}(X), B_{\mu\nu}(X)$ and $\Phi(X)$. 
	Suppose $B_{\mu\nu}(X) \equiv 0$; at 1-loop, we have:
	\begin{equation}
		0 = \beta^\Phi
		= \frac{D - 26}{6}
			- \frac{\alpha'}{2} \nabla^\mu \cdv{\mu} \Phi
			+ \alpha' \nabla^\mu \Phi \cdv{\mu} \Phi
			+ \order{\alpha'^2},
	\quad
		B_{\mu\nu}(X) \equiv 0
	\end{equation}
	We see that it's actually possible to move away from the critical dimension $D = 26$ by turning on a large gradient of $\Phi$, while keeping the theory consistent. 
	
	In particular, the \textit{linear dilaton theory} is an \textit{exact} solution of \eqref{eq:beta_constraints} to \textit{all loops} given by:
	\begin{equation}
		G_{\mu\nu}(X) = \eta_{\mu\nu},
	\quad
		B_{\mu\nu}(X) = 0,
	\quad
		\Phi(X) = V_\mu X^\mu,
	\quad
		V^2 = \frac{26 - D}{6\alpha'}
	\end{equation}
	\vspace{-.5\baselineskip}
	\begin{equation}
		T(z)
		= T^X_{zz} + T^\Phi_{zz}
		= -\frac{1}{\alpha'}\,\normorder{\,\pd X \pd X}
			+ V \pd^2 X
	\end{equation}
	The reason that this holds to \textit{all loops} is that we can \textit{define} an exact CFT with this action and stress tensor $T(z)$. Although $T^\Phi_{ab}$ is generally not traceless, in this case its trace vanishes on-shell: $\pd \pdbar X = 0$, so indeed we have conformal invariance. 
	
	Finally, by computing $TX$ OPE and applying Ward identity, we \textbf{basically \textit{impose} the transformation rule \eqref{eq:dilaton_transformation} on the field $X$}. One can thus interpret $\var_\omega X$ as some sort of \textbf{anomalous quantum ``dimension'' of $X$}; in this case it is not a rescaling of $X$ like what we usually encounter, but is instead a \textit{shift}, introduced to preserve Weyl (conformal) invariance in the quantum theory. 
	
\subsection{Finite Transformation of Linear Dilaton}
	Although we've already obtained the finite transformation \eqref{eq:finite_Weyl} by viewing it as a Weyl rescaling, we would like to repeat the same program for finding finite transformations, just like what we've done for the Schwarzian derivative, and then confirm the above results. 
	
	In the case of linear dilaton, there are no indices of worldsheet coordinates in $X$, so the composition of finite transformations $F[w]_{(z)}$ simply gives the following ``chain rule'':
	\begin{equation}
		F[v]_{(z)} = F[v]_{(w)} + F[w]_{(z)}
	\end{equation}
	There is no extra $\dd{z}^2$ factors like the ones in the Schwarzian chain rule \eqref{eq:schwarzian_chain}. 
	
	For simplicity, we shall restrict to the \textbf{holomorphic part} of the transformation.
	We note that $F$ converts the product in the usual chain rule $
		\dv{v}{z}
		= \dv{v}{w} \dv{w}{z} 
	$ into a sum. This in fact implies that $F$ must be proportional to $\log \pd w$. But let's be patient and again consider the flow $w_t(z)$ as in the Schwarzian case; we now have:
	\begin{equation}
		\pdd{t} F
		= - \frac{\alpha'V}{2}
				\pdv{w_t} \pdd{t} w_t
		= - \frac{\alpha'V}{2}
				\frac{1}{\pd w_t} \pdd{t} \pd w_t
		= - \frac{\alpha'V}{2}
				\pdd{t} \ln \pd w_t
	\end{equation}
	Indeed we have $
		F = - \frac{\alpha'V}{2} \ln \pd w
	$, as expected. If we include the anti-holomorphic part, we recover the finite transformation \eqref{eq:finite_Weyl}:
	\begin{equation}
		X'(w)
		= X(z) - \frac{\alpha' V}{2}\, \ln \abs{\pd w}^2
	\end{equation}
	
%\nocite{Polchinski:1998rq}
	
%\vspace{1.2\baselineskip}
%\pagebreak[4]
\raggedright
\printbibliography[%
%	title = {参考文献} %
	,heading = bibintoc
]
\end{document}
