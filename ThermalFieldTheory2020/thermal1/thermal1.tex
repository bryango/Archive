% !TeX encoding = UTF-8
% !TeX spellcheck = en_US
% !TeX TXS-program:bibliography = biber -l zh__pinyin --output-safechars %

\documentclass[a4paper,10pt]{article}

\newcommand{\hwNumber}{1}

% to be `\input` in subfolders,
% ... therefore the path should be relative to subfolders.

\usepackage[UTF8
	,heading=false
	,scheme=plain % English Document
]{ctex}
\usepackage{indentfirst}

\input{../.modules/basics/macros.tex}
\input{../.modules/preamble_base.tex}
\input{../.modules/preamble_notes.tex}

\newcommand{\legacyReference}{{
	\clearpage\par
	\quad\clearpage
	\renewcommand{\midquote}{\textbf{PAST WORK, AS TEMPLATE}}
	\newparagraph
}}

% Settings
\counterwithout{equation}{section}
\mathtoolsset{showonlyrefs=false}
%\DeclareTextFontCommand{\textbf}{\sffamily}
\renewcommand{\midquote}{\quad}
\resizegathersetup{equations=false}

% Spacing
\geometry{footnotesep=2\baselineskip} % pre footnote split
\setlength{\parskip}{.5\baselineskip}
\renewcommand{\baselinestretch}{1.15}

%Title
	\posttitle{
		\hfill\Large\ccbyncsajp
		\par\end{flushleft}%
		\vspace*{-.7ex}\hrule%
	}
	\preauthor{\vspace{-1.5ex}%
		\flushleft\itshape%
	}
	\postauthor{\hfill}
	\predate{\noindent\ttfamily Compiled @ }
	\postdate{\vspace{.5ex}}

	\title{Finite Temperature Field Theory \textnumero\hwNumber}
	\author{\signature Bryan}
	\date{\today}

% List
	\setlist*{
		listparindent=\parindent
		,labelindent=\parindent
		,parsep=\parskip
		,itemsep=1.2\parskip
	}
	\setlist*[enumerate,1]{
		align=left
		,label=\fbox{\textbf{\arabic*}}
		,itemsep=.5\baselineskip
	}

\input{../.modules/basics/biblatex.tex}

%%% ID: sensitive, do NOT publish!
\InputIfFileExists{../id.tex}{}{}

\begin{document}
\maketitle
\pagestyle{headings}
\pagenumbering{arabic}
\thispagestyle{empty}

\vspace*{-1.5\baselineskip}

\section*{Thermal propagator for harmonic oscilators}
\vspace*{-.8\baselineskip}
	\begin{equation}
		G(\tau)
		= \frac{1}{Z}
			\Tr \pqty{
				e^{-\beta H}
				x(\tau)\,
				x(0)
			}
	\end{equation}

\noindent
	The Hamiltonian is $
		H = \omega\pqty{
			a^\dagger a + \frac{1}{2}
		}
	$, while $
		x = \frac{1}{\sqrt{2m\omega}}\pqty{
			a + a^\dagger
		}
	$. 
\subsection*{Operator formalism}
	Mode expansion of Lorentzian operator in the Heisenberg picture:
	\begin{equation}
		x(t)
		= e^{iHt} \,x(0)\, e^{-iHt}
		= \frac{1}{\sqrt{2m\omega}} \pqty\Big{
				a\,e^{-i\omega t}
				+ a^\dagger e^{i\omega t}
			}
	\end{equation}
	This follows from the commutation relation between $a^{(\dagger)}$ and $H$, or from the quantum EOM in the Heisenberg picture. Wick-rotation to Euclidean signature: $\tau = it,\ t = -i\tau$, and we have:
	\begin{equation}
		x(\tau)
		= e^{H\tau} \,x(0)\, e^{-H\tau}
		= \frac{1}{\sqrt{2m\omega}} \pqty\Big{
				a\,e^{-\omega\tau}
				+ a^\dagger e^{\omega\tau}
			}
	\end{equation}
	
	Insert $x(\tau)\,x(0)$ into the definition of $G(\tau)$, and note that only $aa^\dagger$ and $a^\dagger a$ contribute to non-zero amplitudes, then we have:
	\begin{equation}
	\begin{aligned}
		G(\tau)
		&= \frac{1}{Z} \sum_{n=0}^\infty
			\mel**{n}{
				e^{-\beta\omega(n + \frac{1}{2})}\,
				\frac{
					aa^\dagger e^{-\omega\tau}
					+ a^\dagger a\, e^{\omega\tau}
				}{2m\omega}
			}{n} \\
		&= \frac{1}{Z} \sum_{n=0}^\infty
			\mel**{n}{
				e^{-\beta\omega(n + \frac{1}{2})}\,
				\frac{
					(n+1)\, e^{-\omega\tau}
					+ n\, e^{\omega\tau}
				}{2m\omega}
			}{n} \\
	\end{aligned}
	\end{equation}
	
	Note that:
	\begin{equation}
		\sum_{n=0}^\infty
			n\,e^{-\beta\omega n}
		= -\frac{1}{\beta} \pdv{\omega}
		\sum_{n=0}^\infty
			e^{-\beta\omega n}
	\label{eq:generatingTrick}
	\end{equation}
	The summations in $G(\tau)$ can be completed, and we get:
	\begin{equation}
		G(\tau)
		= \frac{1}{2m\omega}
			\frac{
				\cos \pqty\big{
					\pqty*{\frac{\beta}{2}-\tau} \omega
				}
			}{
				\sin \pqty\big{\frac{\beta\omega}{2}}
			}
	\end{equation}
\subsection*{Path Integral}
	The mathematical trick in \eqref{eq:generatingTrick} is crucial in our path integral derivation of $G(\tau)$. We have:
	\begin{gather}
		G(\tau)
		= \ave[\big]{x(\tau)\,x(0)}
		= \frac{1}{Z} \int \DD{x} e^{-S}
			x(\tau)\,x(0),\\
		S
		= \int_0^\beta \dd{t} \mcal{L}_E,
	\quad
		\mcal{L}_E
		= \frac{1}{2}\, m\dot{x}^2_{(\tau)}
			+ \frac{1}{2}\, m\omega^2 x^2_{(\tau)}
	\end{gather}
	Note that the path integral of a total derivative vanishes: $0 = \int \DD{x} \fdv{x}$ (see \textit{Polchinski}), we have:
	\begin{equation}
	\begin{aligned}
		0 &= \int \DD{x} \fdv{x(\tau)} \Bqty\Big{
				e^{-S} x(\tau')
			}
%		\\ &
		= \int \DD{x} e^{-S}\,\Bqty{
				- \fdv{S}{x(\tau)}\,x(\tau')
				+ \delta(\tau - \tau')
			}
	\end{aligned}
	\end{equation}
%\pagebreak[3]
	
	$\fdv{S}{x(\tau)}$ yields precisely the quantum EOM:
	\begin{equation}
		\fdv{S}{x(\tau)}
		= m\,\pqty{
				-\pdv[2]{}{\tau}
				+ \omega^2
			}\,x(\tau)
	\end{equation}
	After some re-organization, we get the quantum equation for the Green function (which, for such a free theory, is identical to the classical one):
	\begin{equation}
		m\,\pqty{
			-\pdv[2]{}{\tau}
			+ \omega^2
		} \int \DD{x} e^{-S}
			x(\tau)\,x(\tau')
		= \delta(\tau - \tau')
			\int \DD{x} e^{-S},
	\end{equation}
	\vspace{-.5\baselineskip}
	\begin{equation}
		\pqty{
			-\pdv[2]{}{\tau}
			+ \omega^2
		}\, G(\tau,\tau')
		= \frac{1}{m}\, \delta(\tau - \tau'),
	\quad
		G(\tau) = G(\tau,0)
	\end{equation}
	
	Note that $\tau \in [0,\beta]$ and $x(\tau)$ satisfies the periodic boundary condition: $x(\beta) = x(0)$, hence also $G(\beta) = G(0)$; we can extend the domain so that $x(\tau)$ is a function with period $\beta$, defined for $\tau\in\mbb{R}$. Furthermore, note that by definition $G(\tau,\tau') = G(\tau',\tau)$; this is the Feynman propagator! Integrating around $\tau = \tau'$ then setting $\tau' = 0$ gives:
	\begin{equation}
		G'(\beta^-) - G'(0^+)
		= \frac{1}{m}
	\end{equation}
	While $G(\tau)$ away from $\tau = 0$ grows (or decays) exponentially: $
		G(\tau)
		\propto e^{\omega\tau} + e^{\omega(\beta - \tau)}
	$. This fixes $G(\tau)$ uniquely:
	\begin{equation}
		G(\tau)
		= \frac{1}{2m\omega}
			\frac{
				\cos \pqty\big{
					\pqty*{\frac{\beta}{2}-\tau} \omega
				}
			}{
				\sin \pqty\big{\frac{\beta\omega}{2}}
			}
	\end{equation}

\printbibliography[%
%	title = {参考文献} %
	,heading = bibintoc
]
\end{document}
