% !TeX encoding = UTF-8
% !TeX spellcheck = en_US
% !TeX TXS-program:bibliography = biber -l zh__pinyin --output-safechars %

\documentclass[a4paper,10pt]{article}

\newcommand{\hwNumber}{2 [WIP]}

% Templates: 82ccb576e4df24e5eac4194b76230be360b4f733

% to be `\input` in subfolders,
% ... therefore the path should be relative to subfolders.

\usepackage[UTF8
	,heading=false
	,scheme=plain % English Document
]{ctex}
\usepackage{indentfirst}

\input{../.modules/basics/macros.tex}
\input{../.modules/preamble_base.tex}
\input{../.modules/preamble_notes.tex}

\newcommand{\legacyReference}{{
%	\clearpage\par
%	\quad\clearpage
	\renewcommand{\midquote}{\textbf{PAST WORK, AS TEMPLATE}}
	\newparagraph
}}

% Settings
\counterwithout{equation}{section}
\mathtoolsset{showonlyrefs=false}
%\DeclareTextFontCommand{\textbf}{\sffamily}
\renewcommand{\midquote}{\quad}

% Spacing
\geometry{footnotesep=2\baselineskip} % pre footnote split
\setlength{\parskip}{.5\baselineskip}
\renewcommand{\baselinestretch}{1.15}

%Title
	\posttitle{
		\hfill\Large\ccbyncsajp
		\par\end{flushleft}%
		\vspace*{-.7ex}\hrule%
	}
	\preauthor{\vspace{-1.5ex}%
		\flushleft\itshape%
	}
	\postauthor{\hfill}
	\predate{\noindent\ttfamily Compiled @ }
	\postdate{\vspace{.5ex}}

	\title{Advanced QFT \textnumero\hwNumber}
	\author{\signature Bryan}
	\date{\today}

% List
	\setlist*{
		listparindent=\parindent
		,labelindent=\parindent
		,parsep=\parskip
		,itemsep=1.2\parskip
	}

\input{../.modules/basics/biblatex.tex}

%%% ID: sensitive, do NOT publish!
%\InputIfFileExists{../id.tex}{}{}

\begin{document}
\maketitle
\pagestyle{headings}
\pagenumbering{arabic}
\thispagestyle{empty}

\vspace*{-1.5\baselineskip}

\section{Local Transformation}
	\vspace{-.5\baselineskip}
	\begin{equation}
		\var{A^a_\mu}
		= \pdd{\mu} \lambda^a(x)
			+ f\id{^a_{bc}} A^b_\mu \lambda^c(x),
	\label{eq:A_gauge_var}
	\end{equation}
	Here $f_{abc}$ is the totally anti-symmetric structure constant for a semi-simple Lie algebra $\mfrak{g}$, with generators $\{T_a\}_a$ and normalized Killing form $\delta_{ab}$. 
	
	\begin{itemize}
	\item The field strength is defined as follows:
	\begin{equation}
	\begin{aligned}
		F_{\mu\nu}
		\equiv F^a_{\mu\nu} T_a
		= [D_\mu,D_\nu]
		&= \bqty\big{
			\pdd{\mu} + A_\mu,
			\pdd{\nu} + A_\nu
		} \\
		&= \dd{A} + A\wedge A \\
		&= \pdd{\mu} A_\nu
			- \pdd{\nu} A_\mu
			+ f\id{^a_{bc}} A^b_\mu A^c_\nu\,T_a
	\end{aligned}
	\end{equation}
	Adjoint indices $a,b,\cdots$ are sometimes suppressed by contracting with $T_a$'s. By exploiting the anti-symmetric property of $f\id{^a_{bc}}$, along with the Jacobi identity, we get the infinitesimal transformation:
	\begin{gather}
	\qquad
	\begin{aligned}
		\var{F^a_{\mu\nu}}
		&= \pdd{\mu} \var{A^a_\nu}
			- \pdd{\nu} \var{A^a_\mu}
			+ f\id{^a_{bc}}
				\var{(A^b_\mu A^c_\nu)} \\[.5ex]
		&= f\id{^a_{bc}} \pqty\Big{
			\lambda^c \pqty{
				\pdd{\mu} A^b_\nu
				- \pdd{\nu} A^b_\mu
			} + \pqty{
				A^b_\nu\,\pdd{\mu} \lambda^c
				- A^b_\mu\,\pdd{\nu} \lambda^c
			} + \var{(A^b_\mu A^c_\nu)}
		} \\
		&= f\id{^a_{bc}} \pqty\Big{
			\lambda^c \pqty{
				F^b_{\mu\nu}
				- f\id{^b_{de}} A^d_\mu A^e_\nu
			} + \pqty{
				A^b_\nu\,\pqty{
					\smash{\underline{\var{A^c_\mu}}}
					- f\id{^c_{de}}
						A^d_\mu\,\lambda^e
				}
				- A^b_\mu\,\pqty{
					\smash{\underline{\var{A^c_\nu}}}
					- f\id{^c_{de}}
						A^d_\nu\,\lambda^e
				}
			} + \underline{\var{(A^b_\mu A^c_\nu)}}
		} \\
		&= {f\id{^a_{bc}}} \pqty\Big{
			\lambda^c \pqty{
				F^b_{\mu\nu}
				- {f\id{^b_{de}}}
					A^d_\mu A^e_\nu
			} - \pqty{
					{f\id{^c_{de}}}
						A^b_\nu A^d_\mu\,\lambda^e
					- {f\id{^c_{de}}}
						A^b_\mu A^d_\nu\,\lambda^e
			}
		} \\
		&= \underline{f\id{^a_{bc}}} \pqty\Big{
			\lambda^c \pqty{
				F^b_{\mu\nu}
				- \underline{f\id{^b_{de}}}
					A^d_\mu A^e_\nu
			} - \pqty{
					\underline{f\id{^c_{de}}}
						A^b_\nu A^d_\mu\,\lambda^e
					- \underline{f\id{^c_{de}}}
						A^b_\mu A^d_\nu\,\lambda^e
			}
		} \\
		&= f\id{^a_{bc}}
			\lambda^c F^b_{\mu\nu} \\[-4ex]
	\end{aligned}
	\quad
	\end{gather}
	
	When contracted with $T_a$, this yields:
	\begin{gather}
		\var{F_{\mu\nu}}
		= \lambda^c F^b_{\mu\nu}\,f\id{^a_{bc}} T_a
		= \lambda^c F^b_{\mu\nu}\,[T_b,T_c]
		= F_{\mu\nu}\cdot\lambda
			- \lambda\cdot F_{\mu\nu},\\[.8ex]
		\lambda = \lambda^c(x)\, T_c,\quad
		F_{\mu\nu} = F^b_{\mu\nu}\, T_b,\\[1.2ex]
		F_{\mu\nu}
		\ \longmapsto\ %
			e^{-\Lambda^a(x)\,T_a}\,F_{\mu\nu}\,
			e^{\Lambda^a(x)\,T_a}
	\end{gather}
	The exponentiation is valid even for local $
		\lambda = \lambda(x)
	$, since it is produced by integrating along the fiber direction $\lambda\to\Lambda$, not the spacetime direction $x$. This is the finite transformation w.r.t.\ $\Lambda(x)$. 
	
	\item For any matter field $\psi$ furnishing a representation of $\mfrak{g}$, we have:
	\begin{gather}
		T_a \psi = (T_a)\id{^i_j}\,\psi^j,\quad
		\var{\psi} = -\lambda^a(x)\,T_a\psi,\\[.5ex]
		\psi
			\ \longmapsto\ %
			e^{-\Lambda^a(x)\,T_a}\,\psi,\\
		D_\mu \psi
			\ \longmapsto\ %
			e^{-\Lambda^a(x)\,T_a}\,D_\mu\psi,
	\end{gather}
	In fact, \eqref{eq:A_gauge_var} is chosen to ensure that $D_\mu \psi$ transforms gauge covariantly just like $\psi$. Therefore, 
	\begin{gather}
	\begin{aligned}
		D_{\mu}
= \pdd{\mu} + A_\mu
		\ &\longmapsto\ %
			e^{-\Lambda^a(x)\,T_a}
			\circ D_{\mu}\circ
			e^{\Lambda^a(x)\,T_a} \\
		&\qquad\quad
		= e^{-\Lambda} \circ \pqty{
				\pdd{\mu} + A_\mu
			} \circ e^\Lambda\\
		&\qquad\quad
		= e^{-\Lambda}
			\circ \pdd{\mu} \circ e^\Lambda
			+ e^{-\Lambda} A_\mu\,e^\Lambda,\quad
		\Lambda = \Lambda^a(x)\,T_a,
	\end{aligned}\\[1ex]
		A_\mu
		\ \longmapsto\ %
			e^{-\Lambda}\,(\pdd{\mu} e^\Lambda)
			+ e^{-\Lambda} A_\mu\,e^\Lambda
%		\\
%		&\qquad\quad
		= T_a\,\pdd{\mu} \Lambda^a(x)
			+ e^{-\Lambda} A_\mu\,e^\Lambda
	\end{gather}
	
	\item $F^2\equiv F\wedge F$, we have:
	\begin{equation}
	\begin{aligned}
		F^2
		&= (\dd{A} + A\wedge A)
			\wedge (\dd{A} + A\wedge A) \\
		&= \dd{A} \wedge \dd{A}
			+ \dd{A} \wedge A \wedge A
			+ A \wedge A \wedge \dd{A}
			+ A \wedge A \wedge A \wedge A
	\end{aligned}
	\end{equation}
	The last term is proportional to $
		\epsilon_{abcd}\,T^a\,T^b\,T^c\,T^d
	$, hence its trace will vanish; therefore,
	\begin{gather}
	\begin{aligned}
		\tr F^2
		&= \tr \pqty{
			\dd{A} \wedge \dd{A}
			+ \dd{A} \wedge A \wedge A
			+ A \wedge A \wedge \dd{A}
		} \\
		&= \tr \pqty{
			\dd\pqty{\dd{A} \wedge A}
			+ \frac{2}{3}\,\dd\pqty{
				A \wedge A \wedge A
			}
		} \\
		&= \dd \tr \pqty{
			\dd{A} \wedge A
			+ \frac{2}{3}\,A \wedge A \wedge A
		} = \dd{\omega},
	\end{aligned}\\
		\omega = \tr \pqty{
			\dd{A} \wedge A
			+ \frac{2}{3}\,A \wedge A \wedge A
		}
	\end{gather}
	\vspace{-1.8\baselineskip}
	\end{itemize}
	\qedfull
	\vspace{-1\baselineskip}
\section{Relativistic Particle}
	\vspace{-.3\baselineskip}
	\begin{equation}
		L = \frac{1}{2e} \pqty{
			\frac{1}{c}\dv{X}{t}
		}^2 - \frac{e}{2}\,m^2 c^4
	\end{equation}
	\begin{itemize}
	\item For $t\mapsto t' = t - \xi(t)$, we have $
		X'(t') = X(t)
	$, therefore:
	\begin{equation}
		\var{X^\mu}
		= -\var{t} \dv{X^\mu}{t}
		= \xi(t)\,\dot{X}^\mu,
	\end{equation}
	Or more explicitly, $
		X^\mu(t)
		\mapsto X^\mu(t) + \xi(t)\,\dot{X}^\mu
	$. 
	
%	\item We have:
%	\begin{align}
%		\var{S}
%		= \int\dd{t} \var{{L}}
%		&= \int\dd{t} \Bqty{
%			\frac{1}{ec^2}\,
%				\dot{X}_\mu
%				\var{\dot{X}^\mu}
%			- \frac{\var{e}}{2}\,
%				\frac{1}{e^2 c^2} \dot{X}^2
%			- \frac{\var{e}}{2}\,m^2 c^4
%		}
%		\label{eq:particle_var_generic}\\[1ex]
%		&= - \int\dd{t} \Bqty{
%			\dv{t} \pqty{
%				\frac{1}{ec^2}\,
%				\dot{X}_\mu
%			} \var{X^\mu}
%			+ \frac{\var{e}}{2}\,\pqty{
%				\frac{1}{e^2 c^2} \dot{X}^2
%				+ m^2 c^4
%			}
%		} \notag\\[.5ex]
%		&\qquad + \int \dd\mspace{.5mu}\pqty{
%			\frac{1}{ec^2}\,
%				\dot{X}_\mu
%				\var{X^\mu}
%		}
%		\label{eq:particle_var_integrated}
%	\end{align}
%	This gives the equations of motion (EOMs): $
%		\dv{t} \pqty\big{
%			\frac{1}{ec^2}\,
%			\dot{X}_\mu
%		} = 0,\ %
%		\pqty{\frac{1}{c} \dv{X}{t}}^2
%		= -e^2 m^2 c^4
%	$. As we can see, the auxiliary field $e(t)$ is not dynamical; Fixing $e = 1$ is equivalent to setting $t = \tau$: the proper time, or affine parametrization for the massless case. 
	
	\item We have:
	\begin{equation}
	\begin{aligned}
		\var{{L}}
		&= \frac{1}{ec^2}\,
				\dot{X}_\mu
				\var{\dot{X}^\mu}
			- \frac{\var{e}}{2}\,
				\frac{1}{e^2 c^2} \dot{X}^2
			- \frac{\var{e}}{2}\,m^2 c^4 \\[.5ex]
		&= \frac{1}{ec^2}\,
				\xi\dot{X}_\mu\ddot{X}^\mu
			+ \frac{1}{ec^2}\,
				\dot{\xi}\dot{X}^2
			- \frac{\var{e}}{2}\,
				\frac{1}{e^2 c^2} \dot{X}^2
			- \frac{\var{e}}{2}\,m^2 c^4
	\end{aligned}
	\end{equation}
	For $
		S = \int\dd{t} {{L}}
	$ to be invariant, $\var{{L}}$ should be reduced to a total derivative, which can then be reduced to some vanishing boundary terms. 
	
	Consider $
		\var{e} = \dv{t}\pqty{e\xi}
		= \dot{e}\xi + e\dot{\xi}
	$, and we have:
	\begin{equation}
	\begin{aligned}
		\var{L}
		&= \frac{1}{ec^2}\,
				\xi\dot{X}_\mu\ddot{X}^\mu
			+ \frac{1}{2ec^2}\,
				\dot{\xi}\dot{X}^2
			- \frac{\dot{e}}{2e^2 c^2}\,
				\xi \dot{X}^2
			- \dv{t}\pqty{
				\frac{1}{2}\,e\xi\,m^2 c^4
			} \\[.8ex]
		&= \dv{t} \Bqty{\pqty{
			\frac{1}{2ec^2}\dot{X}^2
			- \frac{e}{2}\,m^2 c^4
		}\,\xi}
		= \dv{t} \pqty\big{\xi L}
	\end{aligned}
	\end{equation}
	Indeed we get a total derivative; therefore,
	\begin{gather}
		\var{e} = \dv{t}\pqty{e\xi},\quad
		\var{S} = \int \var{{L}}
		= \int \dd\mspace{.5mu} \pqty\big{
			\xi L
		} = 0
	\end{gather}
	
	\item $e(t)$ can be seen as a gauge field coupled to $X$, which captures the $t$--reparametrization redundancy. A natural gauge choice is fixing $f = e(t) - 1 \equiv 0$, which is equivalent to setting $t = \tau$: the proper time, or affine parametrization for the massless case. 
	
	\end{itemize}
	
	
	
	
	
	
	
	\legacyReference
	
	\begin{itemize}
	\item For $
		\var{X}^\mu
		= a^\mu + \lambda\id{^\mu_\nu} X^\nu
	$, the Lagrangian (density) transforms as follows:
	\begin{equation}
	\begin{aligned}
		\var{\mcal{L}}
		&= -\pd^\alpha X_\mu\,
			\pd_\alpha \var{X^\mu} \\
		&= -\pd^\alpha X_\mu\,
			\pd_\alpha \pqty{
				a^\mu
				+ \lambda\id{^\mu_\nu} X^\nu
			} \\
		&= -\pd^\alpha X_\mu\,
			\pqty{
				\pdd{\alpha} a^\mu
				+ X^\nu\,
					\pdd{\alpha} \lambda\id{^\mu_\nu}
				+ \lambda\id{^\mu_\nu}\,
					\pdd{\alpha} X^\nu
			} \\
		&= -\pd^\alpha X_\mu\,\pdd{\alpha} a^\mu
		- \pd^\alpha X^\mu\,
			\pdd{\alpha} X^\nu\,
			\lambda_{\mu\nu}
		- X^\nu\,\pd^\alpha X^\mu\,
			\pdd{\alpha} \lambda_{\mu\nu} \\
		&= -\pd^\alpha X_\mu\,\pdd{\alpha} a^\mu
		- \pd^\alpha X^\mu\,
			\pdd{\alpha} X^\nu\,
			\lambda_{(\mu\nu)}
		- X^\nu\,\pd^\alpha X^\mu\,
			\pdd{\alpha} \lambda_{\mu\nu}
		\label{eq:lorentz_variation}
	\end{aligned}
	\end{equation}
	Since $a^\mu$ and $\lambda\id{^\mu_\nu}$ are independent, imposing $\var{L} = 0$ yields $\pdd{\alpha} a^\mu = 0,\,a = \mrm{const}$. Furthermore, if $\var{L} = 0$ is to hold for arbitrary $X^\mu$ fields, then $\pdd{\alpha} \lambda_{\mu\nu} = 0,\,\lambda_{(\mu\nu)} = 0$, i.e.\ $\lambda_{\mu\nu}$ is constant and anti-symmetric over its indices. 
	
%	\item Variation over any generic $\var{X}$ yields:
%	\begin{equation}
%	\begin{aligned}
%		\var{S} = \int \dd[2]{x}
%			\var{\mcal{L}}
%		&= - \int \dd[2]{x}
%			\pd^\alpha X_\mu\,
%			\pd_\alpha \var{X^\mu} \\
%		&= \int \dd[2]{x}
%			\pqty{\pd^\alpha \pdd{\alpha} X_\mu}
%			\var{X^\mu}
%		- \int \dd[2]{x} \pdd{\alpha} \pqty\big{
%			\pd^\alpha X_\mu\,
%			\var{X^\mu}
%		} \\
%	\end{aligned}
%	\end{equation}
%	This gives the equation of motion (EOM) $
%		\pd^\alpha \pdd{\alpha} X = 0
%	$. 
	
	\item Promote $
		\var{X}\mapsto \epsilon(x)\var{X}
		= \epsilon(x)\,\pqty{
			a^\mu + \lambda\id{^\mu_\nu} X^\nu
		}
	$, with $\epsilon(x)$ some localized bump function; using \eqref{eq:lorentz_variation} and considering \textit{on-shell} variation, we have:
	\begin{equation}
	\begin{aligned}
		0 = \var{S}
		&= - \int \dd[2]{x} \pqty{
			\pd^\alpha X_\mu\,
				a^\mu\,
				\pdd{\alpha}\epsilon
			+ X^\nu\,\pd^\alpha X^\mu\,
				\lambda_{\mu\nu}\,
				\pdd{\alpha}\epsilon
		} \\
		&= - \int \dd[2]{x} \pqty{
			\pd^\alpha X_\mu\,
				a^\mu\,
			+ X_{[\nu}\,\pd^\alpha X_{\mu]}\,
				\lambda^{[\mu\nu]}\,
		}\,\pdd{\alpha}\epsilon
	\end{aligned}
	\end{equation}
	It is evident (after partial integration) that the following currents are conserved; they are the Noether currents associated with $a^\mu$ and $\lambda^{[\mu\nu]}$:
	\begin{equation}
		j^\alpha_\mu
		= -\pd^\alpha X_\mu,\quad
		j^\alpha_{\mu\nu}
		= -X_{[\nu}\,\pd^\alpha X_{\mu]}
		= \frac{1}{2}\,\pqty{
			X_\mu\,\pd^\alpha X_\nu
			- X_\nu\,\pd^\alpha X_\mu
		}
	\end{equation}
	
	Conserved charge $
		Q = \int\dd[2]{x} j^0(x)
	$, we have:
	\begin{equation}
		P_\mu = - \int\dd{x^1} \pd^0 X_\mu
		= \int\dd{x^1} \pd_0 X_\mu,\quad
		M_{\mu\nu}
		= \frac{1}{2}\,\int\dd{x^1}\pqty{
			X_\nu\,\pd_0 X_\mu
			- X_\mu\,\pd_0 X_\nu
		}
	\end{equation}
	They can be interpreted as spacetime momentum and spacetime angular momentum. 
	\qedfull
	\end{itemize}
\pagebreak[3]
\subsection{Real Scalar in $(3+1)$\,D}
	\vspace{-.8\baselineskip}
	\begin{equation}
		\mcal{L}
		= -\frac{1}{2}\,
			\pd^\mu\phi\,
			\pdd{\mu}\phi
		-\frac{1}{2}\,m^2\phi^2
	\end{equation}
	\begin{itemize}
	\item For $\phi$: scalar, under $
		x' = \lambda\circ x
	$, $\phi(x)\mapsto \phi'(x)$, while:
	\begin{equation}
		\phi'(x') = \phi(x)
		\ \Longrightarrow\ %
		\phi'(x) = \phi(\lambda^{-1}\circ x)
	\end{equation}
	For $\lambda\sim \lambda\id{^\mu_\nu}$: Lorentz transformation, $
		\eta_{\mu\nu}
			\lambda\id{^\mu_\rho}
			\lambda\id{^\nu_\sigma}
		= \eta_{\rho\sigma}
	$, or equivalently, $
		(\lambda^{-1})\id{^\mu_\nu}
		= \lambda\id{_\nu^\mu}
	$. Therefore,
	\begin{equation}
		\phi'(x^\mu) = \phi(\lambda^{-1}\circ x^\mu)
		= \phi(x^\nu\lambda\id{_\nu^\mu})
	\end{equation}
\pagebreak[3]
	
	\item Under $
		x'^\mu = \lambda\id{^\mu_\nu} x^\nu,\ %
		\phi(x)\mapsto\phi'(x)
	$, we have:
	\begin{equation}
	\begin{aligned}
		\mcal{L}'(x')
		&= - \frac{1}{2}\,
			\pd'^\mu\phi'(x')\,
			\pdd{\mu}'\phi'(x')
		- \frac{1}{2}\,m^2\phi'^2(x') \\
		&= - \frac{1}{2}\,
			\pd'^\mu\phi(x)\,
			\pdd{\mu}'\phi(x)
		- \frac{1}{2}\,m^2\phi^2(x) \\
		&= - \frac{1}{2}\,
			\eta^{\mu\nu}
			\pdv{x^\rho}{x'^\mu}\,
			\pdd{\rho}\phi(x)\,
			\pdv{x^\sigma}{x'^\nu}\,
			\pdd{\sigma}\phi(x)
		- \frac{1}{2}\,m^2\phi^2(x) \\
		&= - \frac{1}{2}\,
			\eta^{\rho\sigma}
			\pdd{\rho}\phi(x)\,
			\pdd{\sigma}\phi(x)
		- \frac{1}{2}\,m^2\phi^2(x) \\[.5ex]
		&= \mcal{L}(x)
	\end{aligned}
	\end{equation}
	Here we've used $
		\eta^{\mu\nu}
		\pdv{x^\rho}{x'^\mu}\,
		\pdv{x^\sigma}{x'^\nu}
		= \eta^{\mu\nu}
			\lambda\id{_\mu^\rho}
			\lambda\id{_\nu^\sigma}
		= \eta^{\rho\sigma}
	$. Hence, $\mcal{L}$ is invariant under Lorentz transformation. 
	
	\item Consider an infinitesimal Lorentz transformation: $\lambda \sim \idty + \omega$, then $
		\eta_{\mu\nu}
			\lambda\id{^\mu_\rho}
			\lambda\id{^\nu_\sigma}
		= \eta_{\rho\sigma}
	$ implies that $\omega_{\mu\nu}$ is anti-symmetric: $
		\omega_{\mu\nu} + \omega_{\nu\nu} = 0
	$. For $
		\var{x}^\mu
		= \omega\id{^\mu_\nu} x^\nu
	$, we have:
	\begin{equation}
		\var{\phi} = - \pdv{\phi}{x^\mu}
			\var{x}^\mu
		= - \omega\id{^\mu_\nu} x^\nu\,
			\pdd{\mu}\phi
%		= - \omega_{\rho\sigma} x^\sigma\,
%			\pd^\rho \phi
	\end{equation}
	
	To obtain the corresponding Noether charges, we can simply repeat the operations done in our previous problem; alternatively, we can try to derive a general recipe\footnote{
		References: \arxiv{1601.03616} and \textit{Tong:} \url{http://damtp.cam.ac.uk/user/tong/qft.html}
	}: for $
		\mcal{L} = \mcal{L}(\phi,\pdd{\mu}\phi)
	$ and $
		S = \int \dd[4]{x} \mcal{L}
	$, we have:
	\begin{equation}
	\begin{aligned}
		\var{S}
		&= \int\dd[4]{x} \var{\mcal{L}} \\
		&= \int\dd[4]{x} \pqty{
			\pdv{\mcal{L}}{\phi} \var{\phi}
			+ \pdv{\mcal{L}}{(\pdd{\mu}\phi)}
				\var{\pdd{\mu}\phi}
		} \\[.5ex]
		&= \int\dd[4]{x} \pqty{
				\pdv{\mcal{L}}{\phi}
				- \pdd{\mu}
				\pdv{\mcal{L}}{(\pdd{\mu}\phi)}
			} \var{\phi}
		+ \int\dd[4]{x}
			\pdd{\mu} \pqty{
				\pdv{\mcal{L}}{(\pdd{\mu}\phi)}
				\var{\phi}
			}
	\end{aligned}
	\end{equation}
	If we vary $S$ w.r.t.\ a symmetry of the system, we will have $\var{\mcal{L}} = \pdd{\mu} K^\mu$ some total derivative; when on-shell, such variation gives the conserved current with boundary term $K^\mu$:
	\begin{equation}
		j^\mu
		= \pdv{\mcal{L}}{(\pdd{\mu}\phi)}
			\var{\phi} - K^\mu
	\end{equation}
	
	Back to our Lorentz transformation $
		\var{\phi}
		= -\omega\id{^\mu_\nu} x^\nu\pdd{\mu}\phi
	$, we have symmetry variation:
	\begin{equation}
		\var{\mcal{L}}
		= - \omega\id{^\mu_\nu} x^\nu
			\pdd{\mu} \mcal{L}
		= - \pdd{\mu} \pqty{
			\omega\id{^\mu_\nu} x^\nu
			\mcal{L}
		}
	\end{equation}
	This gives a boundary term $
		K^\mu = -\omega\id{^\mu_\nu} x^\nu \mcal{L}
	$, and the Noether current and its corresponding conserved charge can be calculated as follows:
	\begin{gather}
		j^\mu = -\omega\id{^\sigma_\nu} x^\nu
		\pqty{
			\pdv{\mcal{L}}{(\pdd{\mu}\phi)}
				\,\pdd{\sigma} \phi
			- \delta^\mu_\sigma \mcal{L}
		},\\[1ex]
		Q = \int \dd[3]{x} j^0
		= - \omega\id{^\sigma_\nu} \int \dd[3]{x}
		x^\nu \pqty{
			\pdd{0}\phi\,\pdd{\sigma}\phi
			- \delta^0_\sigma \mcal{L}
		},
	\end{gather}
	
	Note that $\omega\id{^\mu_\nu}$ is arbitrary, therefore $Q$ can be decomposed into independent charges:
	\begin{equation}
		Q = \omega_{\mu\nu} M^{\mu\nu},\quad
		M^{\mu\nu}
		= - \int \dd[3]{x}
		x^{[\mu} \pqty{
			\pdd{0}\phi\,\pd^{\nu]}\phi
			- \eta^{\nu] 0} \mcal{L}
		},
	\end{equation}
	The indices of $M^{\mu\nu}$ are anti-symmetrized to match the degrees of freedom in $\omega_{\mu\nu}$. Note that the $\mcal{L}$ term only appears when one of the indices is 0. 
	
	Canonical quantization:
	\begin{gather}
		\dot{\phi} = \pdd{0}\phi,\quad
		\Pi = \pdv{\mcal{L}}{\dot{\phi}}
		= \dot{\phi},\quad
		[\phi(\vb{x}),\Pi(\vb{y})]
		= [\phi,\dot{\phi}](\vb{x})
		= i\delta(\vb{x} - \vb{y})
	\end{gather}
	Other equal-time commutators between $\phi,\Pi$ all just vanish. We have:
	\begin{equation}
		M^{0i} = -M^{i0}
		= -\frac{1}{2}\,
		\int \dd[3]{x} \pqty{
			\dot{\phi}\,\pqty{
				x^0 \pd^i
				- x^i \pd^0
			}\,\phi
			+ x^i \mcal{L}
		},\\
		M^{ij}
		= -\frac{1}{2}\,\int \dd[3]{x}
			\dot{\phi}\,\pqty{
				x^i \pd^j
				- x^j \pd^i
			}\,\phi
	\end{equation}
	
	Notice that $
		x^{[\mu} \pd^{\nu]}
		= \frac{1}{2}\,\pqty{
			x^\mu \pd^\nu - x^\nu \pd^\mu
		} = \frac{1}{2} D^{ij}
	$ is the Killing vector fields of $\mbb{R}^{3,1}$, hence they naturally follow the commutation relations of $\mfrak{so}(3,1)$ (up to a constant coefficient)\footnote{
		I would like to thank \textit{林般} for pointing this out. 
	}. We have:
	\begin{equation}
	\begin{aligned}
		[M^{ij}, M^{kl}]
		&= \frac{1}{4}\,\int \dd[3]{x}
			\int \dd[3]{y} \bqty{
				\dot{\phi} D^{ij}\phi(x),
				\dot{\phi} D^{kl}\phi(y)
			} \\
		&= \frac{1}{4}\,\int \dd[3]{x}
			\dot{\phi}\,\bqty{
				 D^{ij}, D^{kl}
			}\,\phi
	\end{aligned}
	\end{equation}
	Similar holds for $M^{i0}$. Therefore, $M^{\mu\nu}$'s indeed form the Lie algebra $\mfrak{so}(3,1)$. 
	\qedfull
	
	\end{itemize}

\printbibliography[%
%	title = {参考文献} %
	,heading = bibintoc
]
\end{document}
