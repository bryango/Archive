% !TeX document-id = {b5392a94-51a3-49d1-9ba5-698bc09f9d35}
% !TeX encoding = UTF-8
% !TeX spellcheck = en_US
% !TeX TXS-program:bibliography = biber -l zh__pinyin --output-safechars %

\documentclass[a4paper
	,10pt
%	,twoside
]{article}

% to be `\input` in subfolders,
% ... therefore the path should be relative to subfolders.

\usepackage{iftex}
\ifPDFTeX
\else
	\usepackage[UTF8
		,heading=false
		,scheme=plain % English Document
	]{ctex}
\fi
%\ctexset{autoindent=true}
\usepackage{indentfirst}

\input{../.modules/basics/macros.tex}
\input{../.modules/preamble_base.tex}
\input{../.modules/preamble_beamer.tex}
\input{../.modules/basics/biblatex.tex}


%Misc
	\usepackage{lilyglyphs}
	\newcommand{\indicator}{$\text{\clefG}$}
	\newcommand{\indicatorInline}{$\text{\clefGInline}$}

\newcommand{\legacyReference}{{
%	\clearpage\par
%	\quad\clearpage
	\def{\midquote}{\textbf{PAST WORK, AS TEMPLATE}}
	\newparagraph
}}

% Settings
\counterwithout{equation}{section}
\mathtoolsset{showonlyrefs=false}
%\DeclareTextFontCommand{\textbf}{\sffamily}

% Spacing
\geometry{footnotesep=2\baselineskip} % pre footnote split
\setlength{\parskip}{.5\baselineskip}
\renewcommand{\baselinestretch}{1.15}


%% List
%	\setlist*{
%		listparindent=\parindent
%		,labelindent=\parindent
%		,parsep=\parskip
%		,itemsep=1.2\parskip
%	}


\addtobeamertemplate{navigation symbols}{}{%
    \usebeamerfont{footline}%
%    \usebeamercolor[fg]{footline}%
    \hspace{1em}%
    \normalsize\insertframenumber/\inserttotalframenumber
}

\makeatletter
\setbeamertemplate{headline}
{%
    \begin{beamercolorbox}[wd=\paperwidth,colsep=1.5pt]{upper separation line head}
    \end{beamercolorbox}
    \begin{beamercolorbox}[wd=\paperwidth,ht=2.5ex,dp=1.125ex,%
      leftskip=.3cm,rightskip=.3cm plus1fil]{title in head/foot}
      \usebeamerfont{title in head/foot}\insertshorttitle
    \end{beamercolorbox}
    \begin{beamercolorbox}[wd=\paperwidth,ht=2.5ex,dp=1.125ex,%
      leftskip=.3cm,rightskip=.3cm plus1fil]{section in head/foot}
      \usebeamerfont{section in head/foot}%
      \ifbeamer@tree@showhooks
        \setbox\beamer@tempbox=\hbox{\insertsectionhead}%
        \ifdim\wd\beamer@tempbox>1pt%
          \hskip2pt\raise1.9pt\hbox{\vrule width0.4pt height1.875ex\vrule width 5pt height0.4pt}%
          \hskip1pt%
        \fi%
      \else%  
        \hskip6pt%
      \fi%
      \insertsectionhead
    \end{beamercolorbox}
% Code for subsections removed here
}
\makeatother
\input{../.modules/basics/biblatex.tex}

\title{Notes on Gravitational Entropy}
\addbibresource{entropy.bib}

%%% ID: sensitive, do NOT publish!
\InputIfFileExists{id.tex}{}{}

\begin{document}
\maketitle
\pagenumbering{arabic}
\thispagestyle{empty}

%\vspace*{-.5\baselineskip}

\setlength{\parskip}{.1\baselineskip}
\tableofcontents
\setlength{\parskip}{\parskipnorm}

%\addtocounter{section}{-1}
\section{Entanglement Entropy}
	Ref: \textcite{Apolo:2020qjm}.
	
	A boundary interval $\mcal{A} \Leftrightarrow \rho_\mcal{A} = e^{-\mcal{H}_\mcal{A}}$. $\mcal{H}_\mcal{A}$ is generically nonlocal, but in a highly symmetric background (e.g.~vaccuum) it might be geometrically realized and corresponds to a symmetry of the underlying quantum field theory.
	In this case there is a generalized Rindler transformation $f$ that maps the casual $\mcal{D} \to \{(\tau,\tilde{x})\}$ in some noncompact generalized Rindler spacetime, with $\pdd{\tau}$ an isometry of the spacetime. Then we have:
	\begin{equation}
		\zeta = 2\pi \pdd{\tau}
		= 2\pi a^i H_i,
	\quad
		\tau \sim \tau + 2\pi i
	\end{equation}
	$H_i$: symmetry generators. Vaccuum$|_\mcal{D}\to$ some thermal state with well-defined temperature $2\pi$. 
	
	
\section{Entropy as Noether Charge}
	Ref: \textcite{Wald:1993nt,Lewkowycz:2013nqa,Faulkner:2013ana}. 
	
	

\vspace{1.2\baselineskip}
\raggedright
\printbibliography[%
%	title = {参考文献} %
	,heading = bibintoc
]
\end{document}
