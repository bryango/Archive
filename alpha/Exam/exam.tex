% !TeX document-id = {b5392a94-51a3-49d1-9ba5-698bc09f9d35}
% !TeX encoding = UTF-8
% !TeX spellcheck = en_US
% !TeX TXS-program:bibliography = biber -l zh__pinyin --output-safechars %
% !TeX TS-program = lualatex
%%% LuaLaTeX is required for `tikz-feynman`

\documentclass[a4paper
	,10pt
%	,twoside
]{article}

% to be `\input` in subfolders,
% ... therefore the path should be relative to subfolders.

\usepackage{iftex}
\ifPDFTeX
\else
	\usepackage[UTF8
		,heading=false
		,scheme=plain % English Document
	]{ctex}
\fi
%\ctexset{autoindent=true}
\usepackage{indentfirst}

\input{../.modules/basics/macros.tex}
\input{../.modules/preamble_base.tex}
\input{../.modules/preamble_beamer.tex}
\input{../.modules/basics/biblatex.tex}


%Misc
	\usepackage{lilyglyphs}
	\newcommand{\indicator}{$\text{\clefG}$}
	\newcommand{\indicatorInline}{$\text{\clefGInline}$}

\newcommand{\legacyReference}{{
%	\clearpage\par
%	\quad\clearpage
	\def{\midquote}{\textbf{PAST WORK, AS TEMPLATE}}
	\newparagraph
}}

% Settings
\counterwithout{equation}{section}
\mathtoolsset{showonlyrefs=false}
%\DeclareTextFontCommand{\textbf}{\sffamily}

% Spacing
\geometry{footnotesep=2\baselineskip} % pre footnote split
\setlength{\parskip}{.5\baselineskip}
\renewcommand{\baselinestretch}{1.15}


%% List
%	\setlist*{
%		listparindent=\parindent
%		,labelindent=\parindent
%		,parsep=\parskip
%		,itemsep=1.2\parskip
%	}


\addtobeamertemplate{navigation symbols}{}{%
    \usebeamerfont{footline}%
%    \usebeamercolor[fg]{footline}%
    \hspace{1em}%
    \large\insertframenumber/\inserttotalframenumber
}

\makeatletter
\setbeamertemplate{headline}
{%
    \begin{beamercolorbox}[wd=\paperwidth,colsep=1.5pt]{upper separation line head}
    \end{beamercolorbox}
    \begin{beamercolorbox}[wd=\paperwidth,ht=2.5ex,dp=1.125ex,%
      leftskip=.3cm,rightskip=.3cm plus1fil]{title in head/foot}
      \usebeamerfont{title in head/foot}\insertshorttitle
    \end{beamercolorbox}
    \begin{beamercolorbox}[wd=\paperwidth,ht=2.5ex,dp=1.125ex,%
      leftskip=.3cm,rightskip=.3cm plus1fil]{section in head/foot}
      \usebeamerfont{section in head/foot}%
      \ifbeamer@tree@showhooks
        \setbox\beamer@tempbox=\hbox{\insertsectionhead}%
        \ifdim\wd\beamer@tempbox>1pt%
          \hskip2pt\raise1.9pt\hbox{\vrule width0.4pt height1.875ex\vrule width 5pt height0.4pt}%
          \hskip1pt%
        \fi%
      \else%  
        \hskip6pt%
      \fi%
      \insertsectionhead
    \end{beamercolorbox}
% Code for subsections removed here
}
\makeatother
\input{../.modules/basics/biblatex.tex}

\title{Qualifying Exam\phantom{:}%
	\,\normalsize Physics 2020 @ YMSC, Tsinghua%
}
%\addbibresource{.bib}

%%% ID: sensitive, do NOT publish!
%\InputIfFileExists{id.tex}{}{}

\renewcommand{\ccbyncsajp}{}

\usepackage{cancel}
\usepackage{tikz-feynman}
\usepackage{filecontents}
\begin{filecontents}{\jobname.bib}
	@book{Peskin:1995ev,
	    author = "Peskin, Michael E. and Schroeder, Daniel V.",
	    title = "{An Introduction to Quantum Field Theory}",
	    isbn = "978-0-201-50397-5",
	    publisher = "Addison-Wesley",
	    address = "Reading, USA",
	    year = "1995"
	}
\end{filecontents}
\addbibresource{\jobname.bib}

\begin{document}
\maketitle
\pagenumbering{arabic}
\thispagestyle{empty}

\section{Gravity}
	\vspace{-1.5\baselineskip}
	\begin{equation}
		\dd{s^2}
		= - f(r) \dd{t^2}
			+ \frac{1}{f(r)} \dd{r^2}
			+ r^2 \dd{\Omega^2}
	\end{equation}
	\vspace{-.3\baselineskip}
	\begin{equation}
		\ f(r)
		= 1 - \frac{GM}{r} + \frac{Q^2}{r^2}
		= \frac{(r - r_+)(r - r_-)}{r^2}
	\end{equation}
	
\begin{enumerate}
\item Event horizon(s): $f(r) = 0$, we have:
	\begin{enumerate}
	\item $M > \abs{Q}$, $r_\pm = M \pm \sqrt{M^2 - Q^2}$, 2 event horizons;
	\item $M = \abs{Q}$, $r_\pm = M$, 1 event horizon;
	\item $M < \abs{Q}$, no event horizon! ``Naked'' singularity.
	\end{enumerate}
	
\item New coordinate: $v = t + r^*$,
	\begin{equation}
		r^* = r
			+ \frac{1}{2k_+}
				\ln \frac{\abs{r - r_+}}{r_+}
			+ \frac{1}{2k_-}
				\ln \frac{\abs{r - r_-}}{r_-},
	\quad
		k_\pm
		= \frac{r_\pm - r_\mp}{2r_\pm^2}
	\label{eq:def_coord}
	\end{equation}
	We have:
	\begin{align}
		\dd{t}
		= \dd{v} - \dd{r^*}
		&= \dd{v} - \pqty{
				1 + \frac{1}{2k_+} \frac{1}{r - r_+}
				+ \frac{1}{2k_-} \frac{1}{r - r_-}
			} \dd{r}
		\label{eq:diff_coord}
		\\
		&= \dd{v} - \pqty{
				1 + \frac{1}{r^2 f(r)}
				\frac{
					r_+^2 (r - r_-)
					- r_-^2 (r - r_+)
				}{r_+ - r_-}
			} \dd{r} \notag\\
		&= \dd{v} - \pqty{
				1 + \frac{1}{r^2 f(r)}\,
				\pqty\Big{
					(r_+ + r_-)\,r
					- r_+ r_-
				}
			} \dd{r} \notag\\
		&= \dd{v} - \frac{1}{f(r)} \dd{r}
		\label{eq:diff_coord_goal}
	\end{align}
	
	Therefore,
	\begin{enumerate}[parsep=\parskip]
	\item The new metric:
	\begin{align}
		\dd{s^2}
		&= - f(r) \pqty{
				\dd{v}
				- \frac{1}{f(r)} \dd{r}
			}^{\!\!2}
			+ \frac{1}{f(r)} \dd{r^2}
			+ r^2 \dd{\Omega^2} \\
		&= - f(r) \dd{v^2}
			+ 2 \dd{v} \dd{r}
			+ r^2 \dd{\Omega^2} \notag
	\end{align}
	It is only singular at $r = 0$.
	
	\begin{quote}
	\textbf{Note:} during the exam I panicked when I saw \eqref{eq:def_coord}, and I made a very stupid mistake in step \eqref{eq:diff_coord}. However, I knew what this new coordinate is trying to achieve --- it's aiming to eliminate the coordinate singularities in $\frac{1}{f} \dd{r^2}$ by absorbing it into $\dd{v^2}$, so I guessed the result \eqref{eq:diff_coord_goal} correctly and carried on. I hope they gave me some points for getting the right answer, despite with some wrong process (>\_<).
	\end{quote}
	
	\item $\pdv{v}$ is a Killing vector field, for the metric components are all $v$-independent. More precisely, since $\pdv{v}$ itself is a coordinate basis, we have the Lie derivative:
	\begin{equation}
		\ldv{\pdv{v}} g_{\mu\nu}
		= \pdd{v} g_{\mu\nu}
		= 0
	\end{equation}
	
	\item $
		\norm{\pdv{v}}^2
		= g_{\mu\nu} \delta^\mu_v \delta^\nu_v
		= g_{vv}
		= -f(r)
	$, therefore, for $M > \abs{Q}$ we have:
	\begin{itemize}[topsep=.3\baselineskip]
	\item $\pdv{v}$ timelike: $r > r_+$ and $r < r_-$
	\item spacelike: $r_- < r < r_+$
	\item null: $r = r_+$ and $r = r_-$
	\end{itemize}
	
	\end{enumerate}
\end{enumerate}
\section{QFT}
We shall restore the reasonable convention: $
	\eta_{\mu\nu}
	\sim \pqty{-,+,+,+}
$. 
\begin{enumerate}
\item 1PI: diagrammatic correction to the (1-particle) propagator that cannot be split into 2 disconnected parts by cutting one line; e.g.\,%
	\raisebox{-.4\baselineskip}{
		\feynmandiagram[
			scale=.4
			,small
			,layered layout
			,horizontal=a to d
		]{
			a -- b[{small,dot}] -- c[{small,dot}] -- d,
			b -- [half left,looseness=1.5]
			c -- [half left,looseness=1.5]
			b,
		};
	}
\item Consider the following Lagrangian:
	\begin{equation}
		\mcal{L}
		= -\frac{1}{2} Z(\pd\phi_r)^2
			-\frac{1}{2} m^2 Z\phi_r^2
			-\frac{\lambda}{4!} \phi_r^4
		-\frac{1}{2} \delta_Z(\pd\phi_r)^2
			-\frac{1}{2} \delta_m \phi_r^2
			-\frac{\delta_\lambda}{4!} \phi_r^4
	\label{eq:lagrangian}
	\end{equation}
	The convention here is rather bizarre; normally we write down the UV Lagrangian $\mcal{L}_{\mrm{UV}}$ and split it into 2 parts, one is the effective IR Lagrangian $\mcal{L}_{\mrm{IR}}$ and the other one is the counterterm:
	\begin{equation}
	\begin{aligned}
		\mcal{L}_{\mrm{UV}}
		&= -\frac{1}{2} Z(\pd\phi_r)^2
			-\frac{1}{2} m^2 Z\phi_r^2
			-\frac{\lambda}{4!} \phi_r^4 \\
		&= \pqty{
			-\frac{1}{2} (\pd\phi_r)^2
			-\frac{1}{2} m_p^2 \phi_r^2
			-\frac{\lambda_p}{4!} \phi_r^4
		}
		- \pqty{
			-\frac{1}{2} \delta_Z(\pd\phi_r)^2
			-\frac{1}{2} \delta_m \phi_r^2
			-\frac{\delta_\lambda}{4!} \phi_r^4
		} \\
		&= \mcal{L}_{\mrm{IR}}
			+ \mcal{L}_{\mrm{ct}}
	\end{aligned}
	\end{equation}
	
	Normally, we use $\mcal{L}$ to denote the UV Lagrangian $\mcal{L}_{\mrm{UV}}$; this is the convention adopted by numerous standard textbooks, incl.~\textit{Peskin \& Schroeder} \cite{Peskin:1995ev}, \textit{Weinberg}, and also \textit{Srednicki}. However, the Lagrangian in \eqref{eq:lagrangian} seems to be $\mcal{L}_{\mrm{IR}}$ instead of $\mcal{L}_{\mrm{UV}}$. Anyway, we have:
	\begin{equation}
		Z + \delta_Z = 1,
	\quad
		m^2 Z + \delta_m = m_p^2,
	\quad
		\lambda + \delta_\lambda = \lambda_p
	\end{equation}
	Where $m_p,\lambda_p$ is the physical IR couplings, fixed by the renormalization scheme. The convention here is really confusing and somewhat inconsistent; e.g.\ if we choose to write the UV mass term as $
		-\frac{1}{2} m^2 Z \phi_r^2
	$, then the corresponding UV interaction term should look like $
		-\frac{\lambda}{4!} Z^2 \phi_r^4
	$, but here we do not have the $Z^2$ factor. Also, we usually use $m_0,\lambda_0$ to denote bare couplings, but here it seems that they are denoted by $m,\lambda$.
	
	We can write down the renormalized Feynman rules nonetheless, despite some sign issues due to the conventions; to avoid further confusion, we will adopt the usual notation: $m_0,\lambda_0$ for bare couplings, and $m = m_p, \lambda = \lambda_p$ for physical couplings. We have:
	\begin{itemize}[noitemsep]
	\item Renormalized propagator: $
			\frac{-i}{p^2 + m^2 - i\epsilon}
		$
	\item Renormalized vertex: $-i\lambda$
	\item Counterterm $\phi^2$ vertex: $+i\pqty{\delta_Z (-p^2) + \delta_m}$,
		\raisebox{0.\baselineskip}{
			\feynmandiagram[
				scale=.4
				,small
				,layered layout
				,horizontal=a to c
			]{
				a -- b[{small,crossed dot}] -- c,
			};
		}
	\item Counterterm $\phi^4$ vertex: $+i\delta_\lambda$
	\end{itemize}
\pagebreak
	
\item The sum of all two point 1PI diagrams (no propagator on external legs) is given by:
	\begin{equation}
		-iM(p^2)
		= \raisebox{-.3\height}{
			\feynmandiagram[
				scale=.7
				,small
				,layered layout
			]{
				b[blob],
			};
		}
	\end{equation}
	The full propagator is thus:
	\begin{equation}
	\begin{aligned}
		G(p^2)
		&= \raisebox{+.15\baselineskip}{
			\feynmandiagram[
				scale=.7
				,small
				,layered layout
				,horizontal=a to b
			]{
				a -- b,
			};
		}
		+ \raisebox{-.3\height}{
			\feynmandiagram[
				scale=.7
				,small
				,layered layout
				,horizontal=a to c
			]{
				a -- b[blob] -- c,
			};
		}
		+ \raisebox{-.3\height}{
			\feynmandiagram[
				scale=.7
				,small
				,layered layout
				,horizontal=a to d
			]{
				a -- b[blob] -- c[blob] -- d,
			};
		}
		+ \cdots \\
		&= \tfrac{-i}{p^2 + m^2}
			+ \tfrac{-i}{p^2 + m^2}
				(-iM) \tfrac{-i}{p^2 + m^2}
			+ \tfrac{-i}{p^2 + m^2}
				(-iM) \tfrac{-i}{p^2 + m^2}
				(-iM) \tfrac{-i}{p^2 + m^2}
			+ \cdots
	\end{aligned}
	\end{equation}
%\pagebreak[4]
	
	With $\sum_{n=0}^\infty q^n = \frac{1}{1 - q}$, we get:
	\begin{equation}
		G(p^2)
		= \frac{-i}{p^2 + m^2} \cdot
			\frac{1}{1 - (-iM) \tfrac{-i}{p^2 + m^2}}
		= \frac{-i}{p^2 + m^2 + M(p^2)}
	\end{equation}
	Here we've suppressed the $(-i\epsilon)$ prescription in the above expressions, but it's presence is always implied.
	
	\item On-shell renormalization scheme --- the full propagator:
	\begin{equation}
		G(p^2)
		= \frac{-i}{p^2 + m^2 + M(p^2) - i\epsilon}
		\xrightarrow{\ p^2\to -m^2}
		\frac{-i}{p^2 + m^2 - i\epsilon}
	\end{equation}
	This means that $
		M(p^2 = -m^2) = 0
	$. Furthermore, $M(p^2) \sim \#\,(p^2 + m^2) + \order{p^4}$, to ensure that the residue is 1 at the pole, we should have $\# \sim 0$, i.e.
	\begin{equation}
		M(p^2)\big|_{p^2 = -m^2}
		= 0,
	\quad
		\pdv{(p^2)} M(p^2)\big|_{p^2 = -m^2}
		= 0
	\label{eq:renorm_cond}
	\end{equation}
	
	\item At 1-loop $\order{\lambda}$, if we do not include counterterm contributions, then there is only one diagram contributing to $M(p^2)$:
	\begin{equation}
	%% ref: <https://tex.stackexchange.com/q/387810/193356>
		\raisebox{-.2\height}{
			\feynmandiagram[
				scale=.7
				,small
				,layered layout
				,horizontal=a to b
			]{
				a -- b [dot]
				-- [out=135, in=45, loop
					,min distance=3.5\baselineskip
					] b
				-- c
			};
		}
		= (-i\lambda)\cdot\frac{1}{2}
			\int \frac{\dd[D]{k}}{(2\pi)^D}
				\frac{-i}{k^2 + m^2 - i\epsilon}
	\end{equation}
	Here $\frac{1}{2}$ is the symmetry factor of the diagram; alternative, we can count the distinct ways of connecting the 4 legs of the $\phi^4$ vertex and divide it by $4!$, which is indeed $
		\frac{4\times 3}{4!} = \frac{1}{2}
	$. 
	
	The $p^0$ integral has poles at $p_0^2 = \vb{p}^2 + m^2 - i\epsilon$, i.e.\ $p^0 = \pm\sqrt{\vb{p}^2 + m^2} \mp i\epsilon$, and it's regular everywhere else; we can thus compute the $p^0$ integral on the $\mbb{C}$ plane using a right-tilted 8-shaped contour, which does not enclose the poles. Effectively, we've performed a Wick rotation $p^0 \mapsto ip^0$ so that the integral happens in Euclidean $p$ space:
	\begin{equation}
		\frac{-i\lambda}{2}
			\int \frac{\dd[D]{k}}{(2\pi)^D}
				\frac{1}{k^2 + m^2}
		= \frac{-i\lambda}{2}\,
			\frac{A(S^d)}{(2\pi)^D}
			\int \frac{k^d \dd{k}}{k^2 + m^2}
	\end{equation}
	Here $D = d + 1$, $d$ is the spatial dimension. There are many ways to regularize this integral; if we continue to work in general $D = d + 1$ dimensions, then dimensional regularization is automatically implied. We have:
	\begin{equation}
		A(S^d) = \frac{2\pi^{D/2}}{\Gamma(D/2)},
	\quad
		\int \frac{k^d \dd{k}}{k^2 + m^2}
		= \frac{m^D}{m^2}
			\int \frac{t^d \dd{t}}{1 + t^2}
	\end{equation}
\pagebreak[4]
	
	The $t$-integral is related to Beta functions; consider $t\mapsto \frac{t^2}{1 + t^2}$, and we have:
	\begin{equation}
		\int_0^\infty \frac{t^d \dd{t}}{1 + t^2}
		= \frac{1}{2}
		\int_0^1 t^{\frac{D}{2} - 1}
			(1-t)^{-\frac{D}{2}} \dd{t}
		= \frac{
				\Gamma(\tfrac{D}{2})\,
				\Gamma(1 - \tfrac{D}{2})
			}{2\Gamma(1)}
		= \frac{1}{2} \Gamma(\tfrac{D}{2})\,
			\Gamma(1 - \tfrac{D}{2})
		= \frac{\pi}{2\sin \frac{\pi D}{2}}
	\end{equation}
	The last line is \textit{Euler's reflection formula}, but here we actually don't need that since the $\Gamma(\tfrac{D}{2})$ factor is canceled by $A(S^d)$. In the end we have:
	\begin{equation}
		\int \frac{\dd[D]{k}}{(2\pi)^D}
			\frac{1}{k^2 + m^2}
		= \frac{\pi^{D/2}}{(2\pi)^D}
			\Gamma(1 - \tfrac{D}{2})\,m^{D-2}
		= \frac{1}{(4\pi)^{D/2}}
			\Gamma(1 - \tfrac{D}{2})\,m^{D-2},
	\end{equation}
	\vspace{-.5\baselineskip}
	\begin{equation}
	%% ref: <https://tex.stackexchange.com/q/387810/193356>
		\raisebox{-.2\height}{
			\feynmandiagram[
				scale=.7
				,small
				,layered layout
				,horizontal=a to b
			]{
				a -- b [dot]
				-- [out=135, in=45, loop
					,min distance=3.5\baselineskip
					] b
				-- c
			};
		}
		= \frac{-i\lambda}{2} \frac{1}{(4\pi)^{D/2}}
			\Gamma(1 - \tfrac{D}{2})\,m^{D-2}
	\end{equation}
	
	We then have to include counterterm contributions so that the renormalization condition \eqref{eq:renorm_cond} is satisfies; we have:
	\begin{equation}
	%% ref: <https://tex.stackexchange.com/q/387810/193356>
	\begin{aligned}
		-iM(p^2)
		&\sim \hspace{-1em}
		\raisebox{-.2\height}{
			\feynmandiagram[
				scale=.7
				,small
				,layered layout
				,horizontal=a to b
			]{
				a -- b [dot]
				-- [out=135, in=45, loop
					,min distance=3.5\baselineskip
					] b
				-- c
			};
		}\hspace{-1em}
		+ \raisebox{0.\baselineskip}{
			\feynmandiagram[
				scale=.7
				,small
				,layered layout
				,horizontal=a to c
			]{
				a -- b[{small,crossed dot}] -- c,
			};
		}
		= \frac{-i\lambda}{2} \frac{1}{(4\pi)^{D/2}}
				\Gamma(1 - \tfrac{D}{2})\,m^{D-2}
			+ i\pqty{\delta_Z (-p^2) + \delta_m} \\
		&\sim 0
			+ 0\cdot (p^2 + m^2)
			+ \order{p^4}
	\end{aligned}
	\end{equation}
	Therefore,
	\begin{equation}
		\delta_Z = 0,
	\quad
		\delta_m
		= \frac{\lambda}{2} \frac{1}{(4\pi)^{D/2}}
			\Gamma(1 - \tfrac{D}{2})\,m^{D-2}
	\end{equation}
	
	Alternatively, if we are working in $D = 4 = 3 + 1$ dimensions, it's easier to impose a naïve cutoff $\Lambda$, which gives:
	\begin{equation}
	\begin{aligned}
		\int^\Lambda \frac{k^d \dd{k}}{k^2 + m^2}
		&\sim \int^\Lambda k^{d-2} \dd{k}
			+ \int^\Lambda k^d \dd{k}
				\pqty{
					\frac{1}{k^2 + m^2}
					- \frac{1}{k^2}
				} \\
		&= \int^\Lambda k^{d-2} \dd{k}
			- m^2 \int^\Lambda 
				\frac{k^{d-2} \dd{k}}{k^2 + m^2},
		\quad d = D - 1 = 3 \\
		&= \frac{\Lambda^2}{2}
			- \frac{m^2}{2} \ln \pqty{
				1 + \frac{\Lambda^2}{m^2}
			},
	\end{aligned}
	\end{equation}
	Similarly, with $A(S^3) = 2\pi^2$, we have:
	\begin{equation}
	\begin{aligned}
		\delta_Z = 0,
	\quad
		\delta_m
		&= \frac{\lambda}{2}
			\frac{2\pi^2}{(2\pi)^4} \Bqty{
				\frac{\Lambda^2}{2}
				- \frac{m^2}{2} \ln \pqty{
					1 + \frac{\Lambda^2}{m^2}
				}
			} \\
		&= \frac{\lambda}{32\pi^2} \Bqty{
				\Lambda^2
				- m^2 \ln \pqty{
					1 + \frac{\Lambda^2}{m^2}
				}
			}
	\end{aligned}
	\end{equation}
	
\end{enumerate}
\printbibliography
\end{document}