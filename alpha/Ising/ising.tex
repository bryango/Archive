% !TeX document-id = {b5392a94-51a3-49d1-9ba5-698bc09f9d35}
% !TeX encoding = UTF-8
% !TeX spellcheck = en_US
% !TeX TXS-program:bibliography = biber -l zh__pinyin --output-safechars %

\documentclass[a4paper
	,10pt
%	,twoside
]{article}

% to be `\input` in subfolders,
% ... therefore the path should be relative to subfolders.

\usepackage{iftex}
\ifPDFTeX
\else
	\usepackage[UTF8
		,heading=false
		,scheme=plain % English Document
	]{ctex}
\fi
%\ctexset{autoindent=true}
\usepackage{indentfirst}

\input{../.modules/basics/macros.tex}
\input{../.modules/preamble_base.tex}
\input{../.modules/preamble_beamer.tex}
\input{../.modules/basics/biblatex.tex}


%Misc
	\usepackage{lilyglyphs}
	\newcommand{\indicator}{$\text{\clefG}$}
	\newcommand{\indicatorInline}{$\text{\clefGInline}$}

\newcommand{\legacyReference}{{
%	\clearpage\par
%	\quad\clearpage
	\def{\midquote}{\textbf{PAST WORK, AS TEMPLATE}}
	\newparagraph
}}

% Settings
\counterwithout{equation}{section}
\mathtoolsset{showonlyrefs=false}
%\DeclareTextFontCommand{\textbf}{\sffamily}

% Spacing
\geometry{footnotesep=2\baselineskip} % pre footnote split
\setlength{\parskip}{.5\baselineskip}
\renewcommand{\baselinestretch}{1.15}


%% List
%	\setlist*{
%		listparindent=\parindent
%		,labelindent=\parindent
%		,parsep=\parskip
%		,itemsep=1.2\parskip
%	}


\addtobeamertemplate{navigation symbols}{}{%
    \usebeamerfont{footline}%
%    \usebeamercolor[fg]{footline}%
    \hspace{1em}%
    \large\insertframenumber/\inserttotalframenumber
}

\makeatletter
\setbeamertemplate{headline}
{%
    \begin{beamercolorbox}[wd=\paperwidth,colsep=1.5pt]{upper separation line head}
    \end{beamercolorbox}
    \begin{beamercolorbox}[wd=\paperwidth,ht=2.5ex,dp=1.125ex,%
      leftskip=.3cm,rightskip=.3cm plus1fil]{title in head/foot}
      \usebeamerfont{title in head/foot}\insertshorttitle
    \end{beamercolorbox}
    \begin{beamercolorbox}[wd=\paperwidth,ht=2.5ex,dp=1.125ex,%
      leftskip=.3cm,rightskip=.3cm plus1fil]{section in head/foot}
      \usebeamerfont{section in head/foot}%
      \ifbeamer@tree@showhooks
        \setbox\beamer@tempbox=\hbox{\insertsectionhead}%
        \ifdim\wd\beamer@tempbox>1pt%
          \hskip2pt\raise1.9pt\hbox{\vrule width0.4pt height1.875ex\vrule width 5pt height0.4pt}%
          \hskip1pt%
        \fi%
      \else%  
        \hskip6pt%
      \fi%
      \insertsectionhead
    \end{beamercolorbox}
% Code for subsections removed here
}
\makeatother
\input{../.modules/basics/biblatex.tex}

\title{Understanding the Ising CFT}
\addbibresource{ising.bib}

%%% ID: sensitive, do NOT publish!
\InputIfFileExists{id.tex}{}{}

\makeatletter
\newcommand{\nobeginpar}{\@beginparpenalty=10000}
\makeatother

\begin{document}
\maketitle
\pagenumbering{arabic}
\thispagestyle{empty}

%\vspace*{-.5\baselineskip}

\setlength{\parskip}{.1\baselineskip}
\tableofcontents
\setlength{\parskip}{\parskipnorm}

\addtocounter{section}{-1}
\section{References}
\raggedright
	\begin{itemize}
	\item \cite{Atanasov:2017abc}~\fullcite{Atanasov:2017abc}
		\begin{itemize}
		\item \textbf{Appendix E} of: \par
		\cite{Belavin:1984vu}~\fullcite{Belavin:1984vu}.
		
		\item \textbf{Section \Romannum{4}} of: \par
		\cite{Kogut:1979wt}~\fullcite{Kogut:1979wt}.
		
		\item \textbf{Chapter 5} of:\par
		\cite{Fradkin:1991nr}~\fullcite{Fradkin:1991nr}.
		
		\end{itemize}
	\end{itemize}
\justifying

\section{Primaries and their physical meanings}
	Primaries:
	\begin{itemize}[noitemsep]
	\item $\idty$
	\item Spin field: $\sigma(z,\bar{z})$ w/ weight $(\frac{1}{16},\frac{1}{16})$
	\item Energy operator: $\epsilon(z,\bar{z})$ w/ weight $(\frac{1}{2},\frac{1}{2})$
	\end{itemize}
	Question: how do we get this?
	One can answer this question through bootstrap, namely under the constraints of conformal symmetry and unitarity, this is the only possible situation for a $c = \tilde{c} = \frac{1}{2}$ CFT. But this doesn't quite explain the physical meaning of these fields. 
	
	To really understand the physical origin, we have to go back to the lattice model. Taking the continuous limit of the lattice model is quite a subtle process. We will summarize this procedure here, in a very rough and very schematic language. For a detailed account with exact equations, see e.g.~\cite{Atanasov:2017abc,Kogut:1979wt,Fradkin:1991nr}. 
\subsection{Ising model on the lattice}
	First, we start with the 2D Ising action, with the field $s(x,\tau) = \pm\frac{1}{2}$, defined on a square lattice labeled by $(x,\tau)$. Consider:
	\begin{equation}
		\mcal{Z} = \int \DD{s} e^{-S[s]}
	\end{equation}
	It can be interpreted in the following 2 ways:
	\begin{enumerate}
	\item If we think of $\tau$ as another spatial direction, then $\mcal{Z}$ is precisely the \textbf{classical partition function} of this 2D lattice. The inverse temperature $\beta$ is related with the coupling of the system. When it's tuned to the critical point, we have a CFT. 
	
	\item Alternatively, if the column direction is interpreted as the Euclidean time $\tau$, while the row direction is interpreted as the only spatial direction $x$, then this is the \textbf{Euclidean path integral} of a quantum system, where $\beta$ is the size of the $\tau$ circle. 
	\end{enumerate}
	
	Some choose to call this the \textit{quantum--classical correspondence}\footnote{
		See \cite{Simmons-Duffin:2016gjk} and \https{mcgreevy.physics.ucsd.edu/s19/2019-215C-lectures.pdf}, and also \href{https://mcgreevy.physics.ucsd.edu/s14/239a-lectures.pdf}{\texttt{/s14/239a-lectures.pdf}}. 
	}. Here we shall take the second perspective, and the Hilbert space consists of the 2-component Pauli spinor $s(x)$ located at each site~$x$. 
	
	We first take the continous $\tau$ limit by switching to the Hamiltonian description. We then rewrite the time evolution using the \textit{transfer matrix}:
	\begin{equation}
		e^{-\tau H}
	\end{equation}
	Where $H$ is a $2\times 2$ matrix acting on the vector space of $s(x,\tau)$ values. 
	$H[\sigma^\pm]$ can be written down explicitly using the Pauli matrices $\sigma^\pm$, and it is in fact $\tau$-independent. Roughly speaking, we have:
	\begin{equation}
		\sigma \sim \sigma^z
		= 2\sigma^+ \sigma^- - 1
	\end{equation}
	
	Note that the original Ising action contains only the $s \sim s^z$ field, with no mention of the $s^x,s^y$ or $s^\pm$ component. Why do we have $\sigma^\pm$ now? In fact, this captures the dynamics of spin flips and serves as the momentum term in the Hamiltonian. 
	
	Note that $H = H[\sigma^\pm]$ itself closely resembles a 1D Ising model with $s$ replaced by $\sigma$, defined along the one spatial direction: $\sigma = \sigma(x)$. 
	Some would like to call it the 1D ``quantum'' Ising model, and call this the correspondence between the 2D ``classical'' model and the 1D ``quantum'' model. But we see here that this is simply the Lagrangian and Hamiltonian descriptions of the same system. 
	
\subsection{Fermionization by Jordan--Wigner}
	
	Why the central charge is only $\frac{1}{2}$? We are looking at Weyl fermions. $\mrm{Ising}^2$ is $c = \tilde{c} = 1$ free boson CFT, with the usual fermionization as in string theory. 
	
	
	
	
	
	
\vspace{1.2\baselineskip}
\pagebreak[4]
\raggedright
\printbibliography[%
%	title = {参考文献} %
	,heading = bibintoc
]
\end{document}
