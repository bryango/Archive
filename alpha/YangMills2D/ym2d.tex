% !TeX document-id = {b5392a94-51a3-49d1-9ba5-698bc09f9d35}
% !TeX encoding = UTF-8
% !TeX spellcheck = en_US
% !TeX TXS-program:bibliography = biber -l zh__pinyin --output-safechars %

\documentclass[a4paper
	,10pt
%	,twoside
]{article}

% to be `\input` in subfolders,
% ... therefore the path should be relative to subfolders.

\usepackage{iftex}
\ifPDFTeX
\else
	\usepackage[UTF8
		,heading=false
		,scheme=plain % English Document
	]{ctex}
\fi
%\ctexset{autoindent=true}
\usepackage{indentfirst}

\input{../.modules/basics/macros.tex}
\input{../.modules/preamble_base.tex}
\input{../.modules/preamble_beamer.tex}
\input{../.modules/basics/biblatex.tex}


%Misc
	\usepackage{lilyglyphs}
	\newcommand{\indicator}{$\text{\clefG}$}
	\newcommand{\indicatorInline}{$\text{\clefGInline}$}

\newcommand{\legacyReference}{{
%	\clearpage\par
%	\quad\clearpage
	\def{\midquote}{\textbf{PAST WORK, AS TEMPLATE}}
	\newparagraph
}}

% Settings
\counterwithout{equation}{section}
\mathtoolsset{showonlyrefs=false}
%\DeclareTextFontCommand{\textbf}{\sffamily}

% Spacing
\geometry{footnotesep=2\baselineskip} % pre footnote split
\setlength{\parskip}{.5\baselineskip}
\renewcommand{\baselinestretch}{1.15}


%% List
%	\setlist*{
%		listparindent=\parindent
%		,labelindent=\parindent
%		,parsep=\parskip
%		,itemsep=1.2\parskip
%	}


\addtobeamertemplate{navigation symbols}{}{%
    \usebeamerfont{footline}%
%    \usebeamercolor[fg]{footline}%
    \hspace{1em}%
    \normalsize\insertframenumber/\inserttotalframenumber
}

\makeatletter
\setbeamertemplate{headline}
{%
    \begin{beamercolorbox}[wd=\paperwidth,colsep=1.5pt]{upper separation line head}
    \end{beamercolorbox}
    \begin{beamercolorbox}[wd=\paperwidth,ht=2.5ex,dp=1.125ex,%
      leftskip=.3cm,rightskip=.3cm plus1fil]{title in head/foot}
      \usebeamerfont{title in head/foot}\insertshorttitle
    \end{beamercolorbox}
    \begin{beamercolorbox}[wd=\paperwidth,ht=2.5ex,dp=1.125ex,%
      leftskip=.3cm,rightskip=.3cm plus1fil]{section in head/foot}
      \usebeamerfont{section in head/foot}%
      \ifbeamer@tree@showhooks
        \setbox\beamer@tempbox=\hbox{\insertsectionhead}%
        \ifdim\wd\beamer@tempbox>1pt%
          \hskip2pt\raise1.9pt\hbox{\vrule width0.4pt height1.875ex\vrule width 5pt height0.4pt}%
          \hskip1pt%
        \fi%
      \else%  
        \hskip6pt%
      \fi%
      \insertsectionhead
    \end{beamercolorbox}
% Code for subsections removed here
}
\makeatother
\input{../.modules/basics/biblatex.tex}

\title{2D Yang--Mills \\ \& Cohomological Field Theory}
\addbibresource{ym2d.bib}

%%% ID: sensitive, do NOT publish!
%\InputIfFileExists{../id.tex}{}{}

%\usepackage{cancel}
\usepackage{xspace}
\newcommand{\YM}{{\ensuremath{\mrm{YM}_2}}\xspace}

\begin{document}
\maketitle
\pagenumbering{arabic}
\thispagestyle{empty}

%\vspace*{-.5\baselineskip}

\setlength{\parskip}{.1\baselineskip}
\tableofcontents
\setlength{\parskip}{\parskipnorm}

\addtocounter{section}{-1}
\section{Main References}
	This review is almost entirely based on the following references:
	\begin{enumerate}[
		noitemsep
		,labelindent=\parindent
		,align=left
		,leftmargin=*
	]
	\item[\cite{Cordes:1994fc}] \textit{Cordes, Moore, Ramgoolam}, \arxiv{hep-th/9411210}
	\item[\cite{Witten:1991we}] Witten, \textit{On Quantum Gauge Theories in Two Dimensions}, 1991
	\end{enumerate}
	It is basically a condensed, more pedagocial version of \cite{Cordes:1994fc}. Many basic facts in this review are therefore uncited to avoid repeated citations of \cite{Cordes:1994fc}. All uncited claims, unless otherwise specified, can be traced back to \cite{Cordes:1994fc}. 
\section{Introduction}
	We start by writing down the usual Yang--Mills action in 2D (\YM), in Euclidean signature:
	\begin{equation}
		I_{\YM}
		= +\frac{1}{8e^2}
			\int_{\Sigma_T} \dd[2]{x}
			\sqrt{G}
			\,\Tr \pqty{F_{ij} F^{ij}},
	\quad
		\sqrt{G} = \sqrt{\det G_{ij}}
	\end{equation}
	Here we will try follow the convention of \cite{Cordes:1994fc}, despite the fact that it is, unfortunately, not quite self-consistent. $\Sigma_T$ stands for the 2D \textit{target}; \sidenote{as we shall see,} in the large $N$ limit, it is possible to realize 2D Yang--Mills as a string theory with worldsheet $\Sigma_W$, as is proposed by D.~Gross and W.~Taylor, among others \cite{Gross:1992tu,Gross:1993hu,Gross:1993yt}. 
	
	Note that in 2D, the $\mfrak{g}$-valued curvature form $F = F^a_{ij}\,T_a \dd{x^i} \wedge \dd{x^j}$ is a \textit{top form}; here $T_a$ is the generator of Lie algebra $\mfrak{g} = \mop{Lie} G$, and $G$ is the compact gauge group, e.g.~$G = \mrm{SU}(N)$. This means that in 2D, we have:
	\begin{gather}
		F = f\mu,\quad f = \hodgedual F,
	\\
		\mu = \sqrt{G} \dd[2]{x},\quad
		F^a_{ij} = \sqrt{G}\,\epsilon_{ij} f^a
	\end{gather}
	Here $\mu$ is the volume form on $\Sigma_T$, and $f$ is some $\mfrak{g}$-valued 0-form. The original \YM action can thus be rewritten as:
	\begin{equation}
		I_{\YM}
		= \frac{1}{4e^2}
			\int_{\Sigma_T} \dd[2]{x}
			\sqrt{G}
			\,\Tr\,(f^2)
		= \frac{1}{4e^2}
			\int_{\Sigma_T} \mu \Tr\,(f^2)
	\end{equation}
	
	First we would like to examine the $e^2 \to 0$ limit of this theory. This can be achieved by a \textit{Hubbard--Stratonovich transformation}\footnote{
		See \wikiref{https://en.wikipedia.org/wiki/Hubbard\%E2\%80\%93Stratonovich\_transformation}{Hubbard–Stratonovich transformation}. 
	}; namely, we introduce an additional $\mfrak{g}$-valued field $\phi$ that serves as a Lagrangian multiplier; consider:
	\begin{equation}
		I[\phi,A] = \frac{1}{2} \int \mu\,\pqty{
				i \Tr\,(\phi f)
				+ \frac{1}{2}\, e^2 \Tr\,(\phi^2)
			}
	\end{equation}
	Using the functional version of the integral identity: $
		\int \frac{\dd{x}}{\sqrt{2\pi}}
			e^{
				-\frac{1}{2} \pqty\big{
					\frac{e^2}{2} x^2
					+ ixy
				}
			}
		= e^{-y^2/(4e^2)}
	$, it is straightforward to verify that\footnote{
		See \cite{Witten:1991we} for a more detailed explanation. With $e\mapsto \sqrt{2}\,e,\ y\mapsto 2y$, we recover the original formula in \cite{Witten:1991we}.
	}:
	\begin{equation}
		\int \DD{\phi} e^{-I[\phi,A]}
		= e^{-I_\YM[A]}
	\end{equation}
	
	The advantage of this formulation is that the $e^2 \to 0$ limit becomes non-singular; in fact, now we can simply set $e^2 = 0$, and get:
	\begin{equation}
		I[\phi,A]
		\longto I_0[\phi,A]
		= \frac{1}{2} \int i \Tr\,(\phi F)
	\end{equation}
	This action is in fact \textit{topological}; there is no explicit metric dependence in the action. Integrating out $\phi$ fixes $F = 0$, i.e.~we need only sum over the moduli of \textit{flat connections}. For a principal $G$ bundle $P\to\Sigma_T$, this is given by:
	\begin{gather}
		\mcal{M}_0
		= \mcal{M}\pqty{F = 0, P\to\Sigma_T}
		= \Bqty{
				A\in \mcal{A}(P)
				\,\Big|\,
				F(A) = 0
			} \Big/ \mcal{G}(P)
		\ \subset\ %
		\mcal{A}(P) \Big/ \mcal{G}(P)
	\\[1ex]
%	\begin{aligned}
		\mcal{A}(P)
%		&
		= \Bqty\Big{
			\text{all possible connections $A$ on $P$}
		} \notag\\
		\mcal{G}(P)
%		&
		= \Bqty\Big{
			\text{all possible gauge transformations on $P$}
		} \notag
%		\\
%	\end{aligned}\notag
	\end{gather}
	
	The moduli space $\mcal{M}_0$ is far from trivial. Flatness implies that all contractible loops correspond to trivial holonomy; only non-trivial circles, i.e.~elements of the homotopy group $\pi_1(\Sigma_T)$, may have non-trivial holonomy. Furthermore, holonomies that differ by a global gauge transformation are by definition, equivalent. In fact, we have \cite{michiels2013moduli}:
	\begin{equation}
		\mcal{M}_0 = \Hom \pqty\big{
				\pi_1(\Sigma_T), G
			} \big/ G
	\end{equation}
	Note that this only identifies the topology of $\mcal{M}_0$; to compute the path integral, we need to derive the measure on $\mcal{M}_0$ following the Faddeev--Popov procedure, which is implemented in \cite{Witten:1991we}.
	
	For $e^2 \ne 0$, the action $I[\phi,A]$ is metric dependent. Somewhat surprisingly, the path integral still contains information about the topology of $\mcal{M}$. This is an example of a so-called \textit{cohomological field theory}. 
	The idea of \cite{Cordes:1994fc} is to start from 2D Yang--Mills as a concrete example, and then use its results to motivate a thorough study of {cohomological field theory}.
	
	As is summarized in \cite{Cordes:1994fc}, topological field theories (TFT's), largely introduced by E.~Witten, may be grouped into two classes: \textit{Schwarz type} and \textit{cohomological type}\footnote{
		Cohomological type TFT's are also called \textit{Witten type} TFT's, e.g.~in \cite{Birmingham:1991ty}. However, \cite{Cordes:1994fc} chooses to call them \textit{cohomological}, probably to avoid confusion, since Witten has done wonderful work on both types of the theories.
	}. 
	Cohomological field theories, including 2D Yang--Mills with coupling $e^2 \ne 0$, are \textit{not} manifestly metric independent; however, they have a Grassmann-odd nilpotent BRST operator $Q$, and physical observables are $Q$-cohomology classes; amplitudes involving these observables are metric independent, thus they are indeed \textit{topological}. 
%	because of decoupling of BRST trivial degrees of freedom.
	
	On the other hand, \textit{Schwarz type} theories have Lagrangians which are metric independent and hence, formally, the quantum theory is expected to be topological. 
	Examples of such theories include the $e^2 = 0$ \YM described above, and also the Chern--Simons theory in 3D. Also, there is a 4D analog of the action $I[\phi,A]$, given by:
	\begin{equation}
		\int \pqty\Big{
			\Tr\,(BF)
			+ e^2 \Tr\,(B\wedge\hodgedual B)
		}
	\end{equation}
	The first term with $BF$ is also manifestly topological, similar to $e^2 = 0$ \YM; therefore Schwarz type theories are also called $BF$ type theories.
	
	Following \cite{Cordes:1994fc}, we will first review the exact solution of \YM, and then try to generalize some aspects for a generic cohomological field theory.
%	
%	In chapter 4 we relate the Hilbert space of class functions to some simple conformal field theories. The main goal is to show how bosonization leads to a natural interpretation of the Hilbert space in terms of string states.
%	
%	4.7
%	
\section{Exact Solution of 2D Yang Mills}
\subsection{Canonical Quantization on the Cylinder}
	One can perform the usual canonical quantization with $I_\YM$ on the cylinder, with coordinates $(x^0,x^1) = (t,x) \in \mbb{R}^1\times S^1$. We shall make full use of the gauge redundancies in 2D; recall that a generic gauge transformation can be written as:
	\begin{equation}
		A'_\mu
		= g A_\mu g^{-1}
			+ g\,\pdd{\mu} (g^{-1}),
	\quad
		A_\mu = A^a_\mu(t,x),
	\quad
		g = e^{-\lambda^a(t,x)\,T_a}
	\end{equation}
	It is thus possible to choose the \textit{temporal gauge} $A_0 = 0$, by simply solving a first order ODE of the gauge parameters $\lambda^a(t,x)$, with respect to the variable $x^0 = t$. 
	
	We can further reduce $A_1(t,x)$ with remaining gauge redundancies; with some $t$-independent, but $x$-dependent $g = g(x)$, we can preserve $A_0 = 0$, while reducing $A_1(t,x) = A_1(t)$. This is basically the \textit{Coulomb gauge} in 2D, i.e.~we have $\pdd{1} A_1 = 0$. 
	
	We can further simplify the results by working in the \textit{Schr\"odinger picture}. A nice treatment of QED from this ``novel'' perspective can be found in \cite{Hatfield:234595}. In conventional formulations of QFT, we are used to work in the \textit{Heisenberg} or \textit{interactive picture}, where the fields evolve in time: $A_1 = A_1(t)$ and satisfy some operator equations of motion (EOM's), which for free theories look identical to the classical EOM's. Alternatively, we can take the quantum mechanical approach, and decompose the fields at each time slice $t = t_0$ to a set of time-independent energy eigenstates; in the case of \YM, we have $A_1 = \mrm{const}$. 
	The time evolution is then tracked by the \textit{wave functional} $\Psi_t[A_1]$. Since the gauge-fixed $A_1$ has no spacetime dependence, we've actually obtained a equivalent 0-dimensional field theory, i.e.~a quantum mechanical system. 
	
	There are still remaining gauge redundancies; with another spacetime independent, \textit{global} gauge transformation, we can rotate $
		A_1 = A_1^a\,T_a \in \mfrak{g}
	$ to the \textit{Cartan subalgebra}, i.e.~the maximal abelian subalgebra of $\mfrak{g}$. Finally, we demand that $A_1$ is invariant under the \textit{Weyl group}, which is the symmetry of the Cartan subalgebra. Therefore, the \textit{physical} Hilbert space of \YM consists of states given by:
	\begin{equation}
		\Psi[A_1],\quad
		A_1
		= A_1^a\,T_a
		\in \textsl{Cartan} \,\Big/ \textsl{Weyl}
	\end{equation}
	
	Alternatively, we can also work with a partial gauge fixing, e.g.~we only impose the temporal gauge $A_0 = 0$, and try to solve for $\Psi_t[A_1(x)]$ by looking at the ``Maxwell's equation" in 2D. The time evolution is taken care of by the Schr\"odinger equation for $\Psi_t$; for now we need only look at the spatial constraints. We have:
	\begin{gather}
		D_1 F_{10} = 0
	\label{eq:gauss_law_constraint}
	\\[.5ex]
		D_\mu = \pdd{\mu} + A^a_\mu\,T_a,
	\quad
		F_{\mu\nu} = [D_\mu,D_\nu]
	\end{gather}
	This is simply the \YM version of the \textit{Gauss's law} $\vec{\nabla}\cdot\vec{E} = 0$. 
	
	One can think of the Gauss's law constraint as the result of integrating out $A_0$, which imposes it's EOM $\fdv{I_\YM}{A_0} = 0$, which is precisely \eqref{eq:gauss_law_constraint}. 
	However, for a gauged system, there are subtleties that we need to look out for. One should account for the gauge volume, which can be treated properly with Faddeev--Popov path integral; and the proper way to implement the constraints is through BRST quantization. 
	
	Fortunately, the Gauss's law constraint \textit{does} work in this example. In fact, if we solve the Gauss's law constraint as an operator equation of $A_1$, we will get further gauge-fixing \cite{Hatfield:234595}. 
	Alternatively, if we ignore the constraint and proceed with canonical quantization, which might be more convenient in some cases, we would expect \textit{unphysical} degrees of freedom like \textit{null states} to show up, due to the unfixed gauge redundancies. The constraint can then be utilized to identify \textit{physical} degrees of freedom, by demanding that it annilates physical states:
	\begin{equation}
		D_1 F_{10} \ket{\Psi} = 0
	\label{eq:gauss_law_constraint_state}
	\end{equation}
	Note that this idea is very much similar to \textit{old covariant quantization} and the \textit{Virasoro constraint} in string theory \cite{Polchinski:1998rq}. Again, for a more rigorous treatment, we should turn to the BRST cohomology, but for now this is sufficient. In fact, we can actually proved \eqref{eq:gauss_law_constraint_state} by demanding the wave functional $\Psi[A_1(x)]$ to be gauge-invariant under the remaining gauge transformations \cite{Hatfield:234595}:
	\begin{equation}
		0 = \var{\Psi[A_1(x)]}
		= \sidenote{...}
	\end{equation}
	
	To actually solve \eqref{eq:gauss_law_constraint_state}, we shall work in the $A_1(x)$ basis, with $\Psi[A_1(x)] = \braket{A_1(x)}{\Psi}$. The canonical momentum operator is then given by:
	\begin{equation}
		E = \fdv{A_1},
	\quad
		\bqty{
			E(x), A_1(x')
		} = \delta(x - x')
	\end{equation}
	Just like the usual quantum mechanical $P = - i\pdv{X}$; the $(-i)$ factor is gone since we are working in Euclidean signature. \sidenote{Just like Schr\"odinger, but with path-ordering...}
	
\raggedright

\printbibliography[%
%	title = {参考文献} %
	,heading = bibintoc
]
\end{document}
