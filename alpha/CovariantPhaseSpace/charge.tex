% !TeX TS-program = pdflatex
% !TeX document-id = {b5392a94-51a3-49d1-9ba5-698bc09f9d35}
% !TeX encoding = UTF-8
% !TeX spellcheck = en_US
% !TeX TXS-program:bibliography = biber -l zh__pinyin --output-safechars %

\documentclass[a4paper
	,10pt
%	,twoside
]{article}

% to be `\input` in subfolders,
% ... therefore the path should be relative to subfolders.

\usepackage{iftex}
\ifPDFTeX
\else
	\usepackage[UTF8
		,heading=false
		,scheme=plain % English Document
	]{ctex}
\fi
%\ctexset{autoindent=true}
\usepackage{indentfirst}

\input{../.modules/basics/macros.tex}
\input{../.modules/preamble_base.tex}
\input{../.modules/preamble_beamer.tex}
\input{../.modules/basics/biblatex.tex}


%Misc
	\usepackage{lilyglyphs}
	\newcommand{\indicator}{$\text{\clefG}$}
	\newcommand{\indicatorInline}{$\text{\clefGInline}$}

\newcommand{\legacyReference}{{
%	\clearpage\par
%	\quad\clearpage
	\def{\midquote}{\textbf{PAST WORK, AS TEMPLATE}}
	\newparagraph
}}

% Settings
\counterwithout{equation}{section}
\mathtoolsset{showonlyrefs=false}
%\DeclareTextFontCommand{\textbf}{\sffamily}

% Spacing
\geometry{footnotesep=2\baselineskip} % pre footnote split
\setlength{\parskip}{.5\baselineskip}
\renewcommand{\baselinestretch}{1.15}


%% List
%	\setlist*{
%		listparindent=\parindent
%		,labelindent=\parindent
%		,parsep=\parskip
%		,itemsep=1.2\parskip
%	}


\addtobeamertemplate{navigation symbols}{}{%
    \usebeamerfont{footline}%
%    \usebeamercolor[fg]{footline}%
    \hspace{1em}%
    \normalsize\insertframenumber/\inserttotalframenumber
}

\makeatletter
\setbeamertemplate{headline}
{%
    \begin{beamercolorbox}[wd=\paperwidth,colsep=1.5pt]{upper separation line head}
    \end{beamercolorbox}
    \begin{beamercolorbox}[wd=\paperwidth,ht=2.5ex,dp=1.125ex,%
      leftskip=.3cm,rightskip=.3cm plus1fil]{title in head/foot}
      \usebeamerfont{title in head/foot}\insertshorttitle
    \end{beamercolorbox}
    \begin{beamercolorbox}[wd=\paperwidth,ht=2.5ex,dp=1.125ex,%
      leftskip=.3cm,rightskip=.3cm plus1fil]{section in head/foot}
      \usebeamerfont{section in head/foot}%
      \ifbeamer@tree@showhooks
        \setbox\beamer@tempbox=\hbox{\insertsectionhead}%
        \ifdim\wd\beamer@tempbox>1pt%
          \hskip2pt\raise1.9pt\hbox{\vrule width0.4pt height1.875ex\vrule width 5pt height0.4pt}%
          \hskip1pt%
        \fi%
      \else%  
        \hskip6pt%
      \fi%
      \insertsectionhead
    \end{beamercolorbox}
% Code for subsections removed here
}
\makeatother
\input{../.modules/basics/biblatex.tex}
\input{../.modules/basics/theorem.tex}

\declaretheorem[
	numbered=no
	,style=nonumber
	,name=Remark
]{remark*}

\title{Covariant Phase Space\\[-.2ex]\large\it A Pedagogical Introduction}
\addbibresource{charge.bib}

%%% ID: sensitive, do NOT publish!
\InputIfFileExists{id.tex}{}{}

\makeatletter
\newcommand{\nobeginpar}{\@beginparpenalty=10000}
\makeatother

\usepackage{cancel}

\begin{document}
\maketitle
\pagenumbering{arabic}
\thispagestyle{empty}

%\vspace*{-.5\baselineskip}

\setlength{\parskip}{.1\baselineskip}
\tableofcontents
\setlength{\parskip}{\parskipnorm}

\addtocounter{section}{-1}
\section{Introduction \& Conventions}
	The phase space is understood as the space of \textit{all solutions} of the equations of motion (EoMs), satisfying the boundary conditions. Alternatively, we can think of it as a collection of all possible \textit{initial configurations} of the system, and the additional \textit{symplectic structure} specifies their evolutions. This idea is best explained by \textcite{Crnkovic:1986ex}. 
	
	Traditionally we construct the phase space variables $(\pi,\phi)$ as follows: we first pick a special time coordinate $t$, and then introduce the canonical momentum:
	\begin{equation}
		\pi = \pdv{\mcal{L}}{\dot{\phi}},
	\quad
		\dot{\phi} = \pdd{t} \phi
	\end{equation}
	Unfortunately, this procedure breaks general covariance. We would like to find a construction of phase space that respects general covariance. 
	
	The inspiration comes from the Lagrangian treatment of field theory, where we work with:
	\begin{equation}
		\pi^\mu = \pdv{\mcal{L}}{(\pd_\mu\phi)}
	\end{equation}
	Which preserves general covariance. But $(\pi_\mu,\phi)$ is not a set of independent coordinates; they are constrained by the equations of motion (EoMs). The rough idea is that we start with $(\pi_\mu,\phi)$, and then reduce it to the independent variables $(\pi,\phi)$ by imposing the EoMs, and possibly, additional constraints due to gauge redundancy. 
	
	In this note we review the covariant formalism in a systematic yet pedagogical manner. 
	We follow Prof.~Wei Song's lecture notes on $\mrm{AdS}_3/\mrm{CFT}_2$, which include a very nice introduction to the covariant formalism and its application. 
	When it comes to notations, we try to follow the conventions of \textcite{Harlow:2019yfa}, with the following exceptions and extensions: 
	
	\begin{itemize}
	\item $\mcal{F}$ denotes the field space including \textbf{off-shell} configurations: $\phi^I$,\\
	and also \textbf{background fields} $\phi^B$; \\
	the variation $\delta$ is assumed to be off-shell unless otherwise specified.
	
	\item $\tilde{\mcal{P}}$ denotes the field space including \textbf{on-shell} configurations only; \\
	one can then define a symplectic form $\omega$ on $\tilde{\mcal{P}}$, \\
	but it might be degenerate due to gauge redundancies.
	
	\item $\mcal{P}$ denotes the \textbf{physical phase space}, \\
	with background fields turned off: $\var{\phi^B}|_{\mcal{P}} = 0$, \\
	and with a ``physical'' (non-degenerate) sympletic structure
	
	\item The index $I$ labels \textbf{all fields} in the theory, \\
	including non-dynamical \textbf{background fields}, \\
	which is labeled by the index $B$.
	
	\end{itemize}
\smallskip
\textit{Differential geometry:}
	
	\begin{itemize}
	\item Contraction between a (poly-)vector $V$ with the \textbf{first few indices} of a form $\omega$ is denoted as:
	\begin{equation}
		V\cdot \omega = \iota_V \omega
	\end{equation}
	In differential geometry, this is more commonly referred to as the \textit{interior product} (or interior derivative, interior multiplication, etc), but there is no need for such fancy notion as it's just simple contraction. 
	
	\item Raising and lowering indices are denoted by $\sharp$ and $\flat$ respectively; these are given the cool name of \textit{musical isomorphisms}. 
	
	For simplicity, sometimes we might omit the explicit notations of musical isomorphisms, but they are nonetheless implied; e.g.~the normal (co-)vector $n$ can be understood as $n^\mu$ or $n_\mu$. 
	
	\item The Levi-Civita \textbf{symbol} is denoted as $\epsilon_{\cdots}$ with \textbf{no} $\sqrt{\abs{g}}$ factor; we will try to keep the $\sqrt{\abs{g}}$ explicit. Therefore, the Levi-Civita \textbf{tensor}, i.e.~the standard volume form, is given by:
	\begin{equation}
	\begin{aligned}
		\mrm{Vol}_M
		= \sqrt{\abs{g}} \dd[D]{x}
		&= \sqrt{\abs{g}}\, \epsilon_{\cdots}
			\dd{x^\bullet} \otimes
			\dd{x^\bullet} \otimes
			\cdots \otimes
			\dd{x^\bullet}
		= \sqrt{\abs{g}} \dd{x^1} \wedge
			\cdots \wedge \dd{x^D} \\
		&= \sqrt{\abs{g}}
			\ \frac{1}{D!}\,
			\epsilon_{\cdots}
			\dd{x^\bullet} \wedge
			\dd{x^\bullet} \wedge
			\cdots \otimes
			\dd{x^\bullet}
	\end{aligned}
	\end{equation}
	We see here that our $\mquote{\wedge}$ is defined by anti-symmetrizing $\mquote{\otimes}$ \textit{without} averaging with a $\frac{1}{D!}$ factor; the $\frac{1}{D!}$ in the second line is to cancel an \textit{additional} contraction with the $\epsilon_{\cdots}$ symbol. 
	If we raise its indices, we get:
	\begin{equation}
		(\sqrt{\abs{g}} \dd[D]{x})^{\sharp,\cdots}
		= \frac{(-1)^s}{\sqrt{\abs{g}}}\,
			\epsilon^{\cdots}
	\end{equation}
	Here $\epsilon^{\cdots}$ is still the Levi-Civita \textbf{symbol} without $\sqrt{\abs{g}}$ factor, and we have $\epsilon_{\cdots} \epsilon^{\cdots} = D!$. The $(-1)^s$ comes from signature of the metric; for $
		\eta_{ab} \sim \mop{diag}(-1,1,1,\cdots)
	$ we have $s = 1$. By the way we adopt the \textit{mostly plus} convention for Lorentzian metric, as any sane person should do; through Wick rotation it gets mapped to an \textit{all plus} Euclidean metric. 
	
	\item Hodge dual of a $p$-form $\omega$ is defined as contraction with the volume form:
	\begin{equation}
		\hstar\omega
		= \frac{1}{p!}\,
			\omega^\sharp \cdot \mrm{Vol}_M
	\quad\Longleftrightarrow\quad
		\eta\wedge \hstar\omega
		= \aqty{\eta,\omega} \,\mrm{Vol}_M
	\end{equation}
	$\aqty{\cdot,\cdot}$ is the induced inner product on the space of $p$-forms. 
	The $\frac{1}{p!}$ factor is necessary to guarentee that $\hstar\hstar\omega = (-1)^s\cdot(-1)^{p(D-p)} \omega$. Again the $(-1)^s$ comes from signature of the metric. In particular, for a scalar $f$ and a vector $V$, we have:
	\begin{equation}
		\hstar f
		= f\sqrt{\abs{g}} \dd[D]{x},
	\quad
		\hstar\,(f\sqrt{\abs{g}} \dd[D]{x})
		= (-1)^s f,
	\quad
		\hstar \theta^\flat
		= \theta\cdot \sqrt{\abs{g}} \dd[D]{x}
	\end{equation}
	It's convenient to only act $\hstar$ on the form of a smaller degree to get the form of a larger degree; in this way we don't need to worry about the $(-1)^s$ factor. 
	A particular useful relation is:
	\begin{equation}
		\dd{\hstar \theta^\flat}
		= \hstar\,(\cdv{\mu} \theta^\mu)
	\end{equation}
	This is derived around \eqref{eq:divergence_form}. The inner product $\aqty{\cdot,\cdot}$ is particularly convenient for 1-forms; in this case we have:
	\begin{equation}
		n \wedge\hstar j^\flat
		= n_\mu j^\mu \,\mrm{Vol}_M
	\end{equation}
	
	\end{itemize}
%\smallskip
\textit{Diffeomorphism and the stress tensor:}
	
	\begin{itemize}
	\item A infinitesimal diffeomorphism (diffeo) generated by vector field $\xi$ is given by:
	\begin{equation}
		x\mapsto x - \xi,
	\quad
		\var_\xi \phi
		= X_\xi \cdot \var{\phi}
		= +\ldv{\xi} \phi,
	\label{eq:diffeo_convention}
	\end{equation}
	Here the sign convention is chosen such that $\var_\xi \phi$ corresponds to a positive Lie derivative along $\xi$. Note that \textcite{Banados:2016zim} picks a unconventional Lie derivative which contains a minus sign in its definition; we are \textit{not} going to follow that. In this convention the usual current for spacetime translation is given by:
	\begin{equation}
	\begin{aligned}
		\xi &= -\pdd{\sigma},
	&
		j^\mu_\xi
		&= j^\mu_{-\pdd{\sigma}}
		= T^{\mu\nu} \eta_{\nu\sigma}
		= T^{\mu\nu} (\pdd{\sigma})_\nu,
%	\\
%		\xi &= -\epsilon
%		= -\epsilon^\sigma \pdd{\sigma},
%	\quad
%	&
%		j^\mu_\xi
%		&= \epsilon^\sigma j^\mu_{-\pdd{\sigma}}
%		= \epsilon_\nu T^{\mu\nu}
%		= \epsilon^\nu T\id{^\mu_\nu},
	\end{aligned}
	\label{eq:Noether_stress_convention}
	\end{equation}
	Where $T^{\mu\nu}$ is the Noether stress tensor. 
	
	\item Wick rotation between Lorentzian and Euclidean signatures is defined such that:
	\begin{equation}
		\tau = it,
	\quad
		V^0_E = iV^0_L,
	\quad
		(\dd[D]{x})_E
		= i\,(\dd[D]{x})_L,
	\quad
		e^{iS_L} = e^{-S_E},
	\quad
		\mcal{L}_E
		= -\mcal{L}_L
	\label{eq:Wick_rotation}
	\end{equation}
	Where $
		S = \int \dd[D]{x} \sqrt{\abs{g}}\,\mcal{L}
	$ in both cases, and $V^0$ is the 0-th component of some $V^\mu$. Other scalar quantities, such as $
		g_{\mu\nu} V^\mu W^\nu
	$, are left invariant. This is the convention introduced in \textcite{Polchinski:1998rq}, Appendix A. 
	For example, the Einstein-Hilbert action is given by\footnote{
		For a closed Riemann surface in 2D, by Gauss--Bonnet theorem, we see that the Euclidean action is proportional to genus of the surface:
		\begin{equation}
			S_E = -\frac{1}{16\pi G_N}\,
				4\pi \chi(M)
			= -\frac{1}{4G_N}\,\chi(M),
		\quad
			\chi(M) = 2 - 2g,
		\quad
			e^{-S_E}
			\propto e^{-\frac{1}{2G_N} g}
		\end{equation}
		Note that in \textcite{Polchinski:1998rq}, the dilaton term in the Euclidean worldsheet action  goes like $+R\,\Phi(X)$, namely it's missing the minus sign; this is due to the convention for the dilaton $\Phi$: in the perturbative regime we have $\ave{\Phi} = \lambda < 0$, and thus $
			e^{-S_E} \sim e^{-\lambda \chi}
			\sim (g_c^2)^g
		$, where $
			g_c \sim e^\lambda < 1
		$ is the closed string coupling. 
	}:
	\begin{equation}
		S_L = \frac{1}{16\pi G_N}
			\int \sqrt{-g} \dd[D]{x} R,
	\quad
		S_E = -\frac{1}{16\pi G_N}
			\int \sqrt{g} \dd[D]{x} R
	\end{equation}
	
	In 2D, $z,\bar{z} = x \pm i\tau = x \mp t = -u, v$, where $(u,v)$ are the lightcone coordinates. Following \textcite{Polchinski:1998rq}, the Euclidean volume form is then given by:
	\begin{equation}
		\dd{\tau} \wedge \dd{x}
		= \frac{1}{2i} \dd{z} \wedge \dd{\bar{z}}
		= \frac{1}{2} \dd[2]{z},
	\quad
		\dd[2]{z} = 2 \dd{\tau} \wedge \dd{x}
		= -i \dd{z} \wedge \dd{\bar{z}}
	\end{equation}
	The divergence theorem in complex coordinates can then be expanded as:
	\begin{equation}
		\int_R \dd[2]{z} \pdd{a} V^a
		= (-1) \oint_{\pd R}
			\dd{\ell} n_a V^a
		= (-1) \oint_{\pd R}
			V \cdot \dd[2]{z}
		= (-1)(-i) \oint_{\pd R} \pqty{
				V^z \dd{\bar{z}}
				- V^{\bar{z}} \dd{z}
			}
	\end{equation}
	The $(-1)$ factor comes from the conventional counter-clockwise contour integral $\oint_{\pd R}$ in complex \mbox{analysis}, which differs from the $\dd{\tau} \wedge \dd{x}$ orientation chosen here. 
	
	\item The Hilbert stress tensor is defined with the Lorentzian action:
	\begin{equation}
		T_{\mu\nu}(x)
		= - \frac{2}{\sqrt{-g}}
			\fdv{S_L}{g^{\mu\nu}(x)},
	\quad
		T^{\mu\nu}(x)
		= + \frac{2}{\sqrt{-g}}
			\fdv{S_L}{g_{\mu\nu}(x)}
	\end{equation}\\[-1.25\baselineskip]
	\begin{equation}
		\var{S_L}
		= \int \sqrt{-g} \dd[D]{x} 
			\pqty{
				-\frac{1}{2}\, T_{\mu\nu}
				\var{g^{\mu\nu}}
				+ \cdots
			}
		= \int \sqrt{-g} \dd[D]{x} 
			\pqty{
				+\frac{1}{2}\, T^{\mu\nu}
				\var{g_{\mu\nu}}
				+ \cdots
			}
	\end{equation}
	To understand the variational derivative here, we note that in the space of fields it's convenient to think of $x$ as an index to be contracted; the contraction is implemented by $
		\int \dd[D]{x}
	$, without the $\sqrt{\abs{g}}$ factor\footnote{
		An action can thus be thought of as the contraction between some fields and some operators with multiple $x$ ``indices'': $\Box_{x,x',\cdots}$. For example, we might have some bi-local operator $\Box_{x,x'}$ connecting $\phi(x)$ and $\phi(x')$. 
		In a \textit{local} action all these operators should be \textit{diagonal} in $x$, i.e.~$
			\Box_{x,x'} = \delta^D(x - x')\,\Box_x
		$. 
	}. 
	
	The above definition of stress tensor agrees with \cite{Harlow:2019yfa,Compere:2018aar} and differs from \textcite{Polchinski:1998rq} (the conventional string normalization) by a factor of $(-2\pi)$. We shall see that this definition agrees with \eqref{eq:Noether_stress_convention} up to a total derivative. 
	
	We would like the functional dependence of $
		T_{\mu\nu}
		= T_{\mu\nu}[\phi,\cdv{\mu} \phi,\cdots]
	$ to persist under Wick rotation, just like any other tensorial quantities; therefore we have to add an additional minus sign to the stress tensor computed with the Euclidean action $S_E$, due to our convention \eqref{eq:Wick_rotation}:
	\begin{equation}
		T_{\mu\nu}
		= + \frac{2}{\sqrt{g}}
			\fdv{S_E}{g^{\mu\nu}},
	\quad
		T^{\mu\nu}
		= - \frac{2}{\sqrt{g}}
			\fdv{S_E}{g_{\mu\nu}}
	\end{equation}
	For a gravitational theory (with dynamical $g_{\mu\nu}$), the \textit{total} Hilbert stress tensor is the left-hand side of the EoM, thus it vanishes on-shell. Therefore, we usually use $T_{\mu\nu}$ to denote the non-vanishing \textit{matter} stress tensor. 
	
	\end{itemize}
	
\pagebreak[4]
\section{Covariant formalism and Noether's theorem}
\subsection{Off-shell variation on the field space}
	\noindent
	Let's start by considering a general action with boundary terms \cite{Harlow:2019yfa}:
	\begin{equation}
		S = \int_M L + \int_{\pd M} l,
	\quad
		L = \dd[D]{x} \sqrt{\abs{g}}\,\mcal{L}
	\end{equation}
	
	Variation $\var$ is now treated as an \textit{exterior derivative} on the space of fields. We can think of it as an \textit{enlarged} space with \textit{redundant variables} $(\pi_\mu,\phi)_I$, where $I$ labels all fields in the system. We have:
	\begin{equation}
		\var{S}
		= \int_M \var{L}
			+ \int_{\pd M} \var{l}
		= \int_M E_I \var{\phi^I}
			+ \int_{\pd M} \pqty{\Theta + \var{l}},
	\end{equation}
	We've defined $\Theta$ which comes from the total derivative terms of the bulk $M$ variation, and we've performed an integration by parts and reduced $\int_M \dd{\Theta} = \int_{\pd M} \Theta$. Basically,
	\begin{empheq}{equation}
		\var{L}
		= E_I \var{\phi^I} + \dd{\Theta}
	\end{empheq}
	\vspace{-.3\baselineskip}
	\begin{equation}
		({\cdots})_I \var{(\dd{\phi^I})}
		= ({\cdots})_I \dd{(\var{\phi^I})}
		\sim \dd{(\cdots \var{\phi})}
		= \dd{\Theta},
	\quad
		\Theta \sim ({\cdots})_I \var{\phi^I}
	\end{equation}
	
	Here we haven't imposed the EoMs, thus $\var$ is understood as an \textbf{off-shell variation} on the space of fields. We will hence refers to the whole space including off-shell configurations as the \textbf{field space $\mcal{F}$}, and use \textbf{phase space $\mcal{P}$} for \textbf{on-shell configurations only}. 
	
	Note that $\mcal{F}$ \textbf{includes possible ``background'' fields $\phi^B$} such as the static metric $g^{\mu\nu}$ for a field theory defined on a curved spacetime. These fields are fixed on-shell and therefore are absent in $\mcal{P}$. A proper treatment of background fields is nicely summarized by \textcite{Banados:2016zim}, but we will take a slightly different approach, as we shall see later. 
	
	Also, we will \textit{not} distinguish phase space with $(\pi,\phi) $ coordinates and the so-called \textit{configuration space} with $(\phi,\dot{\phi})$ coordinates; though they are indeed different, they are dual to each other and related by Legendre transforms. 
	
	What we've done here is simply a rewrite of the usual variation procedure; we just re-think it as a differential in the field space. To get to the physical phase space, we
%	\textbf{suppress the variation of background fields $\phi^B$}, and then 
	require the action to be \textbf{stationary, up to
	variation of the background fields, and
	up to terms in the future / past boundary $\Sigma^\pm$:}
	\begin{empheq}{equation}
%		\var{\phi^B}|_{\tilde{\mcal{P}}}
%		= 0,
%	\quad
		\var{S}|_{\tilde{\mcal{P}}}
		= \int_M E_B \var{\phi^B}
			+ \int_{\Sigma^+} ({\cdots})
			- \int_{\Sigma^-} ({\cdots})
		= \int_M E_B \var{\phi^B}
			+ \int_{\Sigma^\pm} ({\cdots})
	\end{empheq}
	
	This differs slightly from what we are used to in most physics literature, where we simply set $\var{S} = 0$ to find the EoMs. The reason is that in physics, we usually \textit{impose} the initial \& final conditions, or in other words, in \& out states, and the variation at $\Sigma^\pm$ is \textit{required} to vanish: $\var{\phi}|_{\Sigma^\pm} = 0$. 
	Consider the simplest example: the (0+1)\,D worldline action of a point particle; we have:
	\begin{equation}
		S = \int L = \int \dd{t} \mcal{L},
	\quad
		\var{S}
		= \int E_I(t) \var{q^I(t)}
			+ \Theta\big|^{\Sigma^+}_{\Sigma^-}
	\end{equation}
	\vspace{-.5\baselineskip}
	\begin{equation}
		E_I = \pqty{
			\pdv{\mcal{L}}{q^I}
			- \dv{t} \pdv{\mcal{L}}{\dot{q}^I}
		} \dd{t},
	\quad
		\Theta
		= \pdv{\mcal{L}}{\dot{q}^I} \var{q^I}
		= p_I \var{q^I},
	\quad
		p_I = \pdv{\mcal{L}}{\dot{q}^I}
	\end{equation}
	Traditionally, we impose $\var{q^I}|_{\Sigma^\pm} = 0$ as we fix the in \& out states, and then we find the EoMs $E_I = 0$. 
	Same applies to the background field variation, where we simply set $\var{\phi^B} = 0$. 
	However, this is slightly clumsy to implement in our new language, since $\var{\phi}$ is now a differential form in the phase space. We thus simply require that $\var{S}|_{\tilde{\mcal{P}}}$ vanish up to boundary terms at $\Sigma^\pm$, which achieve the same result. 
	
	Generally, we have:
	\begin{equation}
		\var{S}|_{\tilde{\mcal{P}}}
		= \int_M E_B \var{\phi^B}
			+ \int_{\Sigma^\pm} ({\cdots})
	\quad\Longrightarrow\quad
		E_{I\ne B}|_{\tilde{\mcal{P}}} = 0,
	\quad
		\int_N \pqty{\Theta + \var{l}}_{\tilde{\mcal{P}}}
		= \int_{\Sigma^\pm} ({\cdots}),
	\end{equation}
	\vspace{-.8\baselineskip}
	\begin{equation}
		\pd M = N \cup \Sigma^+ \cup (-\Sigma^-),
	\end{equation}
	Here $N$ is the spatial boundary. 
	In this way we've obtained the subspace (submanifold) $\tilde{\mcal{P}}$ of \mbox{solutions} constrained by the EoMs. We are one step closer to the physical phase space; in fact this \textit{is} the physical phase space $\mcal{P}$ if there are no gauge redundancies in the fields, and the variation of background fields are turned off: $\var{\phi^B} = 0$. 
	
	Besides the EoMs, the boundary integral at $N$ should also vanish; this means that \textbf{not all boundary conditions are consistent with the theory}. 
	In fact, $N$ is where the dual theory lives for many holographic systems, so one would imagine that it's highly non-trivial. Generally \cite{Harlow:2019yfa},
	\begin{empheq}{gather}
		(\Theta + \var{l})_{\tilde{\mcal{P}},N}
		= \dd{C},
	\quad
		\dd{C}\colon\ %
		\text{some 1-form on $N \subset \pd M$},
	\label{eq:boundary_cond}
	\\
	\Longrightarrow\ %
		\int_{N} (\Theta + \var{l})_{\tilde{\mcal{P}}}
		= \int_{\pd M} \dd{C}
			+ ({\cdots})
		= \int_{\pd \pd M = \varnothing} C
			+ ({\cdots})
		= 0
			+ \int_{\Sigma^\pm} ({\cdots})
	\end{empheq}
	
	\eqref{eq:boundary_cond} gives the allowed boundary conditions for the fields at $N$. A prominent example of this is the worldsheet Polyakov action for an open string with $l = 0$. Explicit calculation shows that:
	\begin{equation}
		\Theta = - \dd{\tau} \sqrt{-\gamma}\,
			\pd^\sigma X \var{X}
	\end{equation}
	up to some overall coefficients; see e.g.~\textsl{\citeauthor{Polchinski:1998rq}} \cite{Polchinski:1998rq}. Here $\phi = X$ is the field, $(\tau,\sigma)$ is the worldsheet coordinates, and $\gamma$ is the worldsheet metric. The allowed boundary conditions are given by $\var{X} = 0$ or $\pd^\sigma X = 0$, which both corresponds to $C = 0$; examples given in \cite{Harlow:2019yfa} shows that a non-trivial $C$ arises for higher derivative theories and gravity. 
	
	By restricting to $\tilde{\mcal{P}}$, the variation $\delta|_{\tilde{\mcal{P}}}$ is understood as \textbf{on-shell}, since we've imposed the EoMs and the boundary conditions. Also, $\tilde{\mcal{P}}$ is where we actually define our symplectic structure. However, we will always try to first work with the \textbf{off-shell variation $\delta$ in the off-shell field space $\mcal{F}$}, keep the EoMs explicit, and then impose them at the very end. 
	
	We shall see that this formalism will help us avoid confusions with boundary terms, which is best explained by \textcite{Banados:2016zim}. Also, off-shell variation will be quite useful if we want to quantize our theory, where we have to integrate over off-shell configurations as well. 
\subsection{Symmetry variation and the Noether current}
	To get back to the familiar Noether's procedure, we need only plug in (contract with) a vector field $X_\xi$ in the field space $\mcal{F}$, labeled by the symmetry $\xi$:
	\begin{equation}
		\var_\xi
		= X_\xi \cdot \var
		= \iota_{X_\xi} \var
	\colon\ \Omega^0_\delta(\mcal{F})
		\to \Omega^0_\delta(\mcal{F})
	\end{equation}
	$\xi\mapsto X_\xi$ is the representation of $\xi$ in the field space, and $\Omega^0_\delta(\mcal{F})$ denotes 0-forms, i.e.~functions on $\mcal{F}$. 
	As an example, for diffeomorphism (diffeo), the induced variation is given by:
	\begin{equation}
		x\mapsto x - \xi,
	\quad
		\var_\xi \phi
		= X_\xi \cdot \var{\phi}
		= \ldv{\xi} \phi,
	\quad
		X_\xi
		= \int_M \dd[D]{x}
			\pqty\big{\ldv{\xi} \phi^I(x)}
			\fdv{\phi^I(x)}
	\label{eq:symmetry_var_field}
	\end{equation}
	
	We've hence defined symmetry variation $\var_\xi$ as some sort of \textit{directional derivative} on $\mcal{F}$, acting on $\Omega^0(\mcal{F})$. For its action on generic $n$-forms $\Omega^\bullet(\mcal{F})$, the more natural definition is the Lie derivative $\mscr{L}$ on the field space, along the flow of $X_\xi$:
	\begin{equation}
		\var_\xi
		= \mscr{L}_{X_\xi}
		= \iota_{X_\xi} \var + \var\circ\iota_{X_\xi}
		= X_\xi \cdot \var + \var\circ\iota_{X_\xi}
	\end{equation}
	Here we've used \textit{Cartan's magic formula} for the Lie derivative. Note that $\mscr{L}_{X_\xi}$ is the Lie derivative on the field space $\mcal{F}$, which generally differs from the Lie derivative $\ldv{\xi}$ on the spacetime $M$. 
	
	By now we've written down the above expressions \eqref{eq:symmetry_var_field} covariantly on the off-shell field space $\mcal{F}$, in terms of the redundant coordinates $\phi^I$. However, we can still interpret it as a variation on $\tilde{\mcal{P}}$, by simple restriction. 
	
	In fact, by the definition of a \textit{symmetry} $\xi$, we have the \textbf{off-shell symmetry variation}:
	\begin{empheq}{equation}
		\var_\xi S = X_\xi \cdot \var{S}
		= \int_{\Sigma^\pm} ({\cdots})
	\end{empheq}
	In other words, the action of symmetries in the field space is characterized by the kernel, or \textit{zero modes} (0-modes) of $\var{S}$, \textbf{again up to boundary terms at $\Sigma^\pm$}. 
	
	The reasoning for keeping the boundary terms here is, however, different from the on-shell case, where physically we require $\var{\phi}|_{\Sigma^\pm} = 0$. Here the reason is that we generally \textbf{do allow symmetries to act no-trivially on the in and out states.} For example, time translation evolves the states at $\Sigma^\pm$, so it is natural to assume that it will lead to a boundary term at $\Sigma^\pm$. We shall see that this is indeed the case.
	
	Also, here we \textbf{do allow ``generalized'' symmetries to act no-trivially on the background fields $\phi^B$}, but it has to combine with other variations and altogether reduce to some boundary terms at $\Sigma^\pm$. This is different from the treatment of \textcite{Banados:2016zim}, but it will prove to be convenient when we deal with background gauge fields, such as a background metric. 
	
	Intuitively, the generalized symmetries which act non-trivially on the background fields come from some ``true'' symmetries in a larger system $\hat{S}$, where the fields $\phi^B$ are dynamically \textit{turned on}. A prominent example is again, the static background metric $g^{\mu\nu}$ for a field theory on a curved spacetime. General diffeomorphism is \textit{not} truly a symmetry of the theory, since it disturbs the background metric; however, it is truly a (gauge) symmetry when we turn on dynamic $g^{\mu\nu}$. We will come back to this example shortly. 
	
	\newparagraph
	Compared to the definition of $\tilde{\mcal{P}}$, namely $\var{S}|_{\tilde{\mcal{P}}}$ vanishes up to $\var{\phi^B}$ and $\Sigma^\pm$ terms, we see that for a symmetry $\xi$, $X_\xi$ is always tangent to $\tilde{\mcal{P}}$, so that the restriction of the off-shell $\var_\xi \phi$ to $\tilde{\mcal{P}}$ (on-shell) is well-defined, as expected. 
	
\pagebreak
	
	To finally arrive at Noether's theorem, we note that:
	\begin{equation}
		\var_\xi S|_{\tilde{\mcal{P}}}
		= (X_\xi \cdot \var{S})\big|_{\tilde{\mcal{P}}}
		= X_\xi \cdot (\var{S}|_{\tilde{\mcal{P}}})
		= \int_M E_B \,\var_\xi{\phi^B}
			+ \int_{\Sigma^\pm} ({\cdots})
	\end{equation}
	Namely, the \textbf{on-shell symmetry variation}, which is generally \textbf{non-zero} but vanishes up to $\Sigma^\pm$ terms, can be obtained in two ways: by first computing the off-shell symmetry variation $\var_\xi S$ and then restricting it on shell, or by first computing the on-shell variation $\var{S}|_{\tilde{\mcal{P}}}$ and then inserting the symmetry. We should get the same result either way, which means that:
	\begin{equation}
	\begin{aligned}
		0
		= X_\xi \cdot (\var{S}|_{\tilde{\mcal{P}}})
		- (X_\xi \cdot \var{S})\big|_{\tilde{\mcal{P}}}
		&= \int_M E_B \,\var_\xi{\phi^B}
			+ \int_{\Sigma^\pm} ({\cdots}) \\
		&= \int_M E_B \,\var_\xi{\phi^B}
			+ \pqty{
				\int_{\Sigma^+} - \int_{\Sigma^-}
			}\, ({\cdots}) \\
		&= \int_M E_B \,\var_\xi{\phi^B}
			+ H_\xi^+ - H_\xi^-
	\end{aligned}
	\label{eq:charge_conserve_schematic}
	\end{equation}
	
	We've thus obtained the charge $H_\xi$ associated with $\xi$. Generally, it is conserved when the background fields are unchanged under the variation w.r.t.~$\xi$:
	\begin{equation}
		\var_\xi {\phi^B} = 0
	\end{equation}
	Otherwise the charge $H_\xi$ might change due to the contribution of $\var_\xi {\phi^B}$. Some examples of this are given in \cite{Banados:2016zim}. Intuitively, one can understand this as follows: through the restriction to a subsystem where the field $\phi^B$ is ``external'' and held fixed, we are \textit{explicitly breaking} some symmetries of the larger system. The remaining symmetries are those that act trivially on $\phi^B$, namely the \textit{stabilizer subgroup}, or the \textit{little group} associated with $\phi^B$. 
	
	The above arguments are a bit schematic; now let's write down the $({\cdots})$ explicitly. Let's assume that the off-shell symmetry variation is given by:
	\begin{empheq}{equation}
		\var_\xi S
		= X_\xi \cdot \var{S}
		= \int_M \dd{K_\xi}
		= \int_{\pd M} K_\xi
		= \int_{\Sigma^\pm} K_\xi,
	\quad
		\dd{K_\xi}
		= \var_\xi{L} + \dd{(\var_\xi l)}
	\label{eq:symmetry_var_action}
	\end{empheq}
	Given $\xi$, such $K_\xi$ can be explicitly computed; see \cite{Banados:2016zim,Harlow:2019yfa} for examples. The fact that we can reduce $\var_\xi S$ to an integral on $\Sigma^\pm$ means that the \textbf{symmetry $\xi$ must respect the boundary $N$; this actually leads to constraints on $\xi$} \cite{Harlow:2019yfa}. For example, not all na\"ive diffeo will keep the boundary $N$ invariant. Only those that respect the boundary will be true symmetries. 
	On the other hand, from on-shell variation, we have:
	\begin{equation}
	\begin{aligned}
		X_\xi \cdot (\var{S}|_{\tilde{\mcal{P}}})
		&= \int_M E_I|_{\tilde{\mcal{P}}} \,\var_\xi {\phi^I}
			+ \int_{\pd M} \pqty{
				X_\xi \cdot \Theta + \var_\xi {l}
			}_{\tilde{\mcal{P}}} \\
		&= 0 
			+ \int_M E_B|_{\tilde{\mcal{P}}}
				\,\var_\xi {\phi^B}
			+ \int_{\Sigma^\pm} \pqty{
				X_\xi \cdot \Theta + \var_\xi {l}
			}
	\end{aligned}
	\end{equation}
	
\pagebreak
	
	We can then define the Noether current (form) by:
	\begin{empheq}{equation}
		J_\xi = X_\xi \cdot \Theta
			+ \var_\xi {l} - K_\xi
	\end{empheq}
	With this we can re-write \eqref{eq:charge_conserve_schematic} explicitly as:
	\begin{empheq}{equation}
	\begin{aligned}
		0
		= X_\xi \cdot (\var{S}|_{\tilde{\mcal{P}}})
		- (X_\xi \cdot \var{S})\big|_{\tilde{\mcal{P}}}
		&= \int_M E_B \,\var_\xi{\phi^B}
			+ \int_{\Sigma^\pm} J_\xi \\
		&= \int_M E_B \,\var_\xi{\phi^B}
			+ \pqty{
				\int_{\Sigma^+} - \int_{\Sigma^-}
			}\, J_\xi \\
		&= \int_M E_B \,\var_\xi{\phi^B}
			+ H_\xi^+ - H_\xi^-
	\end{aligned}
	\label{eq:charge_conserve_global}
	\end{empheq}
	
	The above derivation is a proof of the \textit{global} version of Noether's theorem, namely charge conservation from $\Sigma^-$ to $\Sigma^+$. However, there is also a \textit{local} version, namely current conservation. 
	
	To see this, we note that we can always choose $\var_\xi \phi(x)$ to be compactly supported around some point $x_0$, by weighting it with some arbitrary, localized bump function. We can then shrink $M$ to surround the support of $\var_\xi \phi(x)$. This means that \eqref{eq:charge_conserve_global} actually holds around any point $x_0$, which further implies that $\dd{J_\xi} = 0$. This can be proven rigorously via explicit calculation; first we have:
	\begin{equation}
		\dd{J_\xi}
		= X_\xi\cdot \dd{\Theta}
			+ \dd (\var_\xi l - K_\xi)
	\label{eq:current_differential}
\pagebreak[3]
	\end{equation}
	$\mquote{\dd}$ does not act on $X_\xi$, since $X_\xi$ is a vector in the field space $\mcal{F}$ and the spacetime coordinate $x$ in $X_\xi$ are like contracted indices, so there is actually no spacetime dependence; see e.g.~\eqref{eq:symmetry_var_field}. To make this more explicit, we can write:
	\begin{equation}
		\Theta = \Theta[\phi,\var{\phi}],
	\quad
		\Theta_\xi
		= X_\xi \cdot \Theta
		= \Theta[\phi,\var_\xi \phi],
	\quad
		\dd{\Theta_\xi}
		= \dd{(X_\xi \cdot \Theta)}
		= X_\xi \cdot \dd{\Theta}
		= \dd{\Theta[\phi,\var_\xi \phi]}
	\end{equation}
	
	On the other hand, $\mquote{\dd}$ \textit{does} act on $\Theta$, since $\Theta$ is not only a field space 1-form but also a spacetime 1-form; we have $
		\dd{\Theta}
		= \var{L} - E_I \var{\phi^I}
	$ by definition. Therefore,
	\begin{empheq}{equation}
\begin{aligned}
	&\begin{aligned}
		\dd{J_\xi}
		&= X_\xi\cdot \dd{\Theta}
			+ \dd (\var_\xi l - K_\xi) \\
		&= \var_\xi {L} - E_I \,\var_\xi{\phi^I}
			+ \dd (\var_\xi l - K_\xi) \\
		&= - E_I \,\var_\xi{\phi^I}
			+ \pqty\big{
				\var_\xi {L} 
				+ \dd (\var_\xi l - K_\xi)
			}
	\end{aligned}
	\\
	&\begin{aligned}
		\phantom{\dd{J_\xi}}
		&= - E_I \,\var_\xi{\phi^I},
			&&\textsl{\footnotesize%
				by \eqref{eq:symmetry_var_field},
			} \\
		&\leadsto - E_B \,\var_\xi{\phi^B},
			&&\textsl{\footnotesize%
				in $\tilde{\mcal{P}}$,
			} \\
		&\leadsto 0,
			&&\textsl{\footnotesize%
				in $\mcal{P}$.
			} \\
	\end{aligned}
\end{aligned}
	\end{empheq}
	
	We've finally arrived at the current conservation $\dd{J_\xi}|_{\mcal{P}} = 0$, which only holds \textbf{on-shell,} and with background fields suppressed: $\var \phi^B|_{\mcal{P}} = \var_\xi \phi^B|_{\mcal{P}} = 0$. Note that the boundary term $K_\xi$ is crucial in all of the above derivations, so it cannot be dropped casually. Let's go back to the point particle example, and consider time translation $\xi = \pdd{t}$; we have:
	\begin{equation}
		\var_\xi {q^I} = \dot{q}^I,
	\quad
		X_\xi\cdot\Theta
		= p_I \,\var_\xi{q^I}
		= p_I \dot{q}^I,
	\quad
		\dd{K_\xi}
		= \var_\xi{L}
		= \dd{t}\,\var_\xi \mcal{L}
		= \dd{\mcal{L}},
	\quad
		K_\xi
		= \mcal{L}
	\end{equation}
	We see that $K_\xi$ does \textit{not} vanish, and we have precisely $K_\xi = \mcal{L}$. This leads to the conserved current (and charge) associated with $\pdd{t}$, which is precisely the \textbf{Hamiltonian:}
	\begin{equation}
		H = J_\xi
		= X_\xi \cdot \Theta - K_\xi
		= p_I \dot{q}^I - \mcal{L}
	\end{equation}
\subsection{Diffeomorphism in scalar field theory}
	A generalization of the above point particle results is to consider \textit{field theory}; we have:
	\begin{equation}
	\begin{aligned}
		\var{L}
		&= \dd[D]{x} \var{(\sqrt{\abs{g}}\,\mcal{L})} \\
		&= \dd[D]{x} \pqty{
				\sqrt{\abs{g}} \var{\mcal{L}}
				+ \var{(\sqrt{\abs{g}})} \mcal{L}
			} \\
		&= \dd[D]{x} \sqrt{\abs{g}}\,\pqty{
				\var{\mcal{L}}
				- \frac{1}{2}\,
					g_{\mu\nu} \mcal{L}
					\var{g^{\mu\nu}} \!
			} \\
	\end{aligned}
	\label{eq:Lagrangian_form_to_density}
	\end{equation}
	Here we've used the fact that $
		\var{\sqrt{\abs{g}}}
		= \sqrt{\abs{g}}\,\pqty{
			- \frac{1}{2} g_{\mu\nu} \var{g^{\mu\nu}}
		}
	$, which has the same structure as the second term in the Einstein tensor: $
		G_{\mu\nu}
		= R_{\mu\nu} - \frac{1}{2} g_{\mu\nu} R
	$. More explicitly, the measure variation can be computed with the Jacobi's formula\footnote{
		See \wikiref{https://en.wikipedia.org/wiki/Jacobi\%27s\_formula}{Jacobi's formula}. 
	}; note the minus sign when we switch from $\var{g_{\mu\nu}}$ to $\var{g^{\mu\nu}}$:
	\begin{equation}
		\var{\det g}
		= \pqty{\det g}\,g^{\mu\nu} \var{g_{\mu\nu}},
	\quad
		\var{\abs{g}}
		= \abs{g}\,g^{\mu\nu} \var{g_{\mu\nu}}
		= - \abs{g}\,g_{\mu\nu} \var{g^{\mu\nu}}
	\label{eq:measure_variation}
	\end{equation}
	
	Let's now restrict to scalar field $\phi$, where we have:
	\begin{equation}
		\var{\mcal{L}}
		= \pdv{\mcal{L}}{\phi} \var{\phi}
			+ \pdv{\mcal{L}}{(\cdv{\mu} \phi)}
				\var{\cdv{\mu}\phi}
	\end{equation}
\pagebreak[3]%
	We now use the covariant derivative since it commutes with the metric, which will soon prove to be convenient. We now perform the integration by parts:
	\begin{equation}
		\pdv{\mcal{L}}{(\cdv{\mu} \phi)}
			\var{\cdv{\mu}\phi}
		= \cdv{\mu} \pqty{
				\pdv{\mcal{L}}{(\cdv{\mu} \phi)}
				\var{\phi}
			}
		- \pqty{
				\cdv{\mu}
				\pdv{\mcal{L}}{(\cdv{\mu} \phi)}
			} \var{\phi}
	\label{eq:integrate_by_parts}
	\end{equation}
	
	To convert the divergence into differential forms, we note that the $\mquote{\dd}$ action on the coefficents can be realized by \textbf{any torsion-free derivatives}, including the covariant derivative:
	\begin{equation}
	\begin{aligned}
		\dd{(\theta\cdot \sqrt{\abs{g}} \dd[D]{x})}
		&= \dd{\pqty{
			\theta^\mu \sqrt{\abs{g}}
			\ \frac{1}{d!}\,
			\epsilon_{\mu\mu_1\cdots\mu_d}
			\dd{x^{\mu_1}} \wedge
			\cdots \wedge \dd{x^{\mu_d}}
		}},
	\quad d = D-1 \\
		&= \cdv{\sigma} \theta^\mu \sqrt{\abs{g}}
			\ \frac{1}{d!}\,
			\epsilon_{\mu\mu_1\cdots\mu_d}
			\dd{x^\sigma} \wedge
			\dd{x^{\mu_1}} \wedge
			\cdots \wedge \dd{x^{\mu_d}} \\
		&= \cdv{\sigma} \theta^\mu \sqrt{\abs{g}}\,
			\delta^\sigma_\mu \dd[D]{x} \\
		&= \cdv{\mu} \theta^\mu
			\sqrt{\abs{g}} \dd[D]{x}
	\end{aligned}
	\label{eq:divergence_form}
	\end{equation}
	Here $\cdv{\sigma}$ does not act on $\sqrt{\abs{g}}$ due to metric compatibility. This fact can be conveniently re-written with Hodge dual:
	\begin{equation}
		\dd{(\theta\cdot \sqrt{\abs{g}} \dd[D]{x})}
		= \dd{\hstar \theta^\flat}
		= \hstar \cdv{\mu} \theta^\mu
	\end{equation}
	
	With $\theta^\mu = \pi^\mu \var{\phi}$, $
		\pi^\mu = \pdv{\mcal{L}}{(\cdv{\mu} \phi)}
	$, we can now rephrase our familiar identity \eqref{eq:integrate_by_parts} with the new language of differential forms:
	\begin{equation}
		\sqrt{\abs{g}} \dd[D]{x} \pqty{
			\pi \cdot \var{\dd{\phi}}
		}
		= \dd \pqty\big{
				\var{\phi}
				(\hstar \pi^\flat)
			}
		- \var{\phi} \pqty\big{
				\dd{(\hstar \pi^\flat)}
			}
	\label{eq:integrate_by_parts_forms}
	\end{equation}
	In this form, the identity can be proven directly by noting that the right hand side (RHS) is simply $
		(\dd{\var{\phi}}) \wedge
		(\pi\cdot \sqrt{\abs{g}} \dd[D]{x})
	$ by Leibnitz's rule, and the left hand side (LHS) and RHS are identical since $\sqrt{\abs{g}} \dd[D]{x}$ is a top form, and thus $
		\pi \cdot \pqty{
			(\dd{\var{\phi}}) \wedge
			(\sqrt{\abs{g}} \dd[D]{x})
		}
		= \pi\cdot 0
		= 0
	$. 
	
	Finally, we are ready to deploy Stokes' theorem for differential forms. We have:
	\begin{equation}
		E = \hstar \pdv{\mcal{L}}{\phi}
			- \dd{(\hstar \pi^\flat)}
		= \hstar \pqty{
				\pdv{\mcal{L}}{\phi}
				- (-1)^s \hstar \dd{\star \pi^\flat}
			}
		= \hstar \pqty{
				\pdv{\mcal{L}}{\phi}
				- \cdv{\mu}
				\pdv{\mcal{L}}{(\cdv{\mu} \phi)}
			},
	\end{equation}
	\vspace*{-.3\baselineskip}
	\begin{equation}
		\Theta = (\hstar \pi^\flat) \var{\phi}
		= \pi \var{\phi}
			\cdot \sqrt{\abs{g}} \dd[D]{x}
		= \var{\phi} \pdv{\mcal{L}}{(\nabla \phi)}
			\cdot \sqrt{\abs{g}} \dd[D]{x}
		= \theta \cdot \sqrt{\abs{g}} \dd[D]{x}
		= \hstar \theta
	\end{equation}
	As we've seen before, $\Theta$ is to be integrated along $\pd M$, so we mostly care about its projection $\Theta_{\pd M}$ along the induced volume form $\mrm{Vol}_{\pd M}$. We have\footnote{
		We've been cavaliar about the index of $\theta$ and $n$, treating them as a vector $\theta^\mu, n^\mu$ and also as a co-vector $\theta_\mu, n_\mu$, without raising or lowering indices explicitly using musical isomorphisms. Whether it should be a vector or a co-vector can be inferred from the context. 
	}:
	\begin{equation}
		\mrm{Vol}_{\pd M}
		= \hstar n
		= n \cdot \mrm{Vol}_M,
	\quad
		\mrm{Vol}_M
		= n\wedge \hstar n,
%		= n\wedge \mrm{Vol}_{\pd M}
%		= n\wedge (n^\sharp\cdot \mrm{Vol}_M)
	\quad
		n^2 = 1
	\end{equation}
	\vspace*{-1.2\baselineskip}
	\begin{equation}
		\Theta_{\pd M}
		= \theta\cdot n\,\mrm{Vol}_{\pd M}
		= n_\mu \theta^\mu\,
			(n\cdot \sqrt{\abs{g}} \dd[D]{x})
	\end{equation}
	
	Now let's consider diffeo $x\mapsto x-\xi$. We then have:
	\begin{equation}
		\var_\xi {\phi} = \xi \cdot \dd{\phi},
	\quad
		\Theta_\xi
		= X_\xi\cdot\Theta
		= \hstar \pi \,\var_\xi{\phi},
	\end{equation}
	There are various ways to compute $K_\xi$; the ``modern'' way is to work with $L$ directly, which gives:
	\begin{empheq}{equation}
		\dd{K_\xi}
		= \var_\xi{L}
		= \ldv{\xi} L
		= \dd{(\xi\cdot L)},
	\quad
		K_\xi
		= \xi\cdot L
		= \mcal{L}\,(\hstar \xi)
	\label{eq:diffeo_boundary_term}
	\end{empheq}
	
	We then repeat the same calculation with the ``traditional'' language, namely by plugging $\xi$ into \eqref{eq:Lagrangian_form_to_density}; using the same techniques for \eqref{eq:integrate_by_parts_forms}, we demonstrate that it indeed gives the same result:
	\begin{equation}
	\begin{aligned}
%	\textsl{\small And then,}\ \,
		\dd{K_\xi}
		&= \dd[D]{x} \sqrt{\abs{g}}\,\pqty{
				\var_\xi {\mcal{L}}
				- \frac{1}{2}\,
					g_{\mu\nu} \mcal{L}
					\,\var_\xi {g^{\mu\nu}} \!
			} \\
		&= \dd[D]{x} \sqrt{\abs{g}}\,\pqty\big{
				\xi \cdot \dd{\mcal{L}}
				+ \mcal{L} \cdv{\mu} \xi^\mu
			} \\
		&= \hstar \pqty{\xi \cdot \dd{\mcal{L}}}
			+ \mcal{L} \dd{\hstar \xi} \\
		&= \hstar \pqty{\xi \cdot \dd{\mcal{L}}}
			+ \dd{\pqty\big{
				\mcal{L}\,(\hstar \xi)
			}} - \dd{\mcal{L}} \wedge \hstar \xi \\
		&= \dd{\pqty\big{
				\mcal{L}\,(\hstar \xi)
			}}
			+ \xi \cdot \cancel{
				(\dd{\mcal{L}} \wedge \mrm{Vol}_M)
			} \\
		&= \dd{\pqty\big{
				\mcal{L}\,(\hstar \xi)
			}},
	\,\ \textsl{\small indeed}\ \,
		K_\xi = \mcal{L}\,(\hstar \xi) = \xi\cdot L
	\end{aligned}
	\end{equation}
	Here we see explicitly that \textbf{the $\var_\xi {g^{\mu\nu}}$ contribution combines with $\var_\xi {\mcal{L}}$ to give a total derivative.} This is a hint of the \textbf{diff-invariance} of the theory: if we turn on dynamical $g^{\mu\nu}$, then \textbf{any arbitrary $\xi$} (compatible with the boundary condition) is a ``symmetry'' generator of the theory\footnote{
		More precisely, only some of them generate ``true'' symmetry, while others just generate diffeo redundancies. We will come back to this in the future.
	}. 
	
	On the other hand, if $g^{\mu\nu}$ is a fixed background field, then we have additional \textbf{constraints on $\xi$} for it to be a ``true'' symmetry, namely $\var_\xi {g^{\mu\nu}} = 0$ since $\var{g^{\mu\nu}}|_{\mcal{P}} \equiv 0$. 
	This is precisely the \textit{Killing equation}; $\xi$ has to be a \textit{Killing vector}, which generates some \textit{isometry} instead of arbitrary diffeo:
	\begin{equation}
		\var_\xi {g^{\mu\nu}}
		= \ldv{\xi} {g^{\mu\nu}}
		= 0
	\end{equation}
	
	\newparagraph
	Finally, the conserved current for diffeo in scalar field theory is given by:
	\begin{equation}
		J_\xi = \hstar \pi\,(\xi\cdot \dd{\phi})
			- \xi\cdot L
		= \hstar \pqty{
				\pi\,(\xi\cdot \dd{\phi})
				- \xi \mcal{L}
			}
		= \hstar j_\xi,
	\quad
		j_\xi^\mu = \pi^\mu\,(\xi\cdot \dd{\phi})
			- \xi^\mu \mcal{L}
	\end{equation}
	The energy-momentum \textit{stress tensor} $T^{\mu\nu}$ is then found by plugging in $\xi = \sidenote{-}\pdd{\sigma}$, i.e.~the generator of translations; the minus sign is due to our convention for diffeomorphism:
	\begin{equation}
		\xi = -\pdd{\sigma},
	\quad
		x^\mu \mapsto x^\mu - \xi^\mu
		= x^\mu + \delta_\sigma^\mu,
	\end{equation}
	\begin{empheq}{equation}
		T\id{^\mu_\sigma}
		\equiv j^\mu_{\sidenote{-}\pdd{\sigma}},
	\quad
		j^\mu_{-\pdd{\sigma}}
		= T^{\mu\nu} \eta_{\nu\sigma}
		= T^{\mu\nu} (\pdd{\sigma})_\nu,
	\quad
		T^{\mu\nu}
		= -\pi^\mu \nabla^\nu\phi + g^{\mu\nu} \mcal{L}
	\label{eq:Noether_stress_tensor}
	\end{empheq}
	
	Note that for now $T^{\mu\nu}$ is generally \textit{not} symmetric as $\mu\leftrightarrow \nu$, and only $\mu$ is the index for $j^\mu$, while $\nu$ is an index labeling the $D$ translations. The divergence of the current is then given by:
	\begin{equation}
		\xi = -\pdd{\sigma},
	\quad
		\cdv{\mu} j^\mu_\xi
		= \cdv{\mu} T\id{^\mu_{(\sigma)}}
		= \cdv{\mu} \pqty\Big{
				T^{\mu\nu} \pqty{\pdd{(\sigma)}}_{\!\nu}
			}
		= g_{\nu\sigma} \cdv{\mu} T^{\mu\nu}
			- T^{\mu\nu} \cdv{\mu} \xi_\nu
	\end{equation}
	Here we use the notation $(\sigma)$ to protect the $\sigma$ index from the action of $\cdv{\mu}$; namely, in $\cdv{\mu} T\id{^\mu_{(\sigma)}}$ the derivative only acts on the $\mu,\nu$ index, treating it like a (co-)vector, while $\cdv{\mu} T^{\mu\nu}$ is the usual covariant derivative of a 2-tensor. 
	
	Note that if $T^{\mu\nu}$ \textit{were} a symmetric tensor, i.e.~$T^{\mu\nu} = T^{\nu\mu}$, and if $\xi = -\pdd{\sigma}$ is a Killing vector, namely $\cdv{(\mu} \xi_{\nu)} = 0$,
	then the current conservation gets reduced to $\cdv{\mu} T^{\mu\nu}|_{\tilde{\mcal{P}}} = 0$, since $
		\cdv{\mu} j^\mu_\xi
			|_{\tilde{\mcal{P}}}
		= g_{\nu\sigma} \cdv{\mu} T^{\mu\nu}
			|_{\tilde{\mcal{P}}}
		= 0
	$. This motivates us to construct a symmetric stress tensor, which we will discuss in details shortly.
	
	More explicitly, consider free scalar field in Lorentzian signature, then we have:
	\begin{equation}
		\mcal{L}
		= -\frac{1}{2}
			\cdv{\mu} \phi\,
			\nabla^\mu \phi,
	\quad
		T_{\mu\nu}
		= +\cdv{\mu} \phi \,\cdv{\nu} \phi
			+ g_{\mu\nu} \mcal{L}
		= \cdv{\mu} \phi \,\cdv{\nu} \phi
			-\frac{1}{2} g_{\mu\nu}
				\cdv{\rho} \phi\,
				\nabla^\rho \phi,
	\end{equation}
	We've lowered the indices and restored the metric $g_{\mu\nu}$ for future convenience. In this case the stress tensor is symmetric by accident, but this is generally not true; one counter-example is the Noether stress tensor for QED; see e.g.~\cite{Banados:2016zim}. 
\subsection{More on charge conservation}
	\vspace{-1\baselineskip}
	\begin{empheq}{equation*}
		\textsl{\textbf{Note:} for simplicity, quantities in this sub-section are taken to be \textbf{on-shell} by default. }
	\end{empheq}
	
	We now define the charge on a general codim-1 $\Sigma$ by integrating the \textbf{on-shell} current:
	\begin{equation}
		H_\xi(\Sigma)
		= \int_\Sigma J_\xi,
	\quad
		J_\xi = J_\xi|_{\tilde{\mcal{P}}}
	\end{equation}
	Charge conservation is then formulated as follows: we consider a deformation of $\Sigma$ along the flow of $\alpha$, and examine its effects $\Delta_{\alpha} H_\xi(\Sigma)$; we have:
	\begin{equation}
	\begin{aligned}
		\Delta_{\alpha} H_\xi(\Sigma)
		&= \int_\Sigma
			\ldv{\alpha} J_\xi,
	\quad
		\ldv{\alpha} J_\xi
		= \dd{\pqty\big{
				\alpha \cdot J_\xi
			}}
			+ \alpha \cdot \cancel{\dd{J_\xi}}, \\
		&= \int_{\pd\Sigma}
			\alpha \cdot J_\xi,
	\quad J_\xi = \hstar j_\xi, \\
		&= \int_{\pd\Sigma}
			\alpha \cdot \pqty{\mspace{2mu}
				j_\xi \cdot \mrm{Vol}_M
			},
	\quad
		\iota_\alpha \iota_{j_\xi}
		= -\iota_{j_\xi} \iota_\alpha, \\
		&= - \int_{\pd\Sigma}
			j_\xi \cdot \pqty{\mspace{2mu}
				\alpha \cdot \mrm{Vol}_M
			},
	\end{aligned}
	\end{equation}
	\begin{empheq}{equation}
		\Delta_{\alpha} H_\xi(\Sigma)
		= - \int_{\pd\Sigma}
			j_\xi \cdot \pqty{\mspace{2mu}
				\alpha \cdot \mrm{Vol}_M
			}
	\end{empheq}
	
	This result is very intuitive: the change of charge $H_\xi$ on the codim-1 slice $\Sigma$ is given by the flux entering the codim-2 boundary surface $\pd\Sigma$. 
	To see this more explicitly, consider a non-vanishing $\alpha$ normal to $\Sigma$; we can then decompose the volume form along $\pd\Sigma$:
	\begin{equation}
		\alpha^2 \mrm{Vol}_M|_{\pd\Sigma}
		= \alpha\wedge n_{(\pd\Sigma)}
			\wedge \pqty{
				n_{(\pd\Sigma)}
				\cdot \alpha
				\cdot \mrm{Vol}_M
			}
		= \alpha\wedge n_{(\pd\Sigma)}
			\wedge \abs{\alpha}\,\mrm{Vol}_{\pd\Sigma},
	\quad
		n_{(\pd\Sigma)} \perp \alpha
		\perp \Sigma
	\end{equation}
	Here we do not demand that $\alpha$ has unit norm; this leads to the modified volume form $\abs{\alpha}\,\mrm{Vol}_{\pd\Sigma}$ on $\pd\Sigma$. 
%	On the other hand, for convenience, we can always choose $n_{(\pd\Sigma)}$ such that it's orthogonal to $\alpha$ along $\pd \Sigma$. 
	With such decomposition,
	\begin{equation}
	\begin{aligned}
		\Delta_{\alpha} H_\xi(\Sigma)
		&= - \int_{\pd\Sigma}
			j_\xi \cdot \pqty{\mspace{2mu}
				\alpha \cdot \mrm{Vol}_M
			} \\
		&= - \int_{\pd\Sigma}
			j_\xi \cdot \pqty{
				n_{(\pd\Sigma)} \wedge 
				\abs{\alpha}\,\mrm{Vol}_{\pd\Sigma}
			} \\
		&= - \int_{\pd\Sigma} \abs{\alpha}\,
			j_\xi \cdot n_{(\pd\Sigma)}
			\mrm{Vol}_{\pd\Sigma}
	\end{aligned}
	\end{equation}
	
	In particular, if $\Sigma$ is a constant time slice orthogonal to $\alpha = \pdd{t}$, then:
	\begin{equation}
		\pdd{t} \perp \Sigma,
	\quad
		\dv{t} H_\xi
		= \int_\Sigma \dv{t} J_\xi
		= - \int_\Sigma
			\pqty{\cdv{i}\,j^i_\xi}\,
			\pqty{\mspace{2mu}
				\pdd{t} \cdot \mrm{Vol}_M
			}
		= - \int_{\pd\Sigma} \abs{\pdd{t}}\,
			j_\xi \cdot n_{(\pd\Sigma)}
			\mrm{Vol}_{\pd\Sigma}
	\end{equation}
	On the other hand, if $\pd\Sigma = \varnothing$, then $\Delta_{\alpha} H_\xi(\Sigma) \equiv 0$ for any choice of $\alpha$. This means that $H_\xi(\Sigma)$ is insensitive to small deformations, i.e.~it's a codim-1 \textbf{topological operator}. 
	
\pagebreak[3]
	
	For the free scalar field theory in flat sapcetime, with $\xi^\mu = -\delta^\mu_\nu$, we have:
	\begin{equation}
		P_\nu \equiv H_{\pd_\nu}
		= \int_\Sigma \dd[d]{x} T\id{^0_\nu},
	\quad
		T\id{^0_\nu}
		= T^{0\mu} \eta_{\mu\nu}
	\end{equation}
	The minus sign we've introduced in \eqref{eq:Noether_stress_tensor} guarantees that \textbf{the conserved charge with upper indices is the usual energy-momentum 4-vector, and $T^{0\nu}$ is precisely the energy momentum density:}
	\begin{equation}
		P^\nu = \int_\Sigma \dd[d]{x} T^{0\nu}
	\end{equation}
	
	
%	It is generally safe to assume that the integrand decays fast enough when it approaches $\Sigma^\pm$, so we can drop the contribution from $\Sigma^\pm$. 
	
	
%	We can then safely assume the boundary integral vanishes at $\Sigma^\pm$, thus:
%	\begin{equation}
%		0 = \int_N \pqty{
%				X_\xi \cdot \Theta + \var_\xi {l}
%			},
%	\end{equation}
	
\subsection{Improvement of the stress tensor}
	We've noted above that the stress tensor obtained from the standard Noether's procedure is generally \textit{not} symmetric. In general, we can \textit{improve} the stress tensor by making use of an \textbf{ambiguity in the definition of the Noether current $J_\xi$}; note that:
	\begin{equation}
		\Theta \mapsto \Theta + \dd{Y},
	\quad
		J_\xi \mapsto J_\xi + X_\xi \cdot \dd{Y}
		= J_\xi + \dd{Y_\xi},
	\quad
		\dd{\pqty{
			J_\xi + \dd{Y_\xi}
		}} = \dd{J_\xi} = 0
	\end{equation}
	Here we've commuted $\mquote{\dd}$ past $X_\xi$, using the same arguments as in \eqref{eq:current_differential}. 
	We see that current conservation still holds if we add an arbitrary total derivative to $J_\xi$. 
	As we shall see later, this $Y$-ambiguity actually has non-trivial effects in the phase space; picking the ``canonical'' $Y$ then becomes a delicate issue. 
	
	However, for the stress tensor $T_{\mu\nu}$ in a \textbf{diff-invariant theory}, there is a natural improvement for the current; here the \textbf{diff-invariance of a theory is \textit{defined} through \eqref{eq:diffeo_boundary_term}}: given an arbitrary vector field $\xi$, the theory is said to be diff-invariant iff.~\cite{Harlow:2019yfa,Compere:2018aar}:
	\begin{empheq}{equation}
		\var_\xi \phi^I
		= \ldv{\xi} \phi^I,
	\quad
		\var_\xi L
		= \dd{K_\xi},
	\quad
		K_\xi
		= \xi \cdot L
	\end{empheq}
	Technically we should also require that $\xi$ generates a diffeomorphism that respects the boundary conditions \cite{Harlow:2019yfa}, but we will ignore this subtlety for now. 
	
	For a diff-invariant theory, one can then define the \textit{Hilbert stress tensor} by varying the action w.r.t.~the metric $g_{\mu\nu}$, or equivalently, the metric inverse $g^{\mu\nu}$:
	\begin{empheq}{equation}
		T_{\mu\nu}
		= - \frac{2}{\sqrt{-g}}
			\fdv{S_L}{g^{\mu\nu}},
	\quad
		\var{S_L}
		= \int \sqrt{-g} \dd[D]{x} 
			\pqty{
				-\frac{1}{2}\, T_{\mu\nu}
				\var{g^{\mu\nu}}
				+ \cdots
			}
	\end{empheq}
	Again this has the same structure as the second term in the Einstein tensor: $
		G_{\mu\nu}
		= R_{\mu\nu} - \frac{1}{2} g_{\mu\nu} R
	$. 
	Similar to \eqref{eq:measure_variation}, we have:
	\begin{empheq}{equation}
		- \frac{1}{2}\, T_{\mu\nu}
			\var{g^{\mu\nu}}
		= + \frac{1}{2}\, T^{\mu\nu}
			\var{g_{\mu\nu}},
	\quad
		T^{\mu\nu}
		= + \frac{2}{\sqrt{-g}}
			\fdv{S_L}{g_{\mu\nu}}
	\end{empheq}
	
	We see an immediate consequence of this definition: if the metric $g_{\mu\nu}$ in a theory is dynamical, then the EoM for $g_{\mu\nu}$ is precisely given by:
	\begin{equation}
		T_{\mu\nu} \equiv E_{\mu\nu} = 0
	\end{equation}
\pagebreak[3]%
	Namely, the \textbf{\textit{total} stress tensor} must vanish. This is what happens in gravity; examples include string theory, where $T_{ab} = 0$ is the so-called \textit{Virasoro constraint} on the worldsheet, and Einstein gravity, where we have the Einstein equations:
	\begin{equation}
		0 = E_{\mu\nu}
		= T^M_{\mu\nu}
			- \frac{1}{8\pi G_N}\,G_{\mu\nu}
	\end{equation}
	Here $T^M_{\mu\nu}$ is the \textit{matter} stress tensor. We now have a new puzzle: since the total $T_{\mu\nu} = E_{\mu\nu} = 0$ in gravity, how should we define the stress tensor for the gravitational field itself? As we shall see later, the covariant formalism provides a neat solution for this problem. \textbf{Also, from now on we will use $T_{\mu\nu} = T^M_{\mu\nu}$ to denote the non-vanishing \textit{matter} stress tensor.} 
	
	\newparagraph
	On the other hand, given a fixed background $\hat{g}_{\mu\nu}$, we have another subtlety in the above definition: there could be various ways to promote $
		\hat{g}_{\mu\nu} \mapsto
		g_{\mu\nu}
	$ and vary w.r.t.\ $g_{\mu\nu}$. In particular, if $\hat{g}_{\mu\nu} = \eta_{\mu\nu}$, then the Ricci scalar $R \equiv 0$, thus any term that is proportional to $R$ can be added to the action without spoiling the diff-invariance. 
	
%\pagebreak[4]
	
	One can think of this process as \textit{turning on} a dynamic $g_{\mu\nu}$ field coupled to the matter theory; there is the usual \textbf{minimal coupling}, i.e.~we replace $\eta_{\mu\nu} \mapsto g_{\mu\nu}$, $\pdd{\mu} \mapsto \cdv{\mu}$, and then compute the variation. In this case the matter action is cleanly separated from the gravitational kinetic terms: $S = S^M[\phi^I,g_{\mu\nu}] + S^G[g_{\mu\nu}]$, and there is a clear definition of $T^M_{\mu\nu}$ from $S^M[\phi^I,g_{\mu\nu}]$.
	
	However, we can introduce additional couplings between the matter fields and $g_{\mu\nu}$, e.g.~the dilaton coupling \cite{Polchinski:1998rq}:
	\begin{equation}
%		S \propto \int
			\sqrt{\abs{g}} \dd[D]{x}
			\pqty{R - 2\Lambda}\,\Phi[\phi^I]
	\end{equation}
	Where the dilaton $\Phi = \Phi[\phi^I]$ is a functional of the fields. This term vanishes in flat space, yet it contributes extra improvement terms to the stress tensor. This is an example of \textit{non-minimal} coupling. In this case the gravitational action is mixed with the matter action, and we don't have an explicit formula for $T^M_{\mu\nu}$; but usually it's still possible to split the total (vanishing) stress tensor into two parts:
	\begin{equation}
		0 = E_{\mu\nu}
		= T^M_{\mu\nu}
			- \mcal{G}_{\mu\nu}
	\end{equation}
	Here $\mcal{G}_{\mu\nu}$ is the gravitational contribution. For Einstein gravity, we have $
		\mcal{G}_{\mu\nu}
		= \frac{1}{8\pi G_N}\,G_{\mu\nu}
	$. For dilaton coupling, we have
	
	
	\newparagraph
	To understand the equivalence of the Hilbert stress tensor with the Noether stress tensor, we shall revisit the idea that \textbf{Noether's theorem is in fact a local statement,} even when the symmetries involved are the so-called ``global'' symmetries. 
\subsection{Noether current from ``localized'' variation}
	As we've noted before, we can always choose $\var_\epsilon \phi(x)$ to be compactly supported around some point $x_0$, by weighting it with some arbitrary, localized bump function $\epsilon(x)$. One can think of this as promoting\footnote{
		This process is described in \textcite{Polchinski:1998rq}, Section 2.3 and Exercise 2.5, and a solution of this exercise is given by \cite{Headrick:2008ke}. We will reproduce the solution here with the language of differential forms.
	}:
	\begin{equation}
		\var_\xi \phi
		\leadsto \var_\epsilon \phi
		= \epsilon(x)\,\var_\xi \phi,
	\quad\text{i.e.}\ \,
		X_\xi \cdot \var{\phi}
		\leadsto X_\epsilon \cdot \var{\phi}
		= \epsilon(x)\,X_\xi \cdot \var{\phi}
	\end{equation}
	Here we are considering general $\xi,\eta$ which act locally on the fields; they don't have to be diffeos for this to hold. 
	
	We see that \textbf{when contracted with $\var{\phi}$, we can simply replace $
		X_\epsilon \leadsto \epsilon(x)\,X_\xi
	$}. 
	\textbf{However, note that this does \textit{not} imply that $
		X_\epsilon = \epsilon(x)\, X_\xi 
	$}, 
\pagebreak[3]
	nor does it work for contractions with general variations such as $\var{L}$. 
	In fact, $X_\epsilon$ is a flow in the field space $\mcal{F}$, so it should not have explicit dependence on the $x$ coordinate; see e.g.~\eqref{eq:symmetry_var_field}. By definition, the correct form of $X_\epsilon$ should be:
	\begin{equation}
		X_\epsilon
		= \int_M \dd[D]{x}
			\epsilon(x)\,
			\pqty\big{
				X_\xi \cdot \var{\phi^I(x)}
			}
			\fdv{\phi^I(x)}
	\end{equation}
	
	We then look at the variation $\var_\epsilon L$; by definition,
	\begin{equation}
	\begin{aligned}
		\var_\epsilon L
		&= X_\epsilon \cdot \var{L} \\
		&= \epsilon\,E_I\,\var_\xi {\phi^I}
			+ \dd{(X_\epsilon \cdot \Theta)} \\
		&= \epsilon\,E_I\,\var_\xi {\phi^I}
			+ \dd{\pqty{
				\epsilon(x)\, X_\xi \cdot \Theta
			}} \\
		&= \epsilon\,E_I\,\var_\xi {\phi^I}
			+ \epsilon \dd{(X_\xi \cdot \Theta)}
			+ (\dd{\epsilon}) \wedge
				X_\xi \cdot \Theta \\
		&= \epsilon\,\var_\xi L
			+ (\dd{\epsilon}) \wedge
				X_\xi \cdot \Theta \\
	\end{aligned}
	\end{equation}
	On the other hand, we know that $\var_\xi L$ is a symmetry variation: $\var_\xi L = \dd{(K_\xi - \var_\xi l)}$; we thus have:
	\begin{empheq}{equation}
	\begin{aligned}
		\var_\epsilon L
		&= \epsilon\,\var_\xi L
			+ (\dd{\epsilon}) \wedge
				X_\xi \cdot \Theta \\
		&= \epsilon \dd{(K_\xi - \var_\xi l)}
			+ (\dd{\epsilon}) \wedge
				X_\xi \cdot \Theta \\
		&= \dd{\pqty\big{
				\epsilon(K_\xi - \var_\xi l)
			}} 
			+ (\dd{\epsilon}) \wedge \pqty{
				X_\xi \cdot \Theta
				+ \var_\xi l - K_\xi
			} \\
		&= \dd{\pqty\big{
				\epsilon(K_\xi - \var_\xi l)
			}} 
			+ (\dd{\epsilon}) \wedge J_\xi
	\end{aligned}
	\label{eq:localized_variation}
	\end{empheq}
	
	We see that we've recovered the Noether current $J_\xi$ with this formalism. 
	Our calculation thus far has been \textbf{off-shell;} we see that while $\xi$ is a symmetry by construction: $
		\var_\xi L = \dd{(K_\xi - \var_\xi l)}
	$, \textbf{$\eta$~is generally \textit{not} a symmetry of the action:} by definition, we require an \textbf{off-shell} symmetry variation of the action to be a total derivative; this is not the case for $\var_\epsilon L$, as we have an extra term $
		(\dd{\epsilon}) \wedge J_\xi
	$ which does \textit{not} vanish \textit{off-shell}. 
	
	We can integrate by parts once more and get:
	\begin{empheq}{equation}
	\begin{aligned}
		\var_\epsilon L
		&= \dd{\pqty\big{
				\epsilon(K_\xi - \var_\xi l)
			}} 
			+ (\dd{\epsilon}) \wedge J_\xi \\
		&= \dd{\pqty\big{
				\epsilon X_\xi \cdot \Theta
			}} 
			- \epsilon \dd{J_\xi} \\
		&= \dd{\Theta_\epsilon} 
			- \epsilon \dd{J_\xi}
	\end{aligned}
	\end{empheq}
	Current conservation follows from the fact that since we pick $\epsilon(x)$ to be compactly supported around some point $x_0$, the \textbf{on-shell variation} of the action vanishes completely, as long as $\pd M$ encloses the support of $\epsilon(x)$; his holds for \textit{arbitrary} $\epsilon(x)$ supported within $M$, which guarantees that $\dd{J_\xi}|_{\tilde{\mcal{P}}} = 0$:
	\begin{empheq}{equation}
	\begin{aligned}
		0 = \var_\epsilon S|_{\tilde{\mcal{P}}}
		= \int_M \var_\epsilon L|_{\tilde{\mcal{P}}}
		&= \cancel{
			\int_{\pd M} \epsilon(x)\,
				X_\xi \cdot \Theta|_{\tilde{\mcal{P}}}
			}
			- \int_M \epsilon(x) \wedge
				\dd{J_\xi}|_{\tilde{\mcal{P}}},
	\quad
		\dd{J_\xi}|_{\tilde{\mcal{P}}} = 0
		\mspace{-16mu}
	\end{aligned}
	\end{empheq}
	
	The on-shell condition is necessary to guarantee that $
		\var_\epsilon S|_{\tilde{\mcal{P}}} = 0
	$, and any possible boundary terms at $\Sigma^\pm$ vanishes since it's weighted by $\epsilon(x)$. Alternatively, we can see this locally by noting that:
	\begin{equation}
		\dd{\Theta_\epsilon}
			+ \epsilon\, E_I \,\var_\xi \phi^I
		= \var_\epsilon L
		= \dd{\Theta_\epsilon} 
			- \epsilon \dd{J_\xi}
	\end{equation}
	We see that the boundary term $\dd{\Theta_\epsilon}$ actually cancels between the left and right-hand side, therefore even we choose some $\epsilon(x)$ which does not vanish along $\pd M$, we still have $
		\dd{J_\xi} = -E_I \,\var_\xi \phi^I
	$ which vanishes on-shell, as we've seen before. 
	
	\newparagraph
	This formalism of promoting $
		\var_\xi \phi
		\leadsto \var_\epsilon \phi
		= \epsilon\,\var_\xi \phi
	$ suggests a convenient method of finding the Noether current $J_\xi$, without worrying about the boundary term $K_\xi$; we need only compute $\var_\epsilon L$, throw away the total derivatives, and arrange the final result as some factors of $\dd{\epsilon}$, which will automatically include $K_\xi$ contributions and give us a conserved $J_\xi$. 
	
	In fact, we don't even need to impose $
		\var_\epsilon \phi
		= \epsilon(x)\,\var_\xi \phi
	$; we need only require that $\var_\epsilon \phi$ has the following properties: it's compactly supported around some point $x_0$, within the support of $\epsilon(x)$, and with $\epsilon \to 1$ it reduces to the usual symmetry variation $\var_\xi \phi$:
	\begin{empheq}{equation}
		\var_\epsilon \phi
		= \var_\epsilon \phi_{\,[\epsilon(x)]},
	\quad
		\var_\epsilon \phi_{\,[\epsilon(x) = 1]}
		= \var_\xi \phi
	\label{eq:gauging_conditions}
	\end{empheq}
	We follow the arguments of \textcite{Polchinski:1998rq}: consider a ``Taylor expansion'' in the functional space around the ``point'' $\epsilon(x) = 1$; up to total derivative terms, $\var_\epsilon S$ would vanish if $\epsilon$ were a constant, so its variation must be proportional to the gradient $\dd{\epsilon}$,
	\begin{empheq}{equation}
		\var_\epsilon L
		= \dd{\pqty\big{
				\epsilon(K_\xi - \var_\xi l)
			}} 
			+ (\dd{\epsilon}) \wedge J_\xi
			+ \order{\epsilon^2}
	\end{empheq}
	The total derivative terms must be proportional to $\epsilon(x)$, since it comes from the constant piece (0-mode) of $\epsilon(x)$, and its contribution is simply proportional to the $\epsilon(x) = 1$ contribution. 
	
	We see that we've recover the result of \eqref{eq:localized_variation} just from basic analysis, without assuming the explicit form of $\var_\xi \phi$. The rest of the arguments that leads to $\dd{J_\xi}|_{\mcal{\tilde{P}}} = 0$ follows through with no issues. We see that as long as it satisfies some basic conditions \eqref{eq:gauging_conditions}, how we promote $\var_\xi \phi \leadsto \var_\epsilon \phi$ doesn't actually matter; in the end we will get a valid conserved current $J_\xi$. 
	
	This result comes in handy, as the promotion of $\var_\xi \phi \leadsto \var_\epsilon \phi$ can be taken to be the usual ``gauging'' process that promotes a global symmetry to a local symmetry, by adding coordinates dependence to the symmetry parameter: $\xi \leadsto \epsilon(x)$. 
	
	\newparagraph
	There will still be ambiguities as one can shift $\Theta$ by a total derivative $\dd{Y}$, as we've seen before. In fact it's highly possible that this $\dd{\epsilon}$ procedure described here may produce a different current, compared with the conventional Noether's procedure; we have:
	\begin{equation}
		\dd{\epsilon} \wedge
			\pqty{J_\xi + \dd{Y_\xi}}
		= \dd{\epsilon} \wedge J_\xi
			+ \dd{\pqty{
				\epsilon \dd{Y_\xi}
			}}
	\end{equation}
	Indeed the coefficient of $\dd{\epsilon}$ is only defined up to a total derivative. 
	This actually provides a way to connect the Hilbert and Noether stress tensor. 
	
\subsection{Derivation for the Hilbert stress tensor}
	Again consider $\xi^\mu = \epsilon^\mu$, i.e.~the generator for translations in flat spacetime; we now promote it to some localized diffeomorphism:
	\begin{equation}
		x\mapsto x - \xi \leadsto x - \epsilon(x),
	\quad
		\xi^\mu = \epsilon^\mu
		\leadsto \epsilon^\mu(x),
	\quad
		\var_\xi \phi
		\leadsto \var_\epsilon \phi
		= \ldv{\epsilon} \phi
	\end{equation}
	
	
%\legacyReference
	
%	\begin{figure}[!ht]
%	\centering
%%	\includegraphics[width=.6\linewidth]{img/}
%%	\hspace{2em}
%%	\vspace{-2ex}
%	\caption[]{}
%	\end{figure}
	
	
\vspace{1.2\baselineskip}
\pagebreak[4]
\raggedright
\printbibliography[%
%	title = {参考文献} %
	,heading = bibintoc
]
\end{document}
