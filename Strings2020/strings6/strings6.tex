% !TeX document-id = {39d24ce0-844d-44a8-943b-2a9698211621}
% !TeX encoding = UTF-8
% !TeX spellcheck = en_US
% !TeX TXS-program:bibliography = biber -l zh__pinyin --output-safechars %
% !TeX TS-program = xelatex
%%% LuaLaTeX is required for `tikz-feynman`

\documentclass[a4paper,10pt]{article}

\newcommand{\hwNumber}{6}

% Templates: 82ccb576e4df24e5eac4194b76230be360b4f733

% to be `\input` in subfolders,
% ... therefore the path should be relative to subfolders.

\usepackage[UTF8
	,heading=false
	,scheme=plain % English Document
]{ctex}

\input{../.modules/basics/macros.tex}
\input{../.modules/preamble_base.tex}
\input{../.modules/preamble_notes.tex}

\newcommand{\legacyReference}{{
	\clearpage\par
	\quad\clearpage
	\renewcommand{\midquote}{\textbf{PAST WORK, AS TEMPLATE}}
	\newparagraph
}}

% Settings
\counterwithout{equation}{section}
\mathtoolsset{showonlyrefs=false}
%\DeclareTextFontCommand{\textbf}{\sffamily}
\renewcommand{\midquote}{\quad}

% Spacing
\geometry{footnotesep=2\baselineskip} % pre footnote split
\setlength{\parskip}{.5\baselineskip}
\renewcommand{\baselinestretch}{1.15}

%Title
	\posttitle{
		\hfill\Large\ccbyncsajp
		\par\end{flushleft}%
		\vspace*{-.7ex}\hrule%
	}
	\preauthor{\vspace{-1.5ex}%
		\flushleft\itshape%
	}
	\postauthor{\hfill}
	\predate{\noindent\ttfamily Compiled @ }
	\postdate{\vspace{.5ex}}

	\title{String Theory \textnumero\hwNumber}
	\author{\signature Bryan}
	\date{\today}

% List
	\setlist*{
		listparindent=\parindent
		,labelindent=\parindent
		,parsep=\parskip
		,itemsep=1.2\parskip
		,leftmargin=0pt
		,itemindent=*
	}
	\setlist*[enumerate,1]{
		align=left
		,label=\fbox{\textbf{\arabic*}}
		,itemsep=.5\baselineskip
		,itemindent=*
	}

\input{../.modules/basics/biblatex.tex}

%%% ID: sensitive, do NOT publish!
\InputIfFileExists{../id.tex}{}{}

% NS & R with spacing
\newcommand{\NS}{\ensuremath{\mspace{2mu}\mrm{NS}}}
\newcommand{\R}{\ensuremath{\mspace{2mu}\mrm{R}}}
% representations
\newcommand{\mbf}[1]{\mathbf{#1}}

\addbibresource{strings6.bib}

\begin{document}
\maketitle
\pagestyle{headings}
\pagenumbering{arabic}
\thispagestyle{empty}

	\begin{enumerate}
	
	\item \textbf{Type 0 Superstrings}
	
	A closed superstring theory consists of sectors labeled by the boundary conditions $(-1)^\alpha$ of $(\psi,\tilde{\psi})$ along with suitable GSO projections $(-1)^F = \pm 1$. Here we follow the discussions of \textit{Polchinski}, with $\mrm{R}\colon \alpha = 1$ and $\mrm{NS}\colon \alpha = 0$. 
	
	There are also some consistency conditions: by modular invariance, there must be at least one left-moving R sector and at least one right-moving R sector; on the other hand, the OPE must close, and since $\mrm{R} \times \mrm{R} = \mrm{NS}$ there must be some corresponding NS sector for each R sector. 
	
	If we include only the (NS, NS) and the (R, R) sectors, then both must exist due to the above conditions. In fact, closure of OPE implies that the $(\NS+, \NS+)$ sector must exist. In addition, NS$-$ sector must be paired with another NS$-$ sector due to the level matching condition of the closed string, i.e. it is possible (but not required) to have a $(\NS-, \NS-)$ sector. 
	
	All possibilities can then be generated by enumerating all possible (R, R) sectors (there are $2\times 2 = 4$ of them), while applying an extra consistency check that all pairs of vertex operators $O_1,O_2$ are mutually local, i.e.
	\begin{equation}
		\exp i\pi \pqty{
				  F_1 \alpha_2
				- F_2 \alpha_1
				- \tilde{F}_1 \tilde{\alpha}_2
				+ \tilde{F}_2 \tilde{\alpha}_1
			}
		= 1
	\end{equation}
	If $O_1 \in (\,\NS+, \NS+)$, then we have $\alpha_1 = \tilde{\alpha}_1 = 0 = F_1 = \tilde{F}_1$, hence the above factor is always trivial; for $O_1 \in (\R,\R)$, however, $\alpha_1 = \tilde{\alpha}_1 = 1$, which yields a non-trivial constraint for the second operator: $
		F_2 - \tilde{F}_2
		= F_1 \alpha_2 - \tilde{F}_1 \tilde{\alpha}_2
		= \alpha_2 \pqty\big{F_1 - \tilde{F}_1}
		\ (\mop{mod} 2)
	$, assuming $\alpha_2 = \tilde{\alpha}_2$. With $\alpha_2 = 0$ this gives $F_2 = \tilde{F}_2$, and with $\alpha_2 = 1$ this gives $
		F_2 - \tilde{F}_2
		= F_1 - \tilde{F}_1
	$; this means that all (R, R) sectors have the same sign difference between $F$ and $\tilde{F}$. The possible solutions can then be narrowed down to:
	\begin{gather}
		\mrm{0A}\colon\ %
			(\NS+,\NS+),\ %
			(\NS−,\NS−),\ %
			(\R+,\R-),\ %
			(\R−,\R+), \\
		\mrm{0B}\colon\ %
			(\NS+,\NS+),\ %
			(\NS−,\NS−),\ %
			(\R+,\R+),\ %
			(\R−,\R−), \\
		\text{And additionally, $(\NS+,\NS+)$ with any \textit{single one} of the 4 possible (R, R) sectors.}
		\label{eq:type0_deviant}
	\end{gather}
	
	If there are two (R, R) sectors, then there must be an accompanying $(\NS−,\NS−)$ sector due to the closure of OPE. It is straightforward to check that these possibilities are all valid under the above constraints: (0) level matching of closed strings, (1) mutual locality, (2) closure of OPE, and (3) (apparent) modular invariance (not sufficient yet, to be checked below). 
	
	\begin{enumerate}
	\item The torus partition function of the theory breaks up into a product of independent sums over the bosonic $X$ and fermionic $(\psi,\tilde{\psi})$ oscillators. The bosonic part is identical to the bosonic string situation, therefore modular invariant; to check the total partition function for modular invariance, we will look at the fermionic contributions $Z = Z_{\psi,\tilde{\psi}}$ explicitly. 
	
	Similar to the Type II case, the building block of $Z$ is given by:
	\begin{equation}
		Z\id{^\alpha_\beta}
		= \Tr_\alpha \bqty{
				(-1)^{\beta F} q^H
			},\quad
		q = e^{2\pi i\tau}
	\end{equation}
	Where $\alpha,\beta$ labels the periodicity in the spatial and temporal directions ($\sigma^1,\sigma^2$); note that for fermionic fields, anti-periodicity in the time direction gives the simple trace, while the periodic path integral gives the trace weighted by $(−1)^F̂$, as is explained in \textit{Polchinski}, Appendix A. 
	
	In 10\,D, $\mu = 1,\cdots,10$, in total there are $
		N = 10 - 2 = 4\times 2 = 8
	$ real, \textit{transverse} spinor components in $(\psi^\mu,\tilde{\psi}^\mu)$; pairing them into complex chiral spinors like $\psi^1 \pm \psi^2$, each one of them contributes a factor of $Z\id{^\alpha_0}$ in the total partition function. 
	
	Note that for type II theories, the boundary conditions and GSO projections $(\alpha,F)$ for the left and right movers are ``decoupled''; any possible $(\alpha,F)$ can be paired with any possible $(\tilde{\alpha},\tilde{F})$, hence the left and right contributions can be calculated separately. For type 0 theories, however, the left and right $(\alpha,F)$'s are coupled, hence we have to calculate their contributions together. With the above considerations, we have:
	\begin{equation}
	\renewcommand{\NS}{\mrm{NS}} % no spacing
	\renewcommand{\R}{\mrm{R}}   % no spacing
	\begin{aligned}
		\Tr_{(\NS,\NS)} \bqty{
				\frac{1 + (-1)^{F-\tilde{F}}}{2}\,
				q^H
			}
		&= \frac{1}{2}\, \Bqty\Big{
				\Tr_{(\NS,\NS)} q^H
				+ \Tr_{(\NS,\NS)}
					\bqty{(-1)^{F-\tilde{F}} q^H}
			} \\
		&= \frac{1}{2}\, \Bqty{
				\abs{(Z\id{^0_0})^{N/2}}^2
				+ \abs{(Z\id{^0_1})^{N/2}}^2
			} \\
		&= \frac{1}{2}\, \Bqty{
				\abs{Z\id{^0_0}}^N
				+ \abs{Z\id{^0_1}}^N
			},
	\\[2ex]
		\Tr_{(\R,\R)} \bqty{
				\frac{1 \mp (-1)^{F-\tilde{F}}}{2}\,
				q^H
			}
		&= \frac{1}{2}\, \Bqty\Big{
				\Tr_{(\R,\R)} q^H
				\mp \Tr_{(\R,\R)}
					\bqty{(-1)^{F-\tilde{F}} q^H}
			} \\
		&= \frac{1}{2}\, \Bqty{
				\abs{Z\id{^1_0}}^N
				\mp \abs{Z\id{^1_1}}^N
			},
	\end{aligned}
	\end{equation}
	\begin{equation}
		Z^{\mrm{0A|B}}
		= \frac{1}{2}\, \Bqty{
				\abs{Z\id{^0_0}}^N
				+ \abs{Z\id{^0_1}}^N
				+ \abs{Z\id{^1_0}}^N
				\mp \abs{Z\id{^1_1}}^N
			}
	\end{equation}
	Similarly, for the situation in \eqref{eq:type0_deviant} with no $(\NS-,\NS-)$ sector, 
	depending on the GSO projections $(F,\tilde{F})$ in the single (R, R) sector, we have:
	\begin{equation}
	\let\bar\overline
	\footnotesize
	\begin{aligned}
		Z'
		&= \abs{\frac{1}{2}\,\pqty{
				(Z\id{^0_0})^{N/2}
				+ (Z\id{^0_1})^{N/2}
			}}^2
			+ \frac{1}{2}\,\pqty{
				(Z\id{^1_0})^{N/2}
				+ (-1)^F (Z\id{^1_1})^{N/2}
			}
			\cdot \frac{1}{2}\,\pqty{
				(Z\id{^1_0})^{N/2}
				+ (-1)^{\tilde{F}} (Z\id{^1_1})^{N/2}
			}^* \\
		&= \frac{1}{2}\, \Bqty{
				Z^{0A|B}
				+ \Re\,
					(Z\id{^0_0} \bar{Z\id{^0_1}})^{N/2}
				+ (-1)^{\tilde{F}} \pqty{\Re| \,i\Im}\,
					(Z\id{^1_0} \bar{Z\id{^1_1}})^{N/2}
			}
	\end{aligned}
	\end{equation}
	
	To check for modular invariance, note that\footnote{
		See \textit{Polchinski}, Chapter 10. Note that the factor $
			\exp \pqty{
				- i\pi\, \frac{3\alpha^2 - 1}{12}
			}
		$ comes from a global gravitational anomaly, but does not matter in $Z^{0A|B}$ since we are taking absolute values. 
	}:
	\begin{equation}
		Z\id{^\alpha_\beta}(\tau)
		= Z\id{^\beta_{-\alpha}}(-\tfrac{1}{\tau})
		= Z\id{^\alpha_{\alpha + \beta - 1}}(\tau + 1)\,
			\cdot \exp \pqty{
				- i\pi\, \frac{3\alpha^2 - 1}{12}
			}
	\end{equation}
	We see that $Z^{0A|B}$ is indeed modular invariant, while $
		Z' = \frac{1}{2} Z^{0A|B}
			+ \pqty{\cdots}
	$ is \textit{not} modular invariant, due to the extra ``mixing'' terms in $(\cdots)$. 
	
	\item Consider the ground states in the type 0 theories; the NS ground state is tachyonic: 
	\begin{equation}
		m^2 = -k^2 = -\frac{1}{2\alpha'}
	\end{equation}
	With $(-1)^F = -1$, while the level 1 states are massless and form a vector representation $\mbf{8}_v$ of the massless little group $\mrm{SO}(8)$. After GSO projections, the NS ground state becomes the (NS$-$) ground state, while the level 1 massless states become the (NS$+$) ground states. 
	
	On the other hand, the R ground state is massless. In general, 10\,D Dirac spinors form a representation $\mbf{32}_{\mrm{Dirac}}$ of $\mrm{SO}(9,1)$; however, in the massless case it can be further reduced into two Weyl spinors $\mbf{16} + \mbf{16}'$, labeled by chirality $\Gamma^{11} = (-1)^F$. They are spinor representations of $\mrm{SO}(8)$. The on-shell condition (i.e.\ the Dirac equation) further reduces the representation into $\mbf{8}$ and $\mbf{8}'$, one for $(\R+)$ and one for $(\R-)$. 
	
	The closed string spectrum is then obtained by tensor product of the left and right moving part. For type 0 theories, we see that there is a tachyonic state: the $(\NS-,\NS-)$ ground state is a $\mbf{1}\times \mbf{1} = \mbf{1}$ scalar tachyon; with a momentum rescale $k\mapsto k/2$, the mass is now given by $m^2 = -2/\alpha'$. The remaining massless states are:
	\begin{alignat}{2}
		(\NS+,\NS+)\colon&\quad&
			\mbf{8}_v \times \mbf{8}_v
			&= [0] + [2] + (2) \\
		(\R\pm,\R\pm)\colon&&
			\mbf{8}^{(\prime)} \times \mbf{8}^{(\prime)}
			&= [0] + [2] + [4]_\pm \\
		(\R+,\R-)\colon&&
			\mbf{8} \times \mbf{8}'
			&= [1] + [3]
	\end{alignat}
	Where we've listed the irreducible decompositions of the various $\mbf{8} \times \mbf{8}$ tensor product, following the notations of \textit{Polchinski}. 
	
	\end{enumerate}
	
	\item \textbf{Kaluza--Klein Mechanism}
	
	The $D = d+1$ dimensional metric can be parameterized as follows:
	\begin{gather}
		\dd{s}^2
		= G_{MN}^D \dd{x^M} \dd{x^N}
		= G_{\mu\nu} \dd{x^\mu} \dd{x^\nu}
			+ e^{2\sigma}\,\pqty{
				\dd{x^d} + A_\mu \dd{x^\mu}
			}^2,
	\label{eq:metric_decomp}
	\\
		G^D_{\mu\nu}
		= G_{\mu\nu} + e^{2\sigma} A_\mu A_\nu,\quad
		x^d \cong x^d + 2\pi R
	\label{eq:sub_metric_convention}
	\end{gather}
	Where $\mu = 0,1,\cdots,(d-1)$ labels the noncompact directions, and the $x^d$ direction is compactified. $G_{\mu\nu},\sigma$ and $A_\mu$ should depend only on the noncompact coordinates $x^\mu$, $A^\mu = G^{\mu\nu} A_\nu$. 
	
	$G_{MN}^D$ can be inverted by solving $
		\delta^L_N
		= G^{LM}_D G^D_{MN}
	$, or in components:
	\begin{gather}
		0 = G_D^{\mu\nu}\, e^{2\sigma} A_\nu
			+ G_D^{\mu d}\, e^{2\sigma}
	\quad\Longrightarrow\quad
		G_D^{\mu d}
		= - G_D^{\mu\nu} A_\nu,
	\\
		1 = G_D^{\mu d}\, e^{2\sigma} A_\mu
			+ G_D^{dd}\, e^{2\sigma}
	\quad\Longrightarrow\quad
		G_D^{dd}
		= e^{-2\sigma} - G_D^{\mu d} A_\mu
		= e^{-2\sigma} + G_D^{\mu\nu} A_\mu A_\nu,
	\\
		\delta^\mu_\rho
		= G^{\mu\nu} G_{\nu\rho}
		= G_D^{\mu\nu} G^D_{\nu\rho}
			+ G_D^{\mu d}\, e^{2\sigma} A_\rho,
	\end{gather}
	Contract the last equation with $A^\rho$, and we can solve for $G_D^{\mu d}$ and then all other components. 
	Alternatively, we can use the inversion formula for a block matrix\footnote{
		See e.g.~\wikiref{https://en.wikipedia.org/wiki/Block\_matrix\#Block\_matrix\_inversion}{Block matrix \# Block matrix inversion}. 
	}; either way, we obtain a nice and clean result:
	\begin{equation}
		G_D^{\mu d}
		= -A^\mu,\quad
		G_D^{dd}
		= e^{-2\sigma} + A^2,\quad
		G_D^{\mu\nu}
		= G^{\mu\nu},
	\end{equation}
	There is also a formula\footnote{
		See e.g.~\wikiref{https://en.wikipedia.org/wiki/Determinant\#Block\_matrices}{Determinant \# Block matrices}. 
	} for the determinant $G_D$; we have:
	\begin{equation}
		G_D^{-1}
		= G_d^{-1} \pqty{
				e^{-2\sigma}
				+ A^2 - A^2
			}
		= G_d^{-1} e^{-2\sigma},\quad
		G_D
		= G_d\, e^{2\sigma}
	\end{equation}
	
%	we have:
%	\begin{equation}
%	\begin{aligned}
%		(\Gamma_D)^\lambda_{\mu\nu}
%		&= G_D^{\lambda\rho}\,\Gamma_{\rho\mu\nu}
%			+ G_D^{\lambda d}\cdot
%			\frac{1}{2}\, \pqty\Big{
%				\pdd{\mu} (e^{2\sigma} A_\nu)
%				+ \pdd{\nu} (e^{2\sigma} A_\mu)
%				- 0
%			} \\
%		&= \Gamma^\lambda_{\mu\nu}
%			- A^\lambda e^{2\sigma}
%				\pqty\Big{
%					2A_{(\mu} \pdd{\nu)} \sigma
%					+ \pdd{(\mu} A_{\nu)}
%				},
%	\end{aligned}
%	\end{equation}
%	\begin{equation}
%	\begin{aligned}
%		(\Gamma_D)^d_{\mu\nu}
%		&= G_D^{\rho d}\,\Gamma_{\rho\mu\nu}
%			+ G_D^{dd}\, e^{2\sigma}
%				\pqty\Big{
%					2A_{(\mu} \pdd{\nu)} \sigma
%					+ \pdd{(\mu} A_{\nu)}
%				} \\
%		&= -A_\rho \Gamma^\rho_{\mu\nu}
%			+ (1 + A^2 e^{2\sigma})
%				\pqty\Big{
%					2A_{(\mu} \pdd{\nu)} \sigma
%					+ \pdd{(\mu} A_{\nu)}
%				},
%	\end{aligned}
%	\end{equation}
%	\begin{equation}
%	\begin{aligned}
%		(\Gamma_D)^\lambda_{\mu d}
%		= (\Gamma_D)^\lambda_{d\mu}
%		&= G^{\lambda\rho}\,\Gamma_{\rho\mu d}
%			+ G_D^{\lambda d}\,\Gamma_{d\mu d} \\
%		&= G^{\lambda\rho}\,
%				\pdd{[\mu} \pqty{A_{\rho]} e^{2\sigma}}
%			- A^\lambda\,\tfrac{1}{2}
%				\pdd{\mu} e^{2\sigma} \\
%		&= e^{2\sigma} \pqty\Big{
%				G^{\lambda\rho}\, \pqty{
%						\tfrac{1}{2} F_{\mu\rho}
%						+ 2A_{[\rho} \pdd{\mu]} \sigma
%					}
%				- A^\lambda\,
%					\pdd{\mu} \sigma
%			} \\
%		&= e^{2\sigma} \pqty{
%				\tfrac{1}{2} F\id{_\mu^\lambda}
%				- A_\mu \pd^\lambda \sigma
%			},
%	\end{aligned}
%	\end{equation}
%	\begin{equation}
%	\begin{aligned}
%		(\Gamma_D)^d_{\mu d}
%		= (\Gamma_D)^d_{d\mu}
%		&= e^{2\sigma} \pqty\Big{
%				- A^\rho\, \pqty{
%						\tfrac{1}{2} F_{\mu\rho}
%						+ 2A_{[\rho} \pdd{\mu]} \sigma
%					}
%				+ (e^{-2\sigma} + A^2)\,
%					\pdd{\mu} \sigma
%			} \\
%		&= e^{2\sigma} \pqty{
%				- \tfrac{1}{2} F_{\mu\rho} A^\rho
%				+ A_\mu A^\rho \pdd{\rho} \sigma
%				+ e^{-2\sigma} \pdd{\mu}\sigma
%			},
%	\end{aligned}
%	\end{equation}
%	\begin{equation}
%		(\Gamma_D)^\lambda_{dd}
%		= G_D^{\lambda\rho}\,\Gamma_{\rho dd}
%		= e^{2\sigma} \pd^\lambda \sigma,\quad
%		(\Gamma_D)^d_{dd}
%		= G_D^{d\rho}\,\Gamma_{\rho dd}
%		= A^\lambda e^{2\sigma}
%			\pdd{\rho} \sigma,
%	\end{equation}
	
	\begin{enumerate}
	\item The Christoffel symbols can hence be calculated explicitly, using the $G^D_{MN}$ components; the Ricci scalar can then be computed with brute force\footnote{
		Reference: \http{www.weylmann.com/kaluza.pdf}, and \textit{Polchinski}, Chapter 8. 
	}; in the end, we have:
	\begin{equation}
		R_D = R_d
			- 2e^{-\sigma}\laplacian e^\sigma
			- \frac{1}{4} e^{2\sigma}
				F_{\mu\nu} F^{\mu\nu},
	\end{equation}
	\vspace*{-.8\baselineskip}
	\begin{equation}
	\begin{aligned}
		S
		&= \frac{1}{2\kappa_0^2}
			\int \dd[D]{x} \sqrt{-G_D}\, R_D \\
		&= \frac{1}{2\kappa_0^2}\cdot 2\pi R
			\int \dd[d]{x} \sqrt{-G_d}\, e^{\sigma}
				R_D \\
		&\sim \frac{\pi R}{\kappa_0^2}
			\int \dd[d]{x} \sqrt{-G_d}\, e^{\sigma}
			\pqty{
				R_d
				- \frac{1}{4} e^{2\sigma}
					F_{\mu\nu} F^{\mu\nu}
			}
	\end{aligned}
	\end{equation}
	Here we've dropped the $
		\laplacian e^\sigma
	$ term in the Einstein--Hilbert action, for it is a total derivative:
	\begin{equation}
		\laplacian e^\sigma
		= \frac{1}{\sqrt{-G_d}}\,
			\pdd{\mu} \pqty{
				\sqrt{-G_d}\, G_d^{\mu\nu}
				\pdd{\nu} e^\sigma
			}
	\end{equation}
	However, if there is a $D$-dimensional dilaton $\Phi$ coupled to gravity: $
		\mcal{L}_D \sim e^{-2\Phi} R_D
	$, then the $
		e^{-2\Phi} \laplacian e^\sigma
	$ term cannot be dropped, since it will contribute a $\Phi$--$\sigma$ coupling term. Here we are setting $\Phi\equiv 0$. 
	
	The $e^\sigma$ factor before $R_d$ can be absorbed by rescaling; first we eliminate the zero mode of $\sigma$ by rescaling the coupling $\kappa_0\to\kappa$:
	\begin{gather}
		\sigma = \sigma_0 + \sigma',\quad
		\ave{\sigma} = \sigma_0,\quad
		\ave{\sigma'} = 0,
	\\
		\frac{1}{\kappa_0^2}\, e^\sigma
		= \frac{1}{\kappa^2}\, e^{\sigma'},\quad
		\kappa
		= \kappa_0\, e^{-\sigma_0/2},
	\end{gather}
	Then we work on the remaining $
		\sigma' = \sigma - \sigma_0
	$. Note that:
	\begin{gather}
		G'_{\mu\nu}
		= e^{2\omega(x)} G_{\mu\nu},\quad
		G'
		= e^{2\omega\cdot d} G,\quad
		G'^{\mu\nu}
		= e^{-2\omega} G^{\mu\nu},
	\\
		R'_d
		= e^{-2\omega} \pqty\Big{
				R_d
				- 2\,(d-1) \laplacian\omega
				- (d-2)(d-1)\,
					\pdd{\mu}\omega\,
					\pd^\mu\omega
			},
	\\[1ex]
		\sqrt{-G}\, e^{\sigma'} R_d
		\sim \sqrt{-G'}\, R'_d
		\sim \sqrt{-G}\, e^{(d-2)\omega} R_d,\quad
		\omega
		= \frac{\sigma'}{d - 2},
	\end{gather}
	
	Before we proceed, let's first work out the Weyl transformation of the Laplacian:
	\begin{equation}
	\begin{aligned}
		\nabla'^2 \sigma'
		&= \frac{1}{\sqrt{-G'}}\,
			\pdd{\mu} \pqty{
				\sqrt{-G'} G'^{\mu\nu}
				\pdd{\nu} \sigma'
			} \\
		&= \frac{1}{\sqrt{-G}}\,
			e^{-\omega d}\,
			\pdd{\mu} \pqty{
				\sqrt{-G}\,
					e^{+\omega d}
					e^{-2\omega}\,
				G^{\mu\nu}
				\pdd{\nu} \sigma'
			} \\
		&= e^{-\omega d}\,
				(\pdd{\mu} e^{\sigma'})\,
				G^{\mu\nu} \pdd{\nu} \sigma'
			+ e^{-2\omega} \laplacian \sigma' \\
		&= G'^{\mu\nu}
				\pdd{\mu} \sigma'
				\pdd{\nu} \sigma'
			+ e^{-2\omega} \laplacian \sigma' \\
	\end{aligned}
	\end{equation}
	The transformed Ricci scalar can then be rewritten as:
	\begin{equation}
	\begin{aligned}
		R'_d
		&= e^{-2\omega} R_d
			- 2\,\frac{d-1}{d-2}\, e^{-2\omega}
				\nabla^2 \sigma'
			- \frac{d-1}{d-2}\,
				\pdd{\mu}\sigma'\,
				\pd'^\mu\sigma' \\
		&= e^{-2\omega} R_d
			- 2\,\frac{d-1}{d-2}
				\pqty\Big{
					\nabla'^2 \sigma'
					- \pdd{\mu}\sigma'\,
						\pd'^\mu\sigma'
				}
			- \frac{d-1}{d-2}\,
				\pdd{\mu}\sigma'\,
				\pd'^\mu\sigma' \\
		&= e^{-2\omega} R_d
			- 2\,\frac{d-1}{d-2}\,
				\nabla'^2 \sigma'
			+ \frac{d-1}{d-2}\,
				\pdd{\mu}\sigma'\,
				\pd'^\mu\sigma' \\
	\end{aligned}
	\end{equation}
	
	Again, the $\nabla'^2 \sigma'$ term is a total derivative and can be dropped in the action. In the end, we get:
	\begin{equation}
	\begin{aligned}
		S
		&\sim \frac{\pi R}{\kappa^2}
			\int \dd[d]{x} \sqrt{-G'_d}
			\pqty{
				R'_d
				- \frac{d-1}{d-2}\,
					\pdd{\mu}\sigma'\,
					\pd'^\mu\sigma'
				- \frac{1}{4}
					e^{
						2(\sigma \sidenote{+} \omega)
					} F_{\mu\nu} F'^{\mu\nu}
			}
	\end{aligned}
	\end{equation}
	This is the effective $d$-dimensional theory that we have been looking for, with a gauge field $F_{\mu\nu}$ and a massless dilaton $\sigma'$. Roughly speaking, the dilaton $\sigma'$ can be treated as a Goldstone boson due to the breaking of scale invariance by compactification\footnote{
		For a more careful discussion, see \textit{Polchinski}. See also \https{physics.stackexchange.com/q/138537}. 
	}. 
	
	Following the convention of \textit{Polchinski}, we define $
		A_\mu = R\tilde{A}_\mu,\ %
		\rho = R e^\sigma,\ %
		\rho_0 = \ave{\rho} = R e^{\sigma_0}
	$, then the gravitational and gauge couplings are given by:
	\begin{gather}
		\frac{1}{2\kappa_d^2}
		= \frac{\pi R}{\kappa^2},\quad
		-\frac{1}{4 g_d^2}
		= -\frac{1}{4}\,
			e^{2\ave{\sigma + \omega}} R^2
			\cdot\frac{\pi R}{\kappa^2}
		= -\frac{1}{4}\, e^{2\sigma_0} R^2
			\cdot \frac{1}{2\kappa_d^2},
	\\[.5ex]
		\therefore\quad
		\kappa_d^2
		= \frac{\kappa^2}{2\pi R}
		= \frac{\kappa_0^2}{2\pi \rho_0},\quad
		g_d^2
		= \frac{2\kappa_d^2}{\rho_0^2}
		= \frac{\kappa_0^2}{\pi \rho_0^3},\quad
		\rho_0
		= R e^{\sigma_0}
	\end{gather}
	
	\item The above mechanism provides a natural theory of gravity and electromagnetism in $d = 4$. Note that the gravitational and gauge couplings are related with the radius of the compact dimension:
	\begin{equation}
		\frac{g_d^2}{\kappa_d^2}
		= \frac{2}{\rho_0^2}
	\end{equation}
	In reality gravity is much weaker than electromagnetism, which means that $\rho_0 \to 0$, or $R\to 0$ if we gauge-fix $\sigma_0 \equiv 0$. In other words, the radius is constrained by the ratio of the couplings:
	\begin{equation}
		R \sim \sqrt{2}\,\frac{\kappa_d}{g_d}
	\end{equation}
	
	\end{enumerate}
	
	\item \textbf{Fiberwise T-Duality and the Dilaton}
	\begin{enumerate}
	\item For a bosonic string moving in a general background of massless fields in $D = d + 1 = 26$, its worldsheet action is given by:
	\begin{equation}
		S = \frac{1}{4\pi\alpha'}
			\int \dd[2]{\sigma} \sqrt{g}\,
			\Bqty\Big{
				\pqty{
					g^{ab} G_{MN}(X)
					+ i\epsilon^{ab} B_{MN}(X)
				}\,
				\pdd{a} X^M
				\pdd{b} X^N
				+ \alpha'\mcal{R}\,\Phi(X)
			}
	\end{equation}
	Where $\Phi$ is the worldsheet Ricci scalar. The $X^d \equiv X^{25}$ direction is to be compactified, and the background fields $G_{MN}, B_{MN}$ and $\Phi$ depends only on $X^\mu,\,\mu = 0,1,\cdots,(d-1) = 24$. 
	
	$G_{MN}$ can be further split into $G_{\mu\nu},G_{\mu d}$ and $G_{dd}$ with $d = 25$, in a way similar to \eqref{eq:metric_decomp}, but here we are using a simpler convention, with $G^D_{\mu\nu} = G_{\mu\nu}$ instead of \eqref{eq:sub_metric_convention}. Similar goes for $B_{MN}$, with $B_{\mu\nu},B_{\mu d}$, and $B_{dd} = 0$, due to anti-symmetry. 
	
	\item After replacing $\pdd{a} X^d\mapsto \pdd{a} X^d + A_a$ where $A_a$ is an auxiliary abelian gauge field on the worldsheet, the $X^d$ related parts in the Lagrangian become:
	\begin{equation}
	\small
	\begin{aligned}
		\mcal{L}_d[X^d,A_a]
		&= \frac{\sqrt{g}}{4\pi\alpha'}
			\Bqty\Big{
				2\,\pqty{
					g^{ab} G_{\mu d}
					+ i\epsilon^{ab} B_{\mu d}
				}
					\,(\pdd{a} X^d + A_a)\,
					\pdd{b} X^\mu
				+ g^{ab} G_{dd}
					\,(\pdd{a} X^d + A_a)\,
					\,(\pdd{b} X^d + A_b)
			}
	\end{aligned}
	\end{equation}
	
	Consider a translation $
		X^d\mapsto X^d + \lambda
	$, where $\lambda$ depends on $X^\mu = X^\mu(\sigma)$ and hence depends on the worldsheet coordinates $\sigma$; it is clear that:
	\begin{gather}
		\pdd{a} \pqty{X^d + \lambda}
			+ \pqty{A_a - \pdd{a}\lambda}
		= \pdd{a} X^d + A_a,
	\quad
		\mcal{L}_d \bqty{
			\pqty{X^d + \lambda},
			\pqty{A_a - \pdd{a}\lambda}
		}
		= \mcal{L}_d\bqty{
			X^d, A_a
		}
	\end{gather}
	i.e.\ $X^d$ translation is equivalent to a local gauge transformation $A_a\mapsto A_a - \pdd{a}\lambda$. 
	
	In fact, we would like $A_a$ to be ``pure gauge'', capturing only the $X^d$ translational symmetry and nothing more; this can be achieved by adding yet another auxiliary field $\phi(\sigma)$ and an extra term:
	\begin{equation}
		\mcal{L}_d
		\longmapsto \mcal{L}_d
			+ i\epsilon^{ab} F_{ab}\,\phi,
	\quad
		F_{ab}
		= \pdd{a} A_b - \pdd{b} A_a,
	\quad
		\int \DD{\phi}\,
			e^{-i\epsilon^{ab} F_{ab}\,\phi}
		\sim \delta\bqty{
				\epsilon^{ab} F_{ab}
			}
	\end{equation}
	Which forces $F_{12} \equiv 0$ in the remaining path integral. Note that the only non-zero independent component of $F_{ab}$ in 2D is $F_{12}$, therefore $F_{12} \equiv 0$ implies that $F_{ab} \equiv 0$, or $
		F = \dd{A} = 0
	$. On the plane, this implies that $A = \dd{\lambda}$, i.e.\ it is indeed pure gauge\footnote{
		However, if there are punctures on the worldsheet, then there is non-trivial cohomology, and $A$ need not be $\dd{\lambda}$. 
		Instead, the gauge field can have non-trivial holonomy around the cycles of the worldsheet. One can show that these holonomies are gauge trivial if $\phi$ has periodicity $2\pi$. In this case, the partition function is again equivalent to the original one. Reference: \textit{Tong}, String Theory; see also \arxiv{0812.4408}. 
	}. 
	
%	\begin{equation}
%		\int \DD{X}\,\DD{A_a}\,\DD{\phi}\,
%			e^{-S'}
%	\end{equation}
	
	We can then proceed to integrate out $A_a$. Since $A = \dd{\lambda}$, we can gauge fix $A \equiv 0$, and the action reduces to the original one:
	\begin{equation}
		S'\bqty{X,\phi=0,A_a=0}
		= S\bqty{X}
	\end{equation}
	Following the Faddeev--Popov procedure, we find that the path integral also reduces to the original one, up to some additional gauge volume determinant $\Delta_{FP}$, which is independent of $X$. This implies that the theory for the fields $(X,\phi,A_a)$ is, indeed, equivalent to that of the original string theory which has only the $X$ fields.
	
	\item Following our discussions in (b), we see that:
	\begin{equation}
		\pdd{a} X^d + A_a
		= 0 + A_a - \pdd{a} \lambda,\quad
		\lambda = -\pdd{a} X^d,
%	\\
%		\int \DD{X^d}\,
%			e^{-S'[X,\phi,A]}
%		= \int \DD{\lambda}\,
%			e^{-S'[X^d\equiv 0,\phi,A_\lambda]},\quad
%		A_\lambda = A - \dd{\lambda},
	\end{equation}
%	i.e.\ integrating over $X^d$ is equivalent to integrating over all the gauge configurations of $A$, labeled by $\lambda$, but with $X^d \equiv 0$. 
	Before completing the path integral, we perform a gauge transformation $
		A_a \mapsto A_a - \pdd{a} \lambda
	$, with $\lambda =  -X^d$. Assuming that there is no anomaly, we can ignore the functional Jacobian of the transformation, and the path integral shall be gauge invariant; in this case, the $\pdd{a} X^d$ term is canceled precisely by the gauge transformation, which is equivalent to setting $X^d = 0$ in the action:
	\begin{equation}
		S\bqty{
				X^\mu,\phi,A
			}
		= S'\bqty{
				X^\mu,X^d = 0,\phi,A
			}
	\end{equation}
	
	Furthermore, the $A_a$ related parts in the Lagrangian is now nice and quadratic:
	\begin{equation}
%	\small
	\begin{aligned}
		\mcal{L}_A
		&= \frac{\sqrt{g}}{4\pi\alpha'}
			\Bqty\Big{
				2\,\pqty{
					g^{ab} G_{\mu d}
					+ i\epsilon^{ab} B_{\mu d}
				}\,A_a\,\pdd{b} X^\mu
				+ g^{ab} G_{dd}\, A_a A_b
				+ i\epsilon^{ab} F_{ab}\,\phi
			} \\
		&= \frac{\sqrt{g}}{4\pi\alpha'}
			\Bqty\Big{
				2\,\pqty{
					g^{ab} G_{\mu d}
					- i\epsilon^{ab} B_{\mu d}
				}\,\pdd{a} X^\mu A_b
				+ g^{ab} G_{dd}\, A_a A_b
				+ 2i\epsilon^{ab} \phi\, \pdd{a} A_b
			} \\
		&\sim \frac{1}{4\pi\alpha'}
			\Bqty\Big{
				2\,\pqty{
					\delta^{ab} G_{\mu d}
					- i\epsilon^{ab} B_{\mu d}
				}\,\pdd{a} X^\mu A_b
				+ \delta^{ab} G_{dd}\, A_a A_b
				- 2i\epsilon^{ab}
					\,\pdd{a}\phi\, A_b
			} \\
		&= \frac{1}{4\pi\alpha'}
			\Bqty\Big{
				2\,\pqty\Big{
					\pqty{
						\delta^{ab} G_{\mu d}
						- i\epsilon^{ab} B_{\mu d}
					}\,\pdd{a} X^\mu
					- i\epsilon^{ab} \,\pdd{a}\phi
				} A_b
				+ \delta^{ab} G_{dd}\, A_a A_b
			}
	\end{aligned}
	\end{equation}
	Here we've fixed the conformal gauge $g_{ab} = \delta_{ab}$ and integrated by parts, so that $
		\phi\, \pdd{a} A_b
		\mapsto -\pdd{a}\phi\, A_b
	$. 
	
	It is convenient to define\footnote{
		Reference: \textit{Blumenhagen et al}, Basic Concepts of String Theory, Chapter 14. 
	}:
	\begin{equation}
	\begin{aligned}
		J^b
		&= \frac{1}{G_{dd}} \pqty\Big{
				\pqty{
					\delta^{ab} G_{\mu d}
					- i\epsilon^{ab} B_{\mu d}
				}\,\pdd{a} X^\mu
				- i\epsilon^{ab} \,\pdd{a}\phi
			}
	\end{aligned}
	\end{equation}
	The path integral over $A_a$ can then be completed as a Gaussian integral. Note that this time we integrate out $A_a$ first, leaving $\phi$ in place; therefore we do not impose any gauge fixing. We have:
	\begin{gather}
		\mcal{L}_A
		= \frac{G_{dd}}{4\pi\alpha'}
			\Bqty\Big{
				2J^b A_b
				+ \delta^{ab} A_a A_b
			}
		= \frac{G_{dd}}{4\pi\alpha'}
			\Bqty\Big{
				(A + J)^2
				- \delta^{ab} J_a J_b
			},
	\\[1ex]
		S = \int \dd[2]{\sigma} \pqty\big{
				\mcal{L}_A + \mcal{L}_0
			},
	\quad
		\int \DD{A_a}\,e^{-S}
		\sim \det \bqty{
				\frac{
					G_{dd}\pqty\big{X^\mu(\sigma)}
				}{2\pi\alpha'}
			}^{-\frac{1}{2}} e^{-\tilde{S}},
	\\[.5ex]
		\tilde{S} = \int \dd[2]{\sigma} \pqty{
				\mcal{L}_0
				- \frac{G_{dd}}{4\pi\alpha'}\,
					J^a J_a
			},
	\end{gather}
	
	If we identify $X^d \cong X^d + 2\pi$, then the radius of the $X^d$ circle is given by $R(X^\mu) = \sqrt{G_{dd}}$. When $R^2 \gg \alpha'$ or $R \gg \sqrt{\alpha'}$, the above path integral approaches the classical limit, and its main contribution comes from the classical saddle $A_a = -J_a$, which is included in the $e^{-\tilde{S}}$ factor. The functional determinant is sub-leading and can be ignored. 
	
	Expand the action $\tilde{S}$ in terms of $(X^\mu,\phi)$, we find that:
	\begin{gather}
		\tilde{S} = \frac{1}{4\pi\alpha'}
			\int \dd[2]{\sigma}
			\Bqty\Big{
				\pqty{
					\delta^{ab} \tilde{G}_{MN}
					+ i\epsilon^{ab} \tilde{B}_{MN}
				}\,
				\pdd{a} \tilde{X}^M
				\pdd{b} \tilde{X}^N
%				+ \alpha'\mcal{R}\,\Phi(X)
			},\quad
		\tilde{X}
		= (\tilde{X}^\mu,\tilde{X}^d)
		= (X^\mu,\phi),
	\\
		\tilde{G}_{dd}
		= \frac{1}{G_{dd}},\quad
		\tilde{G}_{\mu d}
		= \frac{1}{G_{dd}} B_{\mu d},\quad
		\tilde{G}_{\mu\nu}
		= G_{\mu\nu}
			- \frac{1}{G_{dd}} \pqty\big{
				G_{\mu d} G_{\nu d}
				- B_{\mu d} B_{\nu d}
			},
	\\
		\tilde{B}_{\mu d}
		= \frac{1}{G_{dd}} G_{\mu d},\quad
		\tilde{B}_{\mu\nu}
		= B_{\mu\nu}
			- \frac{1}{G_{dd}} \pqty\big{
				G_{\mu d} B_{\nu d}
				- B_{\mu d} G_{\nu d}
			},
	\end{gather}
	i.e.\ we've found the T-dual theory with $\tilde{R}\propto \frac{1}{R}$. Rescale $\phi\mapsto \phi/\sqrt{\alpha'}$ and $G_{dd}\mapsto \alpha'G_{dd}$, we recover the usual result: $
		\tilde{R} = \frac{\alpha'}{R}
	$. 
	
	\item Now we return to the determinant; roughly speaking, we have:
	\begin{equation}
		\det \bqty{
				\frac{
					G_{dd}\pqty\big{X^\mu(\sigma)}
				}{2\pi\alpha'}
			}^{-\frac{1}{2}}
		= \exp \pqty{
				- \frac{1}{2}
				\ln \det \bqty{\cdots}
			}
		\sim \exp \pqty{
				- \frac{1}{2}
				\tr \frac{\ln G_{dd}}{\alpha'}
			}
	\end{equation}
	Which appears to add a term $\sim - \frac{1}{2} \ln G_{dd}$ in the Lagrangian. 
	
	However, the $\mquote{\det}$ and $\mquote{\tr}$ in the above equation are divergent and ill-defined, and would only make sense after some careful regularization\footnote{
		I would like to thank \textit{谷夏} for hints about this problem. 
	}, which was introduced by \textit{Buscher} \cite{Buscher:1987qj} and nicely reviewed by \textit{Alvarez et al} \cite{Alvarez:1994dn}. The regularized determinant along with the Jacobian adds the following contribution in the Lagrangian:
	\begin{equation}
		- \alpha'\mcal{R}\cdot \frac{1}{2}
			\ln \frac{G_{dd}}{\alpha'}
	\end{equation}
	Which is equivalent to a dilaton shift:
	\begin{equation}
		\tilde{\Phi}
		= \Phi - \frac{1}{2}
			\ln \frac{G_{dd}}{\alpha'}
	\end{equation}
	
	In the limit of constant size $R = \sqrt{G_{dd}}$, note that the string coupling $g_s \sim e^{\Phi_0}$, and we recover the usual result: $\tilde{g}_s = g_s \sqrt{\alpha'} / R$. 
	
	\end{enumerate}
	
	\item \textbf{Only One Coupling Constant in String Theory}\footnote{
		Reference: \arxiv{0812.4408}. 
	}
	\begin{enumerate}
	\item Consider the open string one-loop diagram, which is topologically a cylinder. To represent such geometry on the place, we start from the ``rectangular'' torus:
	\begin{equation}
		w \cong w + 2\pi \cong w + 2\pi\tau,\quad
		\tau = it
	\end{equation}
	And identify under an \textit{involution}, i.e.\ a reflection through the imaginary axis:
	\begin{equation}
		w' = -\bar{w},
	\quad\Longrightarrow\quad
		0 \le \Re w \le \pi
	\end{equation}
	
	The amplitude is similar to the torus amplitude; first, with a fixed $t = -i\tau$, we have:
	\begin{equation}
		\ave*{
			\prod_i \normorder{e^{ik_i\cdot X_i}}
		}_{\!\!t}
		= iC\, (2\pi)^d\,
			\delta^d\pqty{\textstyle\sum_i k_i}\,
			\exp \Bqty{
				- \sum_{i<j} k_i\cdot k_j\,
					G'(w_i,w_j)
				- \frac{1}{2} \sum_{i} k_i^2
					G'(w_i,w_i)
			}
	\label{eq:amp_generic}
	\end{equation}
	But with a different propagator $G'$, which can be obtained via the method of images; this leads to a doubling of the exponents compared to the torus amplitude:
	\begin{gather}
		G'(w,w')
		= G'_{T^2}(w,w')
			+ G'_{T^2}(w,-\bar{w}'),
	\\
		\ave*{
			\prod_i \normorder{e^{ik_i\cdot X_i}}
		}_{\!\!t}
		= iC\, (2\pi)^d\,
			\delta^d\pqty{\textstyle\sum_i k_i}\,
			\prod_{i < j}
				\abs{W_{ij}(t)}^{
					2\times \alpha'k_i\cdot k_j
				}
	\end{gather}
	Here $W_{ij}(t)$ is the ``corrected'' distance on $T^2$; as $w\to w'$ we have $W_{ij}\sim w_{ij} = w_i - w_j$. 
	
	On the other hand, the vacuum amplitude is given by:
	\begin{equation}
		Z = iV_d \int_0^\infty
			\frac{\dd{t}}{2t}\,
			\pqty{8\pi^2\alpha't}^{-d/2}
			\eta(it)^{-(d-2)}
	\end{equation}
	The $bc$ ghost contributions is included in the $
		\abs{\eta(it)}^2
		= \pqty\big{\eta(it)}^2
	$ factor; note that for $t > 0,\,\eta(it)>0$. Combining this and $
		\ave*{\prod_i \normorder{e^{ik_i\cdot X_i}}}_t
	$, we obtain the final $n$-tachyon amplitude:
	\begin{equation}
		\mcal{A}
		= ig_o^n\, (2\pi)^d\,
			\delta^d\pqty{\textstyle\sum_i k_i}
			\int_0^\infty
				\frac{\dd{t}}{2t}\,
				\pqty{8\pi^2\alpha't}^{-d/2}
				\eta(it)^{-(d-2)}
			\prod_k \int_{\pd M} \dd{w_k}
			\prod_{i<j}
				\abs{W_{ij}(t)}^{
					2\times \alpha'k_i\cdot k_j
				}
	\end{equation}
	Here $\pd M$ is the two ends of the cylinder. 
	
	As is suggested in \arxiv{0812.4408}, it is convenient to introduce the following parametrization for the operator insertions at each boundary:
	\begin{equation}
		w_i = \frac{1 - (-1)^\sigma}{2}\,\pi
			+ 2\pi it\cdot x_i,
	\quad
		0\le x_i\le 1
	\end{equation}
	$\sigma = 0,1$ labels the left and right boundary. 
	
	We want 2 insertions at each boundary, labeled by $
		i = 1,2,\,\sigma = 0
	$, and $
		i = 3,4,\,\sigma = 1
	$; the amplitude can then be reduced to:
	\begin{equation}
	\small
	\begin{gathered}
		\mcal{A}
		= ig_o^n\, (2\pi)^d\,
			\delta^d\pqty{\textstyle\sum_i k_i}
			\prod_k 2\!\int_0^1 \dd{x_k}
			\int_0^\infty
				\frac{\dd{t}}{2t}\,
				\pqty{8\pi^2\alpha't}^{-d/2}
				\eta(it)^{-(d-2)}
				(2\pi t)^n
			\prod_{i<j}
				\abs{W_{ij}(x_{ij},t)}^{
					2 \alpha'k_i\cdot k_j
				},
	\\
		W_{ij}(x_{ij},t)
		= \eta(it)^{-3}\,
			\vartheta_{1,2} \pqty{
				ix_{ij} t \,\big|\, it
			}\,
			\exp \pqty{-\pi x^2_{ij} t}
	\end{gathered}
	\end{equation}
	There is an additional factor of $2$ since $
		\int_{\pd M} \dd{\omega} = 2\int \dd{x}
	$, which includes the contribution after exchange of the two ends $
		12\leftrightarrow 34
	$. 
	Also, $\vartheta_{1,2} = \vartheta_1$ or $\vartheta_2$, depending on whether the vertex operators $i$ and $j$ are on the same boundary or not; this is because:
	\begin{equation}
		\vartheta_1 \pqty{
			ix_{ij} t - \tfrac{1}{2} \,\big|\, it
		}
		= -\vartheta_2 \pqty{
			ix_{ij} t \,\big|\, it
		}
	\end{equation}
	
	The $\eta(it)^{-3}$ factor in $W_{ij}$ can be further extracted using the on-shell condition $\alpha'k_i^2 = 1$. More specifically, we have:
	\begin{equation}
		\sum_{i<j}
			2 \alpha'k_i\cdot k_j
		= \alpha' \pqty\bigg{\sum_i k_i}^{\!\!2}
			- \alpha' \sum_i k_i^2
		= 0 - n = -n
	\end{equation}
	Therefore, we have:
	\begin{equation}
	\small
	\begin{gathered}
		\mcal{A}
		= ig_o^n\, (2\pi)^d\,
			\delta^d\pqty{\textstyle\sum_i k_i}
			\prod_k \int_0^1 \dd{x_k}
			\int_0^\infty
				\frac{\dd{t}}{t}\,
				\pqty{8\pi^2\alpha't}^{-d/2}
				(2\pi t)^n
				\eta(it)^{\bm{3n}-(d-2)}
			\prod_{i<j}
				\abs{W'_{ij}(x_{ij},t)}^{
					2 \alpha'k_i\cdot k_j
				},
	\\
		W'_{ij}(x_{ij},t)
		= \vartheta_{1,2} \pqty{
				ix_{ij} t \,\big|\, it
			}\,
			\exp \pqty{-\pi x^2_{ij} t}
	\end{gathered}
	\end{equation}
	
	To further simplify the expression, collect all the numerical coefficients:
	\begin{equation}
		ig_o^n\, (2\pi)^d\,
			\pqty{8\pi^2\alpha'}^{-d/2}
			(2\pi)^n
		= ig_o^n\, (2\pi)^d\,
			2^{-d/2} (2\pi)^{-d} \alpha'^{-d/2}
			(2\pi)^n
		= ig_o^n\, (2\pi)^n\,
			2^{-d/2} \alpha'^{-d/2}
	\end{equation}
	In our case $n = 4$ and $d = 26$. 
	
	The $t$ integral can be magically simplified using modular transformations of the $\vartheta$ functions\footnote{
		Reference: \arxiv{0812.4408} and \textit{Polchinski}'s summary of $\vartheta$ function properties. 
	}; with $t = \frac{1}{u}$, we have:
	\begin{equation}
	\begin{aligned}
		F(x)
		&= \int_0^\infty \dd{t}
			t^{n-1-d/2}
			\eta(it)^{3n-(d-2)}
			\prod_{i<j}
				\abs{W'_{ij}(x_{ij},t)}^{
					2 \alpha'k_i\cdot k_j
				} \\
		&= \int_0^\infty \dd{u}
			\eta(iu)^{3n-(d-2)}
			\prod_{i<j}
				\abs{
					\vartheta_{1,4} \pqty{
						x_{ij} | iu
					}
				}^{
					2 \alpha'k_i\cdot k_j
				}
	\end{aligned}
	\end{equation}
	The amplitude can then be neatly written as:
	\begin{equation}
		\mcal{A}
		= ig_o^n\, (2\pi)^n\,
			2^{-d/2} \alpha'^{-d/2}\,
			\delta^d\pqty{\textstyle\sum_i k_i}
			\prod_k \int_0^1 \dd{x_k} F(x)
	\end{equation}
	
	\item The ``long cylinder'' limit corresponds to the $t\to 0$ contributions in the above amplitude. Note that the full amplitude is an integral over the moduli $t = \frac{1}{u}$, 
	\begin{equation}
		F(x)
		= \int_0^\infty \dd{u} f(x,u),
	\quad
		u = \frac{1}{t},
	\quad
		f(x,u)
		= \eta(iu)^{3n-(d-2)}
			\prod_{i<j}
				\abs{
					\vartheta_{1,4} \pqty{
						x_{ij} | iu
					}
				}^{
					2 \alpha'k_i\cdot k_j
				}, 
	\end{equation}
	We need only look at the integrand $f(x,u)$ as $u\to\infty$, or $
		q \equiv e^{-2\pi u} \to 0
	$. In this case $f(x,u)$ can be expanded as power series; $\vartheta_4$ contributions turn out to be sub-leading, hence the product only needs to go over $i,j$ on the same side, denoted as $(i,j)_\sigma$. We have:
	\begin{equation}
		f(x,u)
		= q^{\frac{3n-(d-2)}{24}}
			\prod_{(i<j)_\sigma}
				\abs{
					2\sin \pi x_{ij}
				}^{2\alpha'k_i\cdot k_j}
			q^{\frac{2\alpha'k_i\cdot k_j}{8}}
			\pqty\big{
				1 + \order{q}
			}
	\end{equation}
	
	Again we can simplify using on-shell conditions and Mandelstam variables; we have:
	\begin{equation}
	\begin{aligned}
		\sum_{(i<j)_\sigma}
				2\alpha'k_i\cdot k_j
		&= \sum_{i<j}
				2\alpha'k_i\cdot k_j
			- \sum_{i,\sigma=0} \sum_{j,\sigma=1}
				2\alpha'k_i\cdot k_j \\
		&= -n - 2\alpha' \sum_{i,\sigma=0} k_i
			\sum_{j,\sigma=1} k_j \\
		&= -n - 2\alpha's,
	\quad
		s = -\pqty\bigg{\,\sum_{i,\sigma=0} k_i}^{\!\!2}
		= -\pqty\bigg{\,\sum_{i,\sigma=1} k_i}^{\!\!2}
	\end{aligned}
	\end{equation}
	Where $s$ is the mass squared of the intermediate state propagating along the long cylinder, from one end to another. With this we find that:
	\begin{equation}
	\begin{aligned}
		f(x,u)
		&= q^{
				\frac{3n-(d-2)}{24}
				+ \frac{-n-2\alpha's}{8}
			}
			\prod_{(i<j)_\sigma}
				\abs{
					2\sin \pi x_{ij}
				}^{2\alpha'k_i\cdot k_j}
			\pqty\big{
				1 + \order{q}
			} \\
		&= q^{
				-1 - \frac{\alpha's}{4}
			}
			\prod_{(i<j)_\sigma}
				\abs{
					2\sin \pi x_{ij}
				}^{2\alpha'k_i\cdot k_j}
			\pqty\big{
				1 + \order{q}
			}
	\end{aligned}
	\end{equation}
	
	Upon integrating over $u$, each $q^k$ in the power series above produces a pole in $s$:
	\begin{equation}
		\int_0^\infty \dd{u}
			q^{
				-1 - \frac{\alpha's}{4}
			} q^k
		\propto \frac{1}{k - 1 - \frac{\alpha's}{4}}
		\propto \frac{1}{s - \frac{4}{\alpha'}(k-1)}
	\end{equation}
	Apparently every integer power $k$ appears in the expansion, hence we have the full closed string spectrum at $s = \frac{4}{\alpha'}(k-1),\,k=0,1,2,\cdots$. 
	
	\newparagraph
	Consider the tachyon pole, i.e.\ $
		k = 0,\,s = -\frac{4}{\alpha'}
	$; this contribution is represented as two disk diagrams linked by a closed string tachyon propagator; each disk has 3 insertions, two incoming (or outgoing) open string tachyons and one outgoing (or incoming) closed string tachyon. 
	
	By unitarity of the 4-point amplitude\footnote{
		Reference: \textit{Polchinski}, Chapter 9. 
	}, sum of all such factorized diagrams should be equal to the original cylinder diagram; therefore, the strength of the tachyon pole, calculated from such two-disk diagram, should be equal to our previous calculations from the cylinder diagram. 
	
	On the other hand, by unitarity of the 3-point amplitude, the incoming and the outgoing disk diagrams have the same contributions. Therefore, we should expect that the closed string tachyon pole strength is equal to the \textit{square} of the disk amplitude with 3 insertions. 
	
	\item For the disk diagram, we have two open string insertions at the boundary, and one close string insertion in the disk; when mapped to the upper half plane, we can use the 3 CKVs to fix the closed string insertion at $(z,\bar{z})$ and one open string insertions at $x_1$, while integrating over the position $x_2$ of the remaining open string insertion:
	\begin{equation}
	\begin{aligned}
		\mcal{A}_D
		&= g_c g_o^2 e^{-\lambda}
			\int \dd{x_2}
			\ave[\Big]{
				\normorder{
					c\tilde{c}\,
					e^{ik\cdot X}
				}\,
				\normorder{
					c^x_1\,
					e^{ik_1\cdot X_1}
				}\,
				\normorder{
					e^{ik_2\cdot X_2}
				}
			} \\
		&= g_c g_o^2\, iC\,(2\pi)^d\,
			\delta^d\pqty{\textstyle\sum_i k_i}\,
				\abs{z - x_1}
				\abs{\bar{z} - x_1}
				\abs{z - \bar{z}}
		\\ & \hspace{2em} \times
			\int \dd{x_2}
				\abs{z - \bar{z}}^{\alpha'k^2/2}
				\abs{x_1 - x_2}^{2\alpha'k_1\cdot k_2}
			\prod_i
				\abs{z - x_i}^{2\alpha'k\cdot k_i} \\
	\end{aligned}
	\end{equation}
	Again note the doubling of exponents due to the image charges. With on-shell conditions, we get:
	\begin{equation}
	\begin{aligned}
		\mcal{A}_D
		&= iC g_c g_o^2\,
			(2\pi)^d\,
			\delta^d\pqty{\textstyle\sum_i k_i}\,
				\abs{z - x_1}^{-2}
				\abs{z - \bar{z}}^{3}
			\int \dd{x_2}
				\abs{x_1 - x_2}^{2}
				\abs{z - x_2}^{-4} \\
		&= iC g_c g_o^2\,
			(2\pi)^d\,
			\delta^d\pqty{\textstyle\sum_i k_i}\,
			\cdot 4\pi
	\end{aligned}
	\end{equation}
	
	Furthermore, $C$ can be fixed by comparing the 4-tachyon and 3-tachyon open string disk amplitudes\footnote{
		Reference: \textit{Polchinski}, Chapter 6. 
	}; we get:
	\begin{equation}
		C = \frac{1}{\alpha'g_o^2},\quad
		\mcal{A}_D
		= \frac{4\pi i g_c}{\alpha'}\,
			(2\pi)^d\,
			\delta^d\pqty{\textstyle\sum_i k_i}
	\end{equation}
	Combined with the closed string tachyon propagator, the factorized diagram described in (b) is then given by:
	\begin{equation}
		\mcal{A}'_0
		= \pqty{\frac{4\pi i g_c}{\alpha'}}^2
			\frac{i}{s - (-\frac{4}{\alpha'})}\,
			(2\pi)^d\,
			\delta^d\pqty{\textstyle\sum_i k_i}
	\end{equation}
	
	On the other hand, the tachyon pole in (b) is given by:
	\begin{equation}
	\begin{aligned}
		\mcal{A}_0
		&= ig_o^4\, (2\pi)^4\,
			2^{-d/2} \alpha'^{-d/2}\,
			\delta^d\pqty{\textstyle\sum_i k_i}
			\prod_k \int_0^1 \dd{x_k} F(x) \\
		&\simeq ig_o^4\, (2\pi)^4\,
			2^{-d/2} \alpha'^{-d/2}\,
			\delta^d\pqty{\textstyle\sum_i k_i}
			\prod_k \int_0^1 \dd{x_k}
			\pqty{
				-\sin^2 (\pi x_{12})\,
				\sin^2 (\pi x_{34})\,
				\frac{2^5}{\pi\alpha'}
				\frac{1}{s - (-\frac{4}{\alpha'})}
			} \\
		&= -ig_o^4\, (2\pi)^3\,
			2^{-d/2+4} \alpha'^{-d/2-1}\,
			\delta^d\pqty{\textstyle\sum_i k_i}
			\frac{1}{s - (-\frac{4}{\alpha'})}
	\end{aligned}
	\end{equation}
	Imposing $\mcal{A}'_0 = \mcal{A}_0$ gives our desired relation, with $l_s = \sqrt{\alpha'}$:
	\begin{equation}
		g_o^2
		= 2^{3(d-2)/4} \pi^{(d-1)/2} \alpha'^{(d-2)/4} g_c
		= 2^{18} \pi^{25/2} l_s^{12} g_c
	\end{equation}
	
	\end{enumerate}
	
	\end{enumerate}

\printbibliography[%
%	title = {参考文献} %
	,heading = bibintoc
]
\end{document}
