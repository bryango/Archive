% !TeX encoding = UTF-8
% !TeX spellcheck = en_US
% !TeX TXS-program:bibliography = biber -l zh__pinyin --output-safechars %
%% !TeX TS-program = lualatex
%%% LuaLaTeX is required for `tikz-feynman`

\documentclass[a4paper,10pt]{article}

\newcommand{\hwNumber}{5}

% Templates: 82ccb576e4df24e5eac4194b76230be360b4f733

% to be `\input` in subfolders,
% ... therefore the path should be relative to subfolders.

\usepackage[UTF8
	,heading=false
	,scheme=plain % English Document
]{ctex}

\input{../.modules/basics/macros.tex}
\input{../.modules/preamble_base.tex}
\input{../.modules/preamble_notes.tex}

\newcommand{\legacyReference}{{
	\clearpage\par
	\quad\clearpage
	\renewcommand{\midquote}{\textbf{PAST WORK, AS TEMPLATE}}
	\newparagraph
}}

% Settings
\counterwithout{equation}{section}
\mathtoolsset{showonlyrefs=false}
%\DeclareTextFontCommand{\textbf}{\sffamily}
\renewcommand{\midquote}{\quad}

% Spacing
\geometry{footnotesep=2\baselineskip} % pre footnote split
\setlength{\parskip}{.5\baselineskip}
\renewcommand{\baselinestretch}{1.15}

%Title
	\posttitle{
		\hfill\Large\ccbyncsajp
		\par\end{flushleft}%
		\vspace*{-.7ex}\hrule%
	}
	\preauthor{\vspace{-1.5ex}%
		\flushleft\itshape%
	}
	\postauthor{\hfill}
	\predate{\noindent\ttfamily Compiled @ }
	\postdate{\vspace{.5ex}}

	\title{String Theory \textnumero\hwNumber}
	\author{\signature Bryan}
	\date{\today}

% List
	\setlist*{
		listparindent=\parindent
		,labelindent=\parindent
		,parsep=\parskip
		,itemsep=1.2\parskip
		,leftmargin=0pt
		,itemindent=*
	}
	\setlist*[enumerate,1]{
		align=left
		,label=\fbox{\textbf{\arabic*}}
		,itemsep=.5\baselineskip
		,itemindent=*
	}

\input{../.modules/basics/biblatex.tex}

%%% ID: sensitive, do NOT publish!
%\InputIfFileExists{../id.tex}{}{}

\newcommand{\zbar}{\bar{z}}

\begin{document}
\maketitle
\pagestyle{headings}
\pagenumbering{arabic}
\thispagestyle{empty}

	\begin{enumerate}
	
	\item \textbf{Partition Function for Compact Scalars}
	
	\begin{enumerate}
	\item Mode expansion of $X$ CFT is\footnote{
		Again we follow the convention of \textit{Polchinski}. 
	}:
	\begin{gather}
		\pd X(z) =
			-i\,\sqrt{\frac{\alpha'}{2}}
			\sum_{m = -\infty}^\infty
				\frac{\alpha_m}{z^{m+1}},\quad
		\pdbar X(\bar{z}) =
			-i\,\sqrt{\frac{\alpha'}{2}}
			\sum_{m = -\infty}^\infty
				\frac{\tilde{\alpha}_m}{\bar{z}^{m+1}},\\
		X = x
			- i\,\sqrt{\frac{\alpha'}{2}}
			\,\pqty\big{
				\alpha_0 \ln z
				+ \tilde{\alpha}_0 \ln \bar{z}
			}
			+ i\,\sqrt{\frac{\alpha'}{2}}
			\sum_{m \ne 0}
				\frac{1}{m}
				\pqty{
					\frac{\alpha_m}{z^m}
					+ \frac{\tilde{\alpha}_m}{\bar{z}^m}
				},
	\end{gather}
	Momentum $p$ is the charge for \textit{spacetime} translation; we have:
	\begin{gather}
		X\mapsto X + \mrm{const},\quad
		j_a = \frac{i}{\alpha'}\,\pdd{a} X,
	\\
		p = \frac{1}{2\pi i} \oint_C \pqty{
				\dd{z} j - \dd{\bar{z}} \tilde{j}\,
			}
		= \frac{1}{\alpha'} \sqrt{\frac{\alpha'}{2}}\,
			\pqty\big{\alpha_0 + \tilde{\alpha}_0}
		= \sqrt{\frac{1}{2\alpha'}}\,
			\pqty\big{\alpha_0 + \tilde{\alpha}_0}
	\end{gather}
	
	Additionally, for compact free boson, $X$ is only defined modulo $2\pi R$; therefore, states after $X + 2\pi R$ translation should be identical to the original states, i.e.
	\begin{equation}
		e^{ip\,(2\pi R)} = \idty,\quad
		p = \frac{n}{R},\quad n\in\mbb{Z}
	\end{equation}
	This, in fact, holds for any field theory\footnote{
		Reference: discussions in \textit{Polchinski}, Chapter 8. 
	} defined for $X\in S^1$, including the ordinary quantum mechanics (a classical field theory) on $S^1$. 
	
	On the other hand, there are additional constraints in string theory: for the state of a \textit{single} closed string, there is a discrete translational symmetry on the \textit{worldsheet}:
	\begin{equation}
		X(\sigma^1 + 2\pi) \cong X(\sigma^1),\quad
		X(\sigma^1 + 2\pi) = X(\sigma^1) + 2\pi Rw,\quad
		w \in\mbb{Z}
	\end{equation}
	With some definite winding number $w$. In $(z,\bar{z})$ coordinates, we have:
	\begin{gather}
		2\pi Rw
		= X\pqty{
				z\,e^{2\pi i},\zbar\,e^{-2\pi i}\,
			} - X(z,\zbar)
		= -i\sqrt{\frac{\alpha'}{2}}\,
			2\pi i\,\pqty\big{
				\alpha_0 - \tilde{\alpha}_0
			}
		= 2\pi\sqrt{\frac{\alpha'}{2}}\,
			\pqty\big{
				\alpha_0 - \tilde{\alpha}_0
			},
	\\
		p = \frac{p_L + p_R}{2},\quad
		p_L = \sqrt{\frac{2}{\alpha'}}\,\alpha_0,\quad
		p_R = \sqrt{\frac{2}{\alpha'}}\,\tilde{\alpha}_0,
	\\[1ex]
		p_{L,R} = \frac{n}{R} \pm \frac{wR}{\alpha'}\,,
	\\[1ex]
		X = x
			- i\,\frac{\alpha'}{2}\,
			\pqty\big{
				p_L \ln z
				+ p_R \ln \bar{z}
			}
			+ i\,\sqrt{\frac{\alpha'}{2}}
			\sum_{m \ne 0}
				\frac{1}{m}
				\pqty{
					\frac{\alpha_m}{z^m}
					+ \frac{\tilde{\alpha}_m}{\bar{z}^m}
				},
	\end{gather}
	
	For the oscillator expressions for $L_0$, recall that:
	\begin{gather}
		T(z)
		= - \frac{1}{\alpha'}\,
			\normorder{\pd X\,\pd X}
		= \sum_m \frac{L_m}{z^{m+2}},
	\\
		L_{m\ne 0}
		= \frac{1}{2}
			\sum_l \alpha_{m-l} \alpha_l,\quad
		L_0 = \frac{1}{2}\,\normorder{
			\sum_l \alpha_{-l} \alpha_l
		}
		\sim \frac{\alpha'p_L^2}{4}
			+ \sum_{l > 0} \alpha_{-l} \alpha_l,
	\end{gather}
	The $L_0$ expression may be off by some normal ordering constant; this ambiguity can be resolved by considering:
	\begin{equation}
		2L_0 \ket{0,0;n = w = 0}
		= \pqty{
				L_1 L_{-1} - L_{-1} L_1
			} \ket{0,0;p_L = p_R = 0}
		= 0 - 0 = 0
	\end{equation}
	Therefore the normal ordering constant is, in fact, trivial, and we have:
	\begin{equation}
		L_0 = \frac{\alpha'p_L^2}{4}
			+ \sum_{l > 0}
				\alpha_{-l} \alpha_l,\quad
		\tilde{L}_0 = \frac{\alpha'p_R^2}{4}
			+ \sum_{l > 0}
				\tilde{\alpha}_{-l} \tilde{\alpha}_l,
	\end{equation}
	
	\item The torus partition function is given by:
	\begin{equation}
		\ave{\idty}_{T^2}
		\equiv Z(\tau = \tau_1 + i\tau_2)
		= \int \DD X e^{-S}
		= \Tr e^{-(2\pi\tau_2) H} e^{i\,(2\pi\tau_1) P}
	\end{equation}
	Here $P$ generates \textit{worldsheet} translation along $\sigma^1$, not to be confused with $p$ which generates \textit{spacetime} translation; with $
		z = e^{-iw},\ w = \sigma^1 + i\sigma^2
	$, 
	\begin{gather}
	\begin{aligned}
		T\id{^0_1}
		= \eta^{00}\, (\pdd{0} \sigma^2)\, T_{21}
		= -iT_{12}
		&= -i\,\pqty\big{
				T_{ww}\,
					(\pdd{1} w)(\pdd{2} w)
				+ T_{\bar{w}\bar{w}}\,
					(\pdd{1} \bar{w})(\pdd{2} \bar{w})
			} \\
		&= T_{ww} - T_{\bar{w}\bar{w}} \\
		&= \pqty{
				T_{zz}\,
					(\pdd{w} z)^2
				+ \tfrac{c}{24}
			}
			- \pqty{
				T_{\bar{z}\bar{z}}\,
					(\pdd{\bar{w}} \bar{z})^2
				+ \tfrac{\tilde{c}}{24}
			} \\
		&= T(z)\, (-iz)^2
			- \tilde{T}(\bar{z})\, (+i\bar{z})^2
			+ \frac{c - \tilde{c}}{24},
	\end{aligned}
	\\[1ex]
	\begin{aligned}
		P = \int \frac{\dd{\sigma_1}}{2\pi}\,
				\pqty{- T\id{^0_1}\!}
		&= - \int \frac{\dd{\sigma_1}}{2\pi}\,
				T(z)\, (-iz)^2
			+ \int \frac{\dd{\sigma_1}}{2\pi}\,
				\tilde{T}(\bar{z})\, (+i\bar{z})^2
			- \frac{c - \tilde{c}}{24} \\
		&= + \ointctrclockwise
				\frac{\dd{z}}{2\pi\,(-iz)}\,
				T(z)\, (-iz)^2
			+ \ointctrclockwise
				\frac{\dd{\bar{z}}}{2\pi\,(+i\bar{z})}\,
				\tilde{T}(\bar{z})\, (+i\bar{z})^2
			- \frac{c - \tilde{c}}{24} \\
		&= \oint \frac{\dd{z}}{2\pi i}\,
				zT(z)
			- \oint \frac{\dd{\bar{z}}}{2\pi i}\,
				\bar{z}\tilde{T}(\bar{z})
			- \frac{c - \tilde{c}}{24} \\
		&= L_0 - \tilde{L}_0
			- \frac{c - \tilde{c}}{24} \\
		&= \pqty{L_0 - \tfrac{c}{24}}
			- \pqty\big{
				\tilde{L}_0 - \tfrac{\tilde{c}}{24}
			},
	\\[1.5ex]
		H = \int \frac{\dd{\sigma_1}}{2\pi}\,
				T\id{^0_0}
		&= \int \frac{\dd{\sigma_1}}{2\pi}\,
				T_{22} \\
		&= L_0 + \tilde{L}_0
			- \frac{c + \tilde{c}}{24} \\
		&= \pqty{L_0 - \tfrac{c}{24}}
			+ \pqty\big{
				\tilde{L}_0 - \tfrac{\tilde{c}}{24}
			},
	\end{aligned}
	\end{gather}
	Here we've used the fact that $
		\displaystyle\oint \frac{\dd{\bar{z}}}{\bar{z}}
		= \ointctrclockwise
			\frac{\dd{\bar{z}}}{\bar{z}}
		= 2\pi i
	$. Therefore,
	\begin{equation}
		Z(\tau)
		= \Tr e^{-(2\pi\tau_2) H} e^{i\,(2\pi\tau_1) P}
		= \Tr q^{L_0 - \tfrac{c}{24}}\,
			\bar{q}^{\tilde{L}_0 - \tfrac{\tilde{c}}{24}}
		,\quad q = e^{2\pi i\tau}
	\end{equation}
	
	Note that here we are working in the grand canonical ensemble, where we have temperature $\beta = 2\pi\tau_2$ and chemical potential $2\pi\tau_1$. At this stage $P$ is \textit{not} fixed, and we should sum over all states with various $P$. To go to the canonical ensemble, we do a Legendre transform and trade $\tau_1$ for some $P_0$. This is implemented by a Laplace / Fourier transform $\int \dd{\tau_1} e^{-i\,(2\pi\tau_1) P_0}$ of the partition function. 
	
	In string theory, we should actually work with the canonical ensemble (and eventually, the \textit{micro-}canonical ensemble), since we would like to impose the \textit{level matching condition}, namely $P_0 = 0$. 
	
	However, in string theory, the Legendre transform is implicit as we integrate along $\tau_1$ in the moduli space; roughly speaking, we have:
	\begin{equation}
		\int \dd{\tau_1} e^{-i\,(2\pi\tau_1) (P - 0)}
		\propto \delta(P - 0)
	\end{equation}
	As $\tau_1$ gets integrated out, we are effectively computing $\Tr' e^{-\beta H}$ where the trace only goes over the $P = 0$ sector of the Hilbert space\footnote{
		This is nicely explained in David Tong's \href{https://www.damtp.cam.ac.uk/user/tong/string.html}{\textit{String Theory}}. See also section 6.3 of Blumenhagen et al, \textit{Basic Concepts of String Theory}. See also section 2.1, 2.2 of \arxiv{1912.07654} where the switch between ensembles is reviewed in the context of holography. 
	}. 
	
	On the other hand, since we are computing the grand partition function $Z(\tau_1 + i\tau_2)$ for now, we should \textit{not} impose the level matching condition, and the sum should go over all states with various $P$, although this is a much larger Hilbert space than the physical $P = 0$ subspace in string theory. 
	
	Using the expressions in (a), we find that $L_0$ action on a state $\ket{\psi}$ created by $\alpha_{-l},\tilde{\alpha}_{-l}$ yields the sum of occupation number $N_l$ weighted by $l$:
	\begin{equation}
		L_0 \ket{\psi}
		= \pqty{
				\frac{\alpha'k_L^2}{4}
				+ \sum_{l > 0} l\cdot N_l
			} \ket{\psi}
	\end{equation}
	With $c = \tilde{c} = 1$, we obtain:
	\begin{equation}
	\begin{aligned}
		Z(\tau)
		&= (q\bar{q})^{-\frac{1}{24}}
			\sum_{n,w}
				e^{
					- 2\pi \tau_2
					\alpha'\frac{k_L^2 + k_R^2}{4}
				}
				e^{
					2\pi i \tau_1
					\alpha'\frac{k_L^2 - k_R^2}{4}
				}
			\sum_{(N_l),(\tilde{N}_l)}
				q^{
					\,\sum\limits_{l>0}  l\cdot N_l
				}
				\bar{q}^{
					\,\sum\limits_{l>0}  l\cdot \tilde{N}_l
				} \\
		&= (q\bar{q})^{-\frac{1}{24}}
			\sum_{n,w}
				e^{
					- \pi \tau_2 \pqty{
						\frac{\alpha'n^2}{R^2}
						+ \frac{w^2 R^2}{\alpha'}
					}
					+ 2\pi i \tau_1 nw
				}
			\sum_{(N_l),(\tilde{N}_l)} \prod_{l>0}
				q^{l\cdot N_l}
				\bar{q}^{\,l\cdot \tilde{N}_l} \\
		&= \abs{\eta(\tau)}^{-2}
			\sum_{n,w}
				e^{
					- \pi \tau_2 \pqty{
						\frac{\alpha'n^2}{R^2}
						+ \frac{w^2 R^2}{\alpha'}
					}
					+ 2\pi i \tau_1 nw
				}
	\end{aligned}
	\end{equation}
	We've simplified the contributions from the oscillator modes using $\eta(\tau)$, since they are identical to the oscillator contributions of the non-compact $X\in\mbb{R}^1$:
	\begin{equation}
	\begin{aligned}
		(q\bar{q})^{-\frac{1}{24}}\!\!\!\!\!
		\sum_{(N_l),(\tilde{N}_l)} \prod_{l>0}
			q^{l\cdot N_l}
			\bar{q}^{\,l\cdot \tilde{N}_l}
		&= (q\bar{q})^{-\frac{1}{24}}
			\prod_{l>0} \sum_{N_l,\tilde{N}_l = 0}^\infty
				q^{l\cdot N_l}
				\bar{q}^{\,l\cdot \tilde{N}_l} \\
		&= (q\bar{q})^{-\frac{1}{24}}
			\prod_{l>0}
				\frac{1}{1 - q^l}
				\frac{1}{1 - \bar{q}^{\,l}}
		= \abs{\eta(\tau)}^{-2}
	\end{aligned}
	\end{equation}
	
	In the $R\to\infty$ limit, only the $w = 0$ modes survive; all other modes are exponentially suppressed by the $
		e^{
			- \pi \tau_2 w^2 R^2 / \alpha'
		}
	$ factor; i.e.
	\begin{equation}
	\begin{aligned}
		Z(\tau)
		&= \abs{\eta(\tau)}^{-2}
			\sum_{n,w}
				\exp \Bqty{
					- \pi \tau_2 \pqty{
						\frac{\alpha'n^2}{R^2}
						+ \frac{w^2 R^2}{\alpha'}
					}
					+ 2\pi i \tau_1 nw
				} \\
		&\to \abs{\eta(\tau)}^{-2}
			\sum_{n}
				\exp \Bqty{
					- \pi \tau_2\,
						\frac{\alpha'n^2}{R^2}
				},\quad k = \frac{n}{R} \\
		&\to \abs{\eta(\tau)}^{-2}
			\,V\!\!\int \frac{\dd{k}}{2\pi}
				\exp \Bqty{
					- \pi \tau_2\,\alpha' k^2
				} \\
		&= V\,\abs{\eta(\tau)}^{-2}
			\pqty{
				4\pi^2 \alpha' \tau_2
			}^{-\frac{1}{2}} \\
		&\equiv V\cdot Z_X(\tau)
		= 2\pi R\, Z_X(\tau)
	\end{aligned}
	\end{equation}
	We recover the partition function $V\cdot Z_X(\tau)$ for non-compact $X$, as expected. 
	
	\item Using the Poisson resummation formula, we find that:
	\begin{equation}
	\begin{aligned}
		Z(\tau)
		&= 2\pi R\, Z_X(\tau) \sum_{m,w}
			\exp \pqty{
				- \frac{
					\pi R^2 \abs{m - w\tau}^2
				}{\alpha'\tau_2}
			}
	\end{aligned}
	\end{equation}
	$Z_X(\tau)$ is modular invariant by the properties of the Dedekind $\eta(\tau)$ function, as is demonstrated for the non-compact $X$ in \textit{Polchinski}. 
	
	The sum, on the other hand, is naturally invariant under $T\colon \tau \mapsto \tau + 1$, by making a change of variables $m \mapsto m + w$. It is also invariant under $S\colon \tau \mapsto -1/\tau$ with $m \mapsto -w, w \mapsto m$\,\footnote{
		Reference: \textit{Polchinski}. 
	}. Therefore, $Z(\tau)$ is modular invariant. 
	
	\end{enumerate}
	
	\item \textbf{$\mbb{Z}_2$ Orbifold}
	
	The $\mbb{Z}_2$ orbifold is constructed by imposing an additional identification on $X\in S^1$:
	\begin{equation}
		X\cong -X
	\end{equation}
	The target space is then reduced to $S^1/\mbb{Z}_2 \cong [0,\pi R]$. 
	
	\begin{enumerate}
	\item The first contributions to the orbifold partition function comes from the states that are invariant under reflection $r$; we have:
	\begin{equation}
		\Tr_{S^1/\mbb{Z}_2}
		= \Tr_{S^1} \frac{1 + r}{2}
		= \frac{1}{2}\,{\Tr_{S^1}}
			+ \frac{1}{2}\,{\Tr_{S^1}} \circ r
	\end{equation}
	Acting on $
		q^{L_0 - \tfrac{c}{24}}\,
		\bar{q}^{\tilde{L}_0 - \tfrac{\tilde{c}}{24}}
	$, the first term gives $\frac{1}{2}\,Z_{S^1}(\tau)$ where $Z_{S^1}$ is the $S^1$ partition function we've obtained in \boxed{1}\,. 
	
	For the second term, note that:
	\begin{equation}
		r\colon\ %
				\ket{(N_l),(\tilde{N}_l);n,w}
			\longmapsto (-1)^{\sum_l (N_l + \tilde{N}_l)}
				\ket{(N_l),(\tilde{N}_l);-n,-w}
	\end{equation}
	In particular, it reverses $n,w$, hence $r$ insertion gives vanishing amplitude unless $n = w = 0$; the summation is very much similar to the $Z_{S^1}$ case, i.e. we have:
	\begin{equation}
	\begin{aligned}
		\frac{1}{2} \Tr_{S^1} \pqty\Big{
			r\,
			q^{L_0 - \tfrac{c}{24}}\,
			\bar{q}^{\tilde{L}_0 - \tfrac{\tilde{c}}{24}}
		}
		&= \frac{1}{2}\,(q\bar{q})^{-\frac{1}{24}}
			\prod_{l>0} \sum_{N_l,\tilde{N}_l = 0}^\infty
				(-1)^{N_l + \tilde{N}_l}
				q^{l\cdot N_l}
				\bar{q}^{\,l\cdot \tilde{N}_l} \\
		&= \frac{1}{2}\,(q\bar{q})^{-\frac{1}{24}}
			\prod_{l>0}
				\frac{1}{1 - (-q^l)}
				\frac{1}{1 - (-\bar{q}^{\,l})}
		= \abs{\frac{\eta(\tau)}{\theta_2(\tau)}}
	\end{aligned}
	\end{equation}
	Where we've used the fact that\footnote{
		Reference: Blumenhagen \& Plauschinn, \textit{Introduction to CFT}, and also \textit{Polchinski}. 
	}: $
		q^{-\frac{1}{24}}
		\prod_{l>0} \frac{1}{1 - (-q^l)}
		= \sqrt{2} \sqrt{
			\frac{\eta(\tau)}{\theta_2(\tau)}
		}
	$. Therefore, the total contributions from $r$--invariant states are:
	\begin{equation}
		\frac{1}{2}\,Z_{S^1}(\tau)
		+ \abs{\frac{\eta(\tau)}{\theta_2(\tau)}}
	\end{equation}
	
	\item With $X\cong -X$, new possibilities emerge as the boundary condition along $\sigma^1$:
	\begin{equation}
		X(\sigma^1 + 2\pi) \cong X(\sigma^1),\quad
		X(\sigma^1 + 2\pi) = \pm X(\sigma^1) + 2\pi Rw,\quad
		w\in\mbb{Z}
	\end{equation}
	The $\mquote{-}$ sign corresponds to the \textit{twisted states}. Due to the anti-periodicity, $\pd X$ has a half-integer mode expansion:
	\begin{gather}
		\allowdisplaybreaks
		\pd X\pqty{z\,e^{2\pi i}} = - \pd X(z),\\
		\pd X(z) =
			-i\,\sqrt{\frac{\alpha'}{2}}
			\sum_{m = -\infty}^\infty
				\frac{\alpha_{m-\frac{1}{2}}}
					{z^{m+\frac{1}{2}}},\quad
		\pdbar X(\bar{z}) =
			-i\,\sqrt{\frac{\alpha'}{2}}
			\sum_{m = -\infty}^\infty
				\frac{\tilde{\alpha}_{m-\frac{1}{2}}}
					{\bar{z}^{m+\frac{1}{2}}},\\
		X = x + i\,\sqrt{\frac{\alpha'}{2}}
			\sum_{m = -\infty}^\infty
				\frac{1}{m+\frac{1}{2}}
				\pqty{
					\frac{\alpha_{m+\frac{1}{2}}}
						{z^{m+\frac{1}{2}}}
					+ \frac{\tilde{\alpha}_{m+\frac{1}{2}}}
						{\bar{z}^{m+\frac{1}{2}}}
				},
	\end{gather}
	Apply the boundary condition on $X$, and we find that $x = \pi R w'$; however, due to the identification $X + 2\pi R \cong X \cong -X$, there are only two inequivalent choices: $x = 0$ and $x =\pi R$, which correspond to the string localized around either of the two fixed points of the $\mbb{Z}_2$ action. 
	
	Much similar to the case in \boxed{1}\,, we have:
	\begin{gather}
		\bqty{
			\alpha_{\frac{1}{2} + l},
			\alpha_{- \frac{1}{2} - l}
		} = \frac{1}{2} + l,\\[1ex]
		L_{m\ne 0}
		= \frac{1}{2}
			\sum_l \alpha_{m - \frac{1}{2} - l}
					\alpha_{\frac{1}{2} + l},\quad
		L_0 = \frac{1}{2}\,\normorder{
			\sum_l \alpha_{-\frac{1}{2}-l}
					\alpha_{\frac{1}{2}+l}
		}
		\sim \sum_{l \ge 0}
				\alpha_{-\frac{1}{2}-l}
				\alpha_{\frac{1}{2}+l}
	\end{gather}
	We can use the same trick to fix the normal ordering constant in $L_0$; this time it is non-trivial:
	\begin{gather}
		L_{-1} = \frac{1}{2}\,
				\alpha_{-\frac{1}{2}}^2
			+ \sum_{l \ge 0}
				\alpha_{-\frac{1}{2}-l}
				\alpha_{\frac{1}{2}+l},\quad
		L_1 = \frac{1}{2}\,
				\alpha_{\frac{1}{2}}^2
			+ \sum_{l > 0}
				\alpha_{\frac{1}{2}-l}
				\alpha_{\frac{1}{2}+l}, \\
	\begin{aligned}
		L_0 \ket{0,0;x}
		&= \frac{1}{2}\,\pqty{
				L_1 L_{-1} - L_{-1} L_1
			} \ket{0,0;x} \\
		&= \frac{1}{2}\times \frac{1}{4}\,
				\alpha_{\frac{1}{2}}^2
				\alpha_{-\frac{1}{2}}^2
			\ket{0,0;x} - 0 \\
		&= \frac{1}{16} \ket{0,0;x},
	\end{aligned}
	\\[1ex]
		L_0
		= \frac{1}{16}
			+ \sum_{l \ge 0}
				\alpha_{-\frac{1}{2}-l}
				\alpha_{\frac{1}{2}+l}
		= \frac{1}{16}
			+ \sum_{l \ge 0}
				\pqty{l + \frac{1}{2}}\,
				N_{l + \frac{1}{2}}
		= \frac{1}{16}
			+ \sum_{l > 0}
				\pqty{l - \frac{1}{2}}\,
				N_{l - \frac{1}{2}},
	\end{gather}
	
	The trace can then be computed, following the same recipe as before:
	\begin{equation}
	\begin{aligned}
		\Tr_{S^1} \pqty{
			\frac{1 + r}{2}\,
			q^{L_0 - \tfrac{c}{24}}\,
			\bar{q}^{\tilde{L}_0 - \tfrac{\tilde{c}}{24}}
		}
		&= (q\bar{q})^{
				-\frac{1}{24} + \frac{1}{16}
			}
			\prod_{l + \frac{1}{2} \in \mbb{Z}^+}
			\sum_{N_l,\tilde{N}_l = 0}^\infty
				\frac{1 + (-1)^{N_l + \tilde{N}_l}}{2}\,
				q^{l\cdot N_l}
				\bar{q}^{\,l\cdot \tilde{N}_l} \times 2 \\
		&= \frac{1}{2}\,(q\bar{q})^{+\frac{1}{48}}
		\Bqty{\,
			\prod_{l>0}
				\abs{\frac{1}{1 - q^{l - \frac{1}{2}}}}^2
			+ \prod_{l>0}
				\abs{\frac{1}{1 + q^{l - \frac{1}{2}}}}^2
		}\times 2 \\
		&= \abs{\frac{\eta(\tau)}{\theta_4(\tau)}}
			+ \abs{\frac{\eta(\tau)}{\theta_3(\tau)}}
	\end{aligned}
	\end{equation}
	There is an extra factor of 2 from the number of twisted sectors: $x = 0$ and $x = \pi R$. 
	
	\item The full partition function is therefore:
	\begin{equation}
		Z(\tau)
		= \frac{1}{2}\,Z_{S^1}(\tau)
			+ \abs{\frac{\eta(\tau)}{\theta_2(\tau)}}
			+ \abs{\frac{\eta(\tau)}{\theta_4(\tau)}}
			+ \abs{\frac{\eta(\tau)}{\theta_3(\tau)}}
	\end{equation}
	The first term is modular invariant, as is proved in \boxed{1}\,. 
	
	The remaining terms are also modular invariant, due to the transformational properties of $\eta$ and $\theta$ functions\footnote{
		Reference: \textit{Blumenhagen \& Plauschinn}. 
	}:
	\begin{equation}
		T \circlearrowright
		\abs{\frac{\eta(\tau)}{\theta_2(\tau)}}
		\xleftrightarrow{\ S\ }
		\abs{\frac{\eta(\tau)}{\theta_4(\tau)}}
		\xleftrightarrow{\ T\ }
		\abs{\frac{\eta(\tau)}{\theta_3(\tau)}}
		\circlearrowleft S
	\end{equation}
	Therefore, the full partition function is modular invariant. 
	
	\end{enumerate}
	
	\item \textbf{Torus 4-point function in $bc$ CFT}
	\begin{equation}
		\ave[\big]{
			c(w_1)\,b(w_2)\,
			\tilde{c}(\bar{w}_3)\,\tilde{b}(\bar{w}_4)
		}
		= \int \DD{b} \DD{\tilde{b}}
				\DD{c} \DD{\tilde{c}}\,
			c(w_1)\,b(w_2)\,
			\tilde{c}(\bar{w}_3)\,\tilde{b}(\bar{w}_4)\,
			e^{-S'}
		\equiv Z'
	\end{equation}
	First we argue that only the zero modes of the insertions survive the path integral\footnote{
		I would like to thank \textit{谷夏} for some very helpful discussions about this problem.
	}. In fact, as anti-commuting replacements of the gauge degrees of freedom, ghost modes are \textit{defined} to be the eigenvalues of $P^\dagger P$, where $P$ is the conformal Killing differential\footnote{
		Reference: \textit{Polchinski}, Chapter 3 \& 5. 
	}. 
	More specifically, given a conformal Killing vector (CKV) $\var{\sigma^a}$, the conformal Killing equation can be written as:
	\begin{equation}
		P \var{\sigma} = 0
	\end{equation}
	While $P^\dagger \var'\! g = 0$ gives moduli variation $\var'\! g_{ab}$ of the metric. Roughly speaking, $P$ captures the variation of gauge fixing under an arbitrary gauge transformation; naturally, CKV's are given by $(\ker P)$, while $(\det P) \sim \Delta_{FP}$ is the Faddeev--Popov functional measure near the gauge slice. $(\det P)$ can then be calculated with:
	\begin{gather}
		\var{\sigma^a} \mapsto c^a,\quad
		\var'\! g_{ab} \mapsto b_{ab},\quad
		\Delta_{FP}
		\sim \det P
		\sim \int \DD{b} \DD{\tilde{b}}
				\DD{c} \DD{\tilde{c}} e^{-S'},\\
		S'
		= \frac{1}{2\pi} \int \dd[2]{\sigma}
			g^{1/2}\, b_{ab}\,(P\cdot c)^{ab}
		= \frac{1}{2\pi} \int \dd[2]{w}
			\pqty\big{
				b\,\pdbar_w c
				+ \tilde{b}\,\pd_w \tilde{c}
			}
	\end{gather}
	In the end we have chosen conformal gauge, such that\footnote{
		References:
		\begin{itemize}[
			labelindent=3em,labelsep=1pt,
			topsep=.1\baselineskip
		]
		\item Nakahara, \textit{Geometry, Topology and Physics}; 
		\item Blumenhagen et al, \textit{Basic Concepts of String Theory}. 
		\end{itemize}
	} $
		P \sim (\pdbar_w, \pd_w),\ %
		P^\dagger P \sim -\pdbar_w\pd_w = -\laplacian
	$. In the $w = \sigma^1 + i\sigma^2$ coordinates, CKV's are simple translations: $c^a = \mrm{const}$; with $z = e^{-iw}$, it gets mapped to $
		c^z
		= c^w \,\pd_w z
		= c^w \,(-iz)
	$, which agrees with the zero mode $c_0$ in the $c(z)$ expansion:
	\begin{equation}
		c(z)
		= \sum_{m = -\infty}^\infty
			\frac{c_m}{z^{m + 1 - \lambda}}
		= c_0\,z + \sum_{m \ne 0}
			\frac{c_m}{z^{m - 1}}
		,\quad \lambda = 2
		\label{eq:cghost_holomorphic}
	\end{equation}
	
	Now we are finally ready to prove our argument: for anti-commuting variables like $c(z)$, 
	\begin{equation}
		\int \DD{c}
		\sim \prod_m \int \dd{c_m}
		\sim \prod_m \pdv{c_m}
	\end{equation}
	Since $c_0$ corresponds to a CKV, $P \cdot c_0 = 0$, therefore it vanishes in $S' = \int \dd[2]{\sigma}\pqty{b\cdot P\cdot c}$; for the path integral to be non-zero, there has to be some additional $c_0$ insertions, i.e.
	\begin{equation}
		Z'
		\sim \int \DD{b} \DD{\tilde{b}}
				\DD{c} \DD{\tilde{c}}\,
			c_0 b_0
			\tilde{c}_0 \tilde{b}_0\,
			e^{-S'}
		\sim \pqty{\frac{1}{\sqrt{\tau_2}}}^{\!\!4}\!\!
			\int \DD{'b} \DD{'\tilde{b}}
				\DD{'c} \DD{'\tilde{c}}\,
			e^{-S'},\quad
		\int \DD{'c}
		\sim \prod_{m\ne 0} \int \dd{c_m}
	\end{equation}
	
	Note the additional $
		\pqty\big{\frac{1}{\sqrt{\tau_2}}}^4
	$ factor coming from the zero modes\footnote{
		Reference: \textit{Di Francesco et al}. 
	}; this has to do with the normalization of the zero modes, each contributing a factor of $\frac{1}{\sqrt{A}}$, where $A \sim \tau_2$ is the volume (surface area) of the torus. On a different note, since it is very difficult, if not impossible, to keep track of various (often divergent) constant factors in the path integral, we have been and will be calculating $Z'$ up to an overall constant coefficient.
	
	Now we have to deal with the path integral over non-zero modes. Note that the holomorphic mode expansion \eqref{eq:cghost_holomorphic} is incomplete for our purpose: it gives the \textit{on-shell} mode expansion, while our path integral should go over all possible configurations, including the off-shell modes, which is \textit{not} holomorphic. However, on $T^2 = S^1\times S^1$, the full modes are simple\footnote{
		References: (1) \textit{Nakahara}, (2) \textit{Di Francesco et al.}, and (3) \url{http://theory.uchicago.edu/~sethi/Teaching/P483-W2018/p483-sol3.pdf}. 
	}:
	\begin{gather}
		-\laplacian \psi_{n_1,n_2}
		= \lambda_{n_1,n_2} \psi_{n_1,n_2},
	\\[1ex]
	\begin{aligned}
		\psi_{n_1,n_2}
		&= \exp \pqty\big{
				i\,\pqty*{
					n_1\tilde{\sigma}^1
					+ n_2\tilde{\sigma}^2
				}
			},\quad
		\tilde{\sigma}^2
		= \frac{\sigma^2}{\tau_2},\quad
		\tilde{\sigma}^1
		= \sigma^1 - \sigma^2 \frac{\tau_1}{\tau_2}, \\[1ex]
		&= \exp \Bqty{
				i\,\pqty{
					n_1 \sigma^1
					+ \frac{n_2 - n_1 \tau_1}{\tau_2}\,
						\sigma^2
				}
			},
	\end{aligned}
	\end{gather}
	Here we first use the ``rectangular'' coordinates $(\tilde{\sigma}^1, \tilde{\sigma}^2) \in [0,2\pi]^2$ to write down the obvious eigen-functions $\psi_{n_1,n_2}$, and then relate them back to the $(\sigma^1,\sigma^2)$ coordinates. Therefore, we have:
	\begin{gather}
	\begin{aligned}
		\lambda_{n_1,n_2}
		&= \Bqty{
				n_1^2 + \pqty{
					\frac{n_2 - n_1 \tau_1}{\tau_2}
				}^{\!2}
			} \\
		&= \frac{1}{\tau_2^2}\,
			\Bqty{
				(n_1 \tau_2)^2 + \pqty\big{
					n_1 \tau_1 - n_2
				}^2
			} \\
		&= \frac{1}{\tau_2^2}\,
			\abs{n_1\tau - n_2}^2,
	\end{aligned}
	\\[.5ex]
		\det\nolimits' P
		\sim \pqty{\,
				\sideset{}{'}\prod_{n_1,n_2} 
					\sqrt{\lambda_{n_1,n_2}}
%					\frac{1}{\tau_2}\,
%					\abs{n_1\tau - n_2}
			}^2
		\sim \sideset{}{'}\prod_{n_1,n_2} 
			\lambda_{n_1,n_2}
	\end{gather}
	
	The determinant can be computed with $\zeta$-function regularization, as is performed in detail in \textit{Di Francesco}; the result can be nicely summarized using the Eisenstein series, as shown in \textit{Nakahara}:
	\begin{gather}
		E(\tau,s)
		= \sideset{}{'}\sum_{n_1,n_2} 
			\frac{\tau_2^s}{\abs{n_1\tau - n_2}^{2s}},
	\\
		\det\nolimits' P
		\sim \sideset{}{'}\prod_{n_1,n_2}
			\frac{1}{\tau_2^2}\,
			\abs{n_1\tau - n_2}^2
		\sim \tau_2 \exp \Bqty\Big{
				-\pdd{s} E'(\tau,s)_{s = 0}
			}
		= \tau_2^2\,\abs{\eta(\tau)}^4
	\end{gather}
	Finally, we have:
	\begin{equation}
		Z'
		\sim \tau_2^{-2} \det\nolimits' P
		\sim \tau_2^{-2} \tau_2^2\,\abs{\eta(\tau)}^4
		\sim \abs{\eta(\tau)}^4
	\end{equation}
	
	\item \textbf{Torus Propagator as a Trace}
	\begin{equation}
		w'\to 0,\quad
		\ave[\big]{\pdd{w} X(w)\,\pdd{w'} X(w')}
		= \Tr \pqty{
			\pdd{w} X(w)\,\pdd{w'} X(w')\,
			q^{L_0 - \tfrac{c}{24}}\,
			\bar{q}^{\tilde{L}_0 - \tfrac{\tilde{c}}{24}}
		}
	\end{equation}
	Here we've dropped the time ordering in the $w' \to 0$ limit. 
	Recall the mode expansion of $\pd X$ in \boxed{1}\,; we see that only the ``diagonal'' components of $
		\pd X(w)\,\pd X(w')
	$ survive in the trace, i.e.
	\begin{equation}
	\begin{aligned}
		\pdd{w} X(w)\,\pdd{w'} X(w')
		&= (\pdd{w} z)(\pdd{w'} z')\,
			\pdd{z} X(z)\,\pdd{z'} X(z')
		,\quad z = e^{-iw}
		,\quad 1\le\abs{z}\le e^{2\pi\tau_2} \\
		&\sim -\frac{\alpha'}{2}
			\sum_{n = -\infty}^\infty
				\frac{\alpha_{-n} \alpha_n}
					{z^{-n+1} z'^{n+1}}\,
				(-iz)(-iz') \\
		&= \frac{\alpha'}{2} \pqty{
				\alpha_0^2
				+ \sum_{n > 0}
					\pqty{
						\pqty\Big{\frac{z}{z'}}^{\!n}
						+ \pqty\Big{\frac{z'}{z}}^{\!\!n\,}
					}\, \alpha_{-n} \alpha_n
				+ \sum_{n > 0}
					n\,
					\pqty\Big{\frac{z'}{z}}^{\!\!n\,}
			}
		\\
		&= \frac{\alpha'}{2} \pqty{
				\alpha_0^2
				+ \sum_{n > 0}
					\pqty{
						\pqty\Big{\frac{z}{z'}}^{\!n}
						+ \pqty\Big{\frac{z'}{z}}^{\!\!n\,}
					}\, \alpha_{-n} \alpha_n
				+ \frac{zz'}{(z - z')^2}
			}
	\end{aligned}
	\end{equation}
	The last term is a normal ordering constant; here it is naturally regularized by $
		\pqty\big{\frac{z'}{z}}^{\!n\,}
	$. 
	
	The $\alpha_0^2$ term can be substituted with spacetime momentum $p$; we have:
	\begin{gather}
		p = \sqrt{\frac{1}{2\alpha'}}\,
			\pqty\big{\alpha_0 + \tilde{\alpha}_0}
		= \sqrt{\frac{1}{2\alpha'}}\, 2\alpha_0
		= \sqrt{\frac{2}{\alpha'}}\, \alpha_0,
	\\[1ex]
	\begin{aligned}
		\pdd{w} X(w)\,\pdd{w'} X(w')
		&\sim \frac{\alpha'}{2} \pqty{
				\frac{\alpha' p^2}{2}
				+ \sum_{n > 0}
					\pqty{
						\pqty\Big{\frac{z}{z'}}^{\!n}
						+ \pqty\Big{\frac{z'}{z}}^{\!\!n\,}
					}\, n N_n
			}
	\end{aligned}
	\end{gather}
	On the other hand, the partition function is:
	\begin{equation}
	\begin{aligned}
		Z(\tau) = \ave{\idty}
		&= (q\bar{q})^{-\frac{1}{24}}
			\,V\!\!\int \frac{\dd{k}}{2\pi}\,
				e^{- \pi \tau_2\,\alpha' k^2}
			\!\!\!
			\sum_{(N_l),(\tilde{N}_l)}
				q^{
					\,\sum\limits_{l>0}  l\cdot N_l
				}
				\bar{q}^{
					\,\sum\limits_{l>0}  l\cdot \tilde{N}_l
				} \\
		&= (q\bar{q})^{-\frac{1}{24}}
			\,V\!\!\int \frac{\dd{k}}{2\pi}\,
				e^{- \pi \tau_2\,\alpha' k^2}
			\!\!\!
			\sum_{(N_l),(\tilde{N}_l)} \prod_{l>0}
				q^{l\cdot N_l}
				\bar{q}^{\,l\cdot \tilde{N}_l} \\
		&= \abs{\eta(\tau)}^{-2}
			\,V\!\!\int \frac{\dd{k}}{2\pi}\,
				e^{- \pi \tau_2\,\alpha' k^2}
			\!\!\!
	\end{aligned}
	\end{equation}
	
	We can work out $Z^{-1} \ave{\pd X \pd X}$ by considering term by term insertion of the $\pd X \pd X$ mode expansion into the above expression. For the $\frac{\alpha'p^2}{2}$ term, we have a contribution of:
	\begin{equation}
		\frac{\displaystyle
			\int \frac{\dd{k}}{2\pi}\,
				\frac{\alpha'k^2}{2}\,
				e^{- \pi \tau_2\,\alpha' k^2}
		}{\displaystyle
			\int \frac{\dd{k}}{2\pi}\,
				e^{- \pi \tau_2\,\alpha' k^2}
		}
		= \frac{\alpha'}{2}\,
			\frac{1}{2\cdot \pi\alpha'\tau_2}
		= \frac{1}{4\pi\tau_2}
	\end{equation}
	
	For the $nN_n$ insertion, we have a contribution of:
	\begin{equation}
	\begin{aligned}
%		\frac{1}{\abs{\eta(\tau)}^{-2}}
%		\cdot
%		(q\bar{q})^{-\frac{1}{24}}
%			\!\!\!
%			\sum_{(N_l),(\tilde{N}_l)}
%			\!\!
%				nN_n\,
%				q^{
%					\,\sum\limits_{l>0}  l\cdot N_l
%				}
%				\bar{q}^{
%					\,\sum\limits_{l>0}  l\cdot \tilde{N}_l
%				}
		\frac{\displaystyle
			\sum_{(N_l)}
				nN_n\,
				q^{
					\,\sum\limits_{l>0}  l\cdot N_l
				}
		}{\displaystyle
			\sum_{(N_l)}
				q^{
					\,\sum\limits_{l>0}^{\,}  l\cdot N_l
				}
		}
		&= \frac{\displaystyle
			\sum_{(N_l)}
				nN_n\,
				\prod_{l>0} q^{l\cdot N_l}
		}{\displaystyle
			\sum_{(N_l)}^{\,}
				\prod_{l>0} q^{l\cdot N_l}
		}
		= \frac{\displaystyle
			\sum_{N_n = 0}^\infty
				nN_n\, q^{n\cdot N_n}
		}{\displaystyle
			\sum_{N_n = 0}^\infty
				q^{n\cdot N_n}
		}
		= \frac{\displaystyle
			n q^n \pdv{(q^n)}
			\sum_{N_n = 0}^\infty q^{n\cdot N_n}
		}{\displaystyle
			\sum_{N_n = 0}^\infty q^{n\cdot N_n}
		} \\
		&= \frac{
			n q^n \pdv{(q^n)} \frac{1}{1 - q^n}
		}{
			\frac{1}{1 - q^n}
		}
		= \frac{n q^n}{1 - q^n}
	\end{aligned}
	\end{equation}
	Therefore, the complete result is given by:
	\begin{equation}
	\begin{aligned}
		\frac{1}{Z(\tau)}
		\ave[\big]{
			\pdd{w} X(w)\,\pdd{w'} X(w')
		}
		= {} & \frac{\alpha'}{2} \pqty{
				\frac{1}{4\pi\tau_2}
				+ \sum_{n > 0}
					\pqty{
						\pqty\Big{\frac{z}{z'}}^{\!n}
						+ \pqty\Big{\frac{z'}{z}}^{\!\!n\,}
					}\,
					\frac{n q^n}{1 - q^n}
				+ \frac{zz'}{(z - z')^2}
			} \\
		\xrightarrow[z'\to 1]{\, w'\to 0\ }\ &
			\frac{\alpha'}{2} \pqty{
				\frac{1}{4\pi\tau_2}
				+ \sum_{n > 0}
					\pqty{
						z^{n}
						+ z^{-n}
					}\,
					\frac{n q^n}{1 - q^n}
				+ \frac{z}{(z - 1)^2}
			}
	\end{aligned}
	\label{eq:propagator_diff_sum}
	\end{equation}
	
	On the other hand, the torus propagator is given by:
	\begin{gather}
		G'(w,\bar{w};w',\bar{w}')
		= - \frac{\alpha'}{2}\,
				\ln \abs{f(w - w',\tau)}^2
			+ \frac{\alpha'}{4\pi\tau_2}\,
				\pqty\big{\Im \pqty{w - w'}}^2,\\
		f(w,\tau) \equiv
		\theta_1 \pqty{
				\frac{w}{2\pi}
				\,\bigg\vert\,
				\tau
			}
		= 2\,e^{\frac{i\pi\tau}{4}} \sin \frac{w}{2}\,
			\prod_{m > 0}^\infty
				(1 - q^m)
				(1 - z^{-1}q^m)
				(1 - zq^m),\quad
		z = e^{-iw}
	\end{gather}
	We find that $
		\pdd{w} \pdd{w'} G'
	$ contains the same zero mode contribution $
		\frac{\alpha'}{8\pi\tau_2}
	$ and normal ordering contribution $
		\frac{\alpha'}{2} \frac{z}{(z - 1)^2}
	$ as in \eqref{eq:propagator_diff_sum}:
	\begin{gather}
		\pdd{w} \pdd{w'} G'(w,\bar{w};w',\bar{w}')_{
			w' = 0
		}
		= \frac{\alpha'}{8\pi\tau_2}
			+ \frac{\alpha'}{2}
				\pdd{w}^2 \ln f(w,\tau),
	\\[1ex]
		\pdd{w}^2 \ln f(w,\tau)
		= \pdd{w}^2 \ln \sin \frac{w}{2}
			+ \pdd{w}^2 \sum_{m > 0}  \pqty\Big{
				\ln \,(1 - zq^m)
				+ \ln \,(1 - z^{-1}q^m)
			},
	\\
		\pdd{w}^2 \ln \sin \frac{w}{2}
		= \pdd{w}^2 \ln \sin \frac{w}{2}
		= - \frac{1}{4\sin^2 \!\frac{w}{2}}
		= \frac{1}{2\,\pqty{\cos w - 1}}
		= \frac{1}{z + z^{-1} - 2}
		= \frac{z}{(z - 1)^2},
	\end{gather}
	
	The remaining parts come from oscillator modes; they also match with \eqref{eq:propagator_diff_sum}, but the equivalence is less obvious: we have\footnote{
		Reference: \url{http://theory.uchicago.edu/~sethi/Teaching/P483-W2018/p483-sol3.pdf}. I would like to thank {Lucy Smith} for providing this hint. 
	}:
	\begin{gather}
	\begin{aligned}
		\pdd{w}^2 \sum_{m > 0}
			\ln \,(1 - zq^m)
		&= \pdd{w}^2
			\sum_{m > 0}
			\sum_{n > 0}
				-\frac{1}{n}\,\pqty\big{zq^m}^n \\
		&= \sum_{n > 0}
				\pdd{w}^2 \pqty{-\frac{1}{n}\,z^n}
			\sum_{m > 0}
				q^{mn},\quad \pdd{w} = -iz\,\pdd{z} \\
		&= \sum_{n > 0}
				- \frac{(-in)^2}{n}\,z^n
				\cdot
				\frac{q^n}{1 - q^n} \\
		&= \sum_{n > 0}
				z^n \frac{nq^n}{1 - q^n},
	\end{aligned}
	\allowdisplaybreaks
	\\[1ex]
		\pdd{w}^2 \sum_{m > 0}
			\ln \,(1 - z^{-1} q^m)
		= \sum_{n > 0}
				z^{-n} \frac{nq^n}{1 - q^n},
	\end{gather}
	This is precisely the contribution from oscillator modes in \eqref{eq:propagator_diff_sum}. Therefore, we have:
	\begin{equation}
		\frac{1}{Z(\tau)}
		\ave[\big]{
			\pdd{w} X(w)\,\pdd{w'} X(w')
		}_{w' = 0}
		= \pdd{w} \pdd{w'}
			G'(w,\bar{w};w',\bar{w}')_{w' = 0}
	\end{equation}
	
	\end{enumerate}


\printbibliography[%
%	title = {参考文献} %
	,heading = bibintoc
]
\end{document}
