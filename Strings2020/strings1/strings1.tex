% !TeX encoding = UTF-8
% !TeX spellcheck = en_US
% !TeX TXS-program:bibliography = biber -l zh__pinyin --output-safechars %

\documentclass[a4paper,10pt]{article}

\newcommand{\hwNumber}{1}

% Templates: 82ccb576e4df24e5eac4194b76230be360b4f733

% to be `\input` in subfolders,
% ... therefore the path should be relative to subfolders.

\usepackage[UTF8
	,heading=false
	,scheme=plain % English Document
]{ctex}

\input{../.modules/basics/macros.tex}
\input{../.modules/preamble_base.tex}
\input{../.modules/preamble_notes.tex}

\newcommand{\legacyReference}{{
	\clearpage\par
	\quad\clearpage
	\renewcommand{\midquote}{\textbf{PAST WORK, AS TEMPLATE}}
	\newparagraph
}}

% Settings
\counterwithout{equation}{section}
\mathtoolsset{showonlyrefs=false}
%\DeclareTextFontCommand{\textbf}{\sffamily}
\renewcommand{\midquote}{\quad}

% Spacing
\geometry{footnotesep=2\baselineskip} % pre footnote split
\setlength{\parskip}{.5\baselineskip}
\renewcommand{\baselinestretch}{1.15}

%Title
	\posttitle{
		\hfill\Large\ccbyncsajp
		\par\end{flushleft}%
		\vspace*{-.7ex}\hrule%
	}
	\preauthor{\vspace{-1.5ex}%
		\flushleft\itshape%
	}
	\postauthor{\hfill}
	\predate{\noindent\ttfamily Compiled @ }
	\postdate{\vspace{.5ex}}

	\title{String Theory \textnumero\hwNumber}
	\author{\signature Bryan}
	\date{\today}

% List
	\setlist*{
		listparindent=\parindent
		,labelindent=\parindent
		,parsep=\parskip
		,itemsep=1.2\parskip
		,leftmargin=0pt
		,itemindent=*
	}
	\setlist*[enumerate,1]{
		align=left
		,label=\fbox{\textbf{\arabic*}}
		,itemsep=.5\baselineskip
		,itemindent=*
	}

\input{../.modules/basics/biblatex.tex}

%%% ID: sensitive, do NOT publish!
%\InputIfFileExists{../id.tex}{}{}

\newcommand{\oppower}[2]{\mop{{#1}^{#2}}\!}
\newcommand{\sqsinh}{\oppower{\sinh}{2}}
\newcommand{\sqcosh}{\oppower{\cosh}{2}}

\newcommand{\Vir}{\mathbf{V}\mfrak{ir}}

\begin{document}
\maketitle
\pagestyle{headings}
\pagenumbering{arabic}
\thispagestyle{empty}

%{
%	\noindent\itshape%
%	本文约定:度规$\eta\sim\pqty{-,+,+,+}$, 指标$\mu,\nu,\dots = 0,1,2,3,\ i,j,\dots = 1,2,3$.
%}
	\begin{enumerate}
	\item \textbf{Read \textit{Polchinski} Sections 1.3 and 1.4}: 
	
	Read, \textit{mostly} understood. \qed
	
	\item \textbf{Spinning Closed String in AdS Space}:
	
	For a classical spinning string, we have Nambu--Goto action:
	\begin{equation}
		S_{NG}
		= -T \int \dd{\tau} \dd{\sigma}
			\sqrt{-\det \gamma_{ab}},\quad
		\gamma_{ab} = G_{\mu\nu}
			\pdd{a}X^\mu \pdd{b}X^\nu
	\end{equation}
	Here $G_{\mu\nu}$ is the spacetime metric. $\gamma_{ab}$ can be treated as the induced metric on the worldsheet. 
	
	In AdS space we have:
	\begin{equation}
		ds^2
		= R^2\pqty{
			- \sqcosh\rho \dd{t}^2
			+ \dd{\rho}^2
			+ \sqsinh\rho \dd{\Omega}^2
		}
	\end{equation}
	Where $\dd{\Omega}^2$ is the metric of a unit $(d-2)$--sphere $S^{d-2}$. For convenience let's define unit $S^{d-2}$ metric $G^1_{ij}$, and raise or lower the $i,j,\cdots$ indices using $G^1_{ij}$ instead of $G_{ij}$, i.e.,
	\begin{equation}
		G^1_{ij} = G_{ij} /\,\pqty{
			R^2 \sqsinh\rho
		},\quad
		i,j = 2,\cdots,d-1
	\end{equation}
	
	Furthermore, we consider the special case that the closed string is \textit{folded}, like a rubber band stretched along a line; in this case we can choose the worldsheet parameter $(\tau,\sigma) = (t,\rho)$ while $\Omega = \Omega(t,\rho) = \Omega(\tau,\sigma)$, which leads to the following decomposition:
	\begin{align}
		\pdd{a} X^\mu
		&= \delta^\mu_a + \delta^\mu_i\,
			\pdd{a}\Omega^i,\quad
			a = 0,1,\quad
			i = 2,\cdots,d-1,\\[.5ex]
		\gamma_{ab}
		&= G_{\mu\nu}\,
			\pdd{a}X^\mu \pdd{b}X^\nu
		\notag\\
		&= G_{ab} + G_{ij}\,
			\pdd{a}\Omega^i\,\pdd{b}\Omega^j
		\notag\\
		&= G_{ab} + R^2\sqsinh\rho\,G^1_{ij}\,
			\pdd{a}\Omega^i\,\pdd{b}\Omega^j
		\notag\\
		&= R^2\Bqty{\mat{
			-\sqcosh\rho & \\
			& 1 \\
		} + \sqsinh\rho\,\mat{
			(\pdd{a}\Omega)^2 &
			\pdd{a}\Omega\cdot\pdd{b}\Omega \\
			\pdd{b}\Omega\cdot\pdd{a}\Omega &
			(\pdd{b}\Omega)^2 \\
		}}
%		\notag\\
%		&= R^2\,\sqsinh\rho\,\Bqty{\mat{
%			-\oppower{\coth}{2}\rho & \\
%			& \oppower{\csch}{2}\rho \\
%		} + \,\mat{
%			(\pdd{a}\Omega)^2 &
%			\pdd{a}\Omega\cdot\pdd{b}\Omega \\
%			\pdd{b}\Omega\cdot\pdd{a}\Omega &
%			(\pdd{b}\Omega)^2 \\
%		}}
	\end{align}
	Here $
		\pdd{a}\Omega\cdot\pdd{b}\Omega
		\equiv \pdd{a}\Omega^i\,\pdd{b}\Omega_i
		\equiv G^1_{ij}\,
			\pdd{a}\Omega^i\,\pdd{b}\Omega^j
	$, and we have:
	\begin{equation}
	\begin{aligned}
		\det \gamma_{ab}
		&= (R^2)^2\,\Big\{
			\oppower{\sinh}{4}\rho\,
			\det\,(\pdd{a}\Omega^i\pdd{b}\Omega_i)
		\\
		&\qquad\qquad\qquad
			+ \oppower{\sinh}{2}\rho\,\pqty{
				(\pdd{a}\Omega)^2
				- (\pdd{b}\Omega)^2
				\oppower{\cosh}{2}\rho
			}
		\\
		&\qquad\qquad\qquad
			- \oppower{\cosh}{2}\rho\,
		\Big\},\\[.5ex]
		\sqrt{-\det \gamma_{ab}}
		&= R^2\,\Big\{
			\oppower{\cosh}{2}\rho
			- \oppower{\sinh}{2}\rho\,\pqty{
				(\pdd{a}\Omega)^2
				- (\pdd{b}\Omega)^2
				\oppower{\cosh}{2}\rho
			}
		\\[-.5ex]
		&\qquad\qquad\quad
			- \oppower{\sinh}{4}\rho\,
			\det\,(\pdd{a}\Omega^i\pdd{b}\Omega_i)
		\Big\}^{\!1/2}
	\end{aligned}
	\end{equation}
	
	Mark the end points of the string with $\rho = r(t)$, then the total length of such closed folded string is $\ell = 4r$. We then have:
	\begin{equation}
		S = -4TR^2 \int\dd{t} \int_0^r \dd{\rho}
		\sqrt{
			\oppower{\cosh}{2}\rho
			- \oppower{\sinh}{2}\rho\,\pqty{
				(\pdd{a}\Omega)^2
				- (\pdd{b}\Omega)^2
				\oppower{\cosh}{2}\rho
			}
			- \oppower{\sinh}{4}\rho\,
			\det\,(\pdd{a}\Omega^i\pdd{b}\Omega_i)
		}
	\end{equation}
\pagebreak[3]
	
%	Variation of metric determinant $
%		\gamma \equiv \det\gamma_{ab}
%	$ is give by the following formula:
%	\begin{equation}
%		\var{\sqrt{-\gamma}}
%		= \frac{1}{2\sqrt{-\gamma}}
%			\var{(-\gamma)}
%		= \frac{1}{2\sqrt{-\gamma}}\,
%			(-\gamma)
%			\var{\ln\gamma}
%		= \frac{1}{2} \sqrt{-\gamma}\,
%			\gamma^{ab} \var{\gamma_{ab}}
%	\end{equation}
	
	Further simplification comes from the fact that, due to rotational symmetry, the string's motion can be restricted in a plane where its position is characterized by some angle $\theta = \Omega^{i_0}\in \{\Omega^i\}_i$. In this case other angle parameters $\Omega^i|_{i\ne i_0} = 0$, and the action is further reduced to:
	\begin{gather}
		S = -4TR^2 \int\dd{t} \int_0^r \dd{\rho}
			\sqrt{
				\oppower{\cosh}{2}\rho
				- \oppower{\sinh}{2}\rho\,\pqty{
					(\pdd{a}\theta)^2
					- (\pdd{b}\theta)^2
					\oppower{\cosh}{2}\rho
				}
			}
		= \int\dd{t} \int_0^r \dd{\rho} \mcal{L},\\
		\mcal{L} = -4TR^2
			\sqrt{
				\oppower{\cosh}{2}\rho
				- \omega^2\oppower{\sinh}{2}\rho
			},\quad
			\omega = \pdd{t}\theta,\,
			\pdd{\rho}\theta = 0
	\end{gather}
	We consider the special solution $\theta = \omega t$, while in general the endpoint $r = r(t)$ could be dynamical; variation of the action w.r.t.\ $r(t)$ gives\footnote{
		The above reasoning is confirmed in e.g.\ \arxiv{hep-th/0204051}.
	}:
	\begin{gather}
		0 = \var{S} = -4TR^2 \int\dd{t}
			\int\limits_r^{r+\var{r}} \dd{\rho}
			\sqrt{
				\oppower{\cosh}{2}\rho
				- \omega^2\oppower{\sinh}{2}\rho
			}
		= -4TR^2 \int\dd{t}
			\sqrt{
				\oppower{\cosh}{2}r
				- \omega^2\oppower{\sinh}{2}r
			} \var{r},\\
		\omega^2 = \frac{\sqcosh r}{\sqsinh r}
		= \oppower{\coth}{2} r
		\label{eq:string_endpt_constraint}
	\end{gather}
	
	Note that if $\omega$ is constant, then $r$ must be fixed by \eqref{eq:string_endpt_constraint}. 
	Taking $\theta$ as the only dynamical variable, it is then straight-forward to write the energy $E$ and angular momentum $J$ for such folded closed string:
	\begin{gather}
		\omega = \dot{\theta},\quad
		\Pi = \pdv{\mcal{L}}{\omega}
		= 4TR^2\,\frac{\omega\sqsinh\rho}{
			\sqrt{
				\oppower{\cosh}{2}\rho
				- \omega^2\oppower{\sinh}{2}\rho
			}
		},\\
		J = \int_0^r \dd{\rho} \Pi
		= 4TR^2 \int_0^r \dd{\rho}
		\frac{\omega\sqsinh\rho}{
			\sqrt{
				\oppower{\cosh}{2}\rho
				- \omega^2\oppower{\sinh}{2}\rho
			}
		},\\
		E = \int_0^r \dd{\rho} \pqty{
			\Pi\omega - \mcal{L}
		}
		= 4TR^2 \int_0^r \dd{\rho}
		\frac{\sqcosh\rho}{
			\sqrt{
				\oppower{\cosh}{2}\rho
				- \omega^2\oppower{\sinh}{2}\rho
			}
		},
	\end{gather}
	
	In the large string limit, $r\to\infty,\ \omega = \coth r\to 1$. Expand in terms of $\epsilon = \omega - 1 > 0$, we find that $
		r = \frac{1}{2} \ln\pqty{
			1 + \frac{2}{\epsilon}
		}
		\sim \frac{1}{2} \ln\frac{2}{\epsilon}
	$, or alternatively, $e^{2r}\cdot\epsilon\sim 2$. With some help from Mathematica\texttrademark, we get:
	\begin{equation}
	\begin{aligned}
		E - J
		&= 4TR^2 \int_0^r \dd{\rho}
		\frac{\sqcosh\rho - \omega\sqsinh\rho}{
			\sqrt{
				\oppower{\cosh}{2}\rho
				- \omega^2\oppower{\sinh}{2}\rho
			}
		}
%		\\[.5ex]
%		&
		= 4TR^2 \int_0^r \dd{\rho} \pqty\Big{
			1 + \frac{\epsilon^2}{8}\sinh^2(2\rho)
			+ \order{\epsilon^3}
		} \\[1ex]
		&= 4TR^2 \pqty{
			r \pqty\Big{
				1 - \frac{\epsilon^2}{16}
				+ \order{\epsilon^3}
			} + \order{1}
		}
%		\\[.5ex]
%		&
		= \pqty{2TR^2 \ln\frac{2}{\epsilon}}
		\pqty\Big{
			1 - \frac{\epsilon^2}{16}
			+ \order{\epsilon^3}
		} \\[1ex]
		&\sim 2TR^2
			\pqty{\ln\frac{2}{\epsilon}}
	\end{aligned}
	\end{equation}
	Similarly, $
		J \sim 4TR^2 \mathlarger{\int_0^r}
			\dd{\rho} \sqsinh \rho
		\sim TR^2 \pqty{\frac{2}{\epsilon}}
	$, this gives:
	\begin{equation}
		E - J
		\sim 2TR^2 \ln \frac{J}{TR^2}
	\end{equation}
	\qedfull
\pagebreak[4]
	
	\item \textbf{Special Conformal Transformations:}
	\begin{equation}
		x^\mu
		\xmapsto{\ K(a)\ }
		\tilde{x}^\mu
		= \frac{x^\mu + x^2 a^\mu}{
			1 + 2a\cdot x + a^2 x^2
		}
	\end{equation}
	
	\begin{enumerate}
	\item Under special conformal transformation $K(a)$, metric $\delta_{\mu\nu}\mapsto g_{\mu\nu}$ while:
	\begin{equation}
		g_{\alpha\beta}
			\dd{\tilde{x}^\alpha}
			\dd{\tilde{x}^\beta}
		= \delta_{\mu\nu}
			\dd{x^\mu}
			\dd{x^\nu},\quad
		g_{\alpha\beta}
		= \delta_{\mu\nu}
			\pdv{x^\mu}{\tilde{x}^\alpha}
			\pdv{x^\nu}{\tilde{x}^\beta}
		\label{eq:metric_transform}
	\end{equation}
	To calculate this we have to know the inverse transformation $x = K^{-1}(a)\,\tilde{x}$. First, notice the following decomposition\footnote{
		See \textit{Di Francesco et al}, and also \https{github.com/davidsd/ph229}. 
	} of $K(a)$:
	\begin{equation}
		\tilde{x}^\mu
		= \frac{\frac{x^\mu}{x^2} + a^\mu}{
			\frac{1}{x^2}
			+ \frac{2a\cdot x}{x^2} + a^2
		}
		= \frac{\frac{x^\mu}{x^2} + a^\mu}{
			\abs{\frac{x^\mu}{x^2} + a^\mu}^2
		},\\[1ex]
		\text{i.e.}\quad
		K(a)\colon\ %
			x^\mu
			\ \xmapsto{\ I\ }\ %
			\frac{x^\mu}{x^2}
			\ \xmapsto{\ T(a)\ }\ %
			y^\mu = \frac{x^\mu}{x^2} + a^\mu
			\ \xmapsto{\ I\ }\ %
			\tilde{x}^\mu
			= \frac{y^\mu}{y^2},
		\label{eq:special_conformal_decomp} \\[.5ex]
		\text{i.e.}\quad
		\frac{\tilde{x}^\mu}{\tilde{x}^2}
			= \flatfrac{
				\frac{y^\mu}{y^2}
			}{
				\frac{1}{y^2}
			}
			= y^\mu = \frac{x^\mu}{x^2} + a^\mu
		\label{eq:special_conformal_simplified}
	\end{equation}
	
	From \eqref{eq:special_conformal_simplified}, we see that the transformation parameter $a^\mu$ composes linearly: $K(b)\,K(a) = K(a+b)$, therefore $K^{-1}(a) = K(-a)$, and we have:
	\begin{equation}
		x^\mu = K(-a)\,\tilde{x}^\mu
		= \frac{\tilde{x}^\mu - \tilde{x}^2 a^\mu}{
			1 - 2a\cdot \tilde{x} + a^2 \tilde{x}^2
		}
		= \frac{\tilde{y}^\mu}{y^2},\\[1.5ex]
	\begin{aligned}
		\pdv{x^\mu}{\tilde{x}^\alpha}
		&= \pdv{x^\mu}{\tilde{y}^\sigma}
			\pdv{\tilde{y}^\sigma}{\tilde{x}^\alpha}
		= \pqty{
			\pdv{\tilde{y}^\sigma}
			\frac{\tilde{y}^\mu}{\tilde{y}^2}
		} \pdv{\tilde{x}^\alpha} \pqty{
			\frac{\tilde{x}^\sigma}{\tilde{x}^2}
			- a^\sigma\!
		}
		= \pqty{
			\pdv{\tilde{y}^\sigma}
			\frac{\tilde{y}^\mu}{\tilde{y}^2}
		} \pqty{
			\pdv{\tilde{x}^\alpha}
			\frac{\tilde{x}^\sigma}{\tilde{x}^2}
		} \\[1ex]
		&= \pqty{
			\tilde{y}^2\delta^\mu_\sigma
			- 2\tilde{y}^\mu \tilde{y}_\sigma
		} \pqty{
			\tilde{x}^2\delta^\sigma_\alpha
			- 2\tilde{x}^\sigma \tilde{x}_\alpha
		} \Big/ \pqty{\tilde{y}^4 \tilde{x}^4},
	\end{aligned}\\[1ex]
	\begin{aligned}
		g_{\alpha\beta}
		&\xlongequal{\eqref{eq:metric_transform}}
		\delta_{\mu\nu} \pqty{
			\tilde{y}^2\delta^\mu_\sigma
			- 2\tilde{y}^\mu \tilde{y}_\sigma
		} \pqty{
			\tilde{x}^2\delta^\sigma_\alpha
			- 2\tilde{x}^\sigma \tilde{x}_\alpha
		} \pqty{
			\tilde{y}^2\delta^\nu_\rho
			- 2\tilde{y}^\nu \tilde{y}_\rho
		} \pqty\big{
			\tilde{x}^2\delta^\rho_\beta
			- 2\tilde{x}^\rho \tilde{x}_\beta
		} \Big/ \pqty{\tilde{y}^8 \tilde{x}^8}\\
		&\xlongequal{\sum_{\mu,\nu}}
		\tilde{y}^{-4} \delta_{\sigma\rho} \pqty{
			\tilde{x}^2\delta^\sigma_\alpha
			- 2\tilde{x}^\sigma \tilde{x}_\alpha
		} \pqty\big{
			\tilde{x}^2\delta^\rho_\beta
			- 2\tilde{x}^\rho \tilde{x}_\beta
		} \Big/ \tilde{x}^8 \\
		&\xlongequal{\sum_{\sigma,\rho}}
			\tilde{y}^{-4} \tilde{x}^{-4}
			\delta_{\alpha\beta}
	\end{aligned}
	\end{equation}
	We see that $g_{\alpha\beta} = f(x)\,\delta_{\alpha\beta}$, with coefficient:
	\begin{equation}
		f(x) = \tilde{y}^{-4} \tilde{x}^{-4}
		\xlongequal{%
			\eqref{eq:special_conformal_decomp}%
		} \frac{x^4}{\tilde{x}^4}
		\xlongequal{%
			\eqref{eq:special_conformal_simplified}%
		} \pqty{
			1 + 2a\cdot x + a^2 x^2
		}^2
	\end{equation}
	\qed[(a)]
	\vspace*{-1\baselineskip}
	
%	Note that the transformation depends smoothly on parameter $a$, therefore we can consider infinitesimal transformation generator:
%	\begin{equation}
%		k_\nu
%		= k\id{^\mu_\nu}\pdd{\mu}
%		= \pdv{\tilde{x}^\mu}{a^\nu}
%			\bigg|_{a=0} \!\pdd{\mu}
%		= \pqty{x^2\delta^\mu_\nu
%			- 2x^\mu x_\nu}\pdd{\mu}
%	\end{equation}
	
	\item In 2D with $z = x^1 + ix^2,\ x^\mu\sim (z,\bar{z})$, we see from \eqref{eq:special_conformal_simplified} that:
	\begin{equation}
		\frac{x^\mu}{x^2}
		\sim \frac{z}{\abs{z}^2}
		= \frac{1}{\bar{z}}
		\ \longmapsto\ %
		\frac{1}{\bar{z}} + a,\quad
		\text{i.e.}\quad
		z\ \longmapsto\ 
		w = \frac{1}{\frac{1}{z} + \bar{a}}
		= \frac{z}{1 + z\bar{a}}
	\end{equation}
	Expand in the $\bar{a}\to 0$ limit, we find that $
		w = z\,\pqty{
			1 - z\bar{a} - \cdots
		} \sim z - z^2 \bar{a}
	$, i.e.\ it is generated by:
	\begin{equation}
		K_{\bar{z}}
		= -z^2 \pdd{z}
		= -z^2 \pd,\quad
		\pd \equiv \pdd{z}
	\end{equation}
\pagebreak[3]
	
	Note that when considering non-holomorphic functions, we have to consider $(z,\bar{z})$ as \textit{two} independent variables; hence the anti-holomorphic transformation $
		\bar{z}\mapsto\bar{w}
		= \frac{\bar{z}}{1 + \bar{z} a}
		\sim \bar{z} - \bar{z}^2 a
	$ provides another degree of freedom, namely:
	\begin{gather}
		K_\mu\ \sim\ \pqty{
			K_{\bar{z}} = -z^2 \pd,\ %
			K_{{z}} = -\bar{z}^2 \pdbar
		},\\
		\pd \equiv \pdd{z},\ %
		\pdbar \equiv \pdd{\bar{z}}\notag
	\end{gather}
	Similarly, for translation $z\mapsto z+a$ and its conjugate, we have $
		P_\mu\sim
		\pqty{P_z = \pd, P_{\bar{z}} = \pdbar}
	$. However, dilation and rotation are both encoded in a complex rescaling $
		z\mapsto \lambda z,\ %
		\lambda = re^{i\theta}\in\mbb{C}
	$; we have:
	\begin{equation}
		z\mapsto \lambda z,\quad
		\lambda = re^{i\theta}\in\mbb{C},\quad
	\begin{aligned}
		\var{r} \ \longleftrightarrow\ %
		D &= z\,\pd + \bar{z}\,\pdbar,\\[-.5ex]
		\var{\theta} \ \longleftrightarrow\ %
		M &= i\,\pqty{z\,\pd - \bar{z}\,\pdbar},
	\end{aligned}
	\end{equation}
	
	In summary, we have $
		\mop{span}_\mbb{R}\Bqty{
			P_\mu, K_\mu, D, M
		} = \mfrak{so}(3,1)
	$ generating the ``global'' transformation subgroup of the 2D conformal group; here, the $\mfrak{so}(3,1)$ boost is a linear combination\footnote{
		See e.g.\ \https{github.com/davidsd/ph229}. 
	} of $P_\mu$ and $K_\mu$. More specifically, in 2D any holomorphic or anti-holomorphic function gives a conformal transformation, hence the (classical) 2D conformal group is generated by:
	\begin{equation}
		\ell_m = z^{m+1} \pd,\quad
		\bar{\ell}_m = \bar{z}^{m+1} \pdbar,\quad
		m\in\mbb{Z}
	\end{equation}
	i.e.\ the \textit{Witt algebra} (or Virasoro algbera $\Vir_c$ with $c = 0$). It is clear that a (complexified) $\mfrak{so}(3,1)$ lives inside $\Vir_c$, i.e.,
	\begin{equation}
	\begin{aligned}
		\mfrak{so}(3,1)^\mbb{C}
		&= \mop{span}_\mbb{C}\Bqty{
			P_\mu, K_\mu, D, M
		} \\
		&= \mop{span}_\mbb{C}\Bqty{
			\ell_m,\bar{\ell}_m
			\,\big|\, m = 0,\pm 1
		}
		= \mfrak{sl}(2,\mbb{R})^\mbb{C}
			\oplus_\mbb{C}
			\mfrak{sl}(2,\mbb{R})^\mbb{C}
		\subset \Vir_{c}
	\end{aligned}
	\end{equation}
	\end{enumerate}
	\vspace{-1.2\baselineskip}
	
	\qedfull
	
	\item \textbf{$bc$ CFT}:
	\begin{equation}
		S = \frac{1}{2\pi}
			\int \dd[2]{z} b\,\pdbar c
	\end{equation}
	
	Stress tensor of a theory can be obtained via variation over the metric, or equivalently, over the fields $\phi^i$ with $\var{\phi}$ induced by some \textit{local} spacetime translation $
		x^\mu\mapsto x^\mu + \var{x}^\mu,\ %
		\var{x}^\mu = \epsilon(x)\,a^\mu
	$. Here $\epsilon(x)$ is any compactly supported bump function, centered around some point $x_0$. 
	
	In 2D, we have $\mu = z,\bar{z}$; for $\phi(z,\bar{z})$ with conformal weight $(h,\bar{h})$, consider $
		z\mapsto z',\,
		\bar{z}\mapsto \bar{z}'
	$. For convenience, let's first consider a generic variation $\var{z} = \epsilon(z,\bar{z})$ before restricting to spacetime translation; we have:
	\begin{equation}
		\phi'(z',\bar{z}')
		= \pqty\Big{\dv{z'}{z}}^{\!\! -h}
			\pqty\Big{\dv{\bar{z}'}{\bar{z}}}^{
				\!\! -\bar{h}
			}
			\phi(z,\bar{z}),\\[1ex]
		\tilde{\delta}\phi
		= \pqty{
			- h\,\pd\epsilon
			- \bar{h}\,\pdbar\bar{\epsilon}\,
		}\,\phi,\\
		\var{\phi} 
		= \tilde{\delta}\phi
			-\pdv{\phi}{x^\mu} \var{x^\mu}
		= \pqty{
			- h\,\pd\epsilon
			- \bar{h}\,\pdbar\bar{\epsilon}\,
		}\,\phi
			- \epsilon\,\pd\phi
			- \bar{\epsilon}\,\pdbar\phi,
		\label{eq:var_field_compact_supp}
	\end{equation}
	Here we use $\tilde{\delta}\phi$ to denote the ``internal'' variation related to the conformal weights. 
	
	Note that $\phi = b,c$ are anti-commuting Grassmann numbers, variation of the action gives:
	\begin{equation}
	\begin{aligned}
		\var{S}[b,c,\var{b},\var{c}]
		&= \frac{1}{2\pi} \int \dd[2]{z} \pqty{
			\var{b}\,\pdbar c
			+ b\,\pdbar \var{c}
		} \\
		&= \frac{1}{2\pi} \int \dd[2]{z} \pqty{
			- \pdbar c\,\var{b}
			- \pdbar b\,\var{c}
		} + \frac{1}{2\pi} \int \dd[2]{z}
			\pdbar \pqty{b\var{c}}
	\end{aligned}
	\label{eq:bc_action_var}
	\end{equation}
	For \textit{unknown} $b,c$ and {arbitary} $\var{b},\var{c}$, the second term is reduced to a boundary term at infinity and can be dropped; imposing $\var{S} = 0$ gives the equation of motion (EOM): $\pdbar b = \pdbar c = 0$. 
	
	On the other hand, for \textit{on-shell} $b,c$ and compactly supported $\varphi = \var{b},\var{c}$ given in \eqref{eq:var_field_compact_supp}, the first term in \eqref{eq:bc_action_var} vanishes while $\var{S}_0 = 0$ still holds; this gives:
	\begin{equation}
	\begin{aligned}
		0 = \var{S}_0
		&= \frac{1}{2\pi} \int \dd[2]{z}
			\pdbar \pqty{b\var{c}}
		= \frac{1}{2\pi} \int \dd[2]{z}
			\pdbar\,\pqty\big{
				- (1-\lambda)\,bc\,\pd\epsilon
				- b\,\pd c\,\epsilon
			} \\
		&= \frac{1}{2\pi} \int \dd[2]{z}
			\pqty\big{
				- (1-\lambda)\,bc\,
					\pdbar\pd\epsilon
				- b\,\pd c\,\pdbar\epsilon
			}
	\end{aligned}
	\end{equation}
	Here we've distributed the $\pdbar$ operator and dropped all terms that vanish automatically by EOM. Next we shall collect the $\pd\epsilon,\pdbar\epsilon$ terms; integrating by parts on the first integrand gives:
	\begin{equation}
	\begin{aligned}
		0 = \var{S}_0
		&= \frac{1}{2\pi} \int \dd[2]{z}
			\pqty\big{
				(1-\lambda)\,\pd(bc)\,
				- b\,\pd c
			}\,\pdbar\epsilon \\
		&= \frac{1}{2\pi} \int \dd[2]{z}
			\pqty\big{
				(\pd b)\,c
				- \lambda\,\pd(bc)
			}\,\pdbar\epsilon \\
		&= - \frac{1}{2\pi} \int \dd[2]{z}
			\epsilon(z,\bar{z})\,\pd_{\bar{z}}
			\pqty\big{
				(\pd b)\,c
				- \lambda\,\pd(bc)
			}
	\end{aligned}
	\label{eq:conserved_current}
	\end{equation}
	
	Notice that we have obtained a conserved current using a generic $
		\var{z}
		= \epsilon(z,\bar{z}),
		\var{\bar{z}}
		= \bar{\epsilon}(z,\bar{z})
	$; by setting $\epsilon = \epsilon(z)$, we get \textit{a} energy momentum tensor\footnote{
		Note that the energy momentum tensor obtained in this way is generally \textit{not} unique: it can be off by a boundary term; see Luboš' comment at \https{physics.stackexchange.com/a/96100}, also \arxiv{1601.03616}. However, it is possible to fix this redundancy by considering $Tb$ OPE and match its conformal dimension. I would like to thank \textit{林般} for pointing this out. 
	}:
	\begin{equation}
		T(z)
		= \normorder{(\pd b)\,c}
			- \lambda\,\pd(\normorder{bc})
	\end{equation}
	Normal ordering is added manually to remove singular terms. 
	
	To compute $TT$ OPE, we need the OPE of $b(z)\,c(0)$; this is obtained by examining the following path integral, which is zero since the integrand is a total functional derivative:
	\begin{equation}
		0 = \int \mscr{D}b\,\mscr{D}c\,
			\fdv{\phi}\pqty{e^{-S}\,\psi}
	\end{equation}
	Taking $\phi,\psi=b,c$, this generates operator equations such as $
		\pdbar b(z)\,c(0)
		= 2\pi\delta^2(z,\bar{z})
	$. Note that $
		\pdbar(\frac{1}{z})
		= 2\pi\delta^2(z,\bar{z})
	$, which gives:
	\begin{equation}
		b(z)\,c(0) \sim c(z)\,b(0)
		\sim \frac{1}{z},\quad
		b(z)\,b(0) \sim 0
		\sim c(z)\,c(0)
	\end{equation}
	
	With the $bc$ OPE in hand, the $TT$ OPE is computed directly with brute force, by repeatedly applying Wick's theorem. This gives:
	\begin{equation}
		T(z)\,T(0)
		\sim \frac{
			-6\lambda^2 + 6\lambda - 1
		}{z^4} + \cdots
	\end{equation}
	In general we have $
		-6\lambda^2 + 6\lambda - 1
		= \frac{c}{2}
	$; for $\lambda = 2$ this gives $c = -26$. 
	\qedfull
	
	\item \textbf{Free Fermion CFT:}
	\begin{equation}
		S = \int \dd[2]{z}
			\psi_i\,\pdbar\psi^i,\quad
		\psi^i = \psi_i^*,\quad
		\psi_i = \psi_i(z)
	\end{equation}
	
	\begin{enumerate}
	\item Mode expansion of such chiral fermion is given by:
	\begin{equation}
		\psi_i = \sum_{
			k\in\mbb{Z}+\frac{1}{2}
		} \frac{b_{ik}}{z^{k+\frac{1}{2}}},\quad
		b_{ik}
		= \frac{1}{2\pi i} \oint \dd{z}
			z^{k-\frac{1}{2}} \psi_i
	\end{equation}
	Canonical quantization is achieved by simply imposing anti-commutation relations; this is justified by mapping the system onto a cylinder, then $b_{ik}$'s indeed map to modes on the spatial circle\footnote{
		This can be proven rigorously by considering operator equations like in the $bc$ CFT problem. 
	}. The only non-zero commutators are:
	\begin{equation}
		\Bqty{b_{ik},b^{j\dagger}_q}
		= \delta_{k+q,0}\,\delta^j_i
	\end{equation}
	
	This gives the only non-zero 2-point functions:
	\begin{equation}
	\begin{aligned}
		\ave{
			\psi_i(z)\,\psi^j(w)
		}
		&= \sum_{k,q\in\mbb{Z}+\frac{1}{2}}
			\frac{1}{z^{k+\frac{1}{2}}}
			\frac{1}{w^{q+\frac{1}{2}}}
			\ave{b_{ik} b^{j\dagger}_q} \\
		&= \sum_{k,q\in\mbb{Z}+\frac{1}{2}}
			\frac{1}{z^{k+\frac{1}{2}}}
			\frac{1}{w^{q+\frac{1}{2}}}
			\mel{0}{
				\Bqty{b_{ik},b^{j\dagger}_q}
			}{0}
		= \frac{\delta^j_i}{z-w}
	\end{aligned}
	\end{equation}
	Note that $
		b^i_k\ket{0} = 0,\ %
		\forall\ k\ge\frac{1}{2}
	$. 
	
	\item[(b)(c)] Combining two $\psi$ expansions gives the mode expansion of $
		J\id{_i^j}
		= \normorder{\psi_i(z)\,\psi^j(z)}
	$, namely:
	\begin{equation}
		J\id{_i^j}(z)
		= \sum_{k\in\mbb{Z}}
			\frac{(J\id{_i^j})_k}{z^{k+1}},\quad
		(J\id{_i^j})_k
		= \sum_{q\in\mbb{Z}+\frac{1}{2}}
			\normorder{
				b_{iq}\,b^{j\dagger}_{k-q}
			}
	\end{equation}
	It is in fact more convenient to obtain the $JJ$ OPE first, and then use it to find the  $[J_0,J_0]$ mode commutator\footnote{
		I would like to thank \textit{谷夏} for providing this hint. 
	}; note that $
		{
			{\psi}_i(z)\,{\psi}^j(w)
		}
	$ contraction gives $\frac{\delta^j_i}{z-w}$, we have:
	\begin{equation}
		J\id{_i^j}(z)\,J\id{_k^l}(0)
		\sim \frac{
			\delta_i^l\delta_k^j
		}{z^2} + \frac{
			\delta^j_k J\id{_i^l}(0)
			- \delta^l_i J\id{_k^j}(0)
		}{z},\\[1ex]
		\bqty{
			(J\id{_i^j})_0, (J\id{_k^l})_0
		}
		= \frac{1}{(2\pi i)^2}
			\oint_0 \dd{w}
			\oint_w \dd{z}
				J\id{_i^j}(z)\,J\id{_k^l}(w)
		= \delta_i^l\,(J\id{_k^j})_0
			- \delta_k^j\,(J\id{_i^l})_0
	\end{equation}
	
	\item[(d)] Similar to $bc$ CFT, we have:
	\begin{equation}
		T(z) = \frac{1}{2}\,\pqty{
			\normorder{\psi_i\,\pd\psi^i}
			- \normorder{\pd\psi_i\,\psi^i}
		},\quad
		T(z)\,T(w)
		\sim \frac{n/2}{(z-w)^4}
			+ \frac{2T(w)}{(z-w)^2}
			+ \frac{\pd T(w)}{z-w}
	\end{equation}
	With each (complex) field contributing $\frac{1}{2}\times 2$ central charge\footnote{
		In fact a complex (Dirac) fermion can be ``treated like'' (\textit{dual to}) a boson; this is \textit{bosonization}. 
	}. 
	
	\item[(e)] For real fermions, there is an additional reality condition:
	\begin{equation}
		\psi^i = \psi^\star_i = \psi_i 
	\end{equation}
	The canonical quantization still holds without the extra adjoint, same as the 2-point function:
	\begin{equation}
		\ave{\psi_i(z)\,\psi_j(w)}
		= \frac{\delta_{ij}}{z - w}
	\end{equation}
	
	Similar holds for $J_{ij} = \normorder{\psi_i\psi_j}$ and its OPE, but we no longer need to distinguish upper/lower indices; we have:
	\begin{equation}
		J_{ij}(z)\,J_{kl}(0)
		\sim \frac{
			- \delta_{ik}\delta_{jl}
			+ \delta_{il}\delta_{jk}
		}{z^2} + \frac{
			- \delta_{ik} J_{jl}(0)
			+ \delta_{il} J_{jk}(0)
			+ \delta_{jk} J_{il}(0)
			- \delta_{jl} J_{ik}(0)
		}{z}\\[1ex]
		\bqty{
			(J_{ij})_0, (J_{kl})_0
		}
		= - \delta_{ik} (J_{jl})_0
			+ \delta_{il} (J_{jk})_0
			+ \delta_{jk} (J_{il})_0
			- \delta_{jl} (J_{ik})_0
	\end{equation}
	This is precisely the $\mfrak{o}(n)$ algebra. 
	\qedfull
	
	\end{enumerate}
	
	
	\end{enumerate}


\printbibliography[%
%	title = {参考文献} %
	,heading = bibintoc
]
\end{document}
