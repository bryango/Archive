% !TeX encoding = UTF-8
% !TeX spellcheck = en_US
% !TeX TXS-program:bibliography = biber -l zh__pinyin --output-safechars %
%% !TeX TS-program = lualatex
%%% LuaLaTeX is required for `tikz-feynman`

\documentclass[a4paper,10pt]{article}

\newcommand{\hwNumber}{4 [WIP]}

% Templates: 82ccb576e4df24e5eac4194b76230be360b4f733

% to be `\input` in subfolders,
% ... therefore the path should be relative to subfolders.

\usepackage[UTF8
	,heading=false
	,scheme=plain % English Document
]{ctex}

\input{../.modules/basics/macros.tex}
\input{../.modules/preamble_base.tex}
\input{../.modules/preamble_notes.tex}

\newcommand{\legacyReference}{{
	\clearpage\par
	\quad\clearpage
	\renewcommand{\midquote}{\textbf{PAST WORK, AS TEMPLATE}}
	\newparagraph
}}

% Settings
\counterwithout{equation}{section}
\mathtoolsset{showonlyrefs=false}
%\DeclareTextFontCommand{\textbf}{\sffamily}
\renewcommand{\midquote}{\quad}

% Spacing
\geometry{footnotesep=2\baselineskip} % pre footnote split
\setlength{\parskip}{.5\baselineskip}
\renewcommand{\baselinestretch}{1.15}

%Title
	\posttitle{
		\hfill\Large\ccbyncsajp
		\par\end{flushleft}%
		\vspace*{-.7ex}\hrule%
	}
	\preauthor{\vspace{-1.5ex}%
		\flushleft\itshape%
	}
	\postauthor{\hfill}
	\predate{\noindent\ttfamily Compiled @ }
	\postdate{\vspace{.5ex}}

	\title{String Theory \textnumero\hwNumber}
	\author{\signature Bryan}
	\date{\today}

% List
	\setlist*{
		listparindent=\parindent
		,labelindent=\parindent
		,parsep=\parskip
		,itemsep=1.2\parskip
		,leftmargin=0pt
		,itemindent=*
	}
	\setlist*[enumerate,1]{
		align=left
		,label=\fbox{\textbf{\arabic*}}
		,itemsep=.5\baselineskip
		,itemindent=*
	}

\input{../.modules/basics/biblatex.tex}

%%% ID: sensitive, do NOT publish!
%\InputIfFileExists{../id.tex}{}{}

\newcommand{\oppower}[2]{\mop{{#1}^{#2}}\!}
\newcommand{\sqsinh}{\oppower{\sinh}{2}}
\newcommand{\sqcosh}{\oppower{\cosh}{2}}

\newcommand{\Vir}{\mathbf{V}\mfrak{ir}}
\usepackage{tikz-feynman,cancel}

\begin{document}
\maketitle
\pagestyle{headings}
\pagenumbering{arabic}
\thispagestyle{empty}

%{
%	\noindent\itshape%
%	本文约定:度规$\eta\sim\pqty{-,+,+,+}$, 指标$\mu,\nu,\dots = 0,1,2,3,\ i,j,\dots = 1,2,3$.
%}

	\begin{enumerate}
	\item \textbf{Stringy Physics!}
	\begin{gather}
		T(z)
		= - \frac{1}{\alpha'}\,
			\normorder{\pd X^\mu \pd X_\mu}\,,\quad
		\tilde{T}(\bar{z})
		= - \frac{1}{\alpha'}\,
			\normorder{\pdbar X^\mu \pdbar X_\mu}\,,
	\\[.5ex]
		V_k = \normorder{
				e^{ik\cdot X(z,\bar{z})}
			}\,,\quad
		G_{e,k} = e_{\mu\nu}\,\normorder{
				\pd X^\mu_z\,
				\pdbar X^\nu_{\bar{z}}\,
				e^{ik\cdot X(z,\bar{z})}
			}\,,
	\end{gather}
	We sometimes use subscripts like $
		\pd X^\mu_z
	$ to denote variable dependence to avoid clutter. 
	
	\begin{enumerate}
	\item The weight of a primary operator is given by its OPE with $T$ and $\tilde{T}$. For exponential operators, there is a neat formula for cross contractions\footnote{
		Reference: \textit{Polchinski}, and \https{physics.stackexchange.com/a/389193}. 
	}:
	\begin{equation}
	\begin{aligned}
		T(z)\,V_k(w,\bar{w})
		&= \exp \Bqty{
			\int \dd[2]{z'} 
			\int \dd[2]{w'}
				\wick{
					\c X^\mu_{z'}
					\c X^\nu_{w'}
				}
				\fdv{X^\mu_{z'}}
				\fdv{X^\nu_{w'}}
			}\,
			\normorder{T_z\,e^{ik\cdot X_w}} \\
		&= \exp \Bqty{
			\int \dd[2]{z'}
				\wick{
					\c X^\mu_{z'}
					\c X^\nu_{w}
				}
				\fdv{X^\mu_{z'}}
				\,ik_\nu
			}\,
			\normorder{T_z\,e^{ik\cdot X_w}} \\
		&= \normorder{
			\Bqty{
				\exp \pqty{
					ik_\nu \! \int \dd[2]{z'}
						\wick{
							\c X^\mu_{z'}
							\c X^\nu_{w}
						}
						\fdv{X^\mu_{z'}}
				}\,T_z
			}\,e^{ik\cdot X_w}
		} \\
		&\sim -\frac{1}{\alpha'}\,
			\normorder{
				\Bqty{
					2 \pdd{z} \pqty\big{
						ik_\sigma
						\wick{
							\c X^\mu_{z}
							\c X^\sigma_{w}
						}
					}\, \pdd{z} X_\mu
					+ \pdd{z} \pqty\big{
							ik_\rho
							\wick{
								\c X^\mu_{z}
								\c X^\rho_{w}
							}
						}\,
						\pdd{z} \pqty\big{
							ik_\sigma
							\wick{
								\c X_{z,\mu}
								\c X^\sigma_{w}
							}
						}
				}\,e^{ik\cdot X_w}
			} \\
		&\sim -\frac{1}{\alpha'}\,
			\normorder{
				\Bqty{
					2\,\pqty{
						-\frac{\alpha'}{2}
						\frac{ik^\mu}{z - w}
					}\, \pdd{z} X_\mu
					+ \pqty{
						-\frac{\alpha'}{2}
						\frac{ik^\mu}{z - w}
					} \pqty{
						-\frac{\alpha'}{2}
						\frac{ik_\mu}{z - w}
					}
				}\,e^{ik\cdot X_w}
			} \\
		&\sim
			\frac{\alpha'k^2}{4}
				\frac{V_k(w,\bar{w})}{(z - w)^2}
			+ \frac{\pd V_k(w,\bar{w})}{z - w}
	\end{aligned}
	\end{equation}
	Here we've used the result that $
		ik_\sigma
		\wick{
			\c X^\mu_{z}
			\c X^\sigma_{w}
		}
		= ik^\mu (-\frac{\alpha'}{2})
			\ln \abs{z-w}^2
	$. We see that $V_k$ is a primary of weight $(1,1)$ iff.~$
		\frac{\alpha'k^2}{4}
		= 1
	$, or $
		m^2 = -k^2 = -\frac{4}{\alpha'}
	$. This is the mass shell condition for the closed string tachyon (at level 0). On the other hand,
	\begin{gather}
		G_{e,k}
		= e_{\mu\nu} G^{\mu\nu}_k,\\[1ex]
	\begin{aligned}
		T(z)\,G^{\mu\nu}_k(0)
		&\sim \normorder{\wick{
				\c T_z\,
				\pd \c X^\mu_0\,
				\pdbar X^\nu_0\,
				e^{ik\cdot X_0}
			}}
			+ \cancel{\normorder{\wick{
				\c T_z\,
				\pd X^\mu_0\,
				\pdbar \c X^\nu_0\,
				e^{ik\cdot X_0}
			}}}
			+ \normorder{\wick{
				\c T_z\,
				\pd X^\mu_0\,
				\pdbar X^\nu_0\,
				\c e^{ik\cdot X_0}
			}}
			\\ &\qquad
			+ \cancel{\normorder{\wick{
				\bcontraction[1.1ex]{}{T}{_z\,\pd}{X}
				\c T_z\,
				\pd X^\mu_0\,
				\pdbar \c X^\nu_0\,
				e^{ik\cdot X_0}
			}}}
			+ \normorder{\wick{
				\bcontraction[1.1ex]{}{T}{_z\,\pd}{X}
				\c T_z\,
				\pd X^\mu_0\,
				\pdbar X^\nu_0\,
				\c e^{ik\cdot X_0}
			}}
		\\
		&\sim \pqty{
				\frac{1}{z^2}\, G^{\mu\nu}_k(0)
				+ \frac{1}{z}\, \normorder{
					\pd^2 X^\mu_0\,
					\pdbar X^\nu_0\,
					e^{ik\cdot X_0}
				}
			}
			+ \pqty{
				\frac{\alpha'k^2}{4}
				\frac{1}{z^2}\, G^{\mu\nu}_k(0)
				+ \frac{1}{z}\, \normorder{
					\pd X^\mu_0\,
					\pdbar X^\nu_0\,
					\pd\,e^{ik\cdot X_0}
				}
			}
			\\ &\qquad
			- \frac{2}{\alpha'}
				\pqty{
					-\frac{\alpha'}{2}
					\eta^{\sigma\mu}
					\frac{1}{z^2}
				} \pqty{
					-\frac{\alpha'}{2}
					\frac{ik_\sigma}{z}
				}\, \normorder{
					\pdbar X^\nu_0\,
					e^{ik\cdot X_0}
				}
		\\[1ex]
		&\sim
			ik^\mu\,\normorder{
					\pdbar X^\nu_0\,
					e^{ik\cdot X_0}
				}\,
				\pqty{-\frac{\alpha'}{2}}
				\frac{1}{z^3}
			+ \pqty{
					1 + \frac{\alpha'k^2}{4}
				}\, \frac{G^{\mu\nu}_k(0)}{z^2}
			+ \frac{\pd G^{\mu\nu}_k(0)}{z},
		\\[-1.25\baselineskip]
	\end{aligned}\\[.8\baselineskip]
		\tilde{T}(\bar{z})\,G^{\mu\nu}_k(0)
		\sim
			ik^\nu\,\normorder{
					\pd X^\mu_0\,
					e^{ik\cdot X_0}
				}\,
				\pqty{-\frac{\alpha'}{2}}
				\frac{1}{\bar{z}^3}
			+ \pqty{
					1 + \frac{\alpha'k^2}{4}
				}\, 
				\frac{G^{\mu\nu}_k(0)}{\bar{z}^2}
			+ \frac{\pd G^{\mu\nu}_k(0)}{\bar{z}},
	\end{gather}
	Therefore, $G_{e,k}$ is a primary of weight $(1,1)$ iff.~$
		1 + \frac{\alpha'k^2}{4} = 1
	$ and $
		k^\mu e_{\mu\nu}
		= 0 = k^\nu e_{\mu\nu}
	$. The first equation gives the mass shell condition $
		m^2 = -k^2 = 0
	$ for a massless boson, while the second equation constrains the polarization to be transverse. These are the physical constraints for a massless gauge boson, which is the level 1 excitation for a bosonic closed string. 
	
	\item The form of any primary 3-point function is completely fixed by $\mrm{PSL}(2,\mbb{C})$ invariance\footnote{
		Reference: Blumenhagen, \textit{Introduction to CFT}, and also \textit{Di Francesco et al}. 
	}. In fact, for any holomorphic $\phi_i(z_i)$ with weight $h_i$, by translational invariance, we have:
	\begin{equation}
		\ave{
			\phi_1(z_1)\,
			\phi_2(z_2)\,
			\phi_3(z_3)
		}
		= f(z_{12},z_{23},z_{31}),\quad
		z_{ij} = z_i - z_j,
	\end{equation}
%	Here for simplicity we write $\phi_i(z_i)$ in stead of $\phi_i(z_i,\bar{z}_i)$, but the anti-holomorphic dependence is still implied. 
	Furthermore, scaling invariance requires that $f$ is quasi-homogeneous:
	\begin{gather}
	\begin{aligned}
		z\mapsto z' = \lambda^{-1} z,\quad
		f &\mapsto \ave[\big]{
			\lambda^{h_1} \phi_1(\lambda z_1)\,
			\lambda^{h_2} \phi_2(\lambda z_2)\,
			\lambda^{h_3} \phi_3(\lambda z_3)
		} \\[.5ex]
		&= \lambda^{h_1 + h_2 + h_3} f(
				\lambda z_{12},
				\lambda z_{23},
				\lambda z_{31}
			) \\
		&= f(z_{12},z_{23},z_{31}),
	\end{aligned}
	\\[1ex]
		f = \!\! \sum_{
				a + b + c = \sum_i \! h_i
			} f_{abc}
		= \!\! \sum_{
				a + b + c = \sum_i \! h_i
			}
			\frac{C_{abc}}{
				z_{12}^a
				z_{23}^b
				z_{31}^c
			}
	\end{gather}
	
	On the other hand, for special conformal transformation\footnote{
		See \textit{Di Francesco et al}, and also \https{github.com/davidsd/ph229}. 
	} $
		\frac{1}{\bar{z}}
		\mapsto \frac{1}{\bar{z}'}
		= \frac{1}{\bar{z}} + a
	$, we have:
	\begin{gather}
		z\ \longmapsto\ 
		z' = \frac{1}{\frac{1}{z} + \bar{a}}
		= \frac{z}{1 + z\bar{a}}
		= w(z),\quad
		\pdv{z}{z'}
		= \frac{1}{(1 - z\bar{a})^2}
		= \frac{1}{\kappa^2},\quad
		z_{ij}
		= \frac{z'_{ij}}{\kappa_i \kappa_j},\\
		f \ \longmapsto\ %
		f\pqty\Big{
				w^{-1}(z_{12}),
				w^{-1}(z_{23}),
				w^{-1}(z_{31})
			} \frac{1}{
				\kappa_1^{2h_1}
				\kappa_2^{2h_2}
				\kappa_3^{2h_3}
			}
		= f(z_{12},z_{23},z_{31}),\\
		f_{abc}\pqty\Big{
				w^{-1}(z_{12}),
				w^{-1}(z_{23}),
				w^{-1}(z_{31})
			}
		= f_{abc}(z_{12},z_{23},z_{31})\,
			\kappa_1^{c + a}
			\kappa_2^{a + b}
			\kappa_3^{b + c},
	\end{gather}
	We see that $f$ is invariant under special conformal transformation iff.\ $f = f_{abc}$ where:
	\begin{gather}
		c + a = 2h_1,\quad
		a + b = 2h_2,\quad
		b + c = 2h_3,\\
		\text{i.e.}\quad
		a = h_1 + h_2 - h_3,\quad
		b = h_2 + h_3 - h_1,\quad
		c = h_3 + h_1 - h_2,
	\end{gather}
	
	In the above discussions we've restricted $\phi_i$ to be holomorphic; for \textit{spin-less} $
		\phi_i = \phi_i(z,\bar{z}),\ %
		h_i = \tilde{h}_i$, $
		\Delta_i = h_i + \tilde{h}_i
	$, the holomorphic and anti-holomorphic contributions can be nicely combined, and we have:
	\begin{gather}
		f = \frac{C}{
			\abs{z_{12}}^{2a}
			\abs{z_{23}}^{2b}
			\abs{z_{31}}^{2c}
		},\\[1ex]
		2a = \Delta_1 + \Delta_2 - \Delta_3,\quad
		2b = \Delta_2 + \Delta_3 - \Delta_1,\quad
		2c = \Delta_3 + \Delta_1 - \Delta_2,\\
		\ave{
			V_{k_1}(z_1,\bar{z}_1)\,
			V_{k_2}(z_2,\bar{z}_2)\,
			G_{e,k_3}(z_3,\bar{z}_3)
		}
		= \frac{A(k_1,k_2,e)}{
			\abs{z_{12}}^{2}
			\abs{z_{23}}^{2}
			\abs{z_{31}}^{2}
		}
	\end{gather}
	
	\item Following the recipe in (a), we have:
	\begin{equation}
	\begin{aligned}
		V_{k_1}(z_1,\bar{z}_1)\,
		V_{k_2}(z_2,\bar{z}_2)
		&= \normorder{
			\exp \pqty\Big{
				ik_{1,\mu}
				ik_{2,\nu} 
				\wick{
					\c X^\mu_1
					\c X^\nu_2
				}
			}\,
			e^{ik_1\cdot X_1}\,
			e^{ik_2\cdot X_2}
		} \\
		&= \exp \pqty{
				\frac{\alpha'}{2}\,
				k_1\cdot k_2
				\ln \abs{z_{12}}^2
			}\, \normorder{
				e^{ik_1\cdot X_1}\,
				e^{ik_2\cdot X_2}
			} \\
		&= \abs{z_{12}}^{
				\alpha' k_1\cdot k_2
			}\, \normorder{
				e^{ik_1\cdot X_1}\,
				e^{ik_2\cdot X_2}
			}
	\end{aligned}
	\end{equation}
	Apply the on-shell conditions, and we find that:
	\begin{equation}
		\alpha' k_1\cdot k_2
		= \frac{\alpha'}{2} (k_1 + k_2)^2
			- \frac{\alpha'}{2} k_1^2
			- \frac{\alpha'}{2} k_2^2
		= \frac{\alpha'}{2} (-k_3)^2
			- \frac{\alpha'}{2} k_1^2
			- \frac{\alpha'}{2} k_2^2
		= 0 - 2 - 2
		= -4
	\end{equation}
	We are interested in the $\abs{z_{12}}^{-2}$ term in the OPE around $z_2$; it will contribute to the 3-point function discussed in (b). Note that:
	\begin{gather}
	\begin{aligned}
		\normorder{
			e^{ik_1\cdot X_1}\,
			e^{ik_2\cdot X_2}
		}
%		&= \normorder{
%			\exp{ik_1\cdot \pqty{
%				X_2
%				+ z_{12}\,\pd X_2
%				+ \bar{z}_{12}\,\pdbar X_2
%				+ \frac{1}{2}\,z_{12}^2\,
%					\pd^2 X_2
%				+ \frac{1}{2}\,\bar{z}_{12}^2\,
%					\pdbar^2 X_2
%				+ \cdots
%			}}\,
%			e^{ik_2\cdot X_2}
%		} \\
		&= \normorder{
			\pqty{
				\cdots
				+ \frac{1}{2} (ik_1\cdot X_1)^2
				+ \cdots
			}\, e^{ik_2\cdot X_2}
		} \\
		&= \normorder{
			\pqty{
				\cdots
				- \frac{1}{2}\,k_1^\mu k_1^\nu
					\pqty{
						X_2
						+ z_{12}\,\pd X_2
						+ \bar{z}_{12}\,\pdbar X_2
						+ \cdots
					}_\mu 
					\pqty\big{\cdots}_\nu
					+ \cdot
			}\, e^{ik_2\cdot X_2}
		} \\
		&= \normorder{
			\pqty\Big{
				\cdots
				- k_{1,\mu} k_{1,\nu}\,
					\pqty{
						z_{12} \bar{z}_{12}\,
						\pd X^\mu_2\,\pdbar X^\nu_2
					}
				+ \cdots
			}\, e^{ik_2\cdot X_2}
		} \\
		&= \cdots
			- \abs{z_{12}}^2\,
				k_{1,\mu} k_{1,\nu}\,
				G_{k_2}^{\mu\nu}(z_2,\bar{z}_2)
			+ \cdots,
	\end{aligned}\\[1ex]
		V_{k_1}(z_1,\bar{z}_1)\,
		V_{k_2}(z_2,\bar{z}_2)
		= \cdots
			+ \frac{O_{k_1,k_2}(z_2,\bar{z}_2)}{
				\abs{z_{12}}^2
			}
			+ \cdots,\\[.5ex]
		O_{k_1,k_2}(z_2,\bar{z}_2)
		= - k_{1,\rho} k_{1,\sigma}\,
			G_{k_2}^{\rho\sigma}(z_2,\bar{z}_2),
	\end{gather}
	
	Consider the same limit: $z_1 \to z_2$ of the 3-point function, and we find that:
	\begin{equation}
	\begin{aligned}
		z_1 \to z_2,\quad
		\ave{
			V_{k_1}(z_1,\bar{z}_1)\,
			V_{k_2}(z_2,\bar{z}_2)\,
			G_{e,k_3}(z_3,\bar{z}_3)
		}
		&\to \frac{1}{\abs{z_{12}}^2}
			\frac{A(k_1,k_2,e)}{
				\abs{z_{23}}^4
			} \\
		&\sim \frac{1}{\abs{z_{12}}^2}\,
			\ave[\big]{
				O_{k_1,k_2}(z_2,\bar{z}_2)\,
				G_{e,k_3}(z_3,\bar{z}_3)
			},
	\end{aligned}
	\end{equation}
	We see that in the $z_2\to z_3$ limit, we should obtain:
	\begin{equation}
		O_{k_1,k_2}(z_2,\bar{z}_2)\,
		G_{e,k_3}(z_3,\bar{z}_3)
		= \cdots
			+ \frac{A(k_1,k_2,e)}{
				\abs{z_{23}}^4
			}
			+ \cdots,
	\end{equation}
	
	Note that $A(k_1,k_2,e) = A(k_1,k_2,e)\,\idty$ is simply a number; therefore, when finding $A(k_1,k_2,e)$, it is safe to ignore all (non-identity) operator contributions, as they should cancel each other. 
	Similar to (a), we have:
	\begin{equation}
	\begin{aligned}
		G^{\rho\sigma}_{k_2}(z_2, \bar{z}_2)\,
		G^{\mu\nu}_{k_3}(z_3, \bar{z}_3)
		&= \cdots + \normorder{\wick{
				\pd \c1 X^\rho_2\,
				\pdbar \c2 X^\sigma_2\,
				\c3 e^{ik_2\cdot X_2}\,
				\pd \c1 X^\mu_3\,
				\pdbar \c2 X^\nu_3\,
				\c3 e^{ik_3\cdot X_3}
			}}
			\\ & \phantom{{} = \cdots}
			+ \normorder{\wick{
				\pd \c1 X^\rho_2\,
				\pdbar \c2 X^\sigma_2\,
				\c3 e^{ik_2\cdot X_2}\,
				\pd \c1 X^\mu_3\,
				\pdbar \c3 X^\nu_3\,
				\c2 e^{ik_3\cdot X_3}
			}}
			\\ & \phantom{{} = \cdots}
			+ \normorder{\wick{
				\pd \c1 X^\rho_2\,
				\pdbar \c2 X^\sigma_2\,
				\c3 e^{ik_2\cdot X_2}\,
				\pd \c3 X^\mu_3\,
				\pdbar \c2 X^\nu_3\,
				\c1 e^{ik_3\cdot X_3}
			}}
			\\ & \phantom{{} = \cdots}
			+ \normorder{\wick{
				\bcontraction{
					\pd X^\rho_2\,
					\pdbar}{X}{^\sigma_2\,
					e^{ik_2\cdot X_2}\,
					\pd X^\mu_3\,
					\pdbar X^\nu_3\,
				}{e}
				\bcontraction[1.7ex]{
					\pd X^\rho_2\,
					\pdbar X^\sigma_2\,
					}{e}{^{ik_2\cdot X_2}\,
					\pd}{X}
				\pd \c1 X^\rho_2\,
				\pdbar X^\sigma_2\,
				\c2 e^{ik_2\cdot X_2}\,
				\pd X^\mu_3\,
				\pdbar \c2 X^\nu_3\,
				\c1 e^{ik_3\cdot X_3}
			}}
			+ \cdots
	\end{aligned}
	\end{equation}
	\begin{equation*}
	\begin{aligned}
		G^{\rho\sigma}_{k_2}(z_2, \bar{z}_2)\,
		G^{\mu\nu}_{k_3}(z_3, \bar{z}_3)
		&\sim \cdots
			+ \pqty{
					-\frac{\alpha'}{2}
					\eta^{\rho\mu}
					\frac{1}{z^2_{23}}
				}
				\pqty{
					-\frac{\alpha'}{2}
					\eta^{\sigma\nu}
					\frac{1}{\bar{z}^2_{23}}
				}\times 1
			\\ & \phantom{{} = \cdots}
			+ \pqty{
					-\frac{\alpha'}{2}
					\eta^{\rho\mu}
					\frac{1}{z^2_{23}}
				}
				\pqty{
					-\frac{\alpha'}{2}
					\frac{ik^\sigma_3}{\bar{z}_{23}}
				}
				\pqty{
					-\frac{\alpha'}{2}
					\frac{ik^\nu_2}{\bar{z}_{32}}
				}
			+ \pqty\big{
					z\leftrightarrow\bar{z},\ %
					\rho\leftrightarrow\sigma,\ %
					\mu\leftrightarrow\nu
				}
			\\ & \phantom{{} = \cdots}
			+ \pqty{
					-\frac{\alpha'}{2}
					\frac{ik_2^\rho}{{z}_{23}}
				}
				\pqty{
					-\frac{\alpha'}{2}
					\frac{ik_2^\sigma}{\bar{z}_{23}}
				}
				\pqty{
					-\frac{\alpha'}{2}
					\frac{ik_3^\mu}{{z}_{32}}
				}
				\pqty{
					-\frac{\alpha'}{2}
					\frac{ik_3^\nu}{\bar{z}_{32}}
				}
			+ \cdots
%		\\[-1.15\baselineskip]
	\end{aligned}
	\end{equation*}
	\begin{equation}
	\begin{aligned}
		O_{k_1,k_2}(z_2, \bar{z}_2)\,
		G^{\mu\nu}_{k_3}(z_3, \bar{z}_3)
		&\sim \cdots
			- k_1^\mu k_1^\nu \pqty{
					\frac{\alpha'^2}{4}
				} \frac{1}{\abs{z_{23}}^4}
			\\ & \phantom{{} = \cdots}
			- i^2 \pqty{
					k_1^\mu k_2^\nu
					+ k_1^\nu k_2^\mu
				} 
				\pqty{
					\frac{\alpha'}{2}
					(k_1\cdot k_3)
				}
				\pqty{
					\frac{\alpha'^2}{4}
				} \frac{1}{\abs{z_{23}}^4}
			\\ & \phantom{{} = \cdots}
			- i^4 k_3^\mu k_3^\nu \pqty{
					\frac{\alpha'}{2}
					(k_1\cdot k_2)
				}^{\!\!2}
				\pqty{
					\frac{\alpha'^2}{4}
				} \frac{1}{\abs{z_{23}}^4}
			+ \cdots
	\end{aligned}
	\end{equation}
	Again, apply the on-shell conditions, and we find that:
	\begin{gather}
		\frac{\alpha'}{2} k_1\cdot k_2
		= -2,\quad
		\frac{\alpha'}{2} k_1\cdot k_3
		= -\frac{\alpha'}{2} k_1\cdot (k_1 + k_2)
		= -\frac{\alpha'}{2} k_1^2 
			-\frac{\alpha'}{2} k_1\cdot k_2
		= -2 - (-2) = 0,\\[1ex]
%		k_3^\mu k_3^\nu
%		= (k_1 + k_2)^\mu (k_1 + k_2)^\nu
%		= k_1^\mu k_1^\nu + k_2^\mu k_2^\nu
%			+ (k_1^\mu k_2^\mu + k_1^\nu k_2^\mu),
	\begin{aligned}
		A(k_1,k_2,e)
		&= - \frac{\alpha'^2}{4} \pqty\big{
				4 \cancel{
					e_{\mu\nu} k_3^\mu k_3^\nu
				}
				+ e_{\mu\nu} k_1^\mu k_1^\nu
			}
		= - \frac{\alpha'^2}{4}
			e_{\mu\nu} k_1^\mu k_1^\nu \\
		&= - \frac{\alpha'^2}{4}
			e_{\mu\nu} (k_2+k_3)^\mu (k_2+k_3)^\nu
		= - \frac{\alpha'^2}{4}
			e_{\mu\nu} k_2^\mu k_2^\nu \\
		&= - \frac{\alpha'^2}{8}
			e_{\mu\nu} \pqty\big{
				k_1^\mu k_1^\nu
				+ k_2^\mu k_2^\nu
			} \\
		&= - \frac{\alpha'^2}{8}
			e_{\mu\nu} \pqty\Big{
				k_{12}^\mu k_{12}^\nu
				+ \pqty{
					k_1^\mu k_2^\nu
					+ k_1^\nu k_2^\mu
				} 
			},
	\end{aligned}
	\end{gather}
	On the other hand, 
	\begin{gather}
		0 = e_{\mu\nu} k_3^\mu k_3^\nu
		= e_{\mu\nu} (k_1+k_2)^\mu (k_1+k_2)^\nu
		= e_{\mu\nu} \pqty\Big{
				k_{12}^\mu k_{12}^\nu
				+ 2\pqty{
					k_1^\mu k_2^\nu
					+ k_1^\nu k_2^\mu
				} 
			}\\
		A(k_1,k_2,e)
		= - \frac{\alpha'^2}{8}
			e_{\mu\nu} k_{12}^\mu k_{12}^\nu
			\pqty{
				1 - \frac{1}{2}
			}
		= -\frac{\alpha'^2}{16}
			e_{\mu\nu}\, k_{12}^\mu k_{12}^\nu
	\end{gather}
	
	\end{enumerate}
	
	\item \textbf{Strings Scattering Off a Heavy Particle:}
	
	A heavy particle can be modeled by some D0-brane with Neumann boundary condition in the $X_0$ direction\footnote{
		Reference: \arxiv{hep-th/9611214}, \arxiv{hep-th/9605168}, and \textit{Polchinski}. 
	}. The scattering of a closed string tachyon off the heavy particle can then be computed via a disc diagram with two insertions. 
	
	\begin{enumerate}
	\item The conformal Killing group (CKG) of the disc is $\mrm{PSL}(2,\mbb{R})$. It is a 3 dimensional $\mbb{R}$ Lie group, generated by 3 conformal Killing vectors (CKV's); therefore, it is possible to partially fix the positions of the two insertions $V_1,V_2$. On the upper half plane, this can be implemented by putting $z_1,z_2$ on the imaginary axis, with $z_2$ fixed and $z_1$ integrated\footnote{
		Reference: \arxiv{0812.4408}. I would like to thank Lucy Smith for pointing this out. 
	}:
	\begin{equation}
		\mcal{A}
		= g_c^2 e^{-\lambda}
			\int_0^{z_2} \dd{z_1}
			\ave[\Big]{
				\normorder{
					c^x_1
					e^{ik_1\cdot X_1}
				}\,
				\normorder{
					c_2\tilde{c}_2\,
					e^{ik_2\cdot X_2}
				}
			},\quad
		z_2 = i,\quad
		z_1 = iy,\quad y\in[0,1]
	\end{equation}
	Here $c^x$ comes from the CKV that brings $z_1\to iy$. On the disc this can be taken to be a rotation around $z_2$; when mapped to the upper half plane and at around the imaginary axis, this is simply a translation along the $x = \frac{1}{2} \pqty{
		z + \bar{z}
	}$ direction\footnote{
		Reference: \textit{Polchinski}, Chapter 5 \& 6. 
	}, i.e.
	\begin{gather}
		\text{CKV}\colon\ \pdd{x} = \delta^a_x\,\pdd{a}
		\quad\Longrightarrow\quad
		\text{Ghost}\colon\ c^x,\\
		c^x \pdd{x}
		+ c^y \pdd{y}
			= c^z \pdd{z}
			+ c^{\bar{z}} \pdd{\bar{z}},\quad
		c^x = \frac{1}{2} \pqty{
				c^z + c^{\bar{z}}
			}
			= \frac{1}{2} \pqty\big{
				c(z) + \tilde{c}(\bar{z})
			},
	\end{gather}
	The ghost contribution is then:
	\begin{equation}
	\begin{aligned}
		\ave[\big]{
			c^x_1 c_2 \tilde{c}_2
		}
		= \ave[\big]{
			c^x(z_1)\,c(z_2)\,\tilde{c}(\bar{z}_2)
		}
		&= \frac{1}{2} \pqty\Big{
			\ave[\big]{
				c(z_1)\,c(z_2)\,\tilde{c}(\bar{z}_2)
			}
			+ \ave[\big]{
				\tilde{c}(z_1)\,c(z_2)\,
				\tilde{c}(\bar{z}_2)
			}
		} \\[.5ex]
		&= \frac{1}{2} \pqty\Big{
			\ave[\big]{
				c(z_1)\,c(z_2)\,{c}(z'_2)
			}
			+ \ave[\big]{
				{c}(z'_1)\,c(z_2)\,
				{c}(z'_2)
			}
		}, \quad z' = \bar{z}, \\[.5ex]
		&= \frac{C^g_{D^2}}{2} \pqty{
			z_{12} z_{12'} z_{22'}
			+ z_{1'2} z_{1'2'} z_{22'}
		},\quad z_1,z_2\in i\mbb{R}, \\[.2ex]
		&= 2C^g_{D^2} \pqty{
				z_1^2 - z_2^2
			} z_2
	\end{aligned}
	\end{equation}
	
	On the other hand, the $e^{ik_j\cdot X_j}$ contribution is very similar to what we compute in \boxed{1}\,, except that now we should be careful about the boundary conditions of $X^\mu$ on the upper half plane, which affect the $XX$ contraction in the formulae. For Neumann boundary condition: $\pdd{y} X^0 = 0$, the half-plane propagator from $z'$ can be constructed with an image at $\bar{z}'$ with \textit{the same charge}, i.e.\ we have:
	\begin{equation}
		\wick{ \c X^0_1 \c X^0_2 }
		= - \frac{\alpha'}{2}\, \eta^{00}
				\ln \abs{z_1 - z_2}^2
			- \frac{\alpha'}{2}\, \eta^{00}
				\ln \abs{z_1 - \bar{z}_2}^2
	\end{equation}
	While for Dirichlet boundary $X^i = \mrm{const}$, we can always select the origin so that $X^i = 0$, and in this case the image should have \textit{the opposite charge}, i.e.
	\begin{gather}
		\wick{ \c X^i_1 \c X^j_2 }
		= - \frac{\alpha'}{2}\, \delta^{ij}
				\ln \abs{z_1 - z_2}^2
			+ \frac{\alpha'}{2}\, \delta^{ij}
				\ln \abs{z_1 - \bar{z}_2}^2,
	\\
	\begin{aligned}
		\Longrightarrow\quad
		\normorder{
			e^{ik_1\cdot X_1}
		}\,
		\normorder{
			e^{ik_2\cdot X_2}
		}
		&= \exp \pqty{
				ik_{1,\mu}
				ik_{2,\nu} 
				\wick{
					\c X^\mu_1
					\c X^\nu_2
				}
			}\, \normorder{
				e^{ik_1\cdot X_1}\,
				e^{ik_2\cdot X_2}
			} \\
		&= \abs{z_{12}}^{
				\alpha' k_1\cdot k_2
			}
			\abs{z_{1\bar{2}}}^{
				\alpha' \pqty{
					- k_1^0 k_2^0
					- \delta_{ij} k_1^i k_2^j
				}
			}\, \normorder{
				e^{ik_1\cdot X_1}\,
				e^{ik_2\cdot X_2}
			}
	\end{aligned}
	\end{gather}
	
	Before further calculations, we note that the normal ordering defined here on $D^2$ differs from that on the usual $\mbb{C}^2$; in fact, there are also self-contractions with image charge\footnote{
		This is very much similar to the torus situation, where we also have to consider self-contractions with image charges. More rigorous discussion of $G^r$ is given in \textit{Polchinski}. 
	}:
	\begin{gather}
		\wick{ \c X^\mu(z,\bar{z})\, \c X^\nu(\bar{z},z) }
		= G^{\mu\nu}_r(z,\bar{z})
		= \mp \frac{\alpha'}{2}\, \eta^{\mu\nu}
				\ln \abs{z - \bar{z}}^2,
	\\
		\Longrightarrow\quad
		\ave[\Big]{\normorder{
				e^{ik_1\cdot X_1}\,
				e^{ik_2\cdot X_2}
			}}_{D^2}
		= \ave[\Big]{\normorder{
				e^{ik_1\cdot X_1}\,
				e^{ik_2\cdot X_2}
			}}_{\mbb{C}^2} 
			\exp \pqty{
				\frac{1}{2} \sum_n
				ik_{n,\mu}
				ik_{n,\nu} 
				\wick{
					\c X^\mu_n
					\c X^\nu_n
				}
			},\quad n = 1,2
	\end{gather}
	The $\mquote{\mp}$ sign choice depends on the boundary condition. 
	
	Therefore,
	\begin{equation}
	\begin{aligned}
		\ave[\Big]{
			\normorder{
				e^{ik_1\cdot X_1}
			}\,
			\normorder{
				e^{ik_2\cdot X_2}
			}
		}_{D^2}
		&= \ave[\Big]{\normorder{
				e^{ik_1\cdot X_1}\,
				e^{ik_2\cdot X_2}
			}}_{\mbb{C}^2} 
			\exp \pqty{
				ik_{1,\mu}
				ik_{2,\nu} 
				\wick{
					\c X^\mu_1
					\c X^\nu_2
				}
			}
			\exp \pqty{
				\frac{1}{2} \sum_n
				ik_{n,\mu}
				ik_{n,\nu} 
				\wick{
					\c X^\mu_n
					\c X^\nu_n
				}
			} \\
		&= \ave[\Big]{\normorder{
				e^{ik_1\cdot X_1}\,
				e^{ik_2\cdot X_2}
			}}_{\mbb{C}^2} 
			\exp \pqty{
				\frac{1}{2} \sum_{m,n}
				ik_{m,\mu}
				ik_{n,\nu} 
				\wick{
					\c X^\mu_m
					\c X^\nu_n
				}
			} \\
		&= \ave[\Big]{\normorder{
				e^{ik_1\cdot X_1}\,
				e^{ik_2\cdot X_2}
			}}_{\mbb{C}^2}
			\abs{z_{12}}^{
				\alpha' k_1\cdot k_2
			}
			\abs{z_{1\bar{2}}}^{
				\alpha' \pqty{
					- k_1^0 k_2^0
					- \vb{k}_1 \cdot \vb{k}_2
				}
			}
			\prod_n
				\abs{z_{n\bar{n}}}^{
					\frac{\alpha'}{2} \pqty{
						- (k_n^0)^2
						- \vb{k}_n^2
					}
				}
	\end{aligned}
	\end{equation}
	
	Note that $X^i$ has no zero mode due to the Dirichlet boundary, hence $\int \DD X$ gives a delta function in only the Neumann direction: $
		\delta\pqty{k_1^0 + k_2^0}
	$. Physically, this means that only the energy is conversed; the momentum $k^i$ is not conserved since the heavy D0-brane does not recoil. It is therefore convenient to define these on shell variables:
	\begin{gather}
		s = \omega^2 = (k_1^0)^2 = (k_2^0)^2,\quad
		t = - (\vb{k}_1 + \vb{k}_2)^2
		= - \vb{k}_1^2 - \vb{k}_2^2
			- 2\vb{k}_1 \cdot \vb{k}_2
		= 2 \pqty{
				-\omega^2 - \vb{k}_1 \cdot \vb{k}_2
				- \frac{4}{\alpha'}
			},\\
		\vb{k_1} \cdot \vb{k_2}
		= -\frac{t}{2} - \omega^2
			- \frac{4}{\alpha'},\quad
		k_1 \cdot k_2
		= -\omega(-\omega) + \vb{k_1} \cdot \vb{k_2}
	\end{gather}
	Here we've used the on-shell condition: $
		m^2 = -k^2
		= \omega^2 - \vb{k}^2
		= -\frac{4}{\alpha'}
	$ for tachyons. The previous expressions can then be simplified to:
	\begin{gather}
	\begin{aligned}
		\ave[\Big]{
			\normorder{
				e^{ik_1\cdot X_1}
			}\,
			\normorder{
				e^{ik_2\cdot X_2}
			}
		}_{D^2}
		&= \ave[\Big]{\normorder{
				e^{ik_1\cdot X_1}\,
				e^{ik_2\cdot X_2}
			}}_{\mbb{C}^2}
			\abs{z_{12}}^{
				- \frac{\alpha't}{2} - 4
			}
			\abs{z_{1\bar{2}}}^{
				+ \frac{\alpha't}{2} + 4
				+ 2\alpha'\omega^2
			}
			\prod_n
				\abs{2z_n}^{
					- \alpha'\omega^2 - 2
				} \\
		&= {
				i C^X_{D^2}\,
				2\pi\,\delta\pqty{k_1^0 + k_2^0}
			}\,
			\abs{z_{12}}^{
				- \frac{\alpha't}{2} - 4
			}
			\abs{z_{1\bar{2}}}^{
				+ \frac{\alpha't}{2} + 4
				+ 2\alpha'\omega^2
			}
			\prod_n
				\abs{2z_n}^{
					- \alpha'\omega^2 - 2
				} \\
		&= {
				i C^X_{D^2}\,
				2\pi\,\delta\pqty{k_1^0 + k_2^0}
			}\,
			f\pqty\big{
				\abs{z_{12}},\abs{z_{1\bar{2}}},
				\abs{z_1},\abs{z_2}
			},
	\end{aligned}
	\\[2ex]
	\begin{aligned}
		\mcal{A}
		&= g_c^2 \underline{e^{-\lambda}}\cdot {
				i \underline{C^X_{D^2}}\,
				2\pi\,\delta\pqty{k_1^0 + k_2^0}
			}\cdot 2\underline{C^g_{D^2}}\,
			\int_0^{z_2} \dd{z_1}
				\pqty{
					z_1^2 - z_2^2
				} z_2\,
				f\pqty\big{
					\abs{z_{12}},\abs{z_{1\bar{2}}},
					\abs{z_1},\abs{z_2}
				} \\
		&= g_c^2 \underline{C_{D^2}}\,
			2\pi\,\delta\pqty{k_1^0 + k_2^0}
			\cdot 2i
			\int_0^1 i\dd{y}
				\pqty{
					(iy)^2 - i^2
				}\,i\cdot
				f\pqty\big{
					1 - y, 1 + y,
					2y, 2
				} \\
		&= -ig_c^2 C_{D^2}\,
			2\pi\,\delta\pqty{k_1^0 + k_2^0}
			\cdot 2\cdot 2^{-2\alpha'\omega^2 - 4}
			\int_0^1 \dd{y}
				(1 - y^2)\,
				f\pqty\big{
					1 - y, 1 + y,
					y, 1
				},
	\end{aligned}
	\\[1ex]
	\begin{aligned}
		\int_0^1 \dd{y}
			(1 - y^2)\,
			f\pqty\big{
				1 - y, 1 + y,
				y, 1
			}
		&= \int_0^1 \dd{y}
				(1 - y)^{
					- \frac{\alpha't}{2} - 4 + 1
				}
				(1 + y)^{
					+ \frac{\alpha't}{2} + 4
					+ 2\alpha'\omega^2 + 1
				}
				y^{
					- \alpha'\omega^2 - 2
				} \\
		&= \int_0^1 \dd{y}
				y^{a-1}
				(1 - y)^{2b-1}
				(1 + y)^{-2a-2b+1},\quad
			t = \frac{1 - y}{1 + y}, \\
		&= - 2^{1-2a} \int_0^1 \dd{t}
				(-t)^{2b-1}
				(1 - t^2)^{a-1} \\
		&= 2^{-2a} \int_0^1 \dd{(t^2)}
				(t^2)^{b-1}
				(1 - t^2)^{a-1} \\
		&= 2^{-2a} B\pqty{
				a = -\alpha'\omega^2 - 1,\,
				b = -\frac{\alpha't}{4} - 1
			} \\
	\end{aligned}
	\end{gather}
	Here $
		B(a,b)
		= \frac{\Gamma(a)\,\Gamma(b)}{\Gamma(a+b)}
	$ is the Euler Beta function. 
	
	Putting everything together, we obatin:
	\begin{gather}
	\begin{aligned}
		\mcal{A}
		&= -ig_c^2 C_{D^2}\,
			2\pi\,\delta\pqty{k_1^0 + k_2^0}
			\cdot \frac{1}{2}\,
			B\pqty{
				-\alpha'\omega^2 - 1,\,
				-\frac{\alpha't}{4} - 1
			} \\
		&= -ig_c^2 C_{D^2}\,
			\pi\,\delta\pqty{k_1^0 + k_2^0}\,
			B\pqty{
				-\alpha'\omega^2 - 1,\,
				-\frac{\alpha't}{4} - 1
			}
	\end{aligned}
	\end{gather}
	In fact $C_{D^2}$ can be further computed by path integral or by comparing physical results. Here we settle for this generic coefficient since it's already enough for our following discussions\footnote{
		And I have run out of time and energy. 
	}. 
	
	\item The Regge limit is found by taking the high energy limit while keeping the momentum transfer fixed; in this case it is achieved by:
	\begin{gather}
		\text{Regge:}\quad
			s = \omega^2 \to\infty,\quad
			t = - (\vb{k}_1 + \vb{k}_2)^2
				\,\ \text{fixed},
	\\[1ex]
	\begin{aligned}
		\mcal{A} \propto
			B\pqty{
				a = -\alpha's - 1,\,
				b = -\frac{\alpha't}{4} - 1
			}
		&= \frac{\Gamma\pqty{
				-\alpha's - 1
			}}{\Gamma\pqty{
				-\alpha's - \frac{\alpha't}{4} - 2
			}}\,
			\Gamma\pqty{-\frac{\alpha't}{4} - 1} \\
		&\sim \Bqty{
				e\pqty{
					\alpha's
					+ \frac{\alpha't}{4} + 3
				}
			}^{\frac{\alpha't}{4} + 1}\,
			\Gamma\pqty{-\frac{\alpha't}{4} - 1} \\[.5ex]
		&\sim \pqty{
				e\alpha'\omega^2
			}^{\frac{\alpha't}{4} + 1}\,
			\Gamma\pqty{-\frac{\alpha't}{4} - 1} \\
		&\sim \pqty{
				\omega^2
			}^{\frac{\alpha't}{4} + 1}\,
			\Gamma\pqty{-\frac{\alpha't}{4} - 1}
	\end{aligned}
	\end{gather}
	Here we've used the Stirling's approximation\footnote{
		For the validity of Stirling's approximation when $z\in\mbb{C}$ and $\abs{z}\to\infty$, see \wikiref{https://en.wikipedia.org/wiki/Stirling\%27s\_approximation\#Stirling's\_formula\_for\_the\_gamma\_function}{Stirling's formula for the gamma function}. 
	}: $
		\ln \Gamma(z + 1)
		= \ln z!
		\sim z\ln z - z
	$. On the other hand, the hard scattering limit is found by keeping the scattering angle fixed, i.e.
	\begin{gather}
		\text{Hard scattering:}\quad
			s = \omega^2 \to\infty,\quad
			\pqty{t/s} \equiv \lambda\,\ \text{fixed},
	\\[1.5ex]
	\begin{aligned}
		\mcal{A}
		\propto B(a,b)
		= \frac{\Gamma(a)\,\Gamma(b)}{
				\Gamma(a+b)
			}
		&\sim \exp\Bqty{
				-\alpha'\pqty{
					s\ln (\alpha's)
					+ \tfrac{t}{4} \ln \tfrac{\alpha't}{4}
					+ \tfrac{u}{4} \ln \tfrac{\alpha'u}{4}
				}
			},
	\end{aligned}
	\\[1.5ex]
		s = \omega^2 = (k_1^0)^2 = (k_2^0)^2,\quad
		t = - (\vb{k}_1 + \vb{k}_2)^2,\quad
		u = - (\vb{k}_1 - \vb{k}_2)^2,
	\\[.5ex]
		s + \frac{t}{4} + \frac{u}{4} = - \frac{4}{\alpha'},
	\end{gather}
	Here we've introduced an additional $u$ variable, and we see that the result is symmetric under $
		t\leftrightarrow u
	$. We find that the amplititude exhibits similar limits as the Veneziano amplititude. 
	
	\item 
	
	
	
	\end{enumerate}
	
\legacyReference
	
	\item \textbf{Strings on Curved Space:}
	\begin{gather}
		S = \frac{1}{4\pi\alpha'}
			\int_M \dd[2]{\sigma} \sqrt{g}\,
			\pqty\Big{
				i\epsilon^{ab}
				B_{\mu\nu}(X)\,
				\pdd{a} X^\mu
				\pdd{b} X^\nu
				+ \cdots
			},\\
		T\id{^a_a} = -\frac{1}{2\alpha'}\,
			\beta^G_{\mu\nu}\, g^{ab}
				\pdd{a} X^\mu
				\pdd{b} X^\nu
			+ \cdots,\\
		\beta^G_{\mu\nu}
		= \alpha'R_{\mu\nu}
			- \frac{1}{4}\,\alpha'
				H_{\mu\lambda\omega}
				H\id{_\nu^{\lambda\omega}}
			+ \cdots
			+ \order{\alpha'^2}
	\end{gather}
	We want to verify the coefficient of $\alpha' H^2$ term in $\beta^G_{\mu\nu}$; for convenience we've omitted non-related terms in the above expressions. 
	
	Note that at $\order{\alpha'}$ such term does not depend on the metric $G_{\mu\nu}$, and it depends only on the field strength $H = \dd{B}$, not the potential $B$, hence it's safe to assume:
	\begin{gather}
		G_{\mu\nu} = \eta_{\mu\nu},\quad
		B_{\mu\nu}
		= \frac{1}{3} H_{\mu\nu\rho} X^\rho,\quad
		H = \mrm{const}, \\[0ex]
		i\epsilon^{ab}
			B_{\mu\nu}(X)\,
			\pdd{a} X^\mu
			\pdd{b} X^\nu
		= \frac{i}{3} H_{\mu\nu\rho}\,
			X^\rho
			\epsilon^{ab}
			\pdd{a} X^\mu
			\pdd{b} X^\nu, 
	\end{gather}
	We consider small perturbation away from the classical saddle: $
		X = X_0 + \xi
	$, then the 1-loop effective action is obtained by integrating over $\order{\xi^2}$ terms in the perturbed action\footnote{
		Reference: Prof.~Xi Yin's String Notes, see also \arxiv{0812.4408}. 
	}:
	\begin{gather}
		\Gamma^{(1)}[X_0]
		= - \ln \int \DD{\xi}\,
			e^{-S^{(2)}[X_0,\xi]},\\
	\begin{aligned}
		\mcal{L}^{(2)}
		&= \frac{i}{3}
		H_{\mu\nu\rho}\,\epsilon^{ab}
			\pqty\Big{
				\xi^\rho\,
					\pdd{a} X_0^\mu\,
					\pdd{b} \xi^\nu
				+ \xi^\rho\,
					\pdd{a} \xi^\mu\,
					\pdd{b} X_0^\nu
				+ X_0^\rho\,
					\pdd{a} \xi^\mu\,
					\pdd{b} \xi^\nu
			} \\
		&\sim \frac{i}{3}
		H_{\mu\nu\rho}\,\epsilon^{ab}
			\pqty\Big{
				\xi^\rho\,
					\pdd{a} X_0^\mu\,
					\pdd{b} \xi^\nu
				- \xi^\rho\,
					\pdd{a} X_0^\nu\,
					\pdd{b} \xi^\mu
				- \xi^\mu\,
					\pdd{a} X_0^\rho\,
					\pdd{b} \xi^\nu
			} \\
		&= \frac{i}{3}
		H_{\mu\nu\rho}\,\epsilon^{ab}
			\cdot 3\xi^\rho\,
				\pdd{a} X_0^\mu\,
				\pdd{b} \xi^\nu \\
		&= iH_{\mu\nu\rho}\,
			\epsilon^{ab}\,
				\pdd{a} X_0^\mu\,\pqty{
					\xi^\rho
					\pdd{b} \xi^\nu
				}
	\end{aligned}
	\end{gather}
	Here we've used the anti-symmetric properties of $H_{\mu\nu\rho},\epsilon^{ab}$, and ignored any total derivative after integration by parts. This term introduces a cubic interaction vertex in the free background; therefore, $\Gamma^{(1)}$ can be expressed in the following diagram\footnote{
		References: 
		\begin{itemize}[
			labelindent=3em,labelsep=1pt
		]
		\item David Tong, \href{https://www.damtp.cam.ac.uk/user/tong/string.html}{\textit{String Theory}};
		\item Callan \& Thorlacius, \href{https://www.damtp.cam.ac.uk/user/tong/string/sigma.pdf}{\textit{Sigma Models and String Theory}};
		\item Timo Weigand, \href{https://www.thphys.uni-heidelberg.de/\~{}weigand/Strings15-16/Strings.pdf}{\textit{Introduction to String Theory}}. 
		\end{itemize}
	}:
	\begin{gather}
	\feynmandiagram[
		layered layout
		,horizontal=b to c
		,every vertex={dot}
	]{
		a [particle=$\pdd{a} X_0^\mu$]
			-- [photon]
		b
			-- [half left
				,edge label=$\xi$
				,looseness=1.5]
		c
			-- [half left
				,looseness=1.5]
		b,
		c -- [photon]
		d [particle=$\pdd{b} X_0^{\nu}$], 
	}; \notag\\
		\allowdisplaybreaks
		\sim \frac{1}{2!}
			\pqty{\frac{1}{\alpha'}}^2
			\int \dd[2]{p} \pqty\Big{
				iH_{\mu\nu\rho}\,
				\epsilon^{ab}\,
				\pdd{a} X_0^\mu\,
				ip_b
			}
			\frac{2}{p^4}
			\pqty{-\frac{\alpha'}{2}}^2
			\pqty\Big{
				iH\id{_{\mu'}^{\nu\rho}}\,
				\epsilon^{a'b'}\,
				\pdd{a'} X_0^{\mu'}\,
				ip_{b'}
			} \\
		= \frac{2}{2!}
			\pqty{\frac{1}{\alpha'}}^2
			\pqty{-\frac{\alpha'}{2}}^2
			H_{\mu\lambda\omega}
			H\id{_{\nu}^{\lambda\omega}}
				\pdd{a} X_0^\mu\,
				\pdd{b} X_0^\nu\,
			\int \dd[2]{p}
			\frac{p^2 g^{ab} - p^a p^b}{p^4} \\
		= \frac{2}{2!}
			\pqty{-\frac{1}{2}}^2
			H_{\mu\lambda\omega}
			H\id{_{\nu}^{\lambda\omega}}
				\pdd{a} X_0^\mu\,
				\pdd{b} X_0^\nu\,
			\pqty{\frac{1}{2}\,g^{ab}}\!
			\int \dd[2]{p} \frac{1}{p^2} \\
		= \frac{2}{2!}
			\pqty{-\frac{1}{2}}^2
			\pqty{\frac{1}{2}}
			H_{\mu\lambda\omega}
			H\id{_{\nu}^{\lambda\omega}}
				\pdd{a} X_0^\mu\,
				\pdd{b} X_0^\nu\,
			g^{ab}\!
			\int \dd[2]{p} \frac{1}{p^2} \\
		= \frac{1}{8}
			H_{\mu\lambda\omega}
			H\id{_{\nu}^{\lambda\omega}}
				g^{ab}
				\pdd{a} X_0^\mu\,
				\pdd{b} X_0^\nu\,
			\int \dd[2]{p} \frac{1}{p^2}
	\end{gather}
	Here the $
		\pqty{\frac{1}{\alpha'}}^2
	$ coefficient comes from the vertices, while $
		\pqty\big{-\frac{\alpha'}{2}}^2
	$ comes from the propagators. 
	The $p^a p^b$ integral provides an additional $
		(\frac{1}{2})
	$ factor. 
	The overall normalization is chosen to match the $\alpha'R_{\mu\nu}$ coefficient in $\beta^G_{\mu\nu} \subset T\id{^a_a}$, which is $
		\frac{1}{1!}
			\times (-\frac{1}{2})
			\times 1
		= -\frac{1}{2}
	$. Therefore, we have:
	\begin{gather}
		T\id{^a_a}
		\supset \frac{1}{8}
			H_{\mu\lambda\omega}
			H\id{_{\nu}^{\lambda\omega}}
				g^{ab}
				\pdd{a} X_0^\mu\,
				\pdd{b} X_0^\nu,\\
		\beta^G_{\mu\nu}
		\supset -\frac{1}{4}\,\alpha'
			H_{\mu\lambda\omega}
			H\id{_{\nu}^{\lambda\omega}}
	\end{gather}
	\qedfull
	
	\item \textbf{Classical Solutions of 11D SUGRA:}
	Following the convention of \textit{Polchinski}, we have bosonic action:
	\begin{equation}
		S = \frac{1}{2\kappa^2}
			\int \pqty{
				\dd[11]{x} \sqrt{-g}\,
					\mcal{R}
				- \frac{1}{2}\,
					G\wedge *\,G
				- \frac{1}{6}\,
					C\wedge G\wedge G
			},
	\end{equation}
	Here $G = \dd{C}$: a 4-form field. In components, the numerical coefficients would be $
		\frac{1}{2}
		\mapsto
		\frac{1}{2\times 4!} = \frac{1}{48}
	$, and $
		\frac{1}{6}
		\mapsto
		\frac{1}{6\times 3!\times 4!\times 4!}
		= \frac{1}{20736}
	$. 
	
	Variation of the action yields the EOMs of our theory\footnote{
		Reference: \arxiv{hep-th/9912164}. I would like to thank \textit{Lucy Smith} for many helpful discussions. 
	}; Note that:
	\begin{equation}
		\var{\sqrt{-g}}
		= \frac{1}{2}\sqrt{-g}\,
			g^{\mu\nu} \var{g_{\mu\nu}}
		= -\frac{1}{2}\sqrt{-g}\,
			g_{\mu\nu} \var{g^{\mu\nu}}
	\end{equation}
	$\fdv{S}{g^{\mu\nu}}$ is easier to compute in components; note that the $
		C\wedge G\wedge G
	$ term does not depend on $g^{\mu\nu}$, therefore it does not contribute to the EOM. 
	We have the usual Einstein's equations:
	\begin{gather}
		R_{\mu\nu}
			- \frac{1}{2} \mcal{R} g_{\mu\nu}
		= \kappa^2 T_{\mu\nu},\\
	\begin{aligned}
		T_{\mu\nu}
		&= \frac{1}{\kappa^2}\,\pqty{
			\frac{4}{48}
			G_{\mu\sigma_1\sigma_2\sigma_3}
			G\id{_\nu^{\sigma_1\sigma_2\sigma_3}}
			- \frac{1}{2}\,g_{\mu\nu}
				\cdot \frac{1}{48}\,
				G^{\sigma_1\sigma_2\sigma_3\sigma_4}
				G_{\sigma_1\sigma_2\sigma_3\sigma_4}
		} \\
		&= \frac{1}{12\kappa^2}\,\pqty{
			G_{\mu\sigma_1\sigma_2\sigma_3}
			G\id{_\nu^{\sigma_1\sigma_2\sigma_3}}
			- \frac{1}{8}\,g_{\mu\nu}\,
				G^{\sigma_1\sigma_2\sigma_3\sigma_4}
				G_{\sigma_1\sigma_2\sigma_3\sigma_4}
		}
	\end{aligned}
	\label{eq:bosonic_stress_tensor}
	\end{gather}
	On the other hand, $
		\fdv{S}{C}
	$ is best carried out using differential forms:
	\begin{gather}
	\begin{aligned}
		0 = \var_C{S}
		&= -\frac{1}{2\kappa^2}
			\int \pqty{
				\var{G} \wedge *\,G
				+ \frac{1}{6} \pqty\Big{
					\var{C} \wedge G\wedge G
					- 2C\wedge\var{G}\wedge G
				}
			} \\
		&= -\frac{1}{2\kappa^2}
			\int \pqty{
				\var{(\dd{C})} \wedge *\,G
				+ \frac{1}{6} \pqty\big{
					\var{C} \wedge G\wedge G
					+ 2\var{(\dd{C})}
						\wedge C\wedge G
				}
			} \\
		&= -\frac{1}{2\kappa^2}
			\int \pqty{
				- (-1)^3 \var{C} \wedge \dd{* G}
				+ \frac{1}{6} \pqty\big{
					\var{C} \wedge G\wedge G
					- 2\,(-1)^3
					\var{C}\wedge \dd{(C\wedge G)}
				}
			} \\
		&= -\frac{1}{2\kappa^2}
			\int \var{C}\wedge\pqty{
				\dd{* G}
				+ \frac{1}{6} \pqty\Big{
					G\wedge G
					+ 2\pqty\big{
						G\wedge G
						- C\wedge \cancel{\dd[2]{C}}
					}
				}
			} \\
		&= -\frac{1}{2\kappa^2}
			\int \var{C}\wedge\pqty{
				\dd{* G}
				+ \frac{1}{2}\,G\wedge G
			},
	\end{aligned}\\[1ex]
		\dd{* G}
			+ \frac{1}{2}\,G\wedge G
		= 0
	\end{gather}
	
	\begin{enumerate}
	\item We hope to find a spacetime solution which is \textit{maximally symmetric} in \textit{some} directions; assume that these directions form a $d$-dimensional sub-manifold $\mcal{M}_d$ with:
	\begin{equation}
	\begin{aligned}
		\text{Coordinates:} &&&
			x^{\mu'},\ %
			\mu' \in \Delta \subset \Bqty{
				0,1,\cdots, 11
			},\\
		\text{Induced metric:} &&&
			g' = g|_{\mcal{M}_d}
	\end{aligned}
	\end{equation}
	The entire spacetime is then a direct product: $
		\mcal{M}_d
		\times \widetilde{\mcal{M}}_{11-d}
	$. 
	For $\mcal{M}_d$ to be maximally symmetric, we expect that $
		\kappa^2 T_{\mu'\nu'}
		= -\Lambda g'_{\mu'\nu'}
	$, i.e.\ the $G$-field serves as a cosmological constant $\Lambda$. By staring at \eqref{eq:bosonic_stress_tensor} we find that this can be achieved with\footnote{
		This is in fact the famous \textit{Freund--Robin ansatz}; see \wikiref{https://en.wikipedia.org/wiki/Freund\%E2\%80\%93Rubin\_compactification}{Freund–Rubin compactification}, and also the original paper: Freund \& Robin, \href{https://inspirehep.net/literature/154579}{\textit{Dynamics of Dimensional Reduction}}, 1980. 
	}:
	\begin{gather}
		d = 4,\quad
		G_{\sigma_1\sigma_2\sigma_3\sigma_4}
		= \alpha \sqrt{\abs{g'}}\,
			\epsilon_{
				\sigma_1\sigma_2\sigma_3\sigma_4
			},\quad
		G^{\sigma_1\sigma_2\sigma_3\sigma_4}
		= \alpha\,
			\frac{\mop{sgn} g'}{\sqrt{\abs{g'}}}\,
			\epsilon^{
				\sigma_1\sigma_2\sigma_3\sigma_4
			},\quad
		\{\sigma_i\}\subset \Delta,
		\\
		G_{\cdots\,\sigma\,\cdots}
		= 0,\quad
		\sigma \not\in\Delta,
	\\[2ex]
		T_{\mu\nu}
		= (\mop{sgn} g')\,
		\frac{\alpha^2}{12\kappa^2}\,\pqty{
			3!\,g'_{\mu\nu}
			- \frac{4!}{8}\,g_{\mu\nu}
		}
		= (\mop{sgn} g')\,
		\frac{\alpha^2}{2\kappa^2}\,
			\pqty{
				g'_{\mu\nu}
				- \frac{1}{2}\,g_{\mu\nu}
			},
	\\[1.5ex]
		\Lambda g_{\mu\nu}
		= \mp(\mop{sgn} g')\,
			\frac{\alpha^2}{4\kappa^2}\,
			g_{\mu\nu},\quad
	\left\lbrace\vbox to 12.5pt {}
	\right.
	\begin{aligned}
		- &\colon\ %
			\mu=\mu',\nu=\nu' \in \Delta,
			\hspace{-1.5ex}
			&&&\sim \mcal{M}_4\\[-.35ex]
		+ &\colon\ %
			\mu,\nu \not\in \Delta,
			&&&\sim \widetilde{\mcal{M}}_7
	\end{aligned}
	\end{gather}
	Matter EOM is trivially satisfied due to anti-symmetricity. We see that the other component $\widetilde{\mcal{M}}_7$ is also maximally symmetric, but with an opposite sign in its cosmological constant. 
	
	The field equations in $\mcal{M}_4$ and $\widetilde{\mcal{M}}_7$ are both of the form $
		R_{\mu\nu} \propto g_{\mu\nu}
	$. For $\mop{sgn} g' = -1$ i.e.\ Lorentzian signature, the solution is flat, AdS or dS, depending on the sign of $\Lambda$; for $\mop{sgn} g' = -1$, the solution is flat, spherical or hyperbolic. Therefore, we have:
	\begin{equation}
	\begin{aligned}
		\mop{sgn} g' = -1,
		&&&
		\Lambda_{4,7}
		= \pm \frac{\alpha^2}{4\kappa^2},\quad
		\mcal{M}_4 = \mrm{AdS}_{3,1},\quad
		\widetilde{\mcal{M}}_7 = S^7 \\
		\mop{sgn} g' = +1,
		&&&
		\Lambda_{4,7}
		= \mp \frac{\alpha^2}{4\kappa^2},\quad
		\mcal{M}_4 = S^4,\quad
		\widetilde{\mcal{M}}_7 = \mrm{AdS}_{6,1}
	\end{aligned}
	\end{equation}
	
	\item Global supersymmetries of a theory with the above $
		\mrm{AdS}_{4/7}\times S^{4/7}
	$ background are given by the solutions of:
	\begin{gather}
		0 = \var_\eta\psi^\mu
		\equiv D^\mu \eta(x),\quad
		\eta\colon \text{spinor},\\[.5ex]
	\begin{aligned}
		D^\mu &= \nabla^\mu
			+ \frac{1}{288}\,
				G_{\nu\rho\sigma\lambda} \pqty{
					\Gamma^{\mu\nu\rho\sigma\lambda}
					- 8g^{\mu\nu}
						\Gamma^{\rho\sigma\lambda}
				} \\
		&= \nabla^\mu
			+ \frac{1}{288}\,
				G_{\nu'\rho'\sigma'\lambda'}
			\pqty\big{
				\Gamma^{\mu\nu'\rho'\sigma'\lambda'}
				- 8g^{\mu\nu'}
					\Gamma^{\rho'\sigma'\lambda'}
			} \\[.5ex]
		&= \nabla^\mu
			+ \alpha
			\left\lbrace\vbox to 24pt {}\right.
			\begin{aligned}
				&\frac{-8\times 3!}{288}\,
					(-\Gamma^\mu\gamma_5)
				= \frac{1}{6}\,
					\Gamma^\mu\gamma_5,\quad
				\mu=\mu' \in \Delta,
					\hspace{-.5ex}
				&& \sim \mcal{M}_4
				\\[.5ex]
				& \frac{4!}{288}\,
					(-\Gamma^\mu)
				= -\frac{1}{12}\,
					\Gamma^\mu,\quad
				\mu \not\in \Delta,
				&& \sim \widetilde{\mcal{M}}_7
			\end{aligned}
	\end{aligned}
	\end{gather}
	Note that we've replaced the $G$ indices with $\mcal{M}_4$ indices, since $G$ vanish in $\widetilde{\mcal{M}}_7$ directions; due to anti-symmetricity, the $G$-term can be reduced to simple $\Gamma^\mu$ multiplications according to the $\mu$-direction\footnote{
		Reference for $\Gamma$-matrices and spinors: \textit{Polchinski} Vol.~\Romannum{2}, Appendix B. I'm a bit confused about all the complicated conventions, therefore the coefficients might be off by some factors...
	}. Furthermore, the spin connection in $\nabla^\mu$ is also block diagonalized, same as $g_{\mu\nu}$; hence there is a natural separation of variable\footnote{
		See \arxiv{hep-th/9912164} for more detailed discussions. 
	}:
	\begin{gather}
		\eta = \eta'(x')\,
			\eta'' (x''),\quad
		D_{\mu'} \eta' = 0,\quad
		D_{\mu''} \eta'' = 0,\\[.5ex]
		\mu',\eta',x'
			\sim \mcal{M}_4,\quad
		\mu'',\eta'',x''
			\sim \widetilde{\mcal{M}}_7,\quad
	\end{gather}
	
	Due to the presence of an additional $\Gamma$, $
		D_{\mu'} \eta' = 0
	$ has only 4 linearly independent solutions labeled by $\mu'$, while $
		D_{\mu''} \eta'' = 0
	$ is $\mop{Spin}(8)$ (or $\mrm{Spin}(7,1)$, depending on the signature) invariant, and has $
		\frac{8\times 7}{2} = 28
	$ linearly independent solutions\footnote{
		Reference: Achilleas Passias, \textit{Aspects of Supergravity in Eleven
		Dimensions}. 
	}. Hence the total number of SUSYs is $4 + 28 = 32$, for $
		\mrm{AdS}_{4/7}\times S^{4/7}
	$ background.
	\end{enumerate}
	
	\item \textbf{SUSY Sigma Models via Superspace:}
	\begin{gather}
	\begin{aligned}
		D_{\bar{\theta}} \vb{X}^\nu
		&= \pqty{
				\pdd{\bar{\theta}}
				+ \bar{\theta} \pdd{\bar{z}}
			}
			\pqty\big{
				X^\nu
				+ i\theta\psi^\nu
				+ i\bar{\theta}\tilde{\psi}^\nu
				+ \theta\bar{\theta} F^\nu
			} \\
		&= i\tilde{\psi}^\nu
			- \theta F^\nu
			+ \bar{\theta}\,
				\pdbar X^\nu
			- i\theta\bar{\theta}\,
				\pdbar \psi^\nu, \\[.5ex]
		D_\theta \vb{X}^\mu
		&= i\psi^\mu
			+ \bar{\theta} F^\mu
			+ \theta\,
				\pd X^\mu
			+ i\theta\bar{\theta}\,
				\pd \tilde{\psi}^\mu,
	\end{aligned}
	\\[1ex]
	\begin{aligned}
		D_{\bar{\theta}} \vb{X}^\nu
		D_\theta \vb{X}^\mu
		&= \pqty{
				i\tilde{\psi}^\nu
				- \theta F^\nu
				+ \bar{\theta}\,
					\pdbar X^\nu
				- i\theta\bar{\theta}\,
					\pdbar \psi^\nu
			}
			\pqty{
				i\psi^\mu
				+ \bar{\theta} F^\mu
				+ \theta\,
					\pd X^\mu
				+ i\theta\bar{\theta}\,
					\pd \tilde{\psi}^\mu
			} \\
		&= - \tilde{\psi}^\nu \psi^\mu
			- i\theta \pqty{
				\tilde{\psi}^\nu \pd X^\mu
				+ \psi^\mu F^\nu
			}
			+ i\bar{\theta} \pqty{
				\psi^\mu \pdbar X^\nu
				- \tilde{\psi}^\nu F^\mu
			} \\
			&\qquad - \theta\bar{\theta} \pqty{
				\pdbar X^\nu\pd X^\mu
				+ \tilde{\psi}^\nu
					\pd \tilde{\psi}^\mu
				- (\pdbar \psi^\nu)
					\psi^\mu
				+ F^\nu F^\mu
			},
	\end{aligned}
	\\[1ex]
	\begin{aligned}
		G_{\mu\nu}(\vb{X})
		&= G_{\mu\nu}
			+ \pqty{
				i\theta\psi^\lambda
				+ i\bar{\theta}\tilde{\psi}^\lambda
				+ \theta\bar{\theta} F^\lambda
			}\,\pdd{\lambda} G_{\mu\nu}
			+ \frac{1}{2}\,\Bqty{
				i\theta\psi^\rho
					\pdd{\rho},\,
				i\bar{\theta}\tilde{\psi}^\sigma
					\pdd{\sigma}\!
			}\,G_{\mu\nu} \\
		&= G_{\mu\nu}
			+ \pqty{
				i\theta\psi^\lambda
				+ i\bar{\theta}\tilde{\psi}^\lambda
			}\,G_{\mu\nu,\lambda}
			+ \theta\bar{\theta} \pqty{
				F^\lambda G_{\mu\nu,\lambda}
				+ \psi^\rho \tilde{\psi}^\sigma
					G_{\mu\nu,\rho\sigma}
			}, \\
	\end{aligned}
	\end{gather}
	
	Note that $
		\int \dd[2]{\theta}
		= \pdd{\theta} \pdd{\bar{\theta}}
	$, hence we need only focus on the $\theta\bar{\theta}$ term in the Lagrangian:
	\begin{equation}
	\begin{aligned}
		4\pi S_G
		&= \int \dd[2]{z} \dd[2]{\theta}
			G_{\mu\nu}(\vb{X})\,
			D_{\bar{\theta}} \vb{X}^\mu
			D_\theta \vb{X}^\nu
		= \int \dd[2]{z} \dd[2]{\theta}
			(-\theta\bar{\theta}) \pqty\Big{
				G_{\mu\nu} \pqty{
					\pd X^\mu \pdbar X^\nu
					+ \cdots
				} + \cdots
			} \\
		&= \int \dd[2]{z}
		\bigg(
			G_{\mu\nu} \pqty{
				\pd X^\mu \pdbar X^\nu
				+ \tilde{\psi}^\nu
					\pd \tilde{\psi}^\mu
				- (\pdbar \psi^\nu)
					\psi^\mu
				+ F^\nu F^\mu
			}
			\\ & \hspace{5em}
			+ \tilde{\psi}^\nu \psi^\mu \pqty{
				F^\lambda G_{\mu\nu,\lambda}
				+ \psi^\rho \tilde{\psi}^\sigma
					G_{\mu\nu,\rho\sigma}
			}
			\\ & \hspace{5em}
			- G_{\mu\nu,\lambda} \pqty{
				\psi^\lambda \pqty\big{
					\psi^\mu \pdbar X^\nu
					- \tilde{\psi}^\nu F^\mu
				}
				+ \tilde{\psi}^\lambda \pqty\big{
					\tilde{\psi}^\nu \pd X^\mu
					+ \psi^\mu F^\nu
				}
			}
		\bigg)
	\end{aligned}
	\end{equation}
	Similar result holds for the $B$ contribution $S_B$. We see that there is no $\pd F$ term in the action, hence $F$ is not dynamical and can be integrated out; we have:
	\begin{gather}
		0 = \var_F S
		= \var_F \pqty{S_G + S_B},\\
	\begin{aligned}
		4\pi \var{S_G}
		&= \int \dd[2]{z} \pqty{
				2G_{\mu\nu} F^\mu \var{F^\nu}
				+ G_{\mu\nu,\lambda} \pqty\big{
					\tilde{\psi}^\nu
						\psi^\mu
						\var{F^\lambda}
					- \tilde{\psi}^\nu
						\psi^\lambda
						\var{F^\mu}
					- \tilde{\psi}^\lambda
						\psi^\mu
						\var{F^\nu}
				}
			} \\
		&= \int \dd[2]{z} \pqty{
				2F_\lambda
				+ \pqty\big{
					G_{\mu\nu,\lambda}
					- G_{\lambda\mu,\nu}
					- G_{\lambda\nu,\mu}
				}\, \tilde{\psi}^\nu
					\psi^\mu
			} \var{F^\lambda} \\
		&= \int \dd[2]{z} \pqty{
				2F_\lambda
				- 2\Gamma_{\lambda\mu\nu}
					\tilde{\psi}^\nu
					\psi^\mu
			} \var{F^\lambda}, \\
		4\pi \var{S_B}
		&= \int \dd[2]{z} \pqty{
				0 + \pqty\big{
					B_{\mu\nu,\lambda}
					+ B_{\lambda\mu,\nu}
					+ B_{\nu\lambda,\mu}
				}\, \tilde{\psi}^\nu
					\psi^\mu
			} \var{F^\lambda}
		= \int \dd[2]{z}
			H_{\lambda\mu\nu}
				\tilde{\psi}^\nu
				\psi^\mu
				\var{F^\lambda}, \\
	\end{aligned}
	\\
		F_\lambda
		= \pqty{
				\Gamma_{\lambda\mu\nu}
				- \frac{1}{2} H_{\lambda\mu\nu}
			} \tilde{\psi}^\nu
				\psi^\mu,\\
		F^\lambda
		= \pqty{
				\Gamma^\lambda_{\mu\nu}
				- \frac{1}{2}
					H^\lambda_{\mu\nu}
			} \tilde{\psi}^\nu
				\psi^\mu,
	\end{gather}
	Here we've used the (anti-)symmetry of $G_{\mu\nu}$ and $B_{\mu\nu}$, and we adopt the convention that the Levi-Civita connection $
		\Gamma^\lambda_{\mu\nu}
		= \Gamma\id{^\lambda_{\mu\nu}}
		= G^{\lambda\lambda'}
			\Gamma_{\lambda'\mu\nu}
	$; similar holds for $B_{\mu\nu}$ and $H^\lambda_{\mu\nu}$. 
	
	Substitute $F_\lambda$ into $S$, collect the $\psi^0, \psi^2, \tilde{\psi}^2$ and $\psi^2\tilde{\psi}^2$ terms respectively, and we have:
	\begin{equation}
	\begin{aligned}
		4\pi S
		&= \int \dd[2]{z}
		\bigg(
			\pqty{G_{\mu\nu} + B_{\mu\nu}}\,
				\pd X^\mu \pdbar X^\nu
			\\ & \hspace{5em}
			+ \pqty{G_{\mu\nu} + \cancel{B_{\mu\nu}}}
			\pqty{
				\tilde{\psi}^\mu
					\pd \tilde{\psi}^\nu
				- (\pdbar \psi^\mu)
					\psi^\nu
			}
			\\ & \hspace{5em}
			- \pqty{
				G_{\mu\nu,\lambda}
				+ B_{\mu\nu,\lambda}
			} \pqty{
				\psi^\lambda
					\psi^\mu \pdbar X^\nu
				+ \tilde{\psi}^\lambda
					\tilde{\psi}^\nu \pd X^\mu
			}
			\\ & \hspace{5em}
			+ G_{\mu\nu} F^\mu F^\nu
			- 2\pqty{
					\Gamma_{\lambda\mu\nu}
					- \frac{1}{2} H_{\lambda\mu\nu}
				} \tilde{\psi}^\nu \psi^\mu
				F^\lambda
			\\ & \hspace{5em}
			+ \pqty{
					G_{\mu\nu,\rho\sigma}
					+ B_{\mu\nu,\rho\sigma}
				}\,
				\tilde{\psi}^\nu \psi^\mu
				\psi^\rho \tilde{\psi}^\sigma
		\bigg) \\
		&= \int \dd[2]{z}
		\bigg(
			\pqty{G_{\mu\nu} + B_{\mu\nu}}\,
				\pd X^\mu \pdbar X^\nu
			\\ & \hspace{5em}
			+ G_{\mu\nu} \pqty{
				\tilde{\psi}^\mu
					\pd \tilde{\psi}^\nu
				+ \psi^\mu
					\pdbar \psi^\nu
			}
%			\\ & \hspace{5em}
			- \pqty{
				G_{\mu\nu,\lambda}
				+ B_{\mu\nu,\lambda}
			} \pqty{
				\psi^\lambda
					\psi^\mu \pdbar X^\nu
				+ \tilde{\psi}^\lambda
					\tilde{\psi}^\nu \pd X^\mu
			}
			\\ & \hspace{5em}
			- F_\lambda F^\lambda
			+ \pqty{
					G_{\mu\nu,\rho\sigma}
					+ B_{\mu\nu,\rho\sigma}
				}\,
				\psi^\mu
				\psi^\rho
				\tilde{\psi}^\nu
				\tilde{\psi}^\sigma
		\bigg) \\
	\end{aligned}
	\end{equation}
	Here we've performed some integration by parts to clean up the result. 
	Note that some terms involving $B_{\mu\nu}$ vanish conveniently (up to integration by parts) due to anti-symmetricity. 
	
	The $\psi^2,\tilde{\psi}^2$ terms in the integrand can be further simplified as follows:
	\begin{gather}
	\begin{aligned}
		\mcal{L}_{\psi^2}
		&= G_{\mu\nu}
			\psi^\mu
			\pdbar \psi^\nu
		- \pqty{
			G_{\mu\nu,\lambda}
			+ B_{\mu\nu,\lambda}
		} \,
			\psi^\lambda
			\psi^\mu \pdbar X^\nu \\
		&= G_{\mu\nu}
			\psi^\mu
			\pdbar \psi^\nu
		- \pqty{
			G_{\mu[\nu,\lambda]}
			+ B_{\mu[\nu,\lambda]}
		} \,
			\psi^\lambda
			\psi^\mu \pdbar X^\nu \\
		&= G_{\mu\nu}
			\psi^\mu
			\pdbar \psi^\nu
		- \pqty{
			- \Gamma_{\lambda\mu\nu}
			+ \frac{1}{2} H_{\lambda\mu\nu}
		} \,
			\psi^\lambda
			\psi^\mu \pdbar X^\nu \\
		&= G_{\mu\nu} \psi^\mu \pqty{
				\pdbar \psi^\nu
				+ \pqty{
					\Gamma^\nu_{\rho\sigma}
					- \frac{1}{2} H^\nu_{\rho\sigma}
				} \,
				\psi^\rho
				\pdbar X^\sigma\!
			} \\
		&= G_{\mu\nu} \psi^\mu \pqty{
				\pdbar \psi^\nu
				+ \pqty{
					\Gamma^\nu_{\rho\sigma}
					+ \frac{1}{2} H^\nu_{\rho\sigma}
				} \,
				\psi^\sigma
				\pdbar X^\rho\!
			}
		= G_{\mu\nu} \psi^\mu
			\bar{\mcal{D}} \psi^\nu,
	\\[1.5ex]
		\mcal{L}_{\tilde{\psi}^2}
		&= G_{\mu\nu}
			\tilde{\psi}^\mu
			\pd \tilde{\psi}^\nu
		- \pqty{
			G_{\mu\nu,\lambda}
			+ B_{\mu\nu,\lambda}
		} \,
			\tilde{\psi}^\lambda
			\tilde{\psi}^\nu \pd X^\mu \\
		&= G_{\mu\nu} \tilde{\psi}^\mu \pqty{
				\pd \tilde{\psi}^\nu
				+ \pqty{
					\Gamma^\nu_{\rho\sigma}
					- \frac{1}{2} H^\nu_{\rho\sigma}
				} \,
				\tilde{\psi}^\sigma
				\pd X^\rho\!
			}
		= G_{\mu\nu} \tilde{\psi}^\mu
			\mcal{D} \tilde{\psi}^\nu,
	\end{aligned}
	\end{gather}
	For the $\psi^2\tilde{\psi}^2$ term, recall that $
		R_{\mu\nu\rho\sigma}
		= e_\mu [\cdv{\rho\,},\cdv{\sigma\,}]\,
			e_\nu,
		\cdv{\sigma} e_\nu
		= e_{\lambda}
			\Gamma^\lambda_{\sigma\nu}
	$, and we have:
	\begin{gather}
	\begin{aligned}
		\mcal{L}_{\psi^2\tilde{\psi}^2}
		&= \psi^\mu
			\psi^\nu
			\tilde{\psi}^\rho
			\tilde{\psi}^\sigma
		\pqty{
			G_{\mu\rho,\nu\sigma}
			+ B_{\mu\rho,\nu\sigma}
			+ \pqty\Big{
					\Gamma_{\lambda\mu\rho}
					- \frac{1}{2} H_{\lambda\mu\rho}
				}
				\pqty\Big{
					\Gamma^\lambda_{\nu\sigma}
					- \frac{1}{2}
						H^\lambda_{\nu\sigma}
				}
		} \\
		&= \psi^\mu
			\psi^\nu
			\tilde{\psi}^\rho
			\tilde{\psi}^\sigma
		\pqty{
			G_{\mu\rho,\nu\sigma}
				+ \Gamma_{\lambda\mu\rho}
					\Gamma^\lambda_{\nu\sigma}
			+ B_{\mu\rho,\nu\sigma}
			- \frac{1}{2} \pqty\Big{
					\Gamma^\lambda_{\mu\rho}
					H_{\lambda\nu\sigma}
					+ \Gamma^\lambda_{\nu\sigma}
					H_{\lambda\mu\rho}
				}
			+ \frac{1}{4}
				H^\lambda_{\mu\rho}
				H_{\lambda\nu\sigma}
		} \\
		&= \mcal{L}_G + \mcal{L}_B
			+ \frac{1}{4}
				H^\lambda_{\mu\rho}
				H_{\lambda\nu\sigma}\,
				\psi^\mu
				\psi^\nu
				\tilde{\psi}^\rho
				\tilde{\psi}^\sigma,
	\end{aligned}
	\\[1.5ex]
	\begin{aligned}
		\mcal{L}_G
		&= \psi^\mu
			\psi^\nu
			\tilde{\psi}^\rho
			\tilde{\psi}^\sigma
		\pqty{
			G_{\mu\rho,\nu\sigma}
				+ \Gamma_{\lambda\mu\rho}
					\Gamma^\lambda_{\nu\sigma}
		} \\
		&= \psi^{[\mu}
			\psi^{\nu]}
			\tilde{\psi}^{[\rho}
			\tilde{\psi}^{\sigma]}
		\pqty{
			G_{\mu\rho,\nu\sigma}
				+ \Gamma_{\lambda\mu\rho}
					\Gamma^\lambda_{\nu\sigma}
		} \\
		&= \frac{1}{2}\,
			\psi^\mu
			\psi^\nu
			\tilde{\psi}^\rho
			\tilde{\psi}^\sigma
		\Bqty{
			\pqty{
				\frac{1}{2}\,\pqty{
					G_{\mu\rho,\nu\sigma}
					- G_{\mu\sigma,\nu\rho}
				}
				+ \Gamma_{\lambda\mu\rho}
					\Gamma^\lambda_{\nu\sigma}
			}
			- \pqty\Big{\cdots}_{
				\rho\leftrightarrow\sigma
			}
		} \\
		&= \frac{1}{2}
			R_{\mu\nu\rho\sigma}\,
			\psi^\mu
			\psi^\nu
			\tilde{\psi}^\rho
			\tilde{\psi}^\sigma,
	\\[1.5ex]
		\mcal{L}_B
		&= \frac{1}{2}
			\cdv{\rho} H_{\mu\nu\sigma}\,
			\psi^\mu
			\psi^\nu
			\tilde{\psi}^\rho
			\tilde{\psi}^\sigma,
	\end{aligned}
	\end{gather}
	Therefore, the total action is:
	\begin{gather}
	\begin{aligned}
		S
		&= \frac{1}{4\pi} \int \dd[2]{z}
		\bigg(
			\pqty{G_{\mu\nu} + B_{\mu\nu}}\,
				\pd X^\mu \pdbar X^\nu
			\\ & \hspace{7.5em}
			+ G_{\mu\nu} \pqty{
				\tilde{\psi}^\mu
					\mcal{D} \tilde{\psi}^\nu
				+ \psi^\mu
					\bar{\mcal{D}} \psi^\nu
			}
			\\ & \hspace{7.5em}
			+ \pqty{
					\frac{1}{2}
						R_{\mu\nu\rho\sigma}
					+ \frac{1}{2}
						\cdv{\rho} H_{\mu\nu\sigma}
					+ \frac{1}{4}
						H^\lambda_{\mu\rho}
						H_{\lambda\nu\sigma}\!
				}\,
				\psi^\mu
				\psi^\nu
				\tilde{\psi}^\rho
				\tilde{\psi}^\sigma
		\bigg) \\
	\end{aligned}
	\end{gather}
	
	\item \textbf{Mixed Anomaly Between Diffeomorphism and Axial $U(1)$ Symmetry:}
	\begin{enumerate}
	\item Calculations of such anomaly is (schematically) similar to the usual axial anomaly; instead of the $A_\mu$ legs, we now have two $h_{\mu\nu}$ legs in the triangular diagram. 
	
	Again we chose the Pauli--Villars regularization with a regulator field $\psi'$ of mass $M\to\infty$. The $\pd^\mu J^A_\mu$ insertion is then reduced to:
	\begin{equation}
		\pd^\mu J^A_\mu
		= \pdd{\mu} \pqty{
				i\bar{\psi}'
				\gamma^\mu \gamma^5 \psi'
			}
		= i\bar{\psi}' (2M\gamma^5) \psi'
	\end{equation}
	The fermion--fermion--graviton vertex is given by $
		h_{\mu\nu} T^{\mu\nu}
	$, and (up to integration by parts) we have:
	\begin{gather}
		T^{\mu\nu}
		= \frac{i}{2} \bar{\psi}
			\gamma^{(\mu}
			\overleftrightarrow{\pd}^{\nu)}
			\psi
		\sim \frac{i}{2} \bar{\psi}
			\gamma^{(\mu}
			\pqty\big{
				-2\pd^{\nu)}
			} \psi
		= -i \bar{\psi}
			\gamma^{(\mu} \pd^{\nu)}
			\psi,\\
		h_{\mu\nu} T^{\mu\nu}
		= \bar{\psi}\,\pqty{
				-ih_{\mu\nu}
				\gamma^{(\mu} \pd^{\nu)}
			} \psi,
	\end{gather}
	This is very similar to the $A_\mu$ coupling, except that there is an extra derivative $\pd^\nu$. Denote the polarization of graviton as $\varepsilon_{\mu\nu}$, then in momentum space the interaction vertex $
		\sim \epsilon_{\mu\nu} \gamma^\mu \pqty{
			k^\nu_1 + k^\nu_2
		}
	$, and we have:
	\begin{gather}
	\begin{aligned}
		\ave{\pd^\mu J_\mu^A}_h
		&\sim \frac{1}{2!}\times 2\!
		\int \frac{\dd[4]{k}}{(2\pi)^4}
			\Tr \pqty{
				2M\gamma_5\cdot
				\frac{
					\slashed{k} + M
				}{k^2 + M^2}\cdot
				\cancel{\varepsilon_1 (2k+p_1)}
				\cdot
				\frac{
					\slashed{k} + \slashed{p}_1 + M
				}{(k+p_1)^2 + M^2}\cdot
				\cancel{\varepsilon_2 (2k+2p_1+p_2)}
				\cdot
				\frac{
					\slashed{k}
					+ \slashed{p}_1
					+ \slashed{p}_2 + M
				}{(k+p_1+p_2)^2 + M^2}
			} \\
		&\sim \int \frac{\dd[4]{k}}{(2\pi)^4}\,
			2M^2
			(4\epsilon_{\mu\nu\rho\sigma})
			\,
			\varepsilon_1^{\mu\mu'}
				(2k+p_1)_{\mu'}\,
			p_1^\nu
			\,
			\varepsilon_2^{\rho\rho'}
				(2k+2p_1+p_2)_{\rho'}\,
			p_2^\sigma\,
			\pqty{
				\frac{1}{k^2 + M^2}
				\cdots
			} \\
		&\sim 8M^2 \epsilon_{\mu\nu\rho\sigma}\,
			p_1^\nu p_2^\sigma\,
			\varepsilon_1^{\mu\mu'}
			\varepsilon_2^{\rho\rho'}
		\int \frac{\dd[4]{k}}{(2\pi)^4}
			\frac{
				(2k+p_1)_{\mu'}
				(2k+2p_1+p_2)_{\rho'}
			}{
				(k^2 + M^2)
				\pqty\big{(k+p_1)^2 + M^2}
				\pqty\big{(k+p_1+p_2)^2 + M^2}
			} \\
		&\sim 8M^2 \epsilon_{\mu\nu\rho\sigma}\,
			p_1^\nu p_2^\sigma\,
			\varepsilon_1^{\mu\mu'}
			\varepsilon_2^{\rho\rho'}
		\int \frac{\dd[4]{k}}{(2\pi)^4}
			\frac{
				4k_{\mu'}k_{\rho'}
				+ p_{1,\mu'} p_{2,\rho'}
			}{(k^2 + M^2)^3}\\[-.8\baselineskip]
	\end{aligned}
	\end{gather}
	There are, in fact, 2 diagrams accounting for this amplitude with $
		1\leftrightarrow 2
	$ symmetry; here we simply take one contribution with an additional factor of 2, and imply $
		1\leftrightarrow 2
	$ symmetrization in the above expressions. 
%	The $\mquote{4}$ factor, on the other hand, comes from tracing over $\gamma$-matrices to produce $\epsilon_{\mu\nu\rho\sigma}$. 
	
	Note that due to the additional $k_{\mu'}k_{\rho'}$ the integral is no longer finite but logarithmic divergent: $
		\int^\Lambda \dd[4]{k} \frac{k^2}{k^6}
		\sim \ln \Lambda
	$. More specifically\footnote{
		References: 
		\begin{itemize}[
			labelindent=3em,labelsep=1pt
		]
		\item David Tong, \href{https://www.damtp.cam.ac.uk/user/tong/gaugetheory.html}{\textit{Gauge Theory}};
		\item A.~Zee, \textit{QFT in a Nutshell}z;
		\item \arxiv{0802.0634};
		\item \wikiref{https://en.wikipedia.org/wiki/Common\_integrals\_in\_quantum\_field\_theory}{Common integrals in quantum field theory}. 
		\end{itemize}
	}, we have:
	\begin{gather}
	\begin{aligned}
		\ave{\pd^\mu J_\mu^A}_h
		&\sim 8M^2 \epsilon_{\mu\nu\rho\sigma}\,
			p_1^\nu p_2^\sigma\,
			\varepsilon_1^{\mu\mu'}
			\varepsilon_2^{\rho\rho'}
			\frac{\mop{Vol} S^3}{(2\pi)^4}
			\int \pqty{
				\frac{
					4k_{\mu'}k_{\rho'}
					k^3 \dd{k}
				}{(k^2 + M^2)^3}
				+ p_{1,\mu'} p_{2,\rho'}
				\frac{
					k^3 \dd{k}
				}{(k^2 + M^2)^3}
			} \\
		&\sim 8M^2 \epsilon_{\mu\nu\rho\sigma}\,
			p_1^\nu p_2^\sigma\,
			\varepsilon_1^{\mu\mu'}
			\varepsilon_2^{\rho\rho'}
			\frac{2\pi^2}{(2\pi)^4}
			\int \pqty{
				\delta_{\mu'\rho'}
				\frac{
					k^5 \dd{k}
				}{(k^2 + M^2)^3}
				+ p_{1,\mu'} p_{2,\rho'}
				\frac{
					k^3 \dd{k}
				}{(k^2 + M^2)^3}
			} \\
		&\sim 8M^2 \epsilon_{\mu\nu\rho\sigma}\,
			p_1^\nu p_2^\sigma\,
			\varepsilon_1^{\mu\mu'}
			\varepsilon_2^{\rho\rho'}
			\frac{1}{8\pi^2}
			\pqty{
				\delta_{\mu'\rho'}
				\frac{1}{2}
					\ln \frac{\Lambda^2}{M^2}
				+ p_{1,\mu'} p_{2,\rho'}
				\frac{1}{4M^2}
			} \\
		&\sim \frac{1}{4\pi^2}\,
			\epsilon_{\mu\nu\rho\sigma}\,
			p_1^\nu p_2^\sigma\,
			\varepsilon_1^{\mu\mu'}
			\varepsilon_2^{\rho\rho'}
			\pqty{
				2\delta_{\mu'\rho'} M^2
					\ln \frac{\Lambda^2}{M^2}
				+ p_{1,\mu'} p_{2,\rho'}
			}
	\end{aligned}
	\end{gather}
	The second term is very much similar to the axial anomaly result, while the first term diverges.
%	even after Pauli--Villars regularization. 
	
	However, we believe that the divergent term must be canceled by other diagrams; otherwise, it will contribute a $
		p^\nu p^\sigma\,
		\delta_{\mu'\rho'}
			\varepsilon^{\mu\mu'}_1
			\varepsilon^{\rho\rho'}_2
		= p^\nu p^\sigma
		(\varepsilon_1)\id{^\mu_\alpha}
			(\varepsilon_2)^{\rho\alpha}
		\sim (\pd h)^2
	$ term in the final result, which is not diff-invariant. The second term, on the other hand, is diff-invariant:
	\begin{gather}
		R_{\mu\nu\alpha\beta}
		= p_\beta\,p_{[\nu}\,
				\varepsilon_{\mu]\alpha}
			- p_\alpha\,p_{[\nu}\,
				\varepsilon_{\mu]\beta},\\
	\begin{aligned}
		\ave{\pd^\mu J_\mu^A}_h
		&\sim \frac{1}{4\pi^2}\,
			\epsilon_{\mu\nu\rho\sigma}\,
			(\varepsilon^{\mu\mu'}
				p_{1,\mu'} p_1^\nu)
			(\varepsilon^{\rho\rho'}
				 p_{2,\rho'} p_2^\sigma) \\
		&\sim \frac{1}{4\pi^2}\,
			\epsilon_{\mu\nu\rho\sigma}
			\frac{1}{4!\times 2\times 2}\times
			\frac{1}{2}
				R_{\mu\nu\alpha\beta}
				R\id{_{\rho\sigma}^{\alpha\beta}} \\
		&\sim \frac{1}{768\pi^2}\,
			\epsilon_{\mu\nu\rho\sigma}
				R_{\mu\nu\alpha\beta}
				R\id{_{\rho\sigma}^{\alpha\beta}}
	\end{aligned}
	\end{gather}
	
	\item The next order contribution would come from the covariant derivative\footnote{
		Reference: Alvarez-Gaume \& Witten, \textit{Gravitational Anomalies}. 
	}:
	\begin{equation}
		\cdv{\mu} \psi
		= \pdd{\mu} \psi
			+ \frac{1}{2}\,
				\omega\id{_\mu^{ab}}
				\sigma_{ab}
				\psi
	\end{equation}
	Where $\omega\id{_\mu^{ab}}$ is the spin connections, and $
		\sigma_{ab}
		= \frac{1}{4} [\gamma_a,\gamma_b]
	$; when linearized this contributes to the following interaction vertex:
	\begin{gather}
		\mcal{L}'
		= -\frac{i}{4}\,
			h\id{_\lambda^\alpha}
			\pdd{\mu} h_{\nu\alpha}\,
			\bar{\psi}\,
				\Gamma^{\mu\lambda\nu}
			\psi,\quad
		\Gamma^{\mu\lambda\nu}
		= \gamma^{[\mu}
			\gamma^{\lambda}
			\gamma^{\nu]}, \\
		\text{Feynman rule:}\quad
			-\frac{i}{4}\,
				\Gamma^{\mu\lambda\nu}
				(p_1 - p_2)_\mu\,
				(\varepsilon_1)\id{_\lambda^\alpha}
				(\varepsilon_2)_{\nu\alpha},
	\end{gather}
	We see a $
		(\varepsilon_1)\id{_\lambda^\alpha}
		(\varepsilon_2)_{\nu\alpha}
	$ factor, much similar to the factor in the divergent term in (a). Note that this vertex already contains 3 $\gamma$-matrices; by joining it with the anomalous vertex $\pdd{\mu} j^\mu_A$, we obtain a simple 1-loop ``seagull'' diagram (with graviton wings)
%	, whose contribution is
	:
	\begin{gather}
	\feynmandiagram[
		layered layout
		,horizontal=a to b
		,every vertex={dot}
	]{
		a [blob
			,label={left:$\pdd{\mu} J^\mu_A$}]
			-- [half left
				,edge label=$k$
				,out=90,in=100
				,looseness=1.35]
		b
			-- [half left
				,edge label={$%
					\!\!\!\!\!\!\!\!%
					\!\!\!\!\!\!\!\!%
					k+p_1+p_2$}
				,inner sep=1.5ex
				,out=80,in=90
				,looseness=1.35]
		a,
		b -- [gluon]
			c [particle={$\varepsilon_1,p_1$}], 
		b -- [gluon]
			d [particle={$\varepsilon_2,p_2$}], 
	}; \notag\\
	\begin{aligned}
		\ave{\pd^\mu J_\mu^A}'_h
		&\sim 2\!
		\int \frac{\dd[4]{k}}{(2\pi)^4}
			\Tr \pqty{
				2M\gamma_5\cdot
				\frac{
					\slashed{k} + M
				}{k^2 + M^2}\cdot
				\pqty{-\frac{1}{4}}\,
				\cancel{
					\varepsilon_1
					\varepsilon_2
					(p_1 - p_2)}
				\cdot
				\frac{
					\slashed{k}
					+ \slashed{p}_1
					+ \slashed{p}_2 + M
				}{(k+p_1+p_2)^2 + M^2}
			} \\
		&\sim -\int \frac{\dd[4]{k}}{(2\pi)^4}\,
			M^2
			(4\epsilon_{\mu\nu\rho\sigma})\,
			\delta_{\mu'\rho'}
				\varepsilon^{\mu\mu'}_1
				\varepsilon^{\rho\rho'}_2
				(p_1 - p_2)^{\nu}\,
				(p_1 + p_2)^{\sigma}\,
			\pqty{
				\frac{1}{k^2 + M^2}
				\cdots
			} \\
		&\sim -4M^2 \epsilon_{\mu\nu\rho\sigma}\,
			(2p_1^\nu p_2^\sigma)\,
			\varepsilon_1^{\mu\mu'}
			\varepsilon_2^{\rho\rho'}
		\int \frac{\dd[4]{k}}{(2\pi)^4}
			\frac{
				\delta_{\mu'\rho'}
			}{(k^2 + M^2)^2} \\
		&\sim -8M^2 \epsilon_{\mu\nu\rho\sigma}\,
			p_1^\nu p_2^\sigma\,
			\varepsilon_1^{\mu\mu'}
			\varepsilon_2^{\rho\rho'}
			\frac{1}{8\pi^2}
			\pqty{
				\delta_{\mu'\rho'}
				\frac{1}{2}
					\ln \frac{\Lambda^2}{M^2}
			}
	\end{aligned}
	\end{gather}
	Compare with the result in (a), and we see that the divergences cancel each other out precisely. 
	
	\item For an anomalous vertex with hypercharge $Y$, there will be an additional $Y$ factor in the front of $\ave{\pdd{\mu} J^\mu_A}$; summing over a family of matter gives the total anomaly\footnote{
		Reference: \textit{Tong}, and \wikiref{arg1https://en.wikipedia.org/wiki/Anomaly\_(physics)\#Anomaly\_cancellation}{Anomaly (physics) \# Anomaly cancellation}. 
	}:
	\begin{equation}
		\ave{\pdd{\mu} J^\mu_A}
		\propto \sum \Tr T_a T_b Y
		\propto \delta_{ab} \sum Y
	\end{equation}
	When the summation goes over all states in a complete generation, we have $
		\sum Y = 0
	$, i.e.\ the anomaly cancels. 
	
	\end{enumerate}
	
	\end{enumerate}


\printbibliography[%
%	title = {参考文献} %
	,heading = bibintoc
]
\end{document}
