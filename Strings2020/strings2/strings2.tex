% !TeX encoding = UTF-8
% !TeX spellcheck = en_US
% !TeX TXS-program:bibliography = biber -l zh__pinyin --output-safechars %

\documentclass[a4paper,10pt]{article}

\newcommand{\hwNumber}{2}

% Templates: 82ccb576e4df24e5eac4194b76230be360b4f733

% to be `\input` in subfolders,
% ... therefore the path should be relative to subfolders.

\usepackage[UTF8
	,heading=false
	,scheme=plain % English Document
]{ctex}

\input{../.modules/basics/macros.tex}
\input{../.modules/preamble_base.tex}
\input{../.modules/preamble_notes.tex}

\newcommand{\legacyReference}{{
	\clearpage\par
	\quad\clearpage
	\renewcommand{\midquote}{\textbf{PAST WORK, AS TEMPLATE}}
	\newparagraph
}}

% Settings
\counterwithout{equation}{section}
\mathtoolsset{showonlyrefs=false}
%\DeclareTextFontCommand{\textbf}{\sffamily}
\renewcommand{\midquote}{\quad}

% Spacing
\geometry{footnotesep=2\baselineskip} % pre footnote split
\setlength{\parskip}{.5\baselineskip}
\renewcommand{\baselinestretch}{1.15}

%Title
	\posttitle{
		\hfill\Large\ccbyncsajp
		\par\end{flushleft}%
		\vspace*{-.7ex}\hrule%
	}
	\preauthor{\vspace{-1.5ex}%
		\flushleft\itshape%
	}
	\postauthor{\hfill}
	\predate{\noindent\ttfamily Compiled @ }
	\postdate{\vspace{.5ex}}

	\title{String Theory \textnumero\hwNumber}
	\author{\signature Bryan}
	\date{\today}

% List
	\setlist*{
		listparindent=\parindent
		,labelindent=\parindent
		,parsep=\parskip
		,itemsep=1.2\parskip
		,leftmargin=0pt
		,itemindent=*
	}
	\setlist*[enumerate,1]{
		align=left
		,label=\fbox{\textbf{\arabic*}}
		,itemsep=.5\baselineskip
		,itemindent=*
	}

\input{../.modules/basics/biblatex.tex}

%%% ID: sensitive, do NOT publish!
%\InputIfFileExists{../id.tex}{}{}

\usepackage{cancel}

\begin{document}
\maketitle
\pagestyle{headings}
\pagenumbering{arabic}
\thispagestyle{empty}

	\begin{enumerate}
	\item \textbf{BRST Quantization of Bosonic String}: 
	\begin{gather}
		S = S^X + S^{bc},\\
		S^X = \frac{1}{2\pi\alpha'} \int\dd[2]{z}
			\pd X^\mu \pdbar X_\mu,\quad
		S^{bc} = \frac{1}{2\pi} \int\dd[2]{z}
			\pqty\Big{
				b\,\pdbar c
				+ \tilde{b}\,\pd\tilde{c}
			}
	\end{gather}
	This is the gauge fixed action. The corresponding BRST transformation is listed in \textit{Polchinski}; for each of the subsystems, we have its energy-momentum:
	\begin{gather}
		T^X(z)
		= - \frac{1}{\alpha'}\,
			\normorder{\pd X^\mu \pd X_\mu}\,,\quad
		\tilde{T}^X(\bar{z})
		= - \frac{1}{\alpha'}\,
			\normorder{\pdbar X^\mu \pdbar X_\mu}\,,
		\\[.5ex]
		T^{bc}(z) = \normorder{(\pd b)\,c}
			- 2\,\pd(\normorder{bc}),\quad
		\tilde{T}^{bc}(\bar{z}) = \normorder{
			(\pdbar \tilde{b})\,\tilde{c}
		} - 2\,\pdbar(\normorder{
			\tilde{b}\tilde{c}
		}),
	\end{gather}
	
	\begin{enumerate}
	\item To get the energy-momentum of $S$, let's visit each of the subsystems respectively; first, BRST transformation of $X$ is given by:
	\begin{equation}
		\var{X^\mu}
		= i\epsilon\,\pqty{
			c\pd + \tilde{c}\pdbar
		} X^\mu
	\end{equation}
	Compared with the conformal transformation\footnote{
		We follow the convention of \textit{Polchinski} unless otherwise stated. 
	}: $
		\var{X^\mu}
		= -\epsilon\,\pqty{
			v\pd
			+ \tilde{v}\pdbar
		} X^\mu
	$, we see that they are in fact identical under the equivalence $
		-\epsilon v \sim i\epsilon c,\ %
		-\epsilon \tilde{v} \sim i\epsilon \tilde{c}
	$, hence we can simply follow the derivation of conformal current and write down $\var{S}^X$'s contribution to the conserved current:
	\begin{equation}
		j^X = c(z)\,T^X(z)
	\end{equation}
	
	The transformation of $b,c$ is less obvious; for holomorphic current, we need only focus on the holomorphic part of $S^{bc}$; on-shell variation yields:
	\begin{equation}
		0 = \var{S}^{bc}
		= \pqty{\frac{1}{2\pi} \int \dd[2]{z} \pqty{
			- \pdbar c\,\var{b}
			- \pdbar b\,\var{c}
		}}_{\!\!=\mspace{1mu}0}
		+ \frac{1}{2\pi} \int \dd[2]{z}
			\pdbar \pqty{b\var{c}}
		= \frac{1}{2\pi} \int \dd[2]{z}
			\pdbar\epsilon\,\pqty\big{
				-ibc\,\pd c
			} 
	\end{equation}
	Here we've plugged in $
		\var{c} = i\epsilon(z,\bar{z})\,c\pd c
	$, and we have moved $\pdbar\epsilon$ to the beginning of the expression, while respecting the anti-commuting nature of $\epsilon$. With a conventional $i$ coefficient (which agrees with the convention of $j^X$), we have $bc$'s contribution to the conserved current:
	\begin{equation}
		j^{bc} = i\,\pqty\big{
			-ibc\,\pd c
		} = bc\,\pd c
	\end{equation}
	
	Note that $j^{bc}$ is, in fact, related to the energy-momentum (at least classically):
	\begin{equation}
		\frac{1}{2}\,cT^{bc}
		= \frac{1}{2}\,c\,{(\pd b)\,c}
			- c\,\pd({bc})
		= - c\,\pd({bc})
		= - cb\,\pd c
		= bc\,\pd c = j^{bc}
	\end{equation}
	Hence we have the classical BRST current:
	\begin{equation}
		j(z) = c(z)\,\pqty{
			T^X + \frac{1}{2}\,T^{bc}
		}
	\end{equation}
	\vspace*{-1.5\baselineskip}
	
	\qed
	
	For a quantum version, redefine $j(z)$ with normal ordering\footnote{
		Normal ordering between $\ge 3$ operators is in fact \textit{not} associative; this directly leads to the ambiguity we are about to discover. See \textit{Di Francesco et al} for more detailed discussions. 
		Na\"ively, $\normorder{bc\,\pd c}_{(0)}$ is \textit{defined} as $b(0)\,c(z_1)\,\pd c(z_2)$ while $z_1,z_2\to 0$, with singular terms subtracted; however, different ways of taking the limit might lead to different results. For example, we can first take $z_1\to 0$ then $z_2\to 0$, or we can first take $z_1\to z_2$ then $z_2\to 0$. This two procedures will differ by $
			\frac{3}{2}\,\pd^2 c(z)
		$, which is precisely the correction we are about to find out. \sidenote{\textit{I suppose this is somehow related to topology, e.g.\ braid group?}}
	}, and we have:
	\begin{gather}
		T(z)\,j(0)
		\sim T^X(z)\,T^X(0)\,c(0)
			+ T^{bc}(z)\,cT^X(0)
			+ T^{bc}(z)\,
				\normorder{bc\,\pd c}_{(0)},
		\label{eq:BRST_Tj_OPE}\\
		\text{where}\quad
		T^X(z)\,T^X(0)\,c(0)
		\sim \pqty{
			\frac{D}{2z^4}
			+ \frac{2}{z^2}\,{T^X(0)}
			+ \frac{1}{z}\,{\pd T^X(0)}
		}\,c(0),
		\label{eq:BRST_Tj_OPE_partI}
	\end{gather}
	Here we've used the fact that $X$ and $b,c$ is de-coupled in the gauge-fixed action, hence their OPE is trivial. Also, we've expanded the first term using $TT$ OPE of the free boson. Additionally, note that $c(z)$ is primary with weight $(-1,0)$, we have:
	\begin{equation}
	\begin{aligned}
		T^{bc}(z)\,cT^X(0)
		&\sim \Bqty{T^{bc}(z)\,c(0)}\,T^X(0) \\
		&\sim \pqty{
			\frac{-1}{z^2}\,c(0)
			+ \frac{1}{z}\,\pd c(0)
		}\,T^X(0),
	\end{aligned}
	\label{eq:BRST_Tj_OPE_partII}
	\end{equation}
	The last term in \eqref{eq:BRST_Tj_OPE} can be brute-forced as follows:
	\begin{equation}
		T^{bc}(z)\,\normorder{bc\,\pd c}_{(0)}
		= \pqty\big{
			\normorder{(\pd b)\,c}
			- 2\,\pd(\normorder{bc})
		}_{(z)}\,\normorder{bc\,\pd c}_{(0)}\,,
	\end{equation}
	\vspace{-1\baselineskip}
	\begin{align}
		\normorder{(\pd b)\,c}_{(z)}\,
		\normorder{bc\,\pd c}_{(0)}
		&\sim \normorder{
			\wick{
				(\pd \c b)\,c_{(z)}\,
				b\c c\,\pd c_{(0)}
			}
		} + \normorder{
			\wick{
				(\pd \c b)\,c_{(z)}\,
				bc\,\pd \c c_{(0)}
			}
		} + \normorder{
			\wick{
				(\pd b)\,\c c_{(z)}\,
				\c b c\,\pd c_{(0)}
			}
		} \notag\\
		&\qquad + \normorder{
			\wick{
				(\pd \c1 b)\,\c2 c_{(z)}\,
				\c2 b \c1 c\,\pd c_{(0)}
			}
		} + \normorder{
			\wick{
				(\pd \c1 b)\,\c2 c_{(z)}\,
				\c2 bc \,\pd \c1 c_{(0)}
			}
		} \notag\\[.5ex]
		&\sim \frac{-1}{z^2}\,(+1)\,\normorder{
			c_{(z)}\,b\,\pd c_{(0)}
		} + \frac{-2}{z^3}\,(-1)\,\normorder{
			c_{(z)}\,bc_{(0)}
		} + \frac{1}{z}\,(+1)\,\normorder{
			\pd b_{(z)}\,c\pd c_{(0)}
		} \notag\\
		&\qquad + \frac{-1}{z^2}\cdot\frac{1}{z}\,
			(+1)\,\pd c(0)
		+ \frac{-2}{z^3}\cdot\frac{1}{z}\,
			(-1)\,c(0) \notag\\[.5ex]
		&\sim \frac{-1}{z^2}\,\pqty{
			-j^{bc}(0) + \order*{z^2}
		} + \frac{2}{z^3}\,\pqty{
			z\,j^{bc}(0)
			+ \frac{z^2}{2}\,\normorder{
				bc\,\pd^2 c
			}_{(0)}
			+ \order*{z^3}
		} \notag\\[-.3ex]
		&\qquad + \frac{1}{z}\,\pqty\big{
			\normorder{
				(\pd b)\,c\,\pd c
			}_{(0)} + \order{z}
		} + \frac{-1}{z^3}\,\pd c(0)
			+ \frac{2}{z^4}\,c(0) \notag\\[.7ex]
		&\sim \frac{4}{2z^4}\,c(0)
			+ \frac{-1}{z^3}\,\pd c(0)
			+ \frac{3}{z^2}\,j^{bc}(0)
			+ \frac{1}{z}\,\normorder{\pqty{
				bc\,\pd^2 c
				+ (\pd b)\,c\,\pd c
			}}_{(0)}\,, \notag\\[1ex]
		&\sim \frac{4}{2z^4}\,c(0)
			+ \frac{-1}{z^3}\,\pd c(0)
			+ \frac{3}{z^2}\,j^{bc}(0)
			+ \frac{1}{z}\,\pd j^{bc}(0), \\[3ex]
%%%%%%%%%%%%%%%%%%%%%%%%%%%%%%%%
%		THIS IS INSANE!
%		ANOTHER OPE ...
%%%%%%%%%%%%%%%%%%%%%%%%%%%%%%%%
		\normorder{bc}_{(z)}\,
		\normorder{bc\,\pd c}_{(0)}
		&\sim \normorder{
			\wick{
				\c1 b c_{(z)}\,
				b \c1 c\, \pd c_{(0)}
			}
		} + \normorder{
			\wick{
				\c1 b c_{(z)}\,
				b c\, \pd \c1 c_{(0)}
			}
		} + \normorder{
			\wick{
				b \c1 c_{(z)}\,
				\c1 b c\, \pd c_{(0)}
			}
		} \notag\\
		&\qquad + \normorder{
			\wick{
				\c1 b \c2 c_{(z)}\,
				\c2 b \c1 c\, \pd c_{(0)}
			}
		} + \normorder{
			\wick{
				\c1 b \c2 c_{(z)}\,
				\c2 b c\, \pd \c1 c_{(0)}
			}
		} \notag\\[.5ex]
		&\sim \frac{1}{z}\,(+1)\,\normorder{
			c_{(z)}\,b\,\pd c_{(0)}
		} + \frac{1}{z^2}\,(-1)\,\normorder{
			c_{(z)}\,bc_{(0)}
		} + \frac{1}{z}\,(+1)\,\normorder{
			b_{(z)}\,c\pd c_{(0)}
		} \notag\\
		&\qquad + \frac{1}{z}\cdot\frac{1}{z}\,
			(+1)\,\pd c(0)
		+ \frac{1}{z^2}\cdot\frac{1}{z}\,
			(-1)\,c(0) \notag\\[1ex]
		\normorder{bc}_{(z)}\,
		\normorder{bc\,\pd c}_{(0)}
		&\sim \frac{1}{z}\,\pqty\big{
			-j^{bc}(0)
		} + \frac{-1}{z^2}\,\pqty\big{
			z\,j^{bc}(0)
		} + \frac{1}{z}\,\pqty\big{
			j^{bc}(0)
		} + \frac{1}{z^2}\,\pd c(0)
		+ \frac{-1}{z^3}\,c(0) \notag\\[1ex]
		&\sim \frac{-1}{z^3}\,c(0)
			+ \frac{1}{z^2}\,\pd c(0)
			+ \frac{-1}{z}\,j^{bc}(0), \\[2ex]
%%%%%%%%%%%%%%%%%%%%%%%%%%%%%%%%
%		DERIVATIVE
%%%%%%%%%%%%%%%%%%%%%%%%%%%%%%%%
		\pd (\normorder{bc})_{(z)}\,
		\normorder{bc\,\pd c}_{(0)}
		&\sim \frac{6}{2z^4}\,c(0)
			+ \frac{-2}{z^3}\,\pd c(0)
			+ \frac{1}{z^2}\,j^{bc}(0),
	\end{align}
	\begin{gather}
		T^{bc}(z)\,\normorder{bc\,\pd c}_{(0)}
		\sim \frac{-8}{2z^4}\,c(0)
			+ \frac{3}{z^3}\,\pd c(0)
			+ \frac{1}{z^2}\,j^{bc}(0)
			+ \frac{1}{z}\,\pd j^{bc}(0),
		\label{eq:BRST_Tj_OPE_partIII}\\[2ex]
		T(z)\,j(0)
		\sim \pqty\big{
			\eqref{eq:BRST_Tj_OPE_partI}
			+ \eqref{eq:BRST_Tj_OPE_partII}
			+ \eqref{eq:BRST_Tj_OPE_partIII}
		}
		\sim \frac{D-8}{2z^4}\,c(0)
			+ \frac{3}{z^3}\,\pd c(0)
			+ \frac{1}{z^2}\,j(0)
			+ \frac{1}{z}\,\pd j(0),
	\end{gather}
	
	We see that $j(z)$ defined with na\"ive normal ordering is \textit{almost} but \textit{not quite} a primary. It differs from primary OPE at $\order{\frac{1}{z^4}}$ and $\order{\frac{1}{z^3}}$. However, it is possible to make it into a primary by adding extra terms that do not interfere with current conservation $\pdbar j = 0$. To cancel the $\frac{3}{z^3}\,\pd c(0)$ term, notice that $
		b(z)\,\pd^2 c(0) \sim \frac{2}{z^3}
	$, therefore it may be helpful to look at:
	\begin{equation}
	\begin{aligned}
		T(z)\,\pd^2 c(0)
		\sim T^{bc}(z)\,\pd^2 c(0)
		&\sim \pd^2_w \pqty{
			T^{bc}(z)\,c(w)
		}_{w\to 0} \\
		&\sim \pd^2_w \pqty{
			\frac{-1}{(z-w)^2}\,c(w)
			+ \frac{1}{z-w}\,\pd c(w)
		}_{\!w\to 0} \\[.5ex]
		&\sim \frac{-12}{2z^4}\,c(0)
			+ \frac{-2}{z^3}\,\pd c(0)
			+ \frac{1}{z^2}\,\pd^2 c(0)
			+ \frac{1}{z}\,\pd^3 c(0),
	\end{aligned}
	\label{eq:BRST_Td2c_OPE}
	\end{equation}
	Again we've used $Tc$ OPE of the primary $c(w)$. We see that indeed, the $\frac{1}{z^3}\,\pd c(0)$ term can be canceled by shifting $j(z)$:
	\begin{gather}
		j(z)\ \longmapsto\ j(z)
			+ \frac{3}{2}\,\pd^2 c(z),\quad
		j(z) = cT^X + \normorder{bc\,\pd c}
			+ \frac{3}{2}\,\pd^2 c,\\[1ex]
		T(z)\,j(0)
		\sim \frac{D-26}{2z^4}\,c(0)
			+ \frac{1}{z^2}\,j(0)
			+ \frac{1}{z}\,\pd j(0),
	\end{gather}
	
	We see that $j(z)$ defined in this way is a primary of weight $(1,0)$ in $D = 26$. This is the quantum BRST current. 
	\qed
	
%	note that $b,c$ are primaries with weight $(2,0)$ and $(-1,0)$, hence $Tb,Tc$ OPEs take the standard form of a $T\cdot(\text{primary})$ OPE. We have:
%	\begin{equation}
%	\begin{aligned}
%		T^{bc}(z)\,\normorder{bc\,\pd c}_{(0)}
%		&\sim \pqty{
%				\frac{2}{z^2}\,
%					\normorder{bc\,\pd c}
%				+ \frac{1}{z}\,\normorder{
%					(\pd b)\,c\pd c
%				}
%			} \\
%		&\phantom{\sim \quad}
%			+ \pqty{
%				\frac{-1}{z^2}\,
%					\normorder{bc\,\pd c}
%				+ \frac{1}{z}\,\normorder{
%					b\,(\pd c)(\pd c)
%				}
%			} \\
%		&\phantom{\sim \quad}
%			+ \pqty{
%				\frac{-1}{z^2}\,
%					\normorder{bc\,\pd c}
%				+ \frac{1}{z}\,\normorder{
%					bc\,\pd^2 c
%				}
%			} \\
%		&\sim \frac{1}{z}\,\pd\,
%			\normorder{bc\,\pd c}
%		\sim \frac{1}{z}\,\pd\,\pqty{
%			j - cT^X
%		}_{(0)} \\
%		&\sim \frac{1}{z}\,\pd j(0)
%		- \frac{1}{z}\,c(0)\,\pd T^X(0)
%		- \frac{1}{z}\,T^X(0)\,\pd c(0),
%	\end{aligned}\\
%		T(z)\,j(0)
%		\sim \frac{D}{2z^4}\,c(0)
%			+ \frac{2}{z^2}\,T^X(0)\,c(0)
%			+ \frac{1}{z}\,\pqty{
%				\pd j(0)
%				- T^X(0)\,\pd c(0)
%			}
%	\end{equation}
	
	\item For $jj$ OPE, we have:
	\begin{gather}
		j = cT^X + j',\quad
		j' \equiv j^{bc}
			+ \frac{3}{2}\,\pd^2 c,\quad
		j^{bc}
		= \frac{1}{2}\normorder{cT^{bc}}
		= \normorder{bc\,\pd c}\,,\\[1ex]
%		
%		j = \frac{1}{2}\,cT^X
%			+ \frac{1}{2}\,\normorder{cT}
%			+ \frac{3}{2}\,\pd^2 c,\\
%
	\begin{aligned}
		j_z j_0
		&\sim \normorder{\Bqty{T^X_z T^X_0}\,c_z c_0}
			+ \normorder{\Bqty{c_z j'_0}\,T^X_z}
			+ \normorder{\Bqty{j'_z c_0}\,T^X_0}
			+ {j'_z j'_0} \\[.5ex]
		&\sim \normorder{\Bqty{T^X_z T^X_0}\,c_z c_0}
			+ \normorder{\Bqty{c_z j^{bc}_0}\,T^X_z}
			+ \normorder{\Bqty{j^{bc}_z c_0}\,T^X_0}
			+ {j'_z j'_0}\,,
	\end{aligned}
	\end{gather}
	From now on, for convenience and clarity, we will use subscripts to denote variable dependence: $c_z = c(z)$. Let's compute this term by term. We have:
	\begin{align}
		\normorder{\Bqty{T^X_z T^X_0}\,c_z c_0}
		&\sim \normorder{\pqty{
			\frac{D}{2z^4}
			+ \frac{2}{z^2}\,{T^X_0}
			+ \frac{1}{z}\,{\pd T^X_0}
		}\pqty{
			z\,\pd c_0
			+ \frac{z^2}{2}\,\pd^2 c_0
			+ \frac{z^3}{6}\,\pd^3 c_0
		} c_0} \notag\\
		&\sim -\pqty{
			\frac{D}{2z^3}\,c\,\pd c_0
			+ \frac{D}{4z^2}\,c\,\pd^2 c_0
			+ \frac{D}{12z}\,c\,\pd^3 c_0
			+ \frac{2}{z}\,\normorder{T^X c\,\pd c_0}
		}, 
		\label{eq:BRST_TTcc_OPE} \\[2ex]
%%%%%%%%%%%%%%%%%%%%%%%%%%%%%%%%
%		jc OPE ...
%%%%%%%%%%%%%%%%%%%%%%%%%%%%%%%%
		j^{bc}_z c_0
		&\sim \frac{1}{2}\,
			\normorder{cT^{bc}}_{z}\,c_0
		\sim \frac{1}{2}\,c_z \Bqty{
			\normorder{T^{bc}}_{z}\,c_0
		}
		\sim \frac{1}{2}\,c_z \Bqty{
			T_z\,c_0
		} \notag\\
		&\sim -\frac{1}{2}\,\pqty{
			\frac{-1}{z^2}\,c_0
			+ \frac{1}{z}\,\pd c_0
		}\pqty\big{
			c_0 + z\,\pd c_0
		} \sim 0, \\
		j^{bc}_0 c_z
		&\sim 0,
	\end{align}
	\begin{equation}
	\begin{aligned}
		j'_z j'_0
		&\sim j^{bc}_z j^{bc}_0
			+ \frac{3}{2}\,j^{bc}_z\,\pd^2 c_0
			+ \frac{3}{2}\,\pd^2 c_z\,j^{bc}_0 \\
		&\sim \frac{1}{2}\,
			\normorder{cT^{bc}}_{z}\,j^{bc}_0
		+ \frac{3}{2}\,\pqty{
			j^{bc}_z\,\pd^2 c_0
			+ \pd^2 c_z\,j^{bc}_0
		},
	\end{aligned}
	\end{equation}
	
	The task is now reduced to calculating terms in the above $j'j'$ OPE, which can be laboriously computed following a similar procedure as before. Note that there will be a $
		\frac{1}{z}\,\normorder{cT^{bc}}\,\pd c_0
	$ term which combines with the $
		\frac{2}{z}\,\normorder{cT^{X}}\,\pd c_0
	$ term in \eqref{eq:BRST_TTcc_OPE}. In total, we obtain the final $jj$ OPE:
	\begin{equation}
		j_z j_0
		\sim - \frac{D-18}{2z^3}\,c\,\pd c_0
			- \frac{D-18}{4z^2}\,c\,\pd^2 c_0
			- \frac{D-26}{12z}\,c\,\pd^3 c_0
	\end{equation}
	
%	For the first term, we have:
%	\begin{equation}
%	\begin{aligned}
%		\normorder{cT^{bc}}_{z}\,j^{bc}_0
%		&\sim \normorder{\wick{
%			\c1 c T^{bc}_z \c1 j^{bc}_0
%		}} + \normorder{\wick{
%			c \c1 T^{bc}_z \c1 j^{bc}_0
%		}} + \normorder{\wick{
%			\bcontraction[1.1ex]{c}{T}{^{bc}_z}{j}
%			\c1 c T^{bc}_z \c1 j^{bc}_0
%		}} \\[.5ex]
%		&\sim \frac{1}{z}\,(+1)\,\normorder{
%			T^{bc}_z c\,\pd c_0
%		} + \normorder{c_z \Bqty{
%			\eqref{eq:BRST_Tj_OPE_partIII}
%		}} + \frac{1}{z}\,\normorder{\pqty\Big{
%			\Bqty{T^{bc}_z c_0}\,\pd c_0
%			+ (+1)\,
%				c_0\,\Bqty{T^{bc}_z\,\pd c_0}
%		}} \\
%		&\sim \frac{1}{z}\,\normorder{
%			T^{bc} c\,\pd c_0
%		} + \normorder{c_z \Bqty{
%			\eqref{eq:BRST_Tj_OPE_partIII}
%		}} + \frac{1}{z}\,\normorder{\pqty\Big{
%			\Bqty{T^{bc}_z c_0}\,\pd c_0
%			+ c_0\,\Bqty{T^{bc}_z\,\pd c_0}
%		}}
%	\end{aligned}
%	\end{equation}
%	Here we've used the fact that there is only one $b$ in $T^{bc}$, hence all non-singular are accounted for in the above expression. Note again that $c$ is primary; similar to \eqref{eq:BRST_Td2c_OPE}, we have:
%	\begin{equation}
%	\begin{aligned}
%		\normorder{cT^{bc}}_{z}\,j^{bc}_0
%		&\sim \normorder{\pqty{
%			c_0 + z\,\pd c_0
%			+ \frac{z^2}{2}\,\pd^2 c_0
%			+ \frac{z^3}{6}\,\pd^3 c_0
%		} \pqty{
%			\frac{-4}{z^4}\,c_0
%			+ \frac{3}{z^3}\,\pd c_0
%			+ \frac{1}{z^2}\,j^{bc}_0
%			+ \frac{1}{z}\,\pd j^{bc}_0,
%		}} \\
%		&\qquad + \frac{1}{z}\,\pqty{
%			\frac{-1}{z^2}\,c\,\pd c_0
%			+ c_0\cdot\frac{1}{z}\,\pd^2 c_0
%		} \\
%		&\sim \frac{3+4-1}{z^3}\,c\,\pd c_0
%			+ \frac{4/2+1}{z^2}\,c\,\pd^2 c_0
%			+ \frac{4/6}{z}\,c\,\pd^3 c_0 \\
%	\end{aligned}
%	\end{equation}
%	
%	
%	\begin{align}
%%%%%%%%%%%%%%%%%%%%%%%%%%%%%%%%%
%%		ANOTHER OPE ...
%%%%%%%%%%%%%%%%%%%%%%%%%%%%%%%%%
%		j^{bc}_z\,\pd^2 c_0
%		&\sim \frac{1}{2}\,c_z\,\Bqty{
%			T^{bc}_z\,\pd^2 c_0
%		} \\
%		\sim\eqref{eq:BRST_Td2c_OPE}
%		&\sim \frac{1}{2}\,c_z\,\pqty{
%			\frac{-6}{z^4}\,c_0
%			+ \frac{-2}{z^3}\,\pd c_0
%			+ \frac{1}{z^2}\,\pd^2 c_0
%			+ \frac{1}{z}\,\pd^3 c_0
%		} \\
%		\sim (\textsl{Taylor})
%		&\sim \frac{1}{2}\,\pqty{
%			\frac{6-2}{z^3}\,
%				c_0\,\pd c_0
%			+ \frac{6/2+1}{z^2}\,
%				c_0\,\pd^2 c_0
%			+ \frac{6/6+1}{z}\,
%				c_0\,\pd^3 c_0
%		} \\
%		&\sim \frac{1}{2}\,\pqty{
%			\frac{4}{z^3}\,
%				c_0\,\pd c_0
%			+ \frac{4}{z^2}\,
%				c_0\,\pd^2 c_0
%			+ \frac{2}{z}\,
%				c_0\,\pd^3 c_0
%		} \\[2ex]
%%%%%%%%%%%%%%%%%%%%%%%%%%%%%%%%%
%%		ANOTHER OPE ...
%%%%%%%%%%%%%%%%%%%%%%%%%%%%%%%%%
%		\pd^2 c_z\,j^{bc}_0
%		&\sim -j^{bc}_0\,\pd^2 c_z \\
%		&\sim - \frac{1}{2}\,c_0\,\Bqty{
%			T^{bc}_0\,\pd^2 c_z
%		} \\
%		\sim\eqref{eq:BRST_Td2c_OPE}
%		&\sim -\frac{1}{2}\,c_0\,\pqty{
%			\frac{-2}{(-z)^3}\,\pd c_z
%			+ \frac{1}{(-z)^2}\,\pd^2 c_z
%			+ \frac{1}{-z}\,\pd^3 c_z
%		} \\
%		&\sim -\frac{1}{2}\,c_0\,\pqty{
%			\frac{2}{z^3}\,\pd c_z
%			+ \frac{1}{z^2}\,\pd^2 c_z
%			+ \frac{-1}{z}\,\pd^3 c_z
%		} \\
%		\sim (\textsl{Taylor})
%		&\sim -\frac{1}{2}\,\pqty{
%			\frac{2}{z^3}\,c_0\,\pd c_0
%			+ \frac{1+2}{z^2}\,c_0\,\pd^2 c_0
%			+ \frac{-1+1+1}{z}\,c_0\,\pd^3 c_0
%		} \\
%		&\sim -\frac{1}{2}\,\pqty{
%			\frac{2}{z^3}\,c_0\,\pd c_0
%			+ \frac{3}{z^2}\,c_0\,\pd^2 c_0
%			+ \frac{1}{z}\,c_0\,\pd^3 c_0
%		} \\[1ex]
%%%%%%%%%%%%%%%%%%%%%%%%%%%%%%%%%
%		\frac{3}{2}\,\pqty{
%			j^{bc}_z\,\pd^2 c_0
%			+ \pd^2 c_z\,j^{bc}_0
%		} &\sim \frac{3}{4}\,\pqty{
%			\frac{2}{z^3}\,c_0\,\pd c_0
%			+ \frac{1}{z^2}\,c_0\,\pd^2 c_0
%			+ \frac{1}{z}\,c_0\,\pd^3 c_0
%		}
%	\end{align}
	
	\item Following the convention of \textit{Polchinski}, expand $X^\mu,b,c$ into modes $\alpha^\mu_n,b_n,c_n$, then a generic level 2 state of an open string can be created as\footnote{
		Reference: Bram M.~Wouters, \href{https://esc.fnwi.uva.nl/thesis/centraal/files/f1989820784.pdf}{\textit{BRST quantization and string theory spectra}}. 
	}:
	\begin{equation}
	\begin{aligned}
		\ket{\psi} = \big(
			&e_{\mu\nu}
				\alpha^\mu_{-1}
				\alpha^\nu_{-1}
			+ \beta_\mu
				\alpha^\mu_{-1} b_{-1}
			+ \gamma_\mu
				\alpha^\mu_{-1} c_{-1}
			\\ &\qquad
			+ \eta\,b_{-1} c_{-1}
			+ e_\mu \alpha^\mu_{-2}
			+ \beta\,b_{-2}
			+ \gamma\,c_{-2}
		\big)\,\ket{k;0}
	\end{aligned}
	\end{equation}
	Here $e_{\mu\nu}$ is chosen to be symmetric since $
		\alpha^\mu_{-1}
		\alpha^\nu_{-1}
	$ commutes. 
	By acting on $L_0$ (expanded in modes), we find that $
		m^2 = -k^2 = \frac{1}{\alpha'} = l_s
	$: massive. 
	
	The BRST charge $
		Q = \frac{1}{2\pi i} \oint \pqty{
			\dd{z} j(z)
			- \dd{\bar{z}} \tilde{j}(z)
		}
	$ can also be expanded in modes; note that:
	\begin{equation}
		Q^2 = \frac{1}{2}\,\{Q,Q\}
		\propto \oint \frac{\dd{z}}{2\pi i} 
			\operatorname*{Res}\limits_{z'\to z}
			j(z')\,j(z) + \pqty{\mrm{conjugate}}
	\end{equation}
	Compared with the $jj$ OPE, we see that $Q$ is nilpotent iff.\ $D = 26$, i.e.\ the critical dimension of bosonic string theory. This condition is necessary for consistent BRST quantization. 
	
	The physical states are firstly, $Q$-closed; i.e.
	\begin{equation}
		Q_B \ket{\psi} = 0
		\ \Longrightarrow\ %
		4l_s\,k^\mu e_{\mu\nu}
			+ l_s\,k_\nu\eta
			+ e_\nu = 0,\quad
		2\sqrt{2}\,l_s\,k^\mu
			+ e^\nu_\nu e_\mu = 0,\quad
		\beta_\mu = \beta = 0,
	\end{equation}
	This is also the negative-norm states. 
	
	On the other hand, $Q$-exact states generate gauge transformations; this gives:
	\begin{equation}
		\gamma_\nu
		\mapsto \gamma_\nu
			+ \gamma'_\nu,\quad
		\gamma
		\mapsto \gamma
			+ \gamma',\quad
		\eta
		\mapsto \eta + \eta',\quad
		e_{\mu\nu}
		\mapsto e_{\mu\nu} + l_s\,\pqty{
			\beta'_\mu k_\nu
			+ \beta'_\nu k_\mu
		},
%		\quad
%		e_\mu
%		\mapsto e_\mu + \beta k_\mu\,l_s
	\end{equation}
	Here $
		\beta'_\mu,\gamma'_\nu,\gamma',\eta'
	$ are arbitrary gauge parameters. For closed string the result can be obtained by the doubling trick, i.e.\ by introducing anti-holomorphic modes $\tilde{\alpha},\tilde{b},\tilde{c}$ and imposing reality conditions. The result is similar. \qedfull
	
	\end{enumerate}
	
	\item \textbf{Linear Dilaton CFT:}
	
	For $z\mapsto z + \epsilon(z)$, we have:
	\begin{equation}
		\var{X^\mu}
		= - \epsilon\pd X^\mu
			- \bar{\epsilon}\pdbar X^\mu
			- \frac{\alpha'V^\mu}{2}\,\pqty{
				\pd\epsilon
				+ \pdbar\bar{\epsilon}
			}
	\end{equation}
	Note that the $\alpha'$ term has no dependence on $X$. 
	
	\begin{enumerate}
	\item For simplicity, assume for now $X = X(z)$: holomorphic. Note that the $\alpha'$ term comes from the transformation of ``\textit{internal}'' degrees of freedom, associated with the conformal properties of $X$. We have:
	\begin{gather}
		X'(z') - X(z)
		= -\frac{\alpha' V}{2}\,\pd\epsilon
			+ \order{\epsilon^2}
%		\simeq -\frac{\alpha' V}{2}\,\pqty{
%			\dv{z'}{z} - 1
%		},\\[.5ex]
%		X'(z') - \frac{\alpha' V}{2}
%		\simeq X(z) - \frac{\alpha' V}{2}\,
%			\dv{z'}{z},
%		\\[1ex]
%		X'(z')
%		\simeq X(z) + \frac{\alpha' V}{2}\,\pqty{
%				1 - \pd w + \order{\pd w}^2
%			},
		\label{eq:dilaton_holomorphic}
	\end{gather}
	This is a first order approximation of the finite transformation, where the transformation parameters are the modes $\epsilon_n$ of $\epsilon(z)$; namely, we have:
	\begin{equation}
		w(z) = z + \epsilon(z) + \order{\epsilon^2},
	\quad
		\epsilon(z) = \sum_n \epsilon_n z^n
	\end{equation}
	\vspace{-.5\baselineskip}
	\begin{equation}
		F[w(z)] = X'(z') - X(z)
	\ \xlongrightarrow{w\to 0\,}\ %
		-\frac{\alpha' V}{2}\,\pd\epsilon
			+ \order{\epsilon^2}
	\end{equation}
	
%	For the finite ``internal'' transformation induced by $w(z)$, 
	What the above actually means is that:
	\begin{equation}
		\fdv{\epsilon_n} F[w(z)]_{\epsilon\to 0}
		= -\frac{\alpha' V}{2}\,\fdv{\epsilon_n}\,
			\pdd{z} \pqty\big{w(z) - z}
		= -\frac{\alpha' V}{2}\,
			n z^{n-1}
	\end{equation}
	Where $\epsilon\to 0$ corresponds to $w \to z$, i.e.~the transformation goes to the identity. 
	On the other hand,
	\begin{equation}
	\begin{aligned}
		\fdv{F}{\epsilon_n}
		&= \pdv{F}{w} \fdv{w}{\epsilon_n}
			+ \pdv{F}{(\pd w)} \fdv{(\pd w)}{\epsilon_n}
			+ \pdv{F}{(\pd^2 w)} \fdv{(\pd^2 w)}{\epsilon_n}
			+ \cdots \\
		&= \pdv{F}{w}\, z^n
			+ \pdv{F}{(\pd w)}\, n z^{n-1}
			+ \pdv{F}{(\pd^2 w)}\, n(n-1)\,z^{n-2}
			+ \cdots
	\end{aligned}
	\end{equation}
	
	By comparing the two above equations, and noting that $\pdv{F}{(\pd^\bullet w)}$ should have no dependence on $n$, we obtain the following constraints on the form of $F[w(z)]$:
	\begin{equation}
		F|_{w\to z} = 0,
	\quad
		\pdv{F}{w}\bigg|_{w\to z} = 0,
	\quad
		\pdv{F}{(\pd w)}\bigg|_{w\to z}
		= -\frac{\alpha' V}{2},
	\quad
		\pdv{F}{(\pd^k w)}\bigg|_{w\to z} = 0,
	\quad k = 2,3,\cdots
	\end{equation}
	We can think of this as the first order ``Taylor'' coefficients of $F[w]$ in the functional space, around the point $w(z) \to z$. Note that $\pd w|_{w\to z} = 1$, while $
		\pd^k w|_{w\to z} = 0
	$, it is thus natural to consider the following ansatz:
	\begin{equation}
		F = F[\pd w],
	\quad
		F[1] = 0,
	\quad
		\pdv{F[x]}{x}\bigg|_{x\to 1}
		= -\frac{\alpha' V}{2}
	\end{equation}
	
	In the end we shall obtain that\footnote{
		I would like to thank Lucy Smith for helpful discussions. 
	}:
	\begin{equation}
	\begin{aligned}
		X'(z',\bar{z}') - X(z,\bar{z})
		&= -\frac{\alpha' V}{2}\,
			\ln \pqty{
				\dv{z'}{z}
				\dv{\bar{z}'}{\bar{z}}
			}
	\end{aligned}
	\end{equation}
	A better recipe to find finite transformations is to consider its properties under composition, which will lead to some constraints that can be solved to obtain the result\footnote{
		See \https{bryango.github.io/resources/archive/alpha/schwarzian.pdf} for some detailed discussions.
	}. 
	
	\item Perform the usual Noether's procedure on the free boson action, and we have:
	\begin{equation}
		\var{\mcal{L}}
		\propto \frac{1}{\alpha'} \pqty{
				\pd \var{X}^\mu \pdbar X_\mu
				+ \pd X^\mu \pdbar \var{X}_\mu
			}
		\sim \pdbar \epsilon \pqty{
			V^\mu \pd^2 X^\mu
			- \frac{1}{\alpha'}
				\pd X^\mu \pdbar X_\mu
		}
	\end{equation}
	Here we've plugged in the holomorphic part of $\var{X}^\mu$, used integration by parts to move $\pdbar$ before $\epsilon$, and collected the $\pdbar\epsilon$ coefficients. This gives:
	\begin{equation}
		T(z) = - \frac{1}{\alpha'}\,\normorder{
				\pd X^\mu \pdbar X_\mu\!
			} + V^\mu \pd^2 X^\mu
	\end{equation}
\pagebreak[3]
	
	With $
		X^\mu_z X^\nu_0
		\sim -\frac{\alpha'}{2}\,\eta^{\mu\nu}
			\ln \abs{z}^2
	$ unchanged, the $TT$ OPE can be calculated following the usual procedure, as shown in great detail before. Here we can use the known result from free boson theory to speed up our calculation:
	\begin{equation}
	\begin{aligned}
		T_z T_0
		&\sim \pqty{
				V_\mu \pd^2 X^\mu + T'
			}_{\!z}\,\pqty{
				V_\mu \pd^2 X^\mu + T'
			}_{\!0} \\
		&\sim V_\mu V_\nu\,
				\pd^2 X^\mu_z\,\pd^2 X^\nu_0
			+ V_\mu\,\pd^2 X^\mu_z\,T'_0
			+ V_\mu T'_z\,\pd^2 X^\mu_0
			+ T'_z\,T'_0
	\end{aligned}
	\end{equation}
	Here $T'$ is the usual free boson stress tensor. Combining all terms yields:
	\begin{equation}
		T_z T_0
		\sim \frac{D + 6\alpha' V^2}{2z^4}
			+ \frac{2}{z^2}\,T_0
			+ \frac{1}{z}\,\pd T_0,\quad
		c = D + 6\alpha' V^2
	\end{equation}
	\vspace{-1.8\baselineskip}
	\end{enumerate}
	\qedfull
	
	\item \textbf{Bosonic Strings on $S^3$:}
	
	For bosonic strings moving on $S^3$ (radius $R$) with background dilaton $\Phi = \mrm{const.}$ and $B$-field:
	\begin{equation}
		B = R^2\sin\theta\,\pqty{
			\psi - \sin\psi \cos\psi
		}\,\dd{\theta}\wedge \dd{\phi}
	\end{equation}
	The corresponding $\beta$-functions and trace anomaly can be computed using the formulae given in \textit{Polchinski}; here $(\psi,\theta,\phi)$ is the usual spherical coordinates on $S^3$. 
\pagebreak[3]
	
	In fact, field strength:
	\begin{equation}
		H = \dd{B}
		= 2R^2 \sin\theta \sin\psi 
			\dd{\psi}
			\wedge\dd{\theta}
			\wedge\dd{\phi}
	\end{equation}
	While the spacetime curvature for a maximally symmetric sphere\footnote{
		I would like to thank \textit{林般} for some very helpful hints. 
	}: $
		\mscr{R}_{\mu\nu}
		= \frac{2}{R^2}\,g_{\mu\nu},\ %
		\mscr{R} = \frac{6}{R^2}
	$. Plug in these results, and we have:
	\begin{equation}
		\beta^G = \beta^B = 0,\quad
		T\id{^a_a}
		\simeq -\frac{1}{2}\,\beta^{\Phi} \mcal{R}
		= -\frac{D-26 - \alpha'\mscr{R}}{12}\,
			\mcal{R}
	\end{equation}
	
	\begin{enumerate}
	\item Compared with the trace anomaly formula of a CFT: $
		T\id{^a_a} = -\frac{1}{12}\,c\mcal{R}
	$, where $\mcal{R}$ is the worldsheet Ricci scalar, we see that our theory is indeed conformally invariant with Weyl anomaly. Its central charge is given by:
	\begin{equation}
		c \simeq D-26 - \alpha'\mscr{R}
		= 3 - 26 - \frac{6\alpha'}{R^2}
	\end{equation}
	This includes ghost contribution $(-26)$. If we do not gauge the conformal symmetry, then there will not be ghost contribution, and we will have $
		c \simeq 3 - \frac{6\alpha'}{R^2}
	$. 
	
	\item The background $B$ field given above is not single-valued on the $\psi$ circle. Note that we've encountered such difficulty in electromagnetism with a multi-valued $A^\mu(x)$. In fact, the resolution for this issue is very similar to Dirac's quantization of the magnetic monopole\footnote{
		Reference: J.~J.~Sakurai, \textit{Modern Quantum Mechanics}. 
	}: by allowing the action $S$ to be invariant modulo $2\pi$, since $
		e^{-(S + 2\pi i)} = e^{-S}
	$. 
	
	More specifically, for $\psi\mapsto\psi + 2\pi$, we have:
	\begin{equation}
	\begin{aligned}
		2\pi i\, n = \Delta S
%		&= \frac{1}{4\pi\alpha'}
%		\int \dd[2]{\sigma}
%			i\epsilon^{ab} 
%			R^2\sin\theta\,\pqty{
%				2\pi - 0
%			}\,\pqty{
%				\pdd{a} \theta\,
%				\pdd{b} \phi
%				- \pdd{a} \phi\,
%				\pdd{b} \theta
%			} \\[.5ex]
%		&= i\,\frac{R^2}{\alpha'}
%		\int \dd[2]{\sigma}
%			\sin\theta\,\epsilon^{ab}\,
%				\pdd{a} \theta\,
%				\pdd{b} \phi
		= \frac{i}{2\pi\alpha'}
			\,\Delta\! \int_\Sigma
				X^* B
		= \frac{i}{2\pi\alpha'}
			\,\Delta\! \int_{X(\Sigma)}
				B
		= \frac{i}{2\pi\alpha'}
			\,\Delta\! \int_M H
	\end{aligned}
	\end{equation}
	$B$ is a 2-form in $S^3$, $X^* B$ denotes its pullback to the worldsheet, and $X(\Sigma)\subset S^3$ denotes the embedding of $\Sigma$ into $S^3$. Note that $H$ is proportional to the volume form in $S^3$, hence we have:
	\begin{equation}
		\Delta\! \int_M H
		= 2R^2\,\Delta\! \mop{Vol}(M)
		= 2R^2\,\mbb{Z} \mop{Vol}(S^3)
		= 2R^2\,2\pi^2\,\mbb{Z}
		= 4\pi^2 R^2\,\mbb{Z}
	\end{equation}
%	Choose $
%		(\sigma^1,\sigma^2) = (\theta,\phi)
%	$ while $
%		\psi = \psi(\theta,\phi)
%	$, then the integral is reduced to $
%		\tilde{A}
%		= \int \dd[2]{\sigma} \sin\theta
%		= 2\pi R\tilde{\ell}
%	$, where $\tilde{\ell}$ is the string length weighted by $\sin\theta$. 
	This leads to the following quantization:
	\begin{equation}
%		\frac{R^3 \tilde{\ell}}{\alpha'}
		\frac{R^2}{\alpha'}
		= n \in\mbb{Z},\quad
		R \ge \sqrt{\alpha'}
%		R = \pqty\big{
%			{n\alpha'}\big/{\tilde{\ell}}
%			\mspace{3.5mu}
%		}^{1/3}
		\ge \pqty\big{
			{\alpha'}\big/{\ell}
			\mspace{2.5mu}
		}^{1/3}
	\end{equation}
	In particular, in string units: $
		\alpha'
%		= \ell
		= 1
	$, we have $R\ge 1$.
	\qedfull
	\end{enumerate}
	
	\item \textbf{Anomalous Currents:}
	
	\begin{enumerate}
	\item For a conserved current in flat worldsheet to be anomalous in curved worldsheet, then its deviation from conservation must be proportional to the Ricci scalar:
	\begin{equation}
		\cdv{a} j^a = QR,\quad
		Q = \mrm{const}. 
	\end{equation}
	The logic here is similar to the Weyl anomaly\footnote{
		See \textit{Polchinski} for reference. 
	}: $\cdv{a} j^a$ is diff- and Poincaré-invariant with dimension~2, because we have preserved these symmetries, and it vanishes in the flat case; this leaves only one possibility --- $
		\cdv{a} j^a \propto R
	$: the Ricci scalar. 
	
	For conformal transformation $
		z\mapsto z + \epsilon(z),\ %
		\bar{z}\mapsto \bar{z}
			+ \bar{\epsilon}(\bar{z})
	$, we have:
	\begin{equation}
		\delta_\epsilon j(0)
		= -\operatorname*{Res}\limits_{z\to 0}
				\epsilon(z)\,
				T(z)\,j(0)
			-\operatorname*{Res}\limits_{\bar{z}\to 0}
				\bar{\epsilon}(\bar{z})\,
				\tilde{T}(\bar{z})\,j(0)
	\end{equation}
	Hence the $z^{-3},\bar{z}^{-3}$ coefficients of the OPE reflect the $
		\epsilon = z^2,\ %
		\bar{\epsilon} = \bar{z}^2
	$ transformation of $j$.  By comparing the Weyl transformations\footnote{
		Note that $
			(\textit{Conformal}\,)
			= (\textit{Weyl}\,)
				+ (\textit{Translation}\,)
		$. 
	}, this yields a total coefficient of $4Q$. 
	
	\item For $bc$ CFT with $j = \normorder{cb}\,$, the anomaly can be explicitly calculated using our results in (a), i.e.\ by calculating $Tj$ OPE. Following the standard procedure\footnote{
		For more detailed discussions, see Blumenhagen et al, \textit{Basic Concepts of String Theory}. 
	}, we obtain that:
	\begin{equation}
		T_z j_0 \sim \frac{1 - 2\lambda}{z^3}
			+ \order{\frac{1}{z^2}}
	\end{equation}
	Note that the anti-holomorphic part is zero, therefore, we have: $
		Q = \frac{1}{4}\,(1 - 2\lambda)
	$. \qedfull
	
	\end{enumerate}
	
	\end{enumerate}


\printbibliography[%
%	title = {参考文献} %
	,heading = bibintoc
]
\end{document}
