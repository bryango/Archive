% !TeX encoding = UTF-8
% !TeX spellcheck = en_US
% !TeX TXS-program:bibliography = biber -l zh__pinyin --output-safechars %
% !TeX TS-program = lualatex
%%% LuaLaTeX is required for `tikz-feynman`

\documentclass[a4paper,10pt]{article}

\newcommand{\hwNumber}{3}

% Templates: 82ccb576e4df24e5eac4194b76230be360b4f733

% to be `\input` in subfolders,
% ... therefore the path should be relative to subfolders.

\usepackage[UTF8
	,heading=false
	,scheme=plain % English Document
]{ctex}

\input{../.modules/basics/macros.tex}
\input{../.modules/preamble_base.tex}
\input{../.modules/preamble_notes.tex}

\newcommand{\legacyReference}{{
	\clearpage\par
	\quad\clearpage
	\renewcommand{\midquote}{\textbf{PAST WORK, AS TEMPLATE}}
	\newparagraph
}}

% Settings
\counterwithout{equation}{section}
\mathtoolsset{showonlyrefs=false}
%\DeclareTextFontCommand{\textbf}{\sffamily}
\renewcommand{\midquote}{\quad}

% Spacing
\geometry{footnotesep=2\baselineskip} % pre footnote split
\setlength{\parskip}{.5\baselineskip}
\renewcommand{\baselinestretch}{1.15}

%Title
	\posttitle{
		\hfill\Large\ccbyncsajp
		\par\end{flushleft}%
		\vspace*{-.7ex}\hrule%
	}
	\preauthor{\vspace{-1.5ex}%
		\flushleft\itshape%
	}
	\postauthor{\hfill}
	\predate{\noindent\ttfamily Compiled @ }
	\postdate{\vspace{.5ex}}

	\title{String Theory \textnumero\hwNumber}
	\author{\signature Bryan}
	\date{\today}

% List
	\setlist*{
		listparindent=\parindent
		,labelindent=\parindent
		,parsep=\parskip
		,itemsep=1.2\parskip
		,leftmargin=0pt
		,itemindent=*
	}
	\setlist*[enumerate,1]{
		align=left
		,label=\fbox{\textbf{\arabic*}}
		,itemsep=.5\baselineskip
		,itemindent=*
	}

\input{../.modules/basics/biblatex.tex}

%%% ID: sensitive, do NOT publish!
%\InputIfFileExists{../id.tex}{}{}

\usepackage{tikz-feynman,cancel}

\begin{document}
\maketitle
\pagestyle{headings}
\pagenumbering{arabic}
\thispagestyle{empty}

	\begin{enumerate}
	
	\item \textbf{Strings on Curved Space:}
	\begin{gather}
		S = \frac{1}{4\pi\alpha'}
			\int_M \dd[2]{\sigma} \sqrt{g}\,
			\pqty\Big{
				i\epsilon^{ab}
				B_{\mu\nu}(X)\,
				\pdd{a} X^\mu
				\pdd{b} X^\nu
				+ \cdots
			},\\
		T\id{^a_a} = -\frac{1}{2\alpha'}\,
			\beta^G_{\mu\nu}\, g^{ab}
				\pdd{a} X^\mu
				\pdd{b} X^\nu
			+ \cdots,\\
		\beta^G_{\mu\nu}
		= \alpha'R_{\mu\nu}
			- \frac{1}{4}\,\alpha'
				H_{\mu\lambda\omega}
				H\id{_\nu^{\lambda\omega}}
			+ \cdots
			+ \order{\alpha'^2}
	\end{gather}
	We want to verify the coefficient of $\alpha' H^2$ term in $\beta^G_{\mu\nu}$; for convenience we've omitted non-related terms in the above expressions. 
	
	Note that at $\order{\alpha'}$ such term does not depend on the metric $G_{\mu\nu}$, and it depends only on the field strength $H = \dd{B}$, not the potential $B$, hence it's safe to assume:
	\begin{gather}
		G_{\mu\nu} = \eta_{\mu\nu},\quad
		B_{\mu\nu}
		= \frac{1}{3} H_{\mu\nu\rho} X^\rho,\quad
		H = \mrm{const}, \\[0ex]
		i\epsilon^{ab}
			B_{\mu\nu}(X)\,
			\pdd{a} X^\mu
			\pdd{b} X^\nu
		= \frac{i}{3} H_{\mu\nu\rho}\,
			X^\rho
			\epsilon^{ab}
			\pdd{a} X^\mu
			\pdd{b} X^\nu, 
	\end{gather}
	We consider small perturbation away from the classical saddle: $
		X = X_0 + \xi
	$, then the 1-loop effective action is obtained by integrating over $\order{\xi^2}$ terms in the perturbed action\footnote{
		Reference: Prof.~Xi Yin's String Notes, see also \arxiv{0812.4408}. 
	}:
	\begin{gather}
		\Gamma^{(1)}[X_0]
		= - \ln \int \DD{\xi}\,
			e^{-S^{(2)}[X_0,\xi]},\\
	\begin{aligned}
		\mcal{L}^{(2)}
		&= \frac{i}{3}
		H_{\mu\nu\rho}\,\epsilon^{ab}
			\pqty\Big{
				\xi^\rho\,
					\pdd{a} X_0^\mu\,
					\pdd{b} \xi^\nu
				+ \xi^\rho\,
					\pdd{a} \xi^\mu\,
					\pdd{b} X_0^\nu
				+ X_0^\rho\,
					\pdd{a} \xi^\mu\,
					\pdd{b} \xi^\nu
			} \\
		&\sim \frac{i}{3}
		H_{\mu\nu\rho}\,\epsilon^{ab}
			\pqty\Big{
				\xi^\rho\,
					\pdd{a} X_0^\mu\,
					\pdd{b} \xi^\nu
				- \xi^\rho\,
					\pdd{a} X_0^\nu\,
					\pdd{b} \xi^\mu
				- \xi^\mu\,
					\pdd{a} X_0^\rho\,
					\pdd{b} \xi^\nu
			} \\
		&= \frac{i}{3}
		H_{\mu\nu\rho}\,\epsilon^{ab}
			\cdot 3\xi^\rho\,
				\pdd{a} X_0^\mu\,
				\pdd{b} \xi^\nu \\
		&= iH_{\mu\nu\rho}\,
			\epsilon^{ab}\,
				\pdd{a} X_0^\mu\,\pqty{
					\xi^\rho
					\pdd{b} \xi^\nu
				}
	\end{aligned}
	\end{gather}
	Here we've used the anti-symmetric properties of $H_{\mu\nu\rho},\epsilon^{ab}$, and ignored any total derivative after integration by parts. This term introduces a cubic interaction vertex in the free background; therefore, $\Gamma^{(1)}$ can be expressed in the following diagram\footnote{
		References: 
		\begin{itemize}[
			labelindent=3em,labelsep=1pt
		]
		\item David Tong, \href{https://www.damtp.cam.ac.uk/user/tong/string.html}{\textit{String Theory}};
		\item Callan \& Thorlacius, \href{https://www.damtp.cam.ac.uk/user/tong/string/sigma.pdf}{\textit{Sigma Models and String Theory}};
		\item Timo Weigand, \href{https://www.thphys.uni-heidelberg.de/\~{}weigand/Strings15-16/Strings.pdf}{\textit{Introduction to String Theory}}. 
		\end{itemize}
	}:
	\begin{gather}
	\feynmandiagram[
		layered layout
		,horizontal=b to c
		,every vertex={dot}
	]{
		a [particle=$\pdd{a} X_0^\mu$]
			-- [photon]
		b
			-- [half left
				,edge label=$\xi$
				,looseness=1.5]
		c
			-- [half left
				,looseness=1.5]
		b,
		c -- [photon]
		d [particle=$\pdd{b} X_0^{\nu}$], 
	}; \notag\\
		\allowdisplaybreaks
		\sim \frac{1}{2!}
			\pqty{\frac{1}{\alpha'}}^2
			\int \dd[2]{p} \pqty\Big{
				iH_{\mu\nu\rho}\,
				\epsilon^{ab}\,
				\pdd{a} X_0^\mu\,
				ip_b
			}
			\frac{2}{p^4}
			\pqty{-\frac{\alpha'}{2}}^2
			\pqty\Big{
				iH\id{_{\mu'}^{\nu\rho}}\,
				\epsilon^{a'b'}\,
				\pdd{a'} X_0^{\mu'}\,
				ip_{b'}
			} \\
		= \frac{2}{2!}
			\pqty{\frac{1}{\alpha'}}^2
			\pqty{-\frac{\alpha'}{2}}^2
			H_{\mu\lambda\omega}
			H\id{_{\nu}^{\lambda\omega}}
				\pdd{a} X_0^\mu\,
				\pdd{b} X_0^\nu\,
			\int \dd[2]{p}
			\frac{p^2 g^{ab} - p^a p^b}{p^4} \\
		= \frac{2}{2!}
			\pqty{-\frac{1}{2}}^2
			H_{\mu\lambda\omega}
			H\id{_{\nu}^{\lambda\omega}}
				\pdd{a} X_0^\mu\,
				\pdd{b} X_0^\nu\,
			\pqty{\frac{1}{2}\,g^{ab}}\!
			\int \dd[2]{p} \frac{1}{p^2} \\
		= \frac{2}{2!}
			\pqty{-\frac{1}{2}}^2
			\pqty{\frac{1}{2}}
			H_{\mu\lambda\omega}
			H\id{_{\nu}^{\lambda\omega}}
				\pdd{a} X_0^\mu\,
				\pdd{b} X_0^\nu\,
			g^{ab}\!
			\int \dd[2]{p} \frac{1}{p^2} \\
		= \frac{1}{8}
			H_{\mu\lambda\omega}
			H\id{_{\nu}^{\lambda\omega}}
				g^{ab}
				\pdd{a} X_0^\mu\,
				\pdd{b} X_0^\nu\,
			\int \dd[2]{p} \frac{1}{p^2}
	\end{gather}
	Here the $
		\pqty{\frac{1}{\alpha'}}^2
	$ coefficient comes from the vertices, while $
		\pqty\big{-\frac{\alpha'}{2}}^2
	$ comes from the propagators. 
	The $p^a p^b$ integral provides an additional $
		(\frac{1}{2})
	$ factor. 
	The overall normalization is chosen to match the $\alpha'R_{\mu\nu}$ coefficient in $\beta^G_{\mu\nu} \subset T\id{^a_a}$, which is $
		\frac{1}{1!}
			\times (-\frac{1}{2})
			\times 1
		= -\frac{1}{2}
	$. Therefore, we have:
	\begin{gather}
		T\id{^a_a}
		\supset \frac{1}{8}
			H_{\mu\lambda\omega}
			H\id{_{\nu}^{\lambda\omega}}
				g^{ab}
				\pdd{a} X_0^\mu\,
				\pdd{b} X_0^\nu,\\
		\beta^G_{\mu\nu}
		\supset -\frac{1}{4}\,\alpha'
			H_{\mu\lambda\omega}
			H\id{_{\nu}^{\lambda\omega}}
	\end{gather}
	\qedfull
	
	\item \textbf{Classical Solutions of 11D SUGRA:}
	Following the convention of \textit{Polchinski}, we have bosonic action:
	\begin{equation}
		S = \frac{1}{2\kappa^2}
			\int \pqty{
				\dd[11]{x} \sqrt{-g}\,
					\mcal{R}
				- \frac{1}{2}\,
					G\wedge *\,G
				- \frac{1}{6}\,
					C\wedge G\wedge G
			},
	\end{equation}
	Here $G = \dd{C}$: a 4-form field. In components, the numerical coefficients would be $
		\frac{1}{2}
		\mapsto
		\frac{1}{2\times 4!} = \frac{1}{48}
	$, and $
		\frac{1}{6}
		\mapsto
		\frac{1}{6\times 3!\times 4!\times 4!}
		= \frac{1}{20736}
	$. 
	
	Variation of the action yields the EOMs of our theory\footnote{
		Reference: \arxiv{hep-th/9912164}. I would like to thank \textit{Lucy Smith} for many helpful discussions. 
	}; Note that:
	\begin{equation}
		\var{\sqrt{-g}}
		= \frac{1}{2}\sqrt{-g}\,
			g^{\mu\nu} \var{g_{\mu\nu}}
		= -\frac{1}{2}\sqrt{-g}\,
			g_{\mu\nu} \var{g^{\mu\nu}}
	\end{equation}
	$\fdv{S}{g^{\mu\nu}}$ is easier to compute in components; note that the $
		C\wedge G\wedge G
	$ term does not depend on $g^{\mu\nu}$, therefore it does not contribute to the EOM. 
	We have the usual Einstein's equations:
	\begin{gather}
		R_{\mu\nu}
			- \frac{1}{2} \mcal{R} g_{\mu\nu}
		= \kappa^2 T_{\mu\nu},\\
	\begin{aligned}
		T_{\mu\nu}
		&= \frac{1}{\kappa^2}\,\pqty{
			\frac{4}{48}
			G_{\mu\sigma_1\sigma_2\sigma_3}
			G\id{_\nu^{\sigma_1\sigma_2\sigma_3}}
			- \frac{1}{2}\,g_{\mu\nu}
				\cdot \frac{1}{48}\,
				G^{\sigma_1\sigma_2\sigma_3\sigma_4}
				G_{\sigma_1\sigma_2\sigma_3\sigma_4}
		} \\
		&= \frac{1}{12\kappa^2}\,\pqty{
			G_{\mu\sigma_1\sigma_2\sigma_3}
			G\id{_\nu^{\sigma_1\sigma_2\sigma_3}}
			- \frac{1}{8}\,g_{\mu\nu}\,
				G^{\sigma_1\sigma_2\sigma_3\sigma_4}
				G_{\sigma_1\sigma_2\sigma_3\sigma_4}
		}
	\end{aligned}
	\label{eq:bosonic_stress_tensor}
	\end{gather}
	On the other hand, $
		\fdv{S}{C}
	$ is best carried out using differential forms:
	\begin{gather}
	\begin{aligned}
		0 = \var_C{S}
		&= -\frac{1}{2\kappa^2}
			\int \pqty{
				\var{G} \wedge *\,G
				+ \frac{1}{6} \pqty\Big{
					\var{C} \wedge G\wedge G
					- 2C\wedge\var{G}\wedge G
				}
			} \\
		&= -\frac{1}{2\kappa^2}
			\int \pqty{
				\var{(\dd{C})} \wedge *\,G
				+ \frac{1}{6} \pqty\big{
					\var{C} \wedge G\wedge G
					+ 2\var{(\dd{C})}
						\wedge C\wedge G
				}
			} \\
		&= -\frac{1}{2\kappa^2}
			\int \pqty{
				- (-1)^3 \var{C} \wedge \dd{* G}
				+ \frac{1}{6} \pqty\big{
					\var{C} \wedge G\wedge G
					- 2\,(-1)^3
					\var{C}\wedge \dd{(C\wedge G)}
				}
			} \\
		&= -\frac{1}{2\kappa^2}
			\int \var{C}\wedge\pqty{
				\dd{* G}
				+ \frac{1}{6} \pqty\Big{
					G\wedge G
					+ 2\pqty\big{
						G\wedge G
						- C\wedge \cancel{\dd[2]{C}}
					}
				}
			} \\
		&= -\frac{1}{2\kappa^2}
			\int \var{C}\wedge\pqty{
				\dd{* G}
				+ \frac{1}{2}\,G\wedge G
			},
	\end{aligned}\\[1ex]
		\dd{* G}
			+ \frac{1}{2}\,G\wedge G
		= 0
	\end{gather}
	
	\begin{enumerate}
	\item We hope to find a spacetime solution which is \textit{maximally symmetric} in \textit{some} directions; assume that these directions form a $d$-dimensional sub-manifold $\mcal{M}_d$ with:
	\begin{equation}
	\begin{aligned}
		\text{Coordinates:} &&&
			x^{\mu'},\ %
			\mu' \in \Delta \subset \Bqty{
				0,1,\cdots, 11
			},\\
		\text{Induced metric:} &&&
			g' = g|_{\mcal{M}_d}
	\end{aligned}
	\end{equation}
	The entire spacetime is then a direct product: $
		\mcal{M}_d
		\times \widetilde{\mcal{M}}_{11-d}
	$. 
	For $\mcal{M}_d$ to be maximally symmetric, we expect that $
		\kappa^2 T_{\mu'\nu'}
		= -\Lambda g'_{\mu'\nu'}
	$, i.e.\ the $G$-field serves as a cosmological constant $\Lambda$. By staring at \eqref{eq:bosonic_stress_tensor} we find that this can be achieved with\footnote{
		This is in fact the famous \textit{Freund--Robin ansatz}; see \wikiref{https://en.wikipedia.org/wiki/Freund\%E2\%80\%93Rubin\_compactification}{Freund–Rubin compactification}, and also the original paper: Freund \& Robin, \href{https://inspirehep.net/literature/154579}{\textit{Dynamics of Dimensional Reduction}}, 1980. 
	}:
	\begin{gather}
		d = 4,\quad
		G_{\sigma_1\sigma_2\sigma_3\sigma_4}
		= \alpha \sqrt{\abs{g'}}\,
			\epsilon_{
				\sigma_1\sigma_2\sigma_3\sigma_4
			},\quad
		G^{\sigma_1\sigma_2\sigma_3\sigma_4}
		= \alpha\,
			\frac{\mop{sgn} g'}{\sqrt{\abs{g'}}}\,
			\epsilon^{
				\sigma_1\sigma_2\sigma_3\sigma_4
			},\quad
		\{\sigma_i\}\subset \Delta,
		\\
		G_{\cdots\,\sigma\,\cdots}
		= 0,\quad
		\sigma \not\in\Delta,
	\\[2ex]
		T_{\mu\nu}
		= (\mop{sgn} g')\,
		\frac{\alpha^2}{12\kappa^2}\,\pqty{
			3!\,g'_{\mu\nu}
			- \frac{4!}{8}\,g_{\mu\nu}
		}
		= (\mop{sgn} g')\,
		\frac{\alpha^2}{2\kappa^2}\,
			\pqty{
				g'_{\mu\nu}
				- \frac{1}{2}\,g_{\mu\nu}
			},
	\\[1.5ex]
		\Lambda g_{\mu\nu}
		= \mp(\mop{sgn} g')\,
			\frac{\alpha^2}{4\kappa^2}\,
			g_{\mu\nu},\quad
	\left\lbrace\vbox to 12.5pt {}
	\right.
	\begin{aligned}
		- &\colon\ %
			\mu=\mu',\nu=\nu' \in \Delta,
			\hspace{-1.5ex}
			&&&\sim \mcal{M}_4\\[-.35ex]
		+ &\colon\ %
			\mu,\nu \not\in \Delta,
			&&&\sim \widetilde{\mcal{M}}_7
	\end{aligned}
	\end{gather}
	Matter EOM is trivially satisfied due to anti-symmetricity. We see that the other component $\widetilde{\mcal{M}}_7$ is also maximally symmetric, but with an opposite sign in its cosmological constant. 
	
	The field equations in $\mcal{M}_4$ and $\widetilde{\mcal{M}}_7$ are both of the form $
		R_{\mu\nu} \propto g_{\mu\nu}
	$. For $\mop{sgn} g' = -1$ i.e.\ Lorentzian signature, the solution is flat, AdS or dS, depending on the sign of $\Lambda$; for $\mop{sgn} g' = -1$, the solution is flat, spherical or hyperbolic. Therefore, we have:
	\begin{equation}
	\begin{aligned}
		\mop{sgn} g' = -1,
		&&&
		\Lambda_{4,7}
		= \pm \frac{\alpha^2}{4\kappa^2},\quad
		\mcal{M}_4 = \mrm{AdS}_{3,1},\quad
		\widetilde{\mcal{M}}_7 = S^7 \\
		\mop{sgn} g' = +1,
		&&&
		\Lambda_{4,7}
		= \mp \frac{\alpha^2}{4\kappa^2},\quad
		\mcal{M}_4 = S^4,\quad
		\widetilde{\mcal{M}}_7 = \mrm{AdS}_{6,1}
	\end{aligned}
	\end{equation}
	
	\item Global supersymmetries of a theory with the above $
		\mrm{AdS}_{4/7}\times S^{4/7}
	$ background are given by the solutions of:
	\begin{gather}
		0 = \var_\eta\psi^\mu
		\equiv D^\mu \eta(x),\quad
		\eta\colon \text{spinor},\\[.5ex]
	\begin{aligned}
		D^\mu &= \nabla^\mu
			+ \frac{1}{288}\,
				G_{\nu\rho\sigma\lambda} \pqty{
					\Gamma^{\mu\nu\rho\sigma\lambda}
					- 8g^{\mu\nu}
						\Gamma^{\rho\sigma\lambda}
				} \\
		&= \nabla^\mu
			+ \frac{1}{288}\,
				G_{\nu'\rho'\sigma'\lambda'}
			\pqty\big{
				\Gamma^{\mu\nu'\rho'\sigma'\lambda'}
				- 8g^{\mu\nu'}
					\Gamma^{\rho'\sigma'\lambda'}
			} \\[.5ex]
		&= \nabla^\mu
			+ \alpha
			\left\lbrace\vbox to 24pt {}\right.
			\begin{aligned}
				&\frac{-8\times 3!}{288}\,
					(-\Gamma^\mu\gamma_5)
				= \frac{1}{6}\,
					\Gamma^\mu\gamma_5,\quad
				\mu=\mu' \in \Delta,
					\hspace{-.5ex}
				&& \sim \mcal{M}_4
				\\[.5ex]
				& \frac{4!}{288}\,
					(-\Gamma^\mu)
				= -\frac{1}{12}\,
					\Gamma^\mu,\quad
				\mu \not\in \Delta,
				&& \sim \widetilde{\mcal{M}}_7
			\end{aligned}
	\end{aligned}
	\end{gather}
	Note that we've replaced the $G$ indices with $\mcal{M}_4$ indices, since $G$ vanish in $\widetilde{\mcal{M}}_7$ directions; due to anti-symmetricity, the $G$-term can be reduced to simple $\Gamma^\mu$ multiplications according to the $\mu$-direction\footnote{
		Reference for $\Gamma$-matrices and spinors: \textit{Polchinski} Vol.~\Romannum{2}, Appendix B. I'm a bit confused about all the complicated conventions, therefore the coefficients might be off by some factors...
	}. Furthermore, the spin connection in $\nabla^\mu$ is also block diagonalized, same as $g_{\mu\nu}$; hence there is a natural separation of variable\footnote{
		See \arxiv{hep-th/9912164} for more detailed discussions. 
	}:
	\begin{gather}
		\eta = \eta'(x')\,
			\eta'' (x''),\quad
		D_{\mu'} \eta' = 0,\quad
		D_{\mu''} \eta'' = 0,\\[.5ex]
		\mu',\eta',x'
			\sim \mcal{M}_4,\quad
		\mu'',\eta'',x''
			\sim \widetilde{\mcal{M}}_7,\quad
	\end{gather}
	
	Due to the presence of an additional $\Gamma$, $
		D_{\mu'} \eta' = 0
	$ has only 4 linearly independent solutions labeled by $\mu'$, while $
		D_{\mu''} \eta'' = 0
	$ is $\mop{Spin}(8)$ (or $\mrm{Spin}(7,1)$, depending on the signature) invariant, and has $
		\frac{8\times 7}{2} = 28
	$ linearly independent solutions\footnote{
		Reference: Achilleas Passias, \textit{Aspects of Supergravity in Eleven
		Dimensions}. 
	}. Hence the total number of SUSYs is $4 + 28 = 32$, for $
		\mrm{AdS}_{4/7}\times S^{4/7}
	$ background.
	\end{enumerate}
	
	\item \textbf{SUSY Sigma Models via Superspace:}
	\begin{gather}
	\begin{aligned}
		D_{\bar{\theta}} \vb{X}^\nu
		&= \pqty{
				\pdd{\bar{\theta}}
				+ \bar{\theta} \pdd{\bar{z}}
			}
			\pqty\big{
				X^\nu
				+ i\theta\psi^\nu
				+ i\bar{\theta}\tilde{\psi}^\nu
				+ \theta\bar{\theta} F^\nu
			} \\
		&= i\tilde{\psi}^\nu
			- \theta F^\nu
			+ \bar{\theta}\,
				\pdbar X^\nu
			- i\theta\bar{\theta}\,
				\pdbar \psi^\nu, \\[.5ex]
		D_\theta \vb{X}^\mu
		&= i\psi^\mu
			+ \bar{\theta} F^\mu
			+ \theta\,
				\pd X^\mu
			+ i\theta\bar{\theta}\,
				\pd \tilde{\psi}^\mu,
	\end{aligned}
	\\[1ex]
	\begin{aligned}
		D_{\bar{\theta}} \vb{X}^\nu
		D_\theta \vb{X}^\mu
		&= \pqty{
				i\tilde{\psi}^\nu
				- \theta F^\nu
				+ \bar{\theta}\,
					\pdbar X^\nu
				- i\theta\bar{\theta}\,
					\pdbar \psi^\nu
			}
			\pqty{
				i\psi^\mu
				+ \bar{\theta} F^\mu
				+ \theta\,
					\pd X^\mu
				+ i\theta\bar{\theta}\,
					\pd \tilde{\psi}^\mu
			} \\
		&= - \tilde{\psi}^\nu \psi^\mu
			- i\theta \pqty{
				\tilde{\psi}^\nu \pd X^\mu
				+ \psi^\mu F^\nu
			}
			+ i\bar{\theta} \pqty{
				\psi^\mu \pdbar X^\nu
				- \tilde{\psi}^\nu F^\mu
			} \\
			&\qquad - \theta\bar{\theta} \pqty{
				\pdbar X^\nu\pd X^\mu
				+ \tilde{\psi}^\nu
					\pd \tilde{\psi}^\mu
				- (\pdbar \psi^\nu)
					\psi^\mu
				+ F^\nu F^\mu
			},
	\end{aligned}
	\\[1ex]
	\begin{aligned}
		G_{\mu\nu}(\vb{X})
		&= G_{\mu\nu}
			+ \pqty{
				i\theta\psi^\lambda
				+ i\bar{\theta}\tilde{\psi}^\lambda
				+ \theta\bar{\theta} F^\lambda
			}\,\pdd{\lambda} G_{\mu\nu}
			+ \frac{1}{2}\,\Bqty{
				i\theta\psi^\rho
					\pdd{\rho},\,
				i\bar{\theta}\tilde{\psi}^\sigma
					\pdd{\sigma}\!
			}\,G_{\mu\nu} \\
		&= G_{\mu\nu}
			+ \pqty{
				i\theta\psi^\lambda
				+ i\bar{\theta}\tilde{\psi}^\lambda
			}\,G_{\mu\nu,\lambda}
			+ \theta\bar{\theta} \pqty{
				F^\lambda G_{\mu\nu,\lambda}
				+ \psi^\rho \tilde{\psi}^\sigma
					G_{\mu\nu,\rho\sigma}
			}, \\
	\end{aligned}
	\end{gather}
	
	Note that $
		\int \dd[2]{\theta}
		= \pdd{\theta} \pdd{\bar{\theta}}
	$, hence we need only focus on the $\theta\bar{\theta}$ term in the Lagrangian:
	\begin{equation}
	\begin{aligned}
		4\pi S_G
		&= \int \dd[2]{z} \dd[2]{\theta}
			G_{\mu\nu}(\vb{X})\,
			D_{\bar{\theta}} \vb{X}^\mu
			D_\theta \vb{X}^\nu
		= \int \dd[2]{z} \dd[2]{\theta}
			(-\theta\bar{\theta}) \pqty\Big{
				G_{\mu\nu} \pqty{
					\pd X^\mu \pdbar X^\nu
					+ \cdots
				} + \cdots
			} \\
		&= \int \dd[2]{z}
		\bigg(
			G_{\mu\nu} \pqty{
				\pd X^\mu \pdbar X^\nu
				+ \tilde{\psi}^\nu
					\pd \tilde{\psi}^\mu
				- (\pdbar \psi^\nu)
					\psi^\mu
				+ F^\nu F^\mu
			}
			\\ & \hspace{5em}
			+ \tilde{\psi}^\nu \psi^\mu \pqty{
				F^\lambda G_{\mu\nu,\lambda}
				+ \psi^\rho \tilde{\psi}^\sigma
					G_{\mu\nu,\rho\sigma}
			}
			\\ & \hspace{5em}
			- G_{\mu\nu,\lambda} \pqty{
				\psi^\lambda \pqty\big{
					\psi^\mu \pdbar X^\nu
					- \tilde{\psi}^\nu F^\mu
				}
				+ \tilde{\psi}^\lambda \pqty\big{
					\tilde{\psi}^\nu \pd X^\mu
					+ \psi^\mu F^\nu
				}
			}
		\bigg)
	\end{aligned}
	\end{equation}
	Similar result holds for the $B$ contribution $S_B$. We see that there is no $\pd F$ term in the action, hence $F$ is not dynamical and can be integrated out; we have:
	\begin{gather}
		0 = \var_F S
		= \var_F \pqty{S_G + S_B},\\
	\begin{aligned}
		4\pi \var{S_G}
		&= \int \dd[2]{z} \pqty{
				2G_{\mu\nu} F^\mu \var{F^\nu}
				+ G_{\mu\nu,\lambda} \pqty\big{
					\tilde{\psi}^\nu
						\psi^\mu
						\var{F^\lambda}
					- \tilde{\psi}^\nu
						\psi^\lambda
						\var{F^\mu}
					- \tilde{\psi}^\lambda
						\psi^\mu
						\var{F^\nu}
				}
			} \\
		&= \int \dd[2]{z} \pqty{
				2F_\lambda
				+ \pqty\big{
					G_{\mu\nu,\lambda}
					- G_{\lambda\mu,\nu}
					- G_{\lambda\nu,\mu}
				}\, \tilde{\psi}^\nu
					\psi^\mu
			} \var{F^\lambda} \\
		&= \int \dd[2]{z} \pqty{
				2F_\lambda
				- 2\Gamma_{\lambda\mu\nu}
					\tilde{\psi}^\nu
					\psi^\mu
			} \var{F^\lambda}, \\
		4\pi \var{S_B}
		&= \int \dd[2]{z} \pqty{
				0 + \pqty\big{
					B_{\mu\nu,\lambda}
					+ B_{\lambda\mu,\nu}
					+ B_{\nu\lambda,\mu}
				}\, \tilde{\psi}^\nu
					\psi^\mu
			} \var{F^\lambda}
		= \int \dd[2]{z}
			H_{\lambda\mu\nu}
				\tilde{\psi}^\nu
				\psi^\mu
				\var{F^\lambda}, \\
	\end{aligned}
	\\
		F_\lambda
		= \pqty{
				\Gamma_{\lambda\mu\nu}
				- \frac{1}{2} H_{\lambda\mu\nu}
			} \tilde{\psi}^\nu
				\psi^\mu,\\
		F^\lambda
		= \pqty{
				\Gamma^\lambda_{\mu\nu}
				- \frac{1}{2}
					H^\lambda_{\mu\nu}
			} \tilde{\psi}^\nu
				\psi^\mu,
	\end{gather}
	Here we've used the (anti-)symmetry of $G_{\mu\nu}$ and $B_{\mu\nu}$, and we adopt the convention that the Levi-Civita connection $
		\Gamma^\lambda_{\mu\nu}
		= \Gamma\id{^\lambda_{\mu\nu}}
		= G^{\lambda\lambda'}
			\Gamma_{\lambda'\mu\nu}
	$; similar holds for $B_{\mu\nu}$ and $H^\lambda_{\mu\nu}$. 
	
	Substitute $F_\lambda$ into $S$, collect the $\psi^0, \psi^2, \tilde{\psi}^2$ and $\psi^2\tilde{\psi}^2$ terms respectively, and we have:
	\begin{equation}
	\begin{aligned}
		4\pi S
		&= \int \dd[2]{z}
		\bigg(
			\pqty{G_{\mu\nu} + B_{\mu\nu}}\,
				\pd X^\mu \pdbar X^\nu
			\\ & \hspace{5em}
			+ \pqty{G_{\mu\nu} + \cancel{B_{\mu\nu}}}
			\pqty{
				\tilde{\psi}^\mu
					\pd \tilde{\psi}^\nu
				- (\pdbar \psi^\mu)
					\psi^\nu
			}
			\\ & \hspace{5em}
			- \pqty{
				G_{\mu\nu,\lambda}
				+ B_{\mu\nu,\lambda}
			} \pqty{
				\psi^\lambda
					\psi^\mu \pdbar X^\nu
				+ \tilde{\psi}^\lambda
					\tilde{\psi}^\nu \pd X^\mu
			}
			\\ & \hspace{5em}
			+ G_{\mu\nu} F^\mu F^\nu
			- 2\pqty{
					\Gamma_{\lambda\mu\nu}
					- \frac{1}{2} H_{\lambda\mu\nu}
				} \tilde{\psi}^\nu \psi^\mu
				F^\lambda
			\\ & \hspace{5em}
			+ \pqty{
					G_{\mu\nu,\rho\sigma}
					+ B_{\mu\nu,\rho\sigma}
				}\,
				\tilde{\psi}^\nu \psi^\mu
				\psi^\rho \tilde{\psi}^\sigma
		\bigg) \\
		&= \int \dd[2]{z}
		\bigg(
			\pqty{G_{\mu\nu} + B_{\mu\nu}}\,
				\pd X^\mu \pdbar X^\nu
			\\ & \hspace{5em}
			+ G_{\mu\nu} \pqty{
				\tilde{\psi}^\mu
					\pd \tilde{\psi}^\nu
				+ \psi^\mu
					\pdbar \psi^\nu
			}
%			\\ & \hspace{5em}
			- \pqty{
				G_{\mu\nu,\lambda}
				+ B_{\mu\nu,\lambda}
			} \pqty{
				\psi^\lambda
					\psi^\mu \pdbar X^\nu
				+ \tilde{\psi}^\lambda
					\tilde{\psi}^\nu \pd X^\mu
			}
			\\ & \hspace{5em}
			- F_\lambda F^\lambda
			+ \pqty{
					G_{\mu\nu,\rho\sigma}
					+ B_{\mu\nu,\rho\sigma}
				}\,
				\psi^\mu
				\psi^\rho
				\tilde{\psi}^\nu
				\tilde{\psi}^\sigma
		\bigg) \\
	\end{aligned}
	\end{equation}
	Here we've performed some integration by parts to clean up the result. 
	Note that some terms involving $B_{\mu\nu}$ vanish conveniently (up to integration by parts) due to anti-symmetricity. 
	
	The $\psi^2,\tilde{\psi}^2$ terms in the integrand can be further simplified as follows:
	\begin{gather}
	\begin{aligned}
		\mcal{L}_{\psi^2}
		&= G_{\mu\nu}
			\psi^\mu
			\pdbar \psi^\nu
		- \pqty{
			G_{\mu\nu,\lambda}
			+ B_{\mu\nu,\lambda}
		} \,
			\psi^\lambda
			\psi^\mu \pdbar X^\nu \\
		&= G_{\mu\nu}
			\psi^\mu
			\pdbar \psi^\nu
		- \pqty{
			G_{\mu[\nu,\lambda]}
			+ B_{\mu[\nu,\lambda]}
		} \,
			\psi^\lambda
			\psi^\mu \pdbar X^\nu \\
		&= G_{\mu\nu}
			\psi^\mu
			\pdbar \psi^\nu
		- \pqty{
			- \Gamma_{\lambda\mu\nu}
			+ \frac{1}{2} H_{\lambda\mu\nu}
		} \,
			\psi^\lambda
			\psi^\mu \pdbar X^\nu \\
		&= G_{\mu\nu} \psi^\mu \pqty{
				\pdbar \psi^\nu
				+ \pqty{
					\Gamma^\nu_{\rho\sigma}
					- \frac{1}{2} H^\nu_{\rho\sigma}
				} \,
				\psi^\rho
				\pdbar X^\sigma\!
			} \\
		&= G_{\mu\nu} \psi^\mu \pqty{
				\pdbar \psi^\nu
				+ \pqty{
					\Gamma^\nu_{\rho\sigma}
					+ \frac{1}{2} H^\nu_{\rho\sigma}
				} \,
				\psi^\sigma
				\pdbar X^\rho\!
			}
		= G_{\mu\nu} \psi^\mu
			\bar{\mcal{D}} \psi^\nu,
	\\[1.5ex]
		\mcal{L}_{\tilde{\psi}^2}
		&= G_{\mu\nu}
			\tilde{\psi}^\mu
			\pd \tilde{\psi}^\nu
		- \pqty{
			G_{\mu\nu,\lambda}
			+ B_{\mu\nu,\lambda}
		} \,
			\tilde{\psi}^\lambda
			\tilde{\psi}^\nu \pd X^\mu \\
		&= G_{\mu\nu} \tilde{\psi}^\mu \pqty{
				\pd \tilde{\psi}^\nu
				+ \pqty{
					\Gamma^\nu_{\rho\sigma}
					- \frac{1}{2} H^\nu_{\rho\sigma}
				} \,
				\tilde{\psi}^\sigma
				\pd X^\rho\!
			}
		= G_{\mu\nu} \tilde{\psi}^\mu
			\mcal{D} \tilde{\psi}^\nu,
	\end{aligned}
	\end{gather}
	For the $\psi^2\tilde{\psi}^2$ term, recall that $
		R_{\mu\nu\rho\sigma}
		= e_\mu [\cdv{\rho\,},\cdv{\sigma\,}]\,
			e_\nu,
		\cdv{\sigma} e_\nu
		= e_{\lambda}
			\Gamma^\lambda_{\sigma\nu}
	$, and we have:
	\begin{gather}
	\begin{aligned}
		\mcal{L}_{\psi^2\tilde{\psi}^2}
		&= \psi^\mu
			\psi^\nu
			\tilde{\psi}^\rho
			\tilde{\psi}^\sigma
		\pqty{
			G_{\mu\rho,\nu\sigma}
			+ B_{\mu\rho,\nu\sigma}
			+ \pqty\Big{
					\Gamma_{\lambda\mu\rho}
					- \frac{1}{2} H_{\lambda\mu\rho}
				}
				\pqty\Big{
					\Gamma^\lambda_{\nu\sigma}
					- \frac{1}{2}
						H^\lambda_{\nu\sigma}
				}
		} \\
		&= \psi^\mu
			\psi^\nu
			\tilde{\psi}^\rho
			\tilde{\psi}^\sigma
		\pqty{
			G_{\mu\rho,\nu\sigma}
				+ \Gamma_{\lambda\mu\rho}
					\Gamma^\lambda_{\nu\sigma}
			+ B_{\mu\rho,\nu\sigma}
			- \frac{1}{2} \pqty\Big{
					\Gamma^\lambda_{\mu\rho}
					H_{\lambda\nu\sigma}
					+ \Gamma^\lambda_{\nu\sigma}
					H_{\lambda\mu\rho}
				}
			+ \frac{1}{4}
				H^\lambda_{\mu\rho}
				H_{\lambda\nu\sigma}
		} \\
		&= \mcal{L}_G + \mcal{L}_B
			+ \frac{1}{4}
				H^\lambda_{\mu\rho}
				H_{\lambda\nu\sigma}\,
				\psi^\mu
				\psi^\nu
				\tilde{\psi}^\rho
				\tilde{\psi}^\sigma,
	\end{aligned}
	\\[1.5ex]
	\begin{aligned}
		\mcal{L}_G
		&= \psi^\mu
			\psi^\nu
			\tilde{\psi}^\rho
			\tilde{\psi}^\sigma
		\pqty{
			G_{\mu\rho,\nu\sigma}
				+ \Gamma_{\lambda\mu\rho}
					\Gamma^\lambda_{\nu\sigma}
		} \\
		&= \psi^{[\mu}
			\psi^{\nu]}
			\tilde{\psi}^{[\rho}
			\tilde{\psi}^{\sigma]}
		\pqty{
			G_{\mu\rho,\nu\sigma}
				+ \Gamma_{\lambda\mu\rho}
					\Gamma^\lambda_{\nu\sigma}
		} \\
		&= \frac{1}{2}\,
			\psi^\mu
			\psi^\nu
			\tilde{\psi}^\rho
			\tilde{\psi}^\sigma
		\Bqty{
			\pqty{
				\frac{1}{2}\,\pqty{
					G_{\mu\rho,\nu\sigma}
					- G_{\mu\sigma,\nu\rho}
				}
				+ \Gamma_{\lambda\mu\rho}
					\Gamma^\lambda_{\nu\sigma}
			}
			- \pqty\Big{\cdots}_{
				\rho\leftrightarrow\sigma
			}
		} \\
		&= \frac{1}{2}
			R_{\mu\nu\rho\sigma}\,
			\psi^\mu
			\psi^\nu
			\tilde{\psi}^\rho
			\tilde{\psi}^\sigma,
	\\[1.5ex]
		\mcal{L}_B
		&= \frac{1}{2}
			\cdv{\rho} H_{\mu\nu\sigma}\,
			\psi^\mu
			\psi^\nu
			\tilde{\psi}^\rho
			\tilde{\psi}^\sigma,
	\end{aligned}
	\end{gather}
	Therefore, the total action is:
	\begin{gather}
	\begin{aligned}
		S
		&= \frac{1}{4\pi} \int \dd[2]{z}
		\bigg(
			\pqty{G_{\mu\nu} + B_{\mu\nu}}\,
				\pd X^\mu \pdbar X^\nu
			\\ & \hspace{7.5em}
			+ G_{\mu\nu} \pqty{
				\tilde{\psi}^\mu
					\mcal{D} \tilde{\psi}^\nu
				+ \psi^\mu
					\bar{\mcal{D}} \psi^\nu
			}
			\\ & \hspace{7.5em}
			+ \pqty{
					\frac{1}{2}
						R_{\mu\nu\rho\sigma}
					+ \frac{1}{2}
						\cdv{\rho} H_{\mu\nu\sigma}
					+ \frac{1}{4}
						H^\lambda_{\mu\rho}
						H_{\lambda\nu\sigma}\!
				}\,
				\psi^\mu
				\psi^\nu
				\tilde{\psi}^\rho
				\tilde{\psi}^\sigma
		\bigg) \\
	\end{aligned}
	\end{gather}
	
	\item \textbf{Mixed Anomaly Between Diffeomorphism and Axial $U(1)$ Symmetry:}
	\begin{enumerate}
	\item Calculations of such anomaly is (schematically) similar to the usual axial anomaly; instead of the $A_\mu$ legs, we now have two $h_{\mu\nu}$ legs in the triangular diagram. 
	
	Again we chose the Pauli--Villars regularization with a regulator field $\psi'$ of mass $M\to\infty$. The $\pd^\mu J^A_\mu$ insertion is then reduced to:
	\begin{equation}
		\pd^\mu J^A_\mu
		= \pdd{\mu} \pqty{
				i\bar{\psi}'
				\gamma^\mu \gamma^5 \psi'
			}
		= i\bar{\psi}' (2M\gamma^5) \psi'
	\end{equation}
	The fermion--fermion--graviton vertex is given by $
		h_{\mu\nu} T^{\mu\nu}
	$, and (up to integration by parts) we have:
	\begin{gather}
		T^{\mu\nu}
		= \frac{i}{2} \bar{\psi}
			\gamma^{(\mu}
			\overleftrightarrow{\pd}^{\nu)}
			\psi
		\sim \frac{i}{2} \bar{\psi}
			\gamma^{(\mu}
			\pqty\big{
				-2\pd^{\nu)}
			} \psi
		= -i \bar{\psi}
			\gamma^{(\mu} \pd^{\nu)}
			\psi,\\
		h_{\mu\nu} T^{\mu\nu}
		= \bar{\psi}\,\pqty{
				-ih_{\mu\nu}
				\gamma^{(\mu} \pd^{\nu)}
			} \psi,
	\end{gather}
	This is very similar to the $A_\mu$ coupling, except that there is an extra derivative $\pd^\nu$. Denote the polarization of graviton as $\varepsilon_{\mu\nu}$, then in momentum space the interaction vertex $
		\sim \epsilon_{\mu\nu} \gamma^\mu \pqty{
			k^\nu_1 + k^\nu_2
		}
	$, and we have:
	\begin{gather}
	\begin{aligned}
		\ave{\pd^\mu J_\mu^A}_h
		&\sim \frac{1}{2!}\times 2\!
		\int \frac{\dd[4]{k}}{(2\pi)^4}
			\Tr \pqty{
				2M\gamma_5\cdot
				\frac{
					\slashed{k} + M
				}{k^2 + M^2}\cdot
				\cancel{\varepsilon_1 (2k+p_1)}
				\cdot
				\frac{
					\slashed{k} + \slashed{p}_1 + M
				}{(k+p_1)^2 + M^2}\cdot
				\cancel{\varepsilon_2 (2k+2p_1+p_2)}
				\cdot
				\frac{
					\slashed{k}
					+ \slashed{p}_1
					+ \slashed{p}_2 + M
				}{(k+p_1+p_2)^2 + M^2}
			} \\
		&\sim \int \frac{\dd[4]{k}}{(2\pi)^4}\,
			2M^2
			(4\epsilon_{\mu\nu\rho\sigma})
			\,
			\varepsilon_1^{\mu\mu'}
				(2k+p_1)_{\mu'}\,
			p_1^\nu
			\,
			\varepsilon_2^{\rho\rho'}
				(2k+2p_1+p_2)_{\rho'}\,
			p_2^\sigma\,
			\pqty{
				\frac{1}{k^2 + M^2}
				\cdots
			} \\
		&\sim 8M^2 \epsilon_{\mu\nu\rho\sigma}\,
			p_1^\nu p_2^\sigma\,
			\varepsilon_1^{\mu\mu'}
			\varepsilon_2^{\rho\rho'}
		\int \frac{\dd[4]{k}}{(2\pi)^4}
			\frac{
				(2k+p_1)_{\mu'}
				(2k+2p_1+p_2)_{\rho'}
			}{
				(k^2 + M^2)
				\pqty\big{(k+p_1)^2 + M^2}
				\pqty\big{(k+p_1+p_2)^2 + M^2}
			} \\
		&\sim 8M^2 \epsilon_{\mu\nu\rho\sigma}\,
			p_1^\nu p_2^\sigma\,
			\varepsilon_1^{\mu\mu'}
			\varepsilon_2^{\rho\rho'}
		\int \frac{\dd[4]{k}}{(2\pi)^4}
			\frac{
				4k_{\mu'}k_{\rho'}
				+ p_{1,\mu'} p_{2,\rho'}
			}{(k^2 + M^2)^3}\\[-.8\baselineskip]
	\end{aligned}
	\end{gather}
	There are, in fact, 2 diagrams accounting for this amplitude with $
		1\leftrightarrow 2
	$ symmetry; here we simply take one contribution with an additional factor of 2, and imply $
		1\leftrightarrow 2
	$ symmetrization in the above expressions. 
%	The $\mquote{4}$ factor, on the other hand, comes from tracing over $\gamma$-matrices to produce $\epsilon_{\mu\nu\rho\sigma}$. 
	
	Note that due to the additional $k_{\mu'}k_{\rho'}$ the integral is no longer finite but logarithmic divergent: $
		\int^\Lambda \dd[4]{k} \frac{k^2}{k^6}
		\sim \ln \Lambda
	$. More specifically\footnote{
		References: 
		\begin{itemize}[
			labelindent=3em,labelsep=1pt
		]
		\item David Tong, \href{https://www.damtp.cam.ac.uk/user/tong/gaugetheory.html}{\textit{Gauge Theory}};
		\item A.~Zee, \textit{QFT in a Nutshell}z;
		\item \arxiv{0802.0634};
		\item \wikiref{https://en.wikipedia.org/wiki/Common\_integrals\_in\_quantum\_field\_theory}{Common integrals in quantum field theory}. 
		\end{itemize}
	}, we have:
	\begin{gather}
	\begin{aligned}
		\ave{\pd^\mu J_\mu^A}_h
		&\sim 8M^2 \epsilon_{\mu\nu\rho\sigma}\,
			p_1^\nu p_2^\sigma\,
			\varepsilon_1^{\mu\mu'}
			\varepsilon_2^{\rho\rho'}
			\frac{\mop{Vol} S^3}{(2\pi)^4}
			\int \pqty{
				\frac{
					4k_{\mu'}k_{\rho'}
					k^3 \dd{k}
				}{(k^2 + M^2)^3}
				+ p_{1,\mu'} p_{2,\rho'}
				\frac{
					k^3 \dd{k}
				}{(k^2 + M^2)^3}
			} \\
		&\sim 8M^2 \epsilon_{\mu\nu\rho\sigma}\,
			p_1^\nu p_2^\sigma\,
			\varepsilon_1^{\mu\mu'}
			\varepsilon_2^{\rho\rho'}
			\frac{2\pi^2}{(2\pi)^4}
			\int \pqty{
				\delta_{\mu'\rho'}
				\frac{
					k^5 \dd{k}
				}{(k^2 + M^2)^3}
				+ p_{1,\mu'} p_{2,\rho'}
				\frac{
					k^3 \dd{k}
				}{(k^2 + M^2)^3}
			} \\
		&\sim 8M^2 \epsilon_{\mu\nu\rho\sigma}\,
			p_1^\nu p_2^\sigma\,
			\varepsilon_1^{\mu\mu'}
			\varepsilon_2^{\rho\rho'}
			\frac{1}{8\pi^2}
			\pqty{
				\delta_{\mu'\rho'}
				\frac{1}{2}
					\ln \frac{\Lambda^2}{M^2}
				+ p_{1,\mu'} p_{2,\rho'}
				\frac{1}{4M^2}
			} \\
		&\sim \frac{1}{4\pi^2}\,
			\epsilon_{\mu\nu\rho\sigma}\,
			p_1^\nu p_2^\sigma\,
			\varepsilon_1^{\mu\mu'}
			\varepsilon_2^{\rho\rho'}
			\pqty{
				2\delta_{\mu'\rho'} M^2
					\ln \frac{\Lambda^2}{M^2}
				+ p_{1,\mu'} p_{2,\rho'}
			}
	\end{aligned}
	\end{gather}
	The second term is very much similar to the axial anomaly result, while the first term diverges.
%	even after Pauli--Villars regularization. 
	
	However, we believe that the divergent term must be canceled by other diagrams; otherwise, it will contribute a $
		p^\nu p^\sigma\,
		\delta_{\mu'\rho'}
			\varepsilon^{\mu\mu'}_1
			\varepsilon^{\rho\rho'}_2
		= p^\nu p^\sigma
		(\varepsilon_1)\id{^\mu_\alpha}
			(\varepsilon_2)^{\rho\alpha}
		\sim (\pd h)^2
	$ term in the final result, which is not diff-invariant. The second term, on the other hand, is diff-invariant:
	\begin{gather}
		R_{\mu\nu\alpha\beta}
		= p_\beta\,p_{[\nu}\,
				\varepsilon_{\mu]\alpha}
			- p_\alpha\,p_{[\nu}\,
				\varepsilon_{\mu]\beta},\\
	\begin{aligned}
		\ave{\pd^\mu J_\mu^A}_h
		&\sim \frac{1}{4\pi^2}\,
			\epsilon_{\mu\nu\rho\sigma}\,
			(\varepsilon^{\mu\mu'}
				p_{1,\mu'} p_1^\nu)
			(\varepsilon^{\rho\rho'}
				 p_{2,\rho'} p_2^\sigma) \\
		&\sim \frac{1}{4\pi^2}\,
			\epsilon_{\mu\nu\rho\sigma}
			\frac{1}{4!\times 2\times 2}\times
			\frac{1}{2}
				R_{\mu\nu\alpha\beta}
				R\id{_{\rho\sigma}^{\alpha\beta}} \\
		&\sim \frac{1}{768\pi^2}\,
			\epsilon_{\mu\nu\rho\sigma}
				R_{\mu\nu\alpha\beta}
				R\id{_{\rho\sigma}^{\alpha\beta}}
	\end{aligned}
	\end{gather}
	
	\item The next order contribution would come from the covariant derivative\footnote{
		Reference: Alvarez-Gaume \& Witten, \textit{Gravitational Anomalies}. 
	}:
	\begin{equation}
		\cdv{\mu} \psi
		= \pdd{\mu} \psi
			+ \frac{1}{2}\,
				\omega\id{_\mu^{ab}}
				\sigma_{ab}
				\psi
	\end{equation}
	Where $\omega\id{_\mu^{ab}}$ is the spin connections, and $
		\sigma_{ab}
		= \frac{1}{4} [\gamma_a,\gamma_b]
	$; when linearized this contributes to the following interaction vertex:
	\begin{gather}
		\mcal{L}'
		= -\frac{i}{4}\,
			h\id{_\lambda^\alpha}
			\pdd{\mu} h_{\nu\alpha}\,
			\bar{\psi}\,
				\Gamma^{\mu\lambda\nu}
			\psi,\quad
		\Gamma^{\mu\lambda\nu}
		= \gamma^{[\mu}
			\gamma^{\lambda}
			\gamma^{\nu]}, \\
		\text{Feynman rule:}\quad
			-\frac{i}{4}\,
				\Gamma^{\mu\lambda\nu}
				(p_1 - p_2)_\mu\,
				(\varepsilon_1)\id{_\lambda^\alpha}
				(\varepsilon_2)_{\nu\alpha},
	\end{gather}
	We see a $
		(\varepsilon_1)\id{_\lambda^\alpha}
		(\varepsilon_2)_{\nu\alpha}
	$ factor, much similar to the factor in the divergent term in (a). Note that this vertex already contains 3 $\gamma$-matrices; by joining it with the anomalous vertex $\pdd{\mu} j^\mu_A$, we obtain a simple 1-loop ``seagull'' diagram (with graviton wings)
%	, whose contribution is
	:
	\begin{gather}
	\feynmandiagram[
		layered layout
		,horizontal=a to b
		,every vertex={dot}
	]{
		a [blob
			,label={left:$\pdd{\mu} J^\mu_A$}]
			-- [half left
				,edge label=$k$
				,out=90,in=100
				,looseness=1.35]
		b
			-- [half left
				,edge label={$%
					\!\!\!\!\!\!\!\!%
					\!\!\!\!\!\!\!\!%
					k+p_1+p_2$}
				,inner sep=1.5ex
				,out=80,in=90
				,looseness=1.35]
		a,
		b -- [gluon]
			c [particle={$\varepsilon_1,p_1$}], 
		b -- [gluon]
			d [particle={$\varepsilon_2,p_2$}], 
	}; \notag\\
	\begin{aligned}
		\ave{\pd^\mu J_\mu^A}'_h
		&\sim 2\!
		\int \frac{\dd[4]{k}}{(2\pi)^4}
			\Tr \pqty{
				2M\gamma_5\cdot
				\frac{
					\slashed{k} + M
				}{k^2 + M^2}\cdot
				\pqty{-\frac{1}{4}}\,
				\cancel{
					\varepsilon_1
					\varepsilon_2
					(p_1 - p_2)}
				\cdot
				\frac{
					\slashed{k}
					+ \slashed{p}_1
					+ \slashed{p}_2 + M
				}{(k+p_1+p_2)^2 + M^2}
			} \\
		&\sim -\int \frac{\dd[4]{k}}{(2\pi)^4}\,
			M^2
			(4\epsilon_{\mu\nu\rho\sigma})\,
			\delta_{\mu'\rho'}
				\varepsilon^{\mu\mu'}_1
				\varepsilon^{\rho\rho'}_2
				(p_1 - p_2)^{\nu}\,
				(p_1 + p_2)^{\sigma}\,
			\pqty{
				\frac{1}{k^2 + M^2}
				\cdots
			} \\
		&\sim -4M^2 \epsilon_{\mu\nu\rho\sigma}\,
			(2p_1^\nu p_2^\sigma)\,
			\varepsilon_1^{\mu\mu'}
			\varepsilon_2^{\rho\rho'}
		\int \frac{\dd[4]{k}}{(2\pi)^4}
			\frac{
				\delta_{\mu'\rho'}
			}{(k^2 + M^2)^2} \\
		&\sim -8M^2 \epsilon_{\mu\nu\rho\sigma}\,
			p_1^\nu p_2^\sigma\,
			\varepsilon_1^{\mu\mu'}
			\varepsilon_2^{\rho\rho'}
			\frac{1}{8\pi^2}
			\pqty{
				\delta_{\mu'\rho'}
				\frac{1}{2}
					\ln \frac{\Lambda^2}{M^2}
			}
	\end{aligned}
	\end{gather}
	Compare with the result in (a), and we see that the divergences cancel each other out precisely. 
	
	\item For an anomalous vertex with hypercharge $Y$, there will be an additional $Y$ factor in the front of $\ave{\pdd{\mu} J^\mu_A}$; summing over a family of matter gives the total anomaly\footnote{
		Reference: \textit{Tong}, and \wikiref{arg1https://en.wikipedia.org/wiki/Anomaly\_(physics)\#Anomaly\_cancellation}{Anomaly (physics) \# Anomaly cancellation}. 
	}:
	\begin{equation}
		\ave{\pdd{\mu} J^\mu_A}
		\propto \sum \Tr T_a T_b Y
		\propto \delta_{ab} \sum Y
	\end{equation}
	When the summation goes over all states in a complete generation, we have $
		\sum Y = 0
	$, i.e.\ the anomaly cancels. 
	
	\end{enumerate}
	
	\end{enumerate}


\printbibliography[%
%	title = {参考文献} %
	,heading = bibintoc
]
\end{document}
