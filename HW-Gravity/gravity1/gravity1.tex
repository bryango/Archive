% !TeX encoding = UTF-8
% !TeX spellcheck = en_US
% !TeX TXS-program:bibliography = biber -l zh__pinyin --output-safechars %

\documentclass[a4paper,10pt]{article}

\newcommand{\hwNumber}{1}

% Templates: 82ccb576e4df24e5eac4194b76230be360b4f733

% to be `\input` in subfolders,
% ... therefore the path should be relative to subfolders.

\usepackage[UTF8
	,heading=false
	,scheme=plain % English Document
]{ctex}
\usepackage{indentfirst}

\input{../.modules/basics/macros.tex}
\input{../.modules/preamble_base.tex}
\input{../.modules/preamble_notes.tex}

\newcommand{\legacyReference}{{
%	\clearpage\par
%	\quad\clearpage
	\renewcommand{\midquote}{\textbf{PAST WORK, AS TEMPLATE}}
	\newparagraph
}}

% Settings
%\usepackage{tikz-cd}
%\counterwithout{equation}{section}
\mathtoolsset{showonlyrefs=false}
%\DeclareTextFontCommand{\textbf}{\sffamily}
\renewcommand{\midquote}{\quad}

% Spacing
\geometry{footnotesep=2\baselineskip} % pre footnote split
\setlength{\parskip}{.5\baselineskip}
\renewcommand{\baselinestretch}{1.15}

%Title
	\posttitle{
		\hfill\Large\ccbyncsajp
		\par\end{flushleft}%
		\vspace*{-.7ex}\hrule%
	}
	\preauthor{\vspace{-1.5ex}%
		\flushleft\itshape%
	}
	\postauthor{\hfill}
	\predate{\noindent\ttfamily Compiled @ }
	\postdate{\vspace{.5ex}}

	\title{Gravity \textnumero\hwNumber}
	\author{\signature Bryan}
	\date{\today}

% List
%	\setlist*{
%		listparindent=\parindent
%		,labelindent=\parindent
%		,parsep=\parskip
%		,itemsep=1.2\parskip
%	}

\input{../.modules/basics/biblatex.tex}

%%% ID: sensitive, do NOT publish!
\InputIfFileExists{../id.tex}{}{}

\begin{document}
\maketitle
\pagenumbering{arabic}
\thispagestyle{empty}

%\vspace*{-1.5\baselineskip}

%1  Solve the spacelike geodesic equation for AdSd+1in Poincar ́e coordinates.In particular, write down the leading order terms for two points near theboundaryz=with a coordinate separation in a spatial direction,  sayx2−x1=L.
%2  Consider the world-line action of a point particleS=12∫dτ(η−1 ̇Xμ ̇Xμ−ηm2).(1)How doesηtransform under the reparameterization of the parameterτ?Show that it is classically equivalent to the action ̃S=−m∫dτ(−gαβ ̇Xα ̇Xβ)1/2.(2)
%3  Work out the Ricci tensor for the metricds2=−f(r)dt2+h(r)dr2+r2(dθ2+ sin2θdφ2)(3)
%4  Derive the equations of motion for the Jackiw-Teitelboim gravity, a twodimensional theory of gravity coupled with dilatonS=116πG∫dx2√−gΦ(R+ 2)(4)where Φ is a dilaton field.1

\section{Spacelike Geodesic for Poincar\'e AdS}
	The Poincar\'e $\mrm{AdS}_{d+1}$ metric is given by:
	\begin{equation}
		\dd{s}^2
		= G_{IJ} \dd{X^I} \dd{X^J}
		= \frac{
				-\dd{t}^2 + \dd{\vec{x}}^2 + \dd{z}^2
			}{z^2}
%		= \frac{\dd{w}^2}{z^2}
		= \frac{
				\dd{x}^2 + \dd{z}^2
			}{z^2},
	\end{equation}
	\\[-1.8\baselineskip]
	\begin{equation}
		\dd{x}^2
		= -\dd{t}^2 + \dd{\vec{x}}^2
		= \eta_{\mu\nu} \dd{x}^\mu \dd{x}^\nu,
	\quad
		X^I \sim (t,\vec{x},z)
		\sim (x^\mu, z),
	\end{equation}
	Note that here we define $\dd{x}^2$ using the flat metric $\eta$, while $\dd{X}$ has an index that should be raised and lowered using the curved metric $G_{IJ} = \frac{1}{z^2}\,\eta_{IJ}$. Generally upper case tensors are handled with $G_{IJ}$, while lower case ones are handled with $\eta$. 
	The full isometry of such spacetime is then given by $\mrm{SO}(d,2)$ with $\frac{(d+2)(d+1)}{2}$ generators, including:
	\begin{enumerate}[itemsep=.2\baselineskip]
	\item Among the $x^\mu \sim (x^0,\vec{x}) \equiv (t,\vec{x})$ directions:
		\begin{enumerate}[noitemsep]
		\item $\frac{d(d-1)}{2}$ $\mrm{SO}(d-1,1)$ rotations $
			x_\mu \pdd{\nu} - x_\nu \pdd{\mu}
		$, boosts included;
		\item $d$ translations $\pdd{\mu}$;
		\end{enumerate}
	\item Dilation with $X^I \sim (x^\mu,z)\mapsto \lambda X^I = \lambda\,(x^\mu,z)$, generated by $\Delta = X^I \pdd{I} = z\pdd{z} + x^\mu \pdd{\mu}$;
	\item Special conformal transformations; these can be understood as translations conjugated by \textit{inversions}. Note that $\frac{\dd{z}^2}{z^2}$ is invariant under $z\mapsto \frac{1}{z}$; if we include the $x^\mu$ directions, we can consider:
	\begin{equation}
		\mcal{I}\colon\ \chi^I \mapsto
			\frac{\chi^I}{\chi^2},
%			= \frac{\chi^I}{x^2 + z^2}
	\quad
		\chi^2 = -t^2 + \vec{x}^2 + z^2,
	\end{equation}
	\\[-1.5\baselineskip]
	\begin{equation}
		\mcal{I}^2 = \idty,
	\quad
		\dd{s}^2
		\mapsto \pqty{
				\frac{
					\delta^I_J
					- 2\,\frac{\chi^I \chi_J}{\chi^2}
				}{\chi^2}
				\dd{\chi^J}
			}^{\!\!2}
			\bigg/ \pqty{
				\frac{z}{\chi^2}
			}^{\!\!2}
		= \frac{\dd{\chi}^2}{z^2}
		= \dd{s}^2
	\end{equation}
	\end{enumerate}
	We see that inversion $\mcal{I}$ is indeed a (discrete) symmetry of the metric. 
	Here we've defined yet another lower case variable $\chi^I \sim (x^\mu, z)$, which as a contravariant vector has the same components as $X^I$, but with an index that should be lowered by the flat metric $\eta_{IJ}$, i.e.~$\chi_I = \eta_{IJ} \chi^J = \eta_{IJ} X^J$. 
	The $d$ special conformal generators are then given by:
	\begin{equation}
	\begin{aligned}
		K_\mu
		&= \pdv{a^\mu} \pqty{
				\mcal{I}
				\circ e^{a^\nu P_\nu}
				\circ \mcal{I}
				\circ X^I
			}_{\!a = 0} \ \pdv{X^I} \\
		&= \pdv{a^\mu}\pqty{
				\frac{\frac{\chi^I}{\chi^2} + a^I}{
					\abs\big{\frac{\chi^J}{\chi^2} + a^J}^2
				}
			}_{\!\!a = 0} \,\pdv{X^I} \\
		&= \pdv{a^\mu}\pqty{
				\frac{\chi^I + a^I \chi^2}{
					1 + 2a^I \chi_I + a^2 \chi^2
				}
			}_{\!\!a = 0} \,\pdv{X^I} \\
		&= \chi^2 \pdd{\mu}
			- 2x_\mu X^I \pdd{I} \\
		&= \chi^2 \pdd{\mu}
			- 2\eta_{\mu\nu} x^\nu \Delta \\
		&= \pqty{
				\chi^2 \delta^I_\mu
				- 2\eta_{\mu\nu} x^\nu X^I
			} \pdd{I}
	\end{aligned}
	\end{equation}
	
	By Noether's theorem, $Q_\Xi = V_I \Xi^I = G_{IJ} \Xi^I V^J$ is conserved along the geodesic; here $V^I$ is the normalized tangent vector, while $\Xi^I$ is some Killing vector of the spacetime, e.g.~one from the list above. 
	
	We can then write down the conserved charges along a geodesic $\gamma$ in Poincar\'e AdS:
	\begin{subequations}
%	\allowdisplaybreaks
	\begin{align}
		p_\mu &= G_{\mu I} V^I = V_\mu \\
		m\id{^\mu_\nu} &= X^\mu V_\nu - X^\nu V_\mu \\
		\Delta &= X^I V_I
			= \frac{z V^z + x^\mu p_\mu}{z^2}
%	,\quad
%		V^I \sim (v^\mu, v^z)
		\\
		k_\mu &= (z^2 + x^2)\, V_\mu - 2\eta_{\mu\nu} x^\nu \Delta
	\end{align}
	\end{subequations}
	These are all integration constants along $\gamma$. 
	Again note that $X,V$ have indices that should be handled with $G_{IJ}$, in particular,
	\begin{equation}
		X_\mu = G_{\mu I} X^I = \frac{1}{z^2} x_\mu
	\end{equation}
	
	Equation (a) above implies that $X_\mu(\lambda) = X_\mu(0) + p_\mu \lambda$,
%	 linear in the affine parameter $\lambda$, 
	therefore:
	\begin{equation}
		x^\mu(\lambda) = X^\mu(\lambda)
		= X^\mu(0) + p^\mu \lambda z^2,
	\quad z = z(\lambda)
	\end{equation}
	Note the additional $z^2$ factor due to the metric $G$. Plug this into (c)(d)
%	and ensure that $u^\mu$ is properly normalized
	and we have:
	\begin{equation}
	\begin{gathered}
%		\delta = g_{IJ} \dd{X^I} \dd{X^J}
%		= \frac{p^2 + u_z^2}{z^2},
%	\quad
%		\delta = 0,\pm 1,
%	\quad
%		p^2 = p_\mu p^\mu = \eta_{\mu\nu} p^\mu p^\nu,
%	\\
		\Delta = \frac{1}{z} V^z
			+ \frac{1}{z^2}\, x^\mu(0)\, p_\mu
			+ p^2 \lambda,
	\\
		(z^2 + x^2)\,p^\mu
		= k^\mu + 2x^\mu \Delta,
%	\\
%		z^2 + x^2 - 2\Delta\lambda
%		= \frac{p_\mu (k^\mu + 2x^\mu(0))}{p_\mu p^\mu},
%	\\
%		(z^2 + x^2)\,p^\mu
%		= k^\mu + 2x^\mu \Delta
%		= k^\mu + 2x^\mu(0) \Delta + 2\Delta p^\mu \lambda,
	\end{gathered}
	\end{equation}
	
%	For $\delta = 0$, we have $u_z^2 = -p^2$ constant, $z(z) = z(0) + u_z \lambda$ also linear in $\lambda$, i.e.~null geodesics are ``straight'' lines in the $X^I$ coordinates; this agrees with the conformal flatness of the metric. For $\delta \ne 0,\delta^2 = 1$, we have:
	For $p^\mu = 0$, we have $x^\mu = x^\mu(0)$ while (c) implies that $
		\Delta
		= \frac{1}{z}\dv{z}{\lambda}
	$, i.e.~$z(\lambda) = z(0) \,e^{\lambda\Delta}$, with $\norm{V}^2 = \frac{(V^z)^2}{z^2} = \Delta^2$. With the standard normalization $\norm{V}^2 = 1$, we get $\Delta = \pm 1$, $z(\lambda) = z(0) \,e^{\lambda\Delta}$; this is a spacelike ``straight'' line along the $z$ direction.
	
	For $p^\mu \ne 0$, we can always find some $c_\mu$ such that $c_\mu p^\mu \ne 0$, therefore (d) implies that:
	\begin{equation}
		z^2 + x^2 - 2\lambda z^2\Delta
		= \frac{c_\mu \pqty{k^\mu + 2x^\mu(0)\,\Delta}}{c_\mu p^\mu}
	\end{equation}
	The right hand side is constant; note that $x^\mu$ is also linear in $\lambda z^2$, therefore we can complete the square on the left hand side such that:
	\begin{equation}
		z^2 + \pqty{x - a}^2
		= \pqty{\frac{L}{2}}^{\!\!2}
		= \mrm{const.},
	\quad
		\Delta
		= p_\mu a^\mu
	\end{equation}
	
	We see that $\gamma$ lands on a ``sphere'' centered at $z = 0, x = a$ in $\mbb{R}^{d,1}$; with the Lorentzian metric, it is actually a hyperboloid. Note that for now the radius $\pqty{\frac{L}{2}}^{\!2}$ can actually be negative or zero; more specifically,
	\begin{enumerate}[noitemsep]
	\item $L^2 > 0$: one-sheet hyperboloid;
	\item $L^2 < 0$: two-sheet hyperboloid;
	\item $L^2 = 0$: conic surface. 
	\end{enumerate}
	
	
	

\printbibliography[%
%	title = {参考文献} %
	,heading = bibintoc
]
\end{document}
