% !TeX document-id = {09950f11-703a-4175-9f6c-5ae00c774d6a}
% !TeX encoding = UTF-8
% !TeX spellcheck = en_US
% !TeX TXS-program:bibliography = biber -l zh__pinyin --output-safechars %

\documentclass[a4paper,10pt]{article}

\newcommand{\hwNumber}{1}

% Templates: 82ccb576e4df24e5eac4194b76230be360b4f733

% to be `\input` in subfolders,
% ... therefore the path should be relative to subfolders.

\usepackage[UTF8
	,heading=false
	,scheme=plain % English Document
]{ctex}
\usepackage{indentfirst}

\input{../.modules/basics/macros.tex}
\input{../.modules/preamble_base.tex}
\input{../.modules/preamble_notes.tex}

\newcommand{\legacyReference}{{
%	\clearpage\par
%	\quad\clearpage
	\renewcommand{\midquote}{\textbf{PAST WORK, AS TEMPLATE}}
	\newparagraph
}}

% Settings
%\usepackage{tikz-cd}
%\counterwithout{equation}{section}
\mathtoolsset{showonlyrefs=false}
%\DeclareTextFontCommand{\textbf}{\sffamily}
\renewcommand{\midquote}{\quad}

% Spacing
\geometry{footnotesep=2\baselineskip} % pre footnote split
\setlength{\parskip}{.5\baselineskip}
\renewcommand{\baselinestretch}{1.15}

%Title
	\posttitle{
		\hfill\Large\ccbyncsajp
		\par\end{flushleft}%
		\vspace*{-.7ex}\hrule%
	}
	\preauthor{\vspace{-1.5ex}%
		\flushleft\itshape%
	}
	\postauthor{\hfill}
	\predate{\noindent\ttfamily Compiled @ }
	\postdate{\vspace{.5ex}}

	\title{Gravity \textnumero\hwNumber}
	\author{\signature Bryan}
	\date{\today}

% List
%	\setlist*{
%		listparindent=\parindent
%		,labelindent=\parindent
%		,parsep=\parskip
%		,itemsep=1.2\parskip
%	}

\input{../.modules/basics/biblatex.tex}

%%% ID: sensitive, do NOT publish!
\InputIfFileExists{../id.tex}{}{}

\begin{document}
\maketitle
\pagenumbering{arabic}
\thispagestyle{empty}

%\vspace*{-1.5\baselineskip}

%1  Solve the spacelike geodesic equation for AdSd+1in Poincar ́e coordinates.In particular, write down the leading order terms for two points near theboundaryz=with a coordinate separation in a spatial direction,  sayx2−x1=L.
%2  Consider the world-line action of a point particleS=12∫dτ(η−1 ̇Xμ ̇Xμ−ηm2).(1)How doesηtransform under the reparameterization of the parameterτ?Show that it is classically equivalent to the action ̃S=−m∫dτ(−gαβ ̇Xα ̇Xβ)1/2.(2)
%3  Work out the Ricci tensor for the metricds2=−f(r)dt2+h(r)dr2+r2(dθ2+ sin2θdφ2)(3)
%4  Derive the equations of motion for the Jackiw-Teitelboim gravity, a twodimensional theory of gravity coupled with dilatonS=116πG∫dx2√−gΦ(R+ 2)(4)where Φ is a dilaton field.1

\section{Spacelike Geodesic for Poincar\'e AdS}
	The Poincar\'e $\mrm{AdS}_{d+1}$ metric is given by:
	\begin{equation}
		\dd{s}^2
		= G_{IJ} \dd{X^I} \dd{X^J}
		= \frac{
				-\dd{t}^2 + \dd{\vec{x}}^2 + \dd{z}^2
			}{z^2}
		= \frac{
				\dd{x}^2 + \dd{z}^2
			}{z^2},
	\end{equation}
	\\[-1.8\baselineskip]
	\begin{equation}
		\dd{x}^2
		= -\dd{t}^2 + \dd{\vec{x}}^2
		= \eta_{\mu\nu} \dd{x}^\mu \dd{x}^\nu,
	\quad
		X^I \sim (t,\vec{x},z)
		\sim (x^\mu, z),
	\end{equation}
	Note that here we define $\dd{x}^2$ using the flat metric $\eta$, while $\dd{X}$ has an index that should be raised and lowered using the curved metric $G_{IJ} = \frac{1}{z^2}\,\eta_{IJ}$. Generally upper case tensors are handled with $G_{IJ}$, while lower case ones are handled with $\eta$. 
	The full isometry of such spacetime is then given by $\mrm{SO}(d,2)$ with $\frac{(d+2)(d+1)}{2}$ generators, including:
	\begin{enumerate}[noitemsep]
	\item Among the $x^\mu \sim (x^0,\vec{x}) \equiv (t,\vec{x})$ directions:
		\begin{enumerate}[noitemsep,topsep=.5\baselineskip]
		\item $\frac{d(d-1)}{2}$ $\mrm{SO}(d-1,1)$ rotations $
			x_\mu \pdd{\nu} - x_\nu \pdd{\mu}
		$, boosts included;
		\item $d$ translations $\pdd{\mu}$;
		\end{enumerate}
	\item Dilation with $X^I \sim (x^\mu,z)\mapsto \lambda X^I = \lambda\,(x^\mu,z)$, generated by $\Delta = X^I \pdd{I} = z\pdd{z} + x^\mu \pdd{\mu}$;
	\item Special conformal transformations; see \autoref{sect:special_conformal} for an intuitive derivation. We have: 
	\begin{equation}
		k_\mu
		= (z^2 + x^2)\,\pdd{\mu}
			- 2x_\mu \Delta
	\end{equation}
	\end{enumerate}
	
	By Noether's theorem, $Q_\Xi = V_I \Xi^I = G_{IJ} \Xi^I V^J$ is conserved along the geodesic; here $V^I$ is the normalized tangent vector, while $\Xi^I$ is some Killing vector of the spacetime, e.g.~one from the list above. 
	We can then write down the conserved charges along a geodesic $\gamma$ in Poincar\'e AdS:
	\begin{equation}
%	\allowdisplaybreaks
%	\renewcommand{\theequation}{\theparentequation.\alph{equation}}
	\begin{aligned}
		p_\mu
			&= G_{\mu I} V^I = V_\mu \\
		m_{\mu\nu}
			&= x_\mu V_\nu - x_\nu V_\mu \\
		\Delta
			&= X^I V_I
			= \frac{z V^z}{z^2} + x^\mu p_\mu
%	,\quad
%		V^I \sim (v^\mu, v^z)
		\\
		k_\mu &= (z^2 + x^2)\, p_\mu - 2x_\mu \Delta
	\end{aligned}
	\end{equation}
	These are all integration constants along $\gamma$. 
	Again note that $X,V$ have indices that should be handled with $G_{IJ}$, in particular,
	\begin{equation}
		\dd{X}_\mu
		= G_{\mu I} \dd{X}^I
		= \frac{1}{z^2} \dd{x}_\mu,
	\quad
		V^\mu = \dv{X^\mu}{\lambda} = z^2 p^\mu
	\label{eq:p_conserved}
	\end{equation}
	
	On the other hand, $V^\mu$ should be properly normalized, therefore:
	\begin{equation}
	\begin{gathered}
		\norm{V}^2
		= g_{IJ} V^I V^J
		= \frac{(V^z)^2}{z^2} + z^2 p^2
		= \delta = 0,\pm 1,
%	\quad
%		p^2 = p_\mu p^\mu = \eta_{\mu\nu} p^\mu p^\nu,
	\label{eq:tangent_norm}
	\end{gathered}
	\end{equation}
	For $p^2 = 0$, we have $
		\norm{V}^2
		= \frac{(V^z)^2}{z^2}
		\ge 0
	$, thus $\gamma$ can be either spacelike or null, but not timelike. In fact, $
		\frac{1}{z}\dv{z}{\lambda} = 0,\pm 1
	$, along with $\dv{x^\mu}{\lambda} = z^2 p^\mu$, we obtain:
	\begin{equation}
	p^2 = 0,\qquad
	\begin{aligned}
		\text{spacelike}\colon\ &
			z(\lambda)
			= z(0)\,e^{\pm\lambda},
		&&
			x^\mu(\lambda)
			= x^\mu(0) \pm z(0)^2\,p^\mu\,
				\tfrac{
					e^{\pm 2\lambda} - 1
				}{2}, \\
		\text{null}\colon\ &
			z(\lambda)
			= z(0),
		&&
			x^\mu(\lambda)
			= x^\mu(0) + z(0)^2\,p^\mu \lambda,
	\end{aligned}
	\label{eq:null_type}
	\end{equation}
	
	From now on we shall focus on the $p^2 \ne 0$ situation. Note that we can complete the square on the right hand side of the $k^\mu$ conservation such that:
	\begin{equation}
		k_\mu + p_\mu a^2
		= p_\mu \pqty{
			z^2 + (x - a)^2
		},
	\quad \Delta = p_\mu a^\mu
	\end{equation}
	For $p^\mu \ne 0$, we can always find some $c_\mu$ such that $c_\mu p^\mu \ne 0$, therefore we have:
	\begin{equation}
		z^2 + \pqty{x - a}^2
		= \frac{c_\mu k^\mu}{c_\mu p^\mu}
			+ a^2
		= \pqty{\frac{L}{2}}^{\!\!2}
		= \mrm{const.},
	\quad \Delta = p_\mu a^\mu
	\end{equation}
	
	We see that $\gamma$ lands on a ``sphere'' centered at $z = 0, x = a$ in $\mbb{R}^{d+1}$; with the Lorentzian metric, it is actually a hyperboloid. Note that for now the radius $\pqty{\frac{L}{2}}^{\!2}$ can actually be negative or zero; more specifically,
	\begin{enumerate}[noitemsep]
	\item $L^2 > 0$: one-sheet hyperboloid;
	\item $L^2 < 0$: two-sheet hyperboloid;
	\item $L^2 = 0$: conic surface. 
	\end{enumerate}
	On the other hand, we can actually solve $X^I$ completely by combining \eqref{eq:p_conserved}, \eqref{eq:tangent_norm}; we have:
	\begin{equation}
		\dv{x^\mu}{\lambda} = z^2 p^\mu,
	\quad
		\dv{z}{\lambda}
		= \pm z\sqrt{\delta - z^2 p^2},
	\quad
		\dv{x^\mu}{z}
		= \pm \frac{zp^\mu}{\sqrt{\delta - z^2 p^2}},
	\quad
		\delta - z^2 p^2
		= \pqty{\tfrac{V^z}{z}}^{\!2} \ge 0
	\end{equation}
	\\[-1.5\baselineskip]
	\begin{equation}
	\Longrightarrow\quad
		\gamma \ \subset\ % 
		x^\mu
		= a^\mu \pm\frac{p^\mu}{p^2}
			\sqrt{\delta - z^2 p^2},
	\quad
		z^2 + (x - a)^2
		= \frac{\delta}{p^2}
		= \pqty{\frac{L}{2}}^{\!\!2}
	\label{eq:unified_geodesic}
	\end{equation}
	
	This confirms our observation above, and further reveals that $\gamma$ lies in a ``plane'' in the $\mbb{R}^d$ subspace consisting of the $x^\mu$ coordinates. The behavior of $\gamma$ is sensitive to the sign of $\delta$ and $p^2$; more specifically,
	\begin{enumerate}
	\item $\delta = -1$, i.e.~for timelike $\gamma$, we must have $p^2 < 0$, and $z \ge \frac{1}{\abs{\norm{p}}} = \frac{L}{2} > 0$, namely $z$ is bounded from below. This means that timelike geodesic can never reach the asymptotic boundary $z \to 0$. In this case, $\gamma$ is a section of the one-sheet hyperboloid. 
	\item $\delta = +1$, i.e.~for spacelike $\gamma$, we can have $p^2 > 0$ or $p^2 < 0$.
		\begin{enumerate}
		\item For $p^2 > 0$, again we have $\frac{L}{2} = \frac{1}{\norm{p}} > 0$, and $\gamma$ is again a cross section of the one-sheet hyperboloid. However, now we have $z \le \frac{L}{2}$, namely $z$ is bounded from above, and:
		\begin{equation}
			z\to 0,
		\quad
			x^\mu
			= a^\mu \pm \frac{p^\mu}{p^2}
			= a^\mu \pm \frac{L}{2}\,\widehat{p}^\mu,
		\quad
			\widehat{p}^\mu = \frac{p^\mu}{\norm{p}}
		\end{equation}
		We can also nicely parametrize $X^I$ in terms of the proper length $\lambda$; for convenience, set $x^\mu(0) = a^\mu$, $z(0) = \frac{L}{2}$, then we have:
		\begin{equation}
			z(\lambda)
			= \frac{L}{2} \frac{1}{\cosh \lambda},
		\quad
			x^\mu(\lambda)
			= a^\mu \pm \widehat{p}^\mu \frac{L}{2}
				\tanh \lambda
		\label{eq:elliptic}
		\end{equation}
		
		\item For $p^2 < 0$, we have $(\frac{L}{2})^2 = \frac{1}{p^2} < 0$, and $\gamma$ is now a cross section of the two-sheet hyperboloid. Again as $z\to 0$ it lands at $x^\mu = a^\mu \pm \frac{L}{2}\,\widehat{p}^\mu$; however, $\abs{x^\mu}$ grows with $z$ and extends into the bulk instead of returning to the boundary, i.e.~$z,\abs{x^\mu} \to \infty$. 
		
		The differential equation in this case is almost the same as in (a), but now we have to choose a different initial condition, since $\gamma$ won't even reach $x = a$. However, we can actually set $x(0),z(0) \to \infty$; we then have:
		\begin{equation}
			z(\lambda)
			= \abs{\frac{L}{2}} \frac{1}{\sinh \lambda},
		\quad
			x^\mu(\lambda)
			= a^\mu \pm \widehat{p}^\mu \abs{\frac{L}{2}}
				\coth \lambda
		\label{eq:hyperbolic}
		\end{equation}
		This is an evidence that $z\to\infty$ might not be the ``end of the world'' after all; it's likely that $z\to\infty$ is only a horizon, since a spacelike geodesic $\gamma$ can reach it within finite proper length $\lambda$. 
		
		\end{enumerate}
	
	\end{enumerate}
	
	In summary, there are 3 types of spacelike geodesics in Poincar\'e AdS, which closely resemble the 3 types of conic sections:
	\begin{enumerate}
	\item $p^2 = 0$: parabolic, given in \eqref{eq:null_type}, with one end going to $z\to 0$ and the other end going to $z\to\infty$; it takes infinite proper length $\lambda$ for it to reach $0$ or $\infty$. 
	\item $p^2 > 0$: hyperbolic, given in \eqref{eq:unified_geodesic} and \eqref{eq:hyperbolic}, also with one end going to $z\to 0$ and the other end going to $z\to\infty$, but it reaches $\infty$ within finite $\lambda$. 
	\item $p^2 > 0$: elliptic, given in \eqref{eq:unified_geodesic} and \eqref{eq:elliptic}, with both ends going to $z\to 0$, and it takes infinite proper length $\lambda$ for it to reach either end. 
	\end{enumerate}
	
	Now consider two points $x_1,x_2$ near the boundary $z = \epsilon$ connected by a spacelike geodesic $\gamma$. This can only be the ``elliptic type'' discussed above. We have:
	\begin{equation}
		x^\mu
		= a^\mu \pm \widehat{p}^\mu
			\sqrt{\pqty{\frac{L}{2}}^{\!\!2} - z^2},
	\quad
		z^2 + (x - a)^2
		= \pqty{\frac{L}{2}}^{\!\!2},
	\end{equation}
	\\[-1.5\baselineskip]
	\begin{equation}
		a = \frac{x_1 + x_2}{2},
	\quad
		\widehat{p} \propto x_2 - x_1
	\end{equation}
	It's length is then given by:
	\begin{equation}
	\begin{aligned}
		A
		&= \int_{z=-\epsilon}^{z=\epsilon}
			\frac{\dd{x}^2 + \dd{z}^2}{z^2} \\
		&= \int_{-\Lambda}^{\Lambda}
			\frac{
				(\dd{\tanh\lambda})^2
				+ (\dd{\sech\lambda})^2
			}{(\sech\lambda)^2},
		\quad \Lambda
			= \cosh^{-1} (\tfrac{L}{2\epsilon}) \\
		&= \int_{-\Lambda}^{\Lambda}
			\dd{\lambda}
		= 2\Lambda
		= 2\cosh^{-1} \pqty{\frac{L}{2\epsilon}}
		\sim 2\log \frac{L}{\epsilon},
	\quad \epsilon \to 0
	\end{aligned}
	\end{equation}

\section{Einbein Action}
	The einbein action of a point particle is given by:
	\begin{equation}
		S[\eta,X] = \frac{1}{2} \int \dd{\tau} \pqty{
				\eta^{-1} \dot{X}_\mu \dot{X}^\mu
				- \eta m^2
			}
	\end{equation}
	Under worldline reparametrization: $\tau \mapsto \tau' = f(\tau)$, we have $X'(\tau') = X(\tau)$, i.e.~$X^\mu$ transforms like a \textbf{scalar} under worldline diffeomorphism; $
		X(\tau)
		\mapsto X'(\tau)
		= X\pqty{f^{-1}(\tau)}
	$. 
	
	On the other hand, $\eta$ should be treated like an einbein: $\eta = \sqrt{-\gamma}$, here $\gamma = \gamma_{\tau\tau}$ is the worldline metric; $\gamma < 0$ due to the Lorentzian signature. We have:
	\begin{gather}
		\eta = \sqrt{-\gamma}
		\longmapsto \eta' = \eta\,
			\det \pdv{\tau}{\tau'}
		= \eta \pdv{\tau}{\tau'},
	\\[1ex]
		\eta^{-1}
		= - \sqrt{-\gamma}\,\gamma^{-1}
		\longmapsto (\eta')^{-1} = \eta^{-1}
			\pqty{\pdv{\tau}{\tau'}}
			\pqty{
				\pdv{\tau'}{\tau}
				\pdv{\tau'}{\tau}
			}
		= \eta^{-1} \pdv{\tau'}{\tau}
	\end{gather}
	It is clear that the action is invariant under the transformation:
	\begin{equation}
	\begin{aligned}
		S'
		&= \frac{1}{2} \int \dd{\tau'} \pqty\Big{
				(\eta')^{-1}
					\pdd{\tau'} X_\mu
					\pdd{\tau'} X^\mu
				- (\eta') m^2
			} \\
		&= \frac{1}{2} \int \dd{\tau}
			\pdv{\tau'}{\tau}\,
			\pqty{
				\eta^{-1}
					\pdv{\tau'}{\tau}
					\cdot
					\pdv{\tau}{\tau'}
					\pdd{\tau} X_\mu
					\cdot
					\pdv{\tau}{\tau'}
					\pdd{\tau} X^\mu
				- \eta\,\pdv{\tau}{\tau'}\,m^2
			} = S
	\end{aligned}
	\end{equation}
	
	We can eliminate $\eta$ classically by placing it on shell:
	\begin{equation}
		0 = \fdv{S}{\eta}
		= - \eta^{-2} \dot{X}_\mu \dot{X}^\mu
			- m^2,
	\quad
		\eta[X] = \frac{1}{m}
			\sqrt{-\dot{X}_\mu \dot{X}^\mu}
	\end{equation}
	Substitute this back to the action, and we have:
	\begin{equation}
	\begin{aligned}
		S[X] = S[\eta = \eta[X], X]
		&= \frac{1}{2} \int \dd{\tau} \pqty{
				m\,(
					-\dot{X}_\mu \dot{X}^\mu
				)^{-\frac{1}{2}}
				\dot{X}_\mu \dot{X}^\mu
				- m\,(
					-\dot{X}_\mu \dot{X}^\mu
				)^{+\frac{1}{2}}
			} \\
		&= -m \int \dd{\tau}
			\sqrt{-\dot{X}_\mu \dot{X}^\mu}
	\end{aligned}
	\end{equation}

\section{Ricci Tensor for Static Spherical Metric}
	Consider the metric:
	\begin{equation}
	\begin{aligned}
		\dd{s}^2
		= - f(r) \dd{t}^2
			+ h(r) \dd{r}^2
			+ r^2 \pqty{
				\dd{\theta}^2
				+ \sin^2\theta \dd{\phi}^2
			}
		= \eta_{ab}\, e^a e^b
	\end{aligned}
	\end{equation}
	Here we've defined the following vierbein:
	\begin{equation}
	\begin{aligned}
		e^t &= \sqrt{f(r)} \dd{t} \\
		e^r &= \sqrt{h(r)} \dd{r} \\
		e^\theta &= r \dd{\theta} \\
		e^\phi   &= r \sin\theta \dd{\phi} \\
	\end{aligned}
	\end{equation}
	
	
	
	
	
\appendix

\section{Derivation of the special conformal transformations}
\label{sect:special_conformal}

	Special conformal transformations can be understood as translations conjugated by \textit{inversions}. Note that $\frac{\dd{z}^2}{z^2}$ is invariant under $z\mapsto \frac{1}{z}$; if we include the $x^\mu$ directions, we can consider:
	\begin{equation}
		\mcal{I}\colon\ \chi^I \mapsto
			\frac{\chi^I}{\chi^2},
%			= \frac{\chi^I}{x^2 + z^2}
	\quad
		\chi^2 = -t^2 + \vec{x}^2 + z^2,
	\end{equation}
	\\[-1.5\baselineskip]
	\begin{equation}
		\mcal{I}^2 = \idty,
	\quad
		\dd{s}^2
		\mapsto \pqty{
				\frac{
					\delta^I_J
					- 2\,\frac{\chi^I \chi_J}{\chi^2}
				}{\chi^2}
				\dd{\chi^J}
			}^{\!\!2}
			\bigg/ \pqty{
				\frac{z}{\chi^2}
			}^{\!\!2}
		= \frac{\dd{\chi}^2}{z^2}
		= \dd{s}^2
	\end{equation}
	We see that inversion $\mcal{I}$ is indeed a (discrete) symmetry of the metric. 
	Here we've defined yet another lower case variable $\chi^I \sim (x^\mu, z)$, which as a contravariant vector has the same components as $X^I$, but with an index that should be lowered by the flat metric $\eta_{IJ}$, i.e.~$\chi_I = \eta_{IJ} \chi^J = \eta_{IJ} X^J$. 
	The $d$ special conformal generators are then given by:
	\begin{equation}
	\begin{aligned}
		k_\mu
		&= \pdv{a^\mu} \pqty{
				\mcal{I}
				\circ e^{a^\nu P_\nu}
				\circ \mcal{I}
				\circ X^I
			}_{\!a = 0} \ \pdv{X^I} \\
		&= \pdv{a^\mu}\pqty{
				\frac{\frac{\chi^I}{\chi^2} + a^I}{
					\abs\big{\frac{\chi^J}{\chi^2} + a^J}^2
				}
			}_{\!\!a = 0} \,\pdv{X^I} \\
		&= \pdv{a^\mu}\pqty{
				\frac{\chi^I + a^I \chi^2}{
					1 + 2a^I \chi_I + a^2 \chi^2
				}
			}_{\!\!a = 0} \,\pdv{X^I} \\
		&= \chi^2 \pdd{\mu}
			- 2x_\mu X^I \pdd{I} \\
		&= \chi^2 \pdd{\mu}
			- 2x_\mu \Delta
%	\\
%		&= \pqty{
%				\chi^2 \delta^I_\mu
%				- 2x_\mu X^I
%			} \pdd{I}
	\end{aligned}
	\end{equation}
	
	

\printbibliography[%
%	title = {参考文献} %
	,heading = bibintoc
]
\end{document}
